\chapter{運作概述}
\subsection{我們旨在解決的挑戰}

\begin{enumerate}
\item \textbf{集中式控制}:運算(特別是在雲端運算、高效能運算及人工智慧(AI)領域)在現代社會的重要性不言而喻。然而,這些關鍵資源主要由大型企業控制,限制了大多數使用者的優勢。我們認為解決方案在於去中心化市場,該市場將民主化運算資源的存取,促進更開放且包容的環境。在此類系統中,使用者不僅是消費者,更是能影響運算發展軌跡並在運算未來中擁有權益的貢獻者。

\item \textbf{低效率}:當前的運算資源分配模式常導致失衡,造成資源未充分利用或過度飽和。我們的專案旨在創建一個能高效配對運算能力需求與可用資源的平台,從而優化利用率並減少浪費。

\item \textbf{高成本}:目前,大多數使用者面臨不必要的高運算成本。我們的願景是建立一個市場平台,以具競爭力的價格提供對各種運算能力、儲存方案、軟體應用、資料及服務的直接存取。這不僅降低整體成本,也擴大了使用者群。

\item \textbf{缺乏透明度}:現有的運算資源分配系統在價格、可用性及服務品質方面缺乏透明度。我們旨在建立一個開放且公正的平台,賦予使用者基於資源、提供者及價格的可靠資訊做出明智決定的能力。

\item \textbf{缺乏使用者賦能}:對我們大多數人來說,執行需要運算的構想可能是一個繁瑣的過程,通常需要依賴第三方服務。例如,人們必須依賴政府機關進行模擬後的電視天氣預報,或是必須將個人資料委託給集中式機構才能為自己建立數位分身。我們的專案旨在建立一個去中心化市場,提供所有必要的運算資源,使使用者能夠在維持完全控制權的同時執行任何他們想要的運算。
\end{enumerate}

對於現代社會這項重要的發展方向,我們需要解決建立去中心化綜合生態系的挑戰,以實現更易存取、高效的運算資源分配與利用。

\subsection{我們的解決方案大綱}

\begin{enumerate}
\item 我們即將推出一個作為開放且民主組織運作的平台。該平台類似於運算資源的市場,讓人聯想到淘寶等平台(即「算力淘寶平台」)。此架構的所有權由CyberPlaza Token (CPT) 的所有持有者分佈,這些持有者是我們平台的「股東」。

\item \textbf{支付系統}:我們的平台使用USDC進行所有交易,確保法規遵循、價格透明度及熟悉的使用者體驗。這消除了與專屬穩定幣相關的風險,並與全球監管架構保持一致。

\item 在平台上,服務提供者(SPs)將其運算資源(包括運算能力、儲存空間、軟體應用、資料及服務)上架,供使用者根據需求選擇。作為服務回報,SPs直接獲得USDC支付,並根據交易量獲得CPT代幣獎勵。

\item 平台本身並不擁有所上架的運算資源。然而,它可以透過「團購」採購運算資源,再轉售給使用者。此模式類似於拼多多的商業策略,使用去中心化流動性池,社群成員可以存入USDC以獲得回報,同時支援平台營運。

\item 我們的平台是開放且包容的。任何人擔任四種角色中的任一或所有角色都沒有限制:平台「股東」、流動性提供者、SP及使用者。這種彈性使參與者能夠以最適合其需求和能力的方式參與平台。
\end{enumerate}
