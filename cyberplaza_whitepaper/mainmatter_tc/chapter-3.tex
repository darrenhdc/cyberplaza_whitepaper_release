\chapter{營運概述}
\subsection{我們旨在解決的挑戰}

\begin{enumerate}
\item \textbf{集中式控制}:在現代社會中,運算的重要性日益提升,尤其在雲端運算、高效能運算與人工智慧(AI)領域,這一點毋庸置疑。然而,這些關鍵資源主要由大型企業控制,多數使用者難以充分享有其優勢。我們認為解決之道在於去中心化市集,讓運算資源的存取民主化,培養更開放、更具包容性的環境。在這類系統中,使用者不僅是消費者,更是能影響運算發展軌跡、並在運算未來中擁有權益的貢獻者。

\item \textbf{低效問題}:當前的運算資源分配模式常導致失衡,造成資源使用率不足或過度飽和。我們的專案旨在打造一個平台,有效地將運算能力需求與可用資源匹配,從而優化使用率並減少浪費。

\item \textbf{高成本問題}:目前,多數使用者面臨不必要的高運算成本。我們的願景是建立一個市集平台,以具競爭力的價格直接提供多樣化的運算能力、儲存方案、軟體應用、資料與服務。這不僅能降低整體成本,也能擴大使用者群體。

\item \textbf{缺乏透明度}:現有運算資源分配系統在定價、可用性與服務品質方面缺乏透明度。我們旨在建立一個開放且公正的平台,讓使用者能依據關於資源、供應商與定價的可靠資訊,做出明智的決策。

\item \textbf{缺乏使用者賦能}:對我們大多數人來說,執行需要運算的想法可能是繁瑣的過程,往往需要依賴第三方服務。例如,人們必須依賴政府機構進行模擬後製作的電視氣象預報,或是必須將個人資料委託給集中式實體,才能為自己建立數位分身。我們的專案旨在打造一個去中心化市集,提供所有必要的運算資源,讓使用者能執行任何他們想要的運算,同時維持完整的控制權。
\end{enumerate}

針對現代社會這項重要的發展方向,我們需要解決建立去中心化綜合生態系的挑戰,以實現運算資源更易於存取、更高效的分配與利用。

\subsection{我們的解決方案大綱}

\begin{enumerate}
\item 我們將推出一個以開放、民主組織形式運作的平台。該平台類似於運算資源的市集,讓人聯想到淘寶等平台(即「算力淘寶平台」)。此架構的所有權由所有CyberPlaza Token(CPT)持有者共同持有,這些持有者是我們平台的「股東」。

\item \textbf{支付系統}:我們的平台使用USDC進行所有交易,確保符合監管要求、價格透明且提供使用者熟悉的體驗。這消除了與專屬穩定幣相關的風險,並與全球監管架構一致。

\item 在平台上,服務供應商(SP)列出其運算資源──包括運算能力、儲存空間、軟體應用、資料與服務──供使用者依據需求選擇。服務供應商提供服務後,直接收到USDC付款,並根據其交易額獲得CPT代幣獎勵。

\item 平台本身並不擁有列出的運算資源。然而,它可以透過「團購」購買運算資源,再轉售給使用者。此模式類似於拼多多的業務策略,使用去中心化流動性池,社群成員可存入USDC以獲得利潤,同時支援平台營運。

\item 我們的平台是開放且具包容性的。任何人擔任四種角色中的任何一種或所有角色都沒有限制:平台「股東」、流動性提供者、SP與使用者。這種靈活性使參與者能以最符合其需求與能力的方式與平台互動。
\end{enumerate}
