\chapter{營運角色說明}

\section{營運的4種角色}

平台生態系由四種不同角色組成:平台角色、服務提供者(SP)角色、流動性提供者角色及使用者角色。任何人擔任或離開任一或所有4種角色均不受限制。

\subsection{平台角色}

\subsubsection{所有權與參與權}

平台是由所有CyberPlaza代幣(CPT)持有者擁有的開放民主組織。任何人可透過以下方式參與專案以取得CPT:(i)向平台提供服務,(ii)擔任流動性提供者,(iii)擔任服務提供者(SP),(iv)擔任使用者,或(v)在二級市場購買CPT。

\subsubsection{平台功能}

平台扮演分銷商、媒合者與保證者的角色,確保使用者、服務提供者(SP)與流動性提供者之間的信任並促進交易。平台維護「認證服務提供者」(CSP)清單與「一般」服務提供者(SP)清單。CSP指的是服務價值超過特定門檻的業者(目前定義為「針對未來10天內的銷售,每月提供價值\$10,000+ USDC的服務」)。「一般」SP指的是服務價值低於該門檻的業者。平台第一階段僅以CSP為主,稍後再引入一般SP。

平台將評估CSP,考量的因素包括其追蹤記錄、聲譽及CSP的績效指標,並將評估結果列於平台上,供使用者做出明智決策。「一般」SP不會接受評估,使用者可自行選擇使用。平台作為可信賴的中間人,增加了一層問責機制,並提高了SP依據其服務水準協議(SLA)履行承諾的可能性。平台從交易費用中抽取一部分收益(使用者支付價格與SP取得價格之間的差額)。

\subsubsection{準備金管理}

平台負責運作由存入USDC代幣的流動性提供者所建立的USDC「準備金」,該準備金為Web3專案的貨幣,用於市場交易。準備金將透過多種方式為USDC代幣持有者產生利息,進而使USDC代幣的鑄造成為高收益投資。這些方式包括透過團購以折扣價取得運算資源再轉售給使用者,亦即利用準備金經營「算力拼多多(Pinduoduo)」業務。團購將針對全球主要雲端服務商,包括AWS、Azure、Google Cloud、Alibaba Cloud等,以及美國、歐洲及中國等地的超級電腦中心。平台亦透過投資可產生收益的運算資產(例如比特幣挖礦設施)獲取利潤,並透過高流動性的去中心化金融或傳統金融投資獲取利息。

\subsubsection{平台參與與未來擴展}

平台亦可根據需要參與其他角色(流動性提供者、SP及使用者),以引導流動性並確保服務品質。若CPT持有者透過治理機制表決同意,平台未來可將淘寶平台(Taobao)及拼多多營運模式(Pinduoduo)擴展至算力(computing resources)以外的領域。

\subsection{服務提供者(SP)角色}

\subsubsection{註冊與服務上架}

SP在平台上註冊服務,向使用者提供運算能力(核心時數、儲存空間、頻寬、應用軟體、資料及服務等)。SP在平台上列出其不同時段的運算資源可用狀況(例如未來24小時的1,000小時Intel Core i7核心時數、未來一個月的10,000核心時數)及價目表,供使用者使用/預訂。SP亦需上傳其提供資源的各項基準測試結果(依平台要求)以及其服務水準協議(SLA)。

\subsubsection{付款與獎勵}

當SP的服務被使用者選擇並使用時,他們將直接收到USDC付款。此外,他們可獲得依交易額比例給予的CPT代幣獎勵(交易金額的2--5\%等值CPT)。質押CPT代幣的SP亦可獲得平台費用減免及提高在市場上的曝光度。

\subsubsection{品質保證}

平台的評估系統會驗證CSP的品質與可靠性,確保所有上架的主要SP(CSP)均值得信賴。平台透過利用聲譽系統、使用者評論及績效指標,為CSP建立基於績效的排名系統。該評估系統讓使用者在選擇SP進行大量使用時能做出明智決策,降低選擇不可靠或不適合SP的機率。

\subsubsection{彈性服務配置}

使用者可選擇組合多個SP來執行單一運算任務,例如使用CSP負責運算的主要部分,並使用「一般」SP(例如使用者自行提供的筆記型電腦)負責資料分析的最後階段。平台會提供所選擇CSP的評估結果,但不會提供非認證SP的評估結果。

\subsection{流動性提供者角色}

\subsubsection{概述}

流動性提供者是將USDC存入平台去中心化借貸池以支援團購及平台營運所需營運資金的參與者。此角色以更透明且去中心化的模式取代了先前的「啟動者」(Enabler)概念。

\subsubsection{流動性提供機制運作方式}

流動性提供機制運作方式如下:參與者將USDC存入經審核的智慧合約,並收到代表其存款的rUSDC代幣(收據代幣)。平台將集資的資金用於團購營運及營運資金。參與者可依據資金池的流動性狀況提領存款。

\subsubsection{可及性}

任何人(包括SP、使用者及外部投資者)均可透過將USDC存入資金池成為流動性提供者。最低存款額的設計兼顧了可及性與有意義的貢獻。

\subsubsection{報酬與福利}

流動性提供者可透過多種機制獲得報酬。他們可從平台營運利潤中獲得6--8\%年百分率報酬(APY)的USDC利息收入,以及額外2--4\%年百分率報酬(APY)的CPT代幣獎勵(具有歸屬期),合計預期總報酬率為8--12\% APY。除了財務報酬外,他們透過累積CPT獲得具投票權的治理權,以及包括費用減免、優先存取權及產品搶先體驗在內的平台福利。風險保護透過智慧合約審核、保險基金(覆蓋率10\%)及透明追蹤來確保。

\subsection{使用者角色}

\subsubsection{存取運算資源}

使用者可透過簡單的流程存取平台上的運算資源:(i)將USDC存入其平台錢包,(ii)從市場瀏覽並選擇服務提供者,以及(iii)透過平台入口提交運算任務並以USDC付款。

\subsubsection{具競爭力的定價}

透過將平台做為運算資源的淘寶平台(算力淘寶平台)使用,使用者可以具競爭力的價格存取最適合自己的運算資源,由於團購優惠,價格通常比直接向雲端服務商購買低10--30\%。

\subsubsection{付款保護與透明度}

平台實施全面的付款保護與透明度措施。智慧合約託管會持有USDC付款,直到確認服務交付為止,若SP未達到SLA要求,則會自動退款。該系統確保定價透明且無隱藏費用、即時績效監控與報告,以及透過平台治理的爭議解決機制。

\subsubsection{使用者獎勵計畫}

使用者可透過平台互動,透過多種獲取機制賺取CyberPlaza代幣(CPT)。消費獎勵為使用者提供消費金額1--3\%的CPT代幣。推薦獎金讓使用者可透過介紹新使用者或SP加入平台賺取CPT。忠誠等級為持續使用平台的使用者提供更高的獎勵,而品質回饋機制則讓使用者可透過提供詳細的服務評論賺取CPT。

持有與質押CPT的好處相當可觀。使用折扣讓使用者可質押CPT以獲得服務費用5--15\%的折扣。收益分潤讓質押的CPT可獲得平台收益分配。治理權允許使用者對平台參數及功能優先順序進行投票。高級功能提供進階工具、分析工具及API服務的存取權。
