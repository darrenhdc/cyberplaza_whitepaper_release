\chapter{營運角色說明}

\section{營運的4種角色}

平台生態系由4種不同角色組成:平台角色、服務提供商(SP)角色、流動性提供商角色及使用者角色。任何人擔任或離開任一或所有4種角色均無限制。

\subsection{平台的角色}

\subsubsection{所有權與參與}

平台是由所有CyberPlaza代幣(CPT)持有者所擁有的開放民主組織。任何人可透過以下方式參與專案以取得CPT:(i) 向平台提供服務;(ii) 擔任流動性提供商;(iii) 擔任服務提供商(SP);(iv) 擔任使用者;或(v) 在二級市場購買CPT。

\subsubsection{平台功能}

平台扮演分銷商、媒合者及保證者的角色,確保信任並促進使用者、服務提供商(SP)與流動性提供商之間的交易。平台維護一份認證服務提供商(CSP)清單以及一份「一般」SP清單。CSP指的是所提供的服務價值超過特定門檻的業者(目前定義為``在未來10天內的銷售中,每月提供價值\$10,000+ USDC的服務'')。「一般」SP指的是所提供的服務低於該門檻的業者。平台的第一階段將僅以CSP開始運作,稍後再引入SP。

平台將評估CSP,考量其實績紀錄、聲譽及績效指標等因素,並將評估結果列於平台上,讓使用者能做出知情決策。一般SP不接受評估,使用者可自行選擇使用。平台身為可信賴仲介的角色,增添了一層問責機制,並提高SP依照其服務水準協議(SLA)履行承諾的可能性。平台收取交易費的一部分(即使用者支付的價格與SP獲得的價格之間的差額)。

\subsubsection{準備金管理}

平台負責管理由存入USDC代幣的流動性提供商所設立的USDC計價「準備金」,該準備金是Web 3專案的貨幣,用於市場中的交易。準備金將透過多種方式為USDC代幣持有者產生利息,因此使USDC代幣的鑄造成為一種高收益投資。這些方式包括透過團購(團購)以折扣價取得運算資源再轉售給使用者,也就是利用準備金開展算力拼多多(Pinduoduo)業務。團購將來自全球主要雲端服務,包括AWS、Azure、Google Cloud、Alibaba Cloud等,以及美國、歐洲及中國等地區的超級運算中心。平台也透過投資可產生收入的計算資產(例如比特幣挖礦設施)獲利,並透過流動性高的去中心化金融或傳統金融投資獲得利息。

\subsubsection{平台參與與未來擴展}

平台也可依需要參與其他角色(流動性提供商、SP及使用者),以啟動流動性並確保服務品質。若CPT持有者透過治理機制投票同意,平台未來可將淘寶平台(Taobao)及拼多多營運(Pinduoduo)模式延伸至算力(計算資源)以外的領域。

\subsection{服務提供商(SP)的角色}

\subsubsection{註冊與服務上架}

SP在平台上註冊其服務,以向使用者提供運算資源(核心時數、儲存空間、頻寬、應用軟體、資料及服務等)。SP在平台上列出其不同時段的運算資源可用性(例如未來24小時內的1,000小時Intel Core i7核心時數、未來一個月內的10,000核心時數)及價目表,供使用者使用/預訂。SP也將上傳其提供之資源的各種效能基準(依平台要求)以及其SLA。

\subsubsection{付款與獎勵}

SP的服務被使用者選擇並使用時,將直接收到USDC付款。此外,他們將依據交易額獲得CPT代幣獎勵(相當於交易價值的2--5\% CPT)。質押CPT代幣的SP也可享有平台費減免及市場曝光度提升的優惠。

\subsubsection{品質保證}

平台的評估系統驗證CSP的品質與可靠性,確保所有列出的主要SP(CSP)均可信賴。平台透過運用聲譽系統、使用者評價及績效指標,為CSP建立一套以實績為基礎的排名系統。該評估系統讓使用者在選擇SP進行任何大量使用時能做出知情決策,降低選擇不可靠或不適合之SP的機率。

\subsubsection{彈性服務配置}

使用者可針對一項工作選擇組合使用多個SP,例如使用CSP進行主要運算部分,同時使用``一般'' SP(例如使用者自己提供的筆記型電腦)進行最後階段的資料分析。平台為所選擇的CSP提供評估,但不為非認證SP提供評估。

\subsection{流動性提供商的角色}

\subsubsection{概述}

流動性提供商是將USDC存入平台去中心化借貸池的參與者,以支持團購及平台營運的營運資金。此角色實現了透明且去中心化的平台營運資金模式。

\subsubsection{流動性提供機制運作方式}

流動性提供機制運作如下:參與者將USDC存入經審計的智慧合約,並收到代表其存款的rUSDC代幣(收據代幣)。平台將集合資金用於團購營運及營運資金。參與者可依據池內流動性可用情況提取存款。

\subsubsection{可及性}

任何人(包括SP、使用者及外部投資者)均可透過將USDC存入池中成為流動性提供商。最低存款額的設計旨在確保可及性的同時,維持有意義的貢獻。

\subsubsection{收益與福利}

流動性提供商透過多種機制獲得收益。他們將從平台營運利潤中獲得以USDC支付的6--8\%年收益率(APY)利息收入,以及額外提供2--4\% APY CPT代幣的CPT代幣獎勵(帶有鎖倉機制),合計預期總收益率為8--12\% APY。除財務收益外,他們透過累積CPT獲得治理權(享有投票權),以及平台福利,包括費用減免、優先存取權及產品搶先體驗。風險保護透過智慧合約審計、保險基金(10\%覆蓋率)及透明追蹤機制確保。

\subsection{使用者的角色}

\subsubsection{存取運算資源}

使用者可透過簡單流程存取平台上的運算資源:(i) 將USDC存入其平台錢包;(ii) 瀏覽並從市場中選擇服務提供商;(iii) 透過平台入口網站提交工作並以USDC付款。

\subsubsection{具競爭力的定價}

透過將平台做為運算資源的淘寶(算力淘寶平台),使用者可以具競爭力的價格存取最適合他們的運算資源,由於團購優惠,價格通常比直接向雲端提供商購買低10--30\%。

\subsubsection{付款保護與透明化}

平台實施全面的付款保護與透明化措施。智慧合約託管將保留USDC付款,直到服務交付確認為止,若SP未達到SLA要求,將自動退款。該系統確保定價透明且無隱藏費用、即時績效監控與報告,以及透過平台治理的爭議解決機制。

\subsubsection{使用者獎勵計畫}

使用者可透過多種賺取機制參與平台活動以獲得CyberPlaza代幣(CPT)。消費獎勵提供使用者消費金額的1--3\%作為CPT代幣。推薦獎金允許使用者透過引進新使用者或SP至平台賺取CPT。忠誠等級為持續使用平台的使用者提供更高獎勵,而品質回饋機制則讓使用者透過提供詳細的服務評價賺取CPT。

持有及質押CPT的福利相當豐厚。使用折扣允許使用者質押CPT以獲得服務費5--15\%的折扣。收入分潤使質押的CPT能夠獲得平台收入分配。治理權允許對平台參數及功能優先順序進行投票。高級功能提供進階工具、分析及API服務的存取權。
