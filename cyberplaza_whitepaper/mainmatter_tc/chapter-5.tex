\chapter{CyberPlaza Token (CPT) 與代幣經濟學}
\subsection{CPT 代幣概述與實用性}

\subsubsection{支付系統}

平台使用 USDC 作為所有市場交易的主要支付貨幣。這種方式消除了與專有穩定幣相關的監管風險,同時確保符合全球穩定幣框架的監管要求、熟悉的使用者體驗(USDC 被廣泛採用且值得信賴)、透明的美元計價定價、與現有 DeFi 基礎設施的無縫整合,以及沒有演算法穩定幣失敗的風險。

\subsubsection{CyberPlaza Token (CPT)}

CPT 是平台的原生治理與實用代幣,旨在協調所有利益相關者之間的激勵措施並捕捉平台價值增長。

\subsubsection{CPT 核心實用性}

\paragraph{治理權}

CPT 持有者可以對平台參數(費用結構、收入分配比例等)進行投票,提出並投票支持新功能、合作夥伴關係和戰略方向,並參與財庫管理和資本配置決策。投票權重基於質押的 CPT 數量和鎖定期限(veToken 模型)。平台實施季度治理會議和透明的提案流程。

\paragraph{透過質押分享收入}

持有者可以質押 CPT 來獲取平台收入分配(以 USDC 支付)。平台收入的 30\% 分配給質押獎勵池(針對可持續性進行優化)。質押獎勵每周或每月分配一次(由治理決定)。較長的質押期限可獲得獎勵乘數(4 年鎖定最高可達 2.5 倍)。目標 APY 基於平台績效(更具可持續性)和質押比例,為 6--10\%。沒有無常損失風險(單資產質押)。

\paragraph{使用優惠}

質押 CPT 可提供平台服務的 5--15\% 折扣(分級系統)、獲得高級功能(包括進階分析、API 存取和優先支援)、高交易量使用者的交易費用減免、新服務和測試版功能的搶先體驗,以及高需求運算資源的優先分配。

\paragraph{生態系統激勵}

平台為所有使用者類別提供激勵。使用者可獲得消費金額的 1--3\% 的 CPT(現金回饋計畫)。服務提供者可獲得交易額的 2--5\% 的 CPT 獎勵。流動性提供者可獲得 2--4\% APY 的 CPT 代幣作為額外收益。推薦新使用者或服務提供者至平台可獲得 CPT。社區貢獻透過漏洞獎勵、內容創作和程式碼貢獻進行獎勵。

\paragraph{通縮機制}

平台收入的 20\% 用於在公開市場回購 CPT。購買的 CPT 代幣將被永久銷毀(發送至 0x0 位址),隨時間減少流通供應,創造稀缺性。平台實施透明的季度銷毀活動並提供鏈上驗證,預計 5 年內供應量將減少 30--40\%。這使所有 CPT 持有者受益,不僅限於質押者。

\subsection{收入模型與分配機制}

\subsubsection{平台收入來源}

平台透過如表~\ref{tab:revenue}所示的多個來源產生收入。

\begin{table}[htbp]
\centering
\caption{平台收入預測}
\label{tab:revenue}
\begin{tabular}{lccccr}
\hline
\textbf{收入來源} & \textbf{費率/金額} & \textbf{第1年} & \textbf{第2年} & \textbf{第3年} & \textbf{佔總額百分比} \\
\hline
\textbf{SaaS 訂閱} & \$50--500/月 & \$1.5M & \$4M & \$8--10M & \textbf{40--50\%} \\
交易費 & GMV 的 2--5\% & \$0.8M & \$2.5M & \$5--7M & \textbf{25--30\%} \\
API 與資料服務 & 變動 & \$0.3M & \$1.5M & \$3--4M & \textbf{15--20\%} \\
認證服務 & 每個 SP \$5K--50K & \$0.3M & \$0.8M & \$1--2M & \textbf{5--8\%} \\
團購利潤 & 5--10\% 利潤 & \$0.2M & \$0.7M & \$1.5--2M & \textbf{5--10\%} \\
\hline
\textbf{總收入} & --- & \textbf{\$3.1M} & \textbf{\$9.5M} & \textbf{\$19--25M} & \textbf{100\%} \\
\hline
\end{tabular}
\end{table}

與原始模型的主要變化包括:SaaS 訂閱現在作為主要收入來源(40--50\%)以確保可預測性,團購減少為補充(5--10\%),鑒於早期規模這是現實的,以及強調 API 服務(15--20\%)作為高邊際、可擴展的收入。保守預測基於第 3 年 0.01\% 的市場滲透率。

\subsubsection{SaaS 訂閱等級}

平台提供如表~\ref{tab:saas}所示的分級訂閱計畫。

\begin{table}[htbp]
\centering
\caption{SaaS 訂閱等級(示意)}
\label{tab:saas}
\begin{tabularx}{\textwidth}{llXXr}
\hline
\textbf{等級} & \textbf{每月價格} & \textbf{目標使用者} & \textbf{功能} & \textbf{預估使用者數量(第3年)} \\
\hline
免費 & \$0 & 個人 & 2 個雲端帳戶、基本監控 & 10,000+ \\
入門版 & \$50 & 小型團隊 & 5 個帳戶、成本追蹤、1\% CPT 現金回饋 & 2,000 \\
專業版 & \$200 & 開發團隊 & 10 個帳戶、AI 最佳化、API、3\% CPT & 500 \\
企業版 & \$500--2000 & 公司 & 無限、自訂整合、5\% CPT & 50--100 \\
\hline
\end{tabularx}
\end{table}

這種分級模型提供可預測的重複性收入,同時仍允許免費增值的使用者獲取。

\textbf{重要說明}:這些預測代表我們的目標場景。我們還模擬了保守場景,第 1 年收入為 \$500K--1M,以確保即使初始增長較慢,也能保持財務可持續性。我們的商業模式不依賴於立即實現大規模的團購折扣。

\subsubsection{收入分配模型}

平台收入(100\%)分配如下:質押獎勵池獲得 30\%(為可持續性而降低),並以 USDC 按比例分配給 CPT 質押者。營運與開發獲得 35\%(為增長而增加),分配給工程與產品開發(15\%)、行銷與業務開發(10\%)以及基礎設施與安全(5\%)。回購與銷毀獲得 20\%,其中 CPT 從 DEX 購買並永久銷毀。團隊與基金會獲得 10\%,用於核心團隊報酬(5\%)和基金會營運(5\%)。緊急準備金獲得 5\%,作為新的波動緩衝。

\subsubsection{質押獎勵計算範例}

考慮第 3 年平台每月收入為 \$1,500,000 的成熟平台場景。質押池分配(40\%)提供 \$600,000。如果質押的總 CPT 為 40,000,000(供應量的 40\%),而您的質押額為 10,000 CPT(質押供應量的 0.025\%),那麼您的每月獎勵為 \$600,000 $\times$ 0.025\% = \$150 USDC,您的年度獎勵為 \$150 $\times$ 12 = \$1,800 USDC。

如果 CPT 價格 = \$2,那麼您的質押價值為 \$20,000,您的 APY 為 \$1,800 / \$20,000 = 9\%。此外,您還可以獲得平台治理的投票權、服務折扣(5--15\%)以及回購/銷毀帶來的價格上漲。

\subsubsection{USDC 存款者的流動性池回報}

將 USDC 存入借貸池的流動性提供者可獲得如表~\ref{tab:liquidity}所示的回報。

\begin{table}[htbp]
\centering
\caption{流動性提供者回報}
\label{tab:liquidity}
\begin{tabular}{llll}
\hline
\textbf{組件} & \textbf{APY} & \textbf{支付貨幣} & \textbf{來源} \\
\hline
基本利息 & 6--8\% & USDC & 平台營運利潤 \\
CPT 激勵 & 2--4\% & CPT & 代幣發行(歸屬) \\
\textbf{總預期} & \textbf{8--12\%} & \textbf{混合} & \textbf{可持續收益} \\
\hline
\end{tabular}
\end{table}

主要功能包括:存款用於團購營運(鏈上透明追蹤)、逐步提款系統防止擠兌情況、保險基金覆蓋高達 10\% 的池 TVL、智慧合約由領先廠商審計,以及根據池利用率的實時 APY 更新。

\subsection{代幣分配與歸屬計劃}

\subsubsection{總供應量與分配}

總供應量為 100,000,000 CPT(固定,無通脹)。分配細分如表~\ref{tab:allocation}所示。
\begin{table}[htbp]
\centering
\caption{CPT 代幣分配}
\label{tab:allocation}
\begin{tabularx}{\textwidth}{lrrr>{\raggedright\arraybackslash}X}
\hline
\textbf{類別} & \textbf{分配} & \textbf{代幣數量} & \textbf{百分比} & \textbf{鎖定與歸屬條款} \\
\hline
\textbf{社區激勵} & \textbf{總額} & \textbf{55,000,000} & \textbf{55\%} & \textbf{基於績效的發放} \\
\quad - 使用者獎勵 & & 25,000,000 & 25\% & 基於平台 GMV 里程碑發放 \\
\quad - SP 激勵 & & 20,000,000 & 20\% & 基於交易額發放 \\
\quad - LP 獎勵 & & 10,000,000 & 10\% & 5 年發行,前端加載 \\
\textbf{基金會} & & \textbf{17,500,000} & \textbf{17.5\%} & \textbf{TGE 時發放 10\%,其餘 90\% 24 個月線性歸屬} \\
\textbf{私人銷售} & & \textbf{12,500,000} & \textbf{12.5\%} & \textbf{6 個月鎖定期,其餘 18 個月線性歸屬} \\
\textbf{團隊} & & \textbf{15,000,000} & \textbf{15\%} & \textbf{12 個月鎖定期,其餘 36 個月線性歸屬} \\
\hline
\textbf{總計} & & \textbf{100,000,000} & \textbf{100\%} & \\
\hline
\end{tabularx}
\end{table}

與原始版本的主要變化包括:社區分配從 50\% 增加到 55\%(移除了 USDC 持有者分配),投資者分配從 15\% 減少到 12.5\%(社區優先方針),團隊分配從 17.5\% 減少到 15\%(更強的一致性),以及取消了「流動性提供者」類別(替換為流動性提供者激勵)。

\subsubsection{歸屬細節}

\paragraph{社區激勵(55\%)}

使用者獎勵(25M CPT)每月根據平台 GMV 目標發放。公式為:每月發放 = 基礎金額 $\times$(實際 GMV / 目標 GMV)。分配期為 5 年,未領取的代幣結轉至下一期。

SP 激勵(20M CPT)每季度根據交易額發放。更高質量的 SP(CSP)可獲得獎勵乘數。分配期為 5 年,可基於績效加速。

LP 獎勵(10M CPT)採用前端加載發行:第 1 年(40\%)、第 2 年(30\%)、第 3--5 年(30\%)。每周分配給活躍的流動性提供者,較長期限的存款可獲得獎勵。歸屬為 50\% 立即發放,50\% 在 6 個月內歸屬。

\paragraph{團隊分配(15\%)}

團隊分配包括 12 個月的鎖定期(第一年不發放代幣)。鎖定期後,36 個月線性歸屬。總歸屬期為 4 年。歸屬合約是透明且公開可驗證的。

\paragraph{基金會分配(17.5\%)}

TGE 時發放 10\% 用於初始營運(多簽控制)。剩餘 90\% 在 24 個月內線性歸屬。這些資金用於合作夥伴關係、審計、法律、行銷和補助金,並提供季度透明度報告。

\paragraph{私人銷售(12.5\%)}

私人銷售包括 6 個月的鎖定期和鎖定期後的 18 個月線性歸屬。總歸屬期為 2 年。反拋售機制限制每日成交量的最高 5\% 出售。

\subsection{流動性激勵與 veToken 質押模型}

\subsubsection{veToken 機制(投票鎖定 CPT)}

我們實施了受 Curve Finance 啟發的 veToken 模型,該模型已證明可以協調長期激勵。使用者鎖定 CPT 以獲得 veCPT(不可轉讓)。鎖定期決定 veCPT 乘數,如表~\ref{tab:vetoken}所示。

\begin{table}[htbp]
\centering
\caption{按鎖定期劃分的 veToken 乘數}
\label{tab:vetoken}
\begin{tabular}{ll}
\hline
\textbf{鎖定期} & \textbf{veCPT 乘數} \\
\hline
1 週 & 0.01x \\
1 個月 & 0.04x \\
3 個月 & 0.25x \\
6 個月 & 0.50x \\
1 年 & 1.00x \\
2 年 & 1.50x \\
4 年 & 2.50x(最大值) \\
\hline
\end{tabular}
\end{table}

\subsubsection{veCPT 的好處}

增強的治理權力提供 1 veCPT = 1 票(與標準 CPT 相比:除非鎖定,否則 0 票),其中更長的鎖定期等於在平台方向上更強的話語權。

增強的質押獎勵包括:1 年鎖定的基礎 APY 為 8--12\%,4 年鎖定的最大增強為 2.5x,最大鎖定的增強 APY 高達 20--30\%。

費用分享優先權意味著 veCPT 持有者首先獲得收入分配,且 veCPT 餘額越高,費用池的份額越高。

專屬好處包括最大服務折扣(15\%)、過度訂閱資源的優先存取,以及專屬治理提案權(需要最低 veCPT)。

\subsubsection{流動性挖礦計劃}

第 1 階段:啟動激勵(第 1--6 個月)提供高 CPT 發行量以引導流動性。Uniswap V3 上的 CPT/USDC 池每天獲得 2000 CPT。CPT 單質押每天獲得 1500 CPT。USDC 借貸池每天獲得相當於 1000 CPT 的獎勵。

第 2 階段:增長(第 7--24 個月)涉及減少發行量,專注於可持續收益。總發行量約為每天 2500 CPT,並增加了 USDC 借貸池的權重(激勵流動性)。

第 3 階段:成熟期(第 25 個月及以後)的新發行量最少(約每天 1000 CPT)。由收入驅動的收益成為主要吸引力,回購和銷毀創造供應稀缺性。

\subsubsection{反巨鯨與公平發行機制}

平台實施了多種保護機制,包括私人銷售中單筆購買的最大限額為 \$100K、確保 TGE 時無大規模拋售的歸屬、防止治理攻擊的時間加權投票、防止挖礦和拋售的逐步發行,以及社區分配大於團隊 + 投資者(55\% > 27.5\%)。

\subsubsection{比較:傳統 vs. veCPT 模型}

表~\ref{tab:comparison}比較了傳統質押與 veCPT 模型。

\begin{table}[htbp]
\centering
\caption{傳統質押 vs. veCPT 模型}
\label{tab:comparison}
\begin{tabular}{lll}
\hline
\textbf{指標} & \textbf{傳統質押} & \textbf{veCPT 模型} \\
\hline
最低承諾 & 無 & 1 週 \\
最大獎勵 & 固定 APY & 最高 2.5x 增強 \\
治理權力 & 線性(1 代幣 = 1 票) & 時間加權 \\
長期一致性 & 低 & 高 \\
僱傭資本風險 & 高 & 低 \\
價格穩定性 & 較低 & 較高 \\
\hline
\end{tabular}
\end{table}

為什麼這個模型有效:它經過 Curve(\$CRV)的驗證,並自 2020 年以來經過實戰測試。它協調了長期持有者的激勵,降低了短期挖礦者的賣壓,創造了強大的治理參與度,並提供了不依賴於永久通脹的可持續代幣經濟學。

\subsection{上市策略與保守場景}

\subsubsection{冷啟動策略}

成功啟動雙邊市場需要仔細的順序安排。我們的方法包括三個階段。

\paragraph{第 0 階段:種子使用者(第 1--3 個月)}

目標是 50--100 個付費使用者。來源包括 ClusterTech 現有的客戶群以及 Web3 專案。激勵包括 3 個月免費試用、早期採用者終身 50\% 折扣,以及初始 CPT 空投(總預算 100K CPT)。預算約為 \$150K(行銷 + 激勵)。

\paragraph{第 1 階段:早期採用者(第 3--12 個月)}

目標是 500--1000 個付費使用者和 10 個企業客戶。策略包括推薦計劃(推薦者和被推薦者均可獲得 \$50 信用額度)、透過技術部落格和 YouTube 教學的內容行銷、黑客松贊助(Web3 社區),以及雲端轉售合作夥伴關係。預算約為 \$500K(行銷 + 銷售)。

\paragraph{第 2 階段:增長(第 12--24 個月)}

目標是 2000--5000 個使用者和 50 個企業客戶。策略包括完全啟動 CPT 質押激勵、戰略合作夥伴關係(Infura、Alchemy 等),以及會議出席和思想領導力。預算為 \$1M+(隨收入擴展)。

\subsubsection{財務場景}

為了向投資者提供透明度,我們模擬了三種場景。

\paragraph{保守場景(高機率)}

表~\ref{tab:conservative}展示了保守的財務場景。

\begin{table}[htbp]
\centering
\caption{保守財務場景}
\label{tab:conservative}
\begin{tabular}{lrrr}
\hline
\textbf{指標} & \textbf{第1年} & \textbf{第2年} & \textbf{第3年} \\
\hline
付費使用者 & 200 & 1,000 & 3,000 \\
ARPU(\$/月) & \$40 & \$60 & \$80 \\
MRR & \$8K & \$60K & \$240K \\
年度收入 & \$96K & \$720K & \$2.9M \\
營運成本 & \$600K & \$900K & \$1.5M \\
淨利 & -\$504K & -\$180K & +\$1.4M \\
累積現金 & -\$500K & -\$680K & +\$720K \\
\hline
\end{tabular}
\end{table}

\paragraph{基礎案例場景(中機率)}

表~\ref{tab:basecase}展示了基礎案例的財務場景。

\begin{table}[htbp]
\centering
\caption{基礎案例財務場景}
\label{tab:basecase}
\begin{tabular}{lrrr}
\hline
\textbf{指標} & \textbf{第1年} & \textbf{第2年} & \textbf{第3年} \\
\hline
付費使用者 & 500 & 2,500 & 8,000 \\
ARPU(\$/月) & \$50 & \$75 & \$100 \\
MRR & \$25K & \$188K & \$800K \\
年度收入 & \$300K & \$2.25M & \$9.6M \\
營運成本 & \$800K & \$1.5M & \$3M \\
淨利 & -\$500K & +\$750K & +\$6.6M \\
\hline
\end{tabular}
\end{table}

\paragraph{樂觀場景(低機率)}

表~\ref{tab:optimistic}展示了樂觀的財務場景。

\begin{table}[htbp]
\centering
\caption{樂觀財務場景}
\label{tab:optimistic}
\begin{tabular}{lrrr}
\hline
\textbf{指標} & \textbf{第1年} & \textbf{第2年} & \textbf{第3年} \\
\hline
付費使用者 & 1,000 & 5,000 & 20,000 \\
ARPU(\$/月) & \$75 & \$100 & \$150 \\
MRR & \$75K & \$500K & \$3M \\
年度收入 & \$900K & \$6M & \$36M \\
營運成本 & \$1M & \$2.5M & \$8M \\
淨利 & -\$100K & +\$3.5M & +\$28M \\
\hline
\end{tabular}
\end{table}

\paragraph{主要假設}

場景反映了不同的市場滲透率和定價能力。營運成本隨增長而擴展,但受益於規模經濟。保守場景假設團購貢獻最小。所有場景都假設主要收入來自 SaaS 和交易費。CPT 激勵成本包含在營運成本中。

\paragraph{資金需求}

種子/天使投資 \$500K--1M 將覆蓋第 1 年的虧損和產品開發。如果確認基礎案例軌跡,計劃在第 2 年進行 \$3--5M 的 A 輪融資。計劃在第 3 年及以後進行 \$10--20M 的 B 輪融資,用於國際擴張。

\paragraph{損益平衡分析}

保守場景在第 30--36 個月達到損益平衡。基礎案例在第 18--24 個月達到損益平衡。樂觀場景在第 12--18 個月達到損益平衡。

這一範圍為投資者提供了現實的預期,同時展示了可擴展性潛力。
