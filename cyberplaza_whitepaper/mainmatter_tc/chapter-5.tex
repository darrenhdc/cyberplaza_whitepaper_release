\chapter{CyberPlaza代幣(CPT)與代幣經濟學}
\subsection{CPT代幣概述與效用}

\subsubsection{支付系統}

平台使用USDC作為所有市集交易的主要支付貨幣。此方法消除了與專有穩定幣相關的監管風險,同時確保符合全球穩定幣框架的監管規定、熟悉的使用者體驗(USDC廣泛被採用且受信任)、透明的美元計價定價、與現有DeFi基礎設施的無縫整合,以及無演算法穩定幣失敗的風險。

\subsubsection{CyberPlaza代幣(CPT)}

CPT是平台的原生治理與效用代幣,旨在使所有利害關係人的誘因一致,並捕捉平台價值成長。

\subsubsection{CPT核心效用}

\paragraph{治理權}

CPT持有者可以對平台參數(費用結構、收入分配比例等)進行表決,對新功能、合作夥伴關係和戰略方向提出提案並表決,並參與財庫管理和資金配置決策。投票權重基於質押的CPT數量和鎖倉期限(veToken模型)。平台實施每季治理會議和透明的提案流程。

\paragraph{透過質押的收入分潤}

持有者可以質押CPT以獲得平台收入分配(以USDC支付)。平台收入的30\%分配到質押獎勵池(針對永續性優化)。質押獎勵每周或每月分配(由治理決定)。較長的質押期限可獲得獎勵倍數(4年鎖倉最高2.5倍)。目標年化收益率(APY)為6--10\%,基於平台績效(更永續)和質押比例。無無常損失風險(單一資產質押)。

\paragraph{使用優惠}

質押CPT可享平台服務5--15\%折扣(分級系統),存取高級功能(包括進階分析、API存取權和優先支援),高交易量使用者可享降低的交易費用,優先存取新服務和測試版功能,以及高需求計算資源的優先配置。

\paragraph{生態誘因}

平台為所有使用者類別提供誘因。使用者可獲得消費金額1--3\%的CPT(現金回饋計畫)。服務提供者的交易額可獲2--5\%的CPT獎金。流動性提供者可獲得2--4\%的CPT年化收益率作為額外收益。推薦人邀請新使用者或服務提供者至平台可獲得CPT。社群貢獻透過漏洞獎勵、內容創作和程式碼貢獻獲得回報。

\paragraph{通縮機制}

平台收入的20\%用於從公開市場買回CPT。購回的CPT代幣將永久銷毀(發送至0x0地址),這會隨時間減少流通供應量,創造稀缺性。平台實施透明的每季銷毀活動並進行鏈上驗證,預計5年內供應量減少30--40\%。這有利於所有CPT持有者,不僅限於質押者。

\subsection{收入模型與分配機制}

\subsubsection{平台收入來源}

平台透過多種管道產生收入,如表\ref{tab:revenue}所示。

\begin{table}[htbp]
\centering
\caption{平台收入預測}
\label{tab:revenue}
\begin{tabular}{lccccr}
\hline
\textbf{Revenue Stream} & \textbf{Rate/Amount} & \textbf{Year 1} & \textbf{Year 2} & \textbf{Year 3} & \textbf{\% of Total} \\
\hline
\textbf{SaaS Subscriptions} & \$50--500/month & \$1.5M & \$4M & \$8--10M & \textbf{40--50\%} \\
Transaction Fees & 2--5\% of GMV & \$0.8M & \$2.5M & \$5--7M & \textbf{25--30\%} \\
API \& Data Services & Variable & \$0.3M & \$1.5M & \$3--4M & \textbf{15--20\%} \\
Certification Services & \$5K--50K per SP & \$0.3M & \$0.8M & \$1--2M & \textbf{5--8\%} \\
Group-Buying Margins & 5--10\% margins & \$0.2M & \$0.7M & \$1.5--2M & \textbf{5--10\%} \\
\hline
\textbf{Total Revenue} & --- & \textbf{\$3.1M} & \textbf{\$9.5M} & \textbf{\$19--25M} & \textbf{100\%} \\
\hline
\end{tabular}
\end{table}

與原始模型相比的主要變更包括:SaaS訂閱現在作為主要收入來源(40--50\%)以提高可預測性;團購縮減為輔助(5--10\%),考慮到早期規模這是合理的;強調API服務(15--20\%),因其具有高利潤和可擴展性。保守預測基於第3年0.01\%的市場滲透率。

\subsubsection{SaaS訂閱級別}

平台提供分級訂閱方案,如表\ref{tab:saas}所示。

\begin{table}[htbp]
\centering
\caption{SaaS訂閱級別(說明性)}
\label{tab:saas}
\begin{tabularx}{\textwidth}{llXXr}
\hline
\textbf{Tier} & \textbf{Price/Month} & \textbf{Target Users} & \textbf{Features} & \textbf{Est. Users (Y3)} \\
\hline
Free & \$0 & Individuals & 2 cloud accounts, basic monitoring & 10,000+ \\
Starter & \$50 & Small teams & 5 accounts, cost tracking, 1\% CPT cashback & 2,000 \\
Professional & \$200 & Dev teams & 10 accounts, AI optimization, API, 3\% CPT & 500 \\
Enterprise & \$500--2000 & Companies & Unlimited, custom integration, 5\% CPT & 50--100 \\
\hline
\end{tabularx}
\end{table}

這種分級模型提供可預測的定期收入,同時仍允許免費增值使用者取得。

\textbf{Important Note}: These projections represent our target scenario. We also model conservative scenarios with Year 1 revenue of \$500K--1M to ensure financial sustainability even with slower initial growth. Our business model does not depend on achieving large-scale group-buying discounts immediately.

\subsubsection{收入分配模型}

平台收入(100\%)分配如下:質押獎勵池獲得30\%(為永續性而降低),並以USDC按比例分配給CPT質押者。營運與開發獲得35\%(為成長而增加),分配給工程與產品開發(15\%)、行銷與業務開發(10\%)以及基礎設施與安全(5\%)。買回與銷毀獲得20\%,用於從去中心化交易所(DEX)購回CPT並永久銷毀。團隊與基金會獲得10\%,用於核心團隊薪酬(5\%)和基金會營運(5\%)。緊急準備金獲得5\%,作為波動的新緩衝。

\subsubsection{質押獎勵計算範例}

考慮第3年平台成熟的情境,平台月收入為1,500,000美元。質押池分配(40\%)提供600,000美元。如果質押的總CPT為40,000,000(供應量的40\%),而您的質押量為10,000 CPT(質押供應量的0.025\%),那麼您的月獎勵為600,000美元 × 0.025\% = 150美元USDC,年獎勵為150美元 × 12 = 1,800美元USDC。

如果CPT價格=2美元,那麼您的質押價值為20,000美元,年化收益率為1,800美元 / 20,000美元 = 9\%。此外,額外福利包括平台治理的投票權、服務折扣(5--15\%)以及買回/銷毀帶來的價格上漲。

\subsubsection{USDC存款人流動性池報酬}

將USDC存入借貸池的流動性提供者可獲得報酬,如表\ref{tab:liquidity}所示。

\begin{table}[htbp]
\centering
\caption{流動性提供者報酬}
\label{tab:liquidity}
\begin{tabular}{llll}
\hline
\textbf{Component} & \textbf{APY} & \textbf{Paid In} & \textbf{Source} \\
\hline
Base Interest & 6--8\% & USDC & Platform operational profits \\
CPT Incentives & 2--4\% & CPT & Token emission (vesting) \\
\textbf{Total Expected} & \textbf{8--12\%} & \textbf{Mixed} & \textbf{Sustainable yields} \\
\hline
\end{tabular}
\end{table}

主要功能包括:存款用於團購營運(透明的鏈上追蹤);逐步提款系統防止擠兌情境;保險基金涵蓋最高10\%的池TVL;智慧合約由領先公司審計;實時年化收益率更新基於池利用率。

\subsection{代幣分配與鎖定發放時程表}

\subsubsection{總供應量與分配}

總供應量為100,000,000 CPT(固定,無通貨膨脹)。分配細節如表\ref{tab:allocation}所示。
\begin{table}[htbp]
\centering
\caption{CPT代幣分配}
\label{tab:allocation}
\begin{tabularx}{\textwidth}{lrrr>{\raggedright\arraybackslash}X}
\hline
\textbf{Category} & \textbf{Allocation} & \textbf{Tokens} & \textbf{\%} & \textbf{Lock \& Vesting Terms} \\
\hline
\textbf{Community Incentives} & \textbf{Total} & \textbf{55,000,000} & \textbf{55\%} & \textbf{Performance-based release} \\
\quad - User Rewards & & 25,000,000 & 25\% & Released based on platform GMV milestones \\
\quad - SP Incentives & & 20,000,000 & 20\% & Released based on transaction volume \\
\quad - LP Rewards & & 10,000,000 & 10\% & 5-year emission, front-loaded \\
\textbf{Foundation} & & \textbf{17,500,000} & \textbf{17.5\%} & \textbf{10\% at TGE, 90\% linear vest 24 months} \\
\textbf{Private Sale} & & \textbf{12,500,000} & \textbf{12.5\%} & \textbf{6-month cliff, 18-month linear vest} \\
\textbf{Team} & & \textbf{15,000,000} & \textbf{15\%} & \textbf{12-month cliff, 36-month linear vest} \\
\hline
\textbf{Total} & & \textbf{100,000,000} & \textbf{100\%} & \\
\hline
\end{tabularx}
\end{table}

與原始版本相比的主要變更包括:社群分配從50\%增加到55\%(移除USDC持有者分配);投資者分配從15\%減少到12.5\%(社群優先方法);團隊分配從17.5\%減少到15\%(更強的一致性);取消「流動性提供者」類別(替換為流動性提供者誘因)。

\subsubsection{鎖定發放細節}

\paragraph{社群誘因(55\%)}

使用者獎勵(25M CPT)每月根據平台GMV目標發放。公式為:每月發放量 = 基礎金額 ×(實際GMV / 目標GMV)。分配期為5年,未領取的代幣滾動到下一期。

服務提供者誘因(20M CPT)每季度根據交易額發放。高品質服務提供者(CSPs)可獲得獎勵倍數。分配期為5年,可能有基於績效的加速。

流動性提供者獎勵(10M CPT)採用前載發行:第1年(40\%)、第2年(30\%)、第3--5年(30\%)。每周分配給活躍的流動性提供者,長期存款可獲得獎金。鎖定發放為50\%立即發放,50\%在6個月內線性鎖定發放。

\paragraph{團隊分配(15\%)}

團隊分配包含12個月禁售期(第一年無代幣發放)。禁售期後,36個月線性鎖定發放。總鎖定發放期為4年。鎖定發放合約是透明且公開可驗證的。

\paragraph{基金會分配(17.5\%)}

10\%在TGE時發放用於初期營運(多簽控制)。剩餘90\%在24個月內線性鎖定發放。這些資金用於合作夥伴關係、審計、法律、行銷和補助,每季發布透明度報告。

\paragraph{私募(12.5\%)}

私募包含6個月禁售期,禁售期後18個月線性鎖定發放。總鎖定發放期為2年。防砸盤機制將賣出限制為每日成交量的最高5\%。

\subsection{流動性誘因與veToken質押模型}

\subsubsection{veToken機制(投票託管CPT)}

我們實施受Curve Finance啟發的veToken模型,該模型經證實可使長期誘因一致。使用者鎖倉CPT以獲得veCPT(不可轉讓)。鎖倉期限決定veCPT倍數,如表\ref{tab:vetoken}所示。

\begin{table}[htbp]
\centering
\caption{veToken倍數(依鎖倉期限)}
\label{tab:vetoken}
\begin{tabular}{ll}
\hline
\textbf{Lock Duration} & \textbf{veCPT Multiplier} \\
\hline
1 week & 0.01x \\
1 month & 0.04x \\
3 months & 0.25x \\
6 months & 0.50x \\
1 year & 1.00x \\
2 years & 1.50x \\
4 years & 2.50x (maximum) \\
\hline
\end{tabular}
\end{table}

\subsubsection{veCPT的優勢}

增強的治理權力:1 veCPT = 1票(相較於標準CPT:除非鎖倉否則0票),鎖倉越長,在平台方向上的影響力越強。

提升的質押獎勵:基礎年化收益率8--12\%(1年鎖倉),最高2.5倍提升(4年鎖倉),最大鎖倉可獲得高達20--30\%的提升年化收益率。

費用分潤優先權:veCPT持有者優先獲得收入分配,veCPT餘額越高,費用池份額越高。

獨家優勢:最高服務折扣(15\%),優先存取超額認購資源,以及獨家治理提案權(需要最低veCPT)。

\subsubsection{流動性挖礦計畫}

階段1:發布誘因(第1--6個月):高CPT發行以引導流動性。Uniswap V3上的CPT/USDC池每日獲得2000 CPT。CPT單一質押每日獲得1500 CPT。USDC借貸池每日獲得相當於1000 CPT的獎勵。

階段2:成長(第7--24個月):降低發行量,專注於永續收益。總發行量約為每日2500 CPT,增加USDC借貸池的權重(激勵流動性)。

階段3:成熟(第25個月起):最低新發行量(約每日1000 CPT)。收入驅動的收益成為主要吸引力,買回與銷毀創造供應稀缺性。

\subsubsection{防巨鯨與公平發布機制}

平台實施多項保護機制,包括私募中的單次購買上限為10萬美元、鎖定發放確保TGE時無大規模砸盤、時間加權投票防止治理攻擊、逐步發行防止挖礦後砸盤,以及社群分配大於團隊+投資者(55\% > 27.5\%)。

\subsubsection{比較:傳統 vs. veCPT模型}

表\ref{tab:comparison}比較傳統質押與veCPT模型。

\begin{table}[htbp]
\centering
\caption{傳統質押 vs. veCPT模型}
\label{tab:comparison}
\begin{tabular}{lll}
\hline
\textbf{Metric} & \textbf{Traditional Staking} & \textbf{veCPT Model} \\
\hline
Minimum commitment & None & 1 week \\
Maximum rewards & Fixed APY & Up to 2.5x boost \\
Governance power & Linear (1 token = 1 vote) & Time-weighted \\
Long-term alignment & Low & High \\
Mercenary capital risk & High & Low \\
Price stability & Lower & Higher \\
\hline
\end{tabular}
\end{table}

為什麼這個模型有效:它由Curve(\$CRV)證實,並自2020年以來經過實戰測試。它使長期持有者的誘因一致,減少短期挖礦者的賣出壓力,創造強大的治理參與度,並提供不依賴永久通貨膨脹的永續代幣經濟學。

\subsection{上市策略與保守情境}

\subsubsection{冷啟動策略}

成功發布雙邊市集需要謹慎的順序安排。我們的方法包含三個階段。

\paragraph{階段0:種子使用者(第1--3個月)}

目標為50--100名付費使用者。來源包括ClusterTech現有客戶群以及Web3專案。誘因包括3個月免費試用、早期採用者可享50\%終身折扣,以及初始CPT空投(總預算100K CPT)。預算約為15萬美元(行銷+誘因)。

\paragraph{階段1:早期採用者(第3--12個月)}

目標為500--1000名付費使用者和10名企業客戶。策略包括推薦計畫(推薦者和被推薦者均可獲得50美元信用額度)、透過技術部落格和YouTube教學進行內容行銷、駭客松贊助(Web3社群),以及雲端轉售商合作夥伴關係。預算約為50萬美元(行銷+銷售)。

\paragraph{階段2:成長(第12--24個月)}

目標為2000--5000名使用者和50名企業客戶。策略包括CPT質押誘因全面啟用、戰略合作夥伴關係(Infura、Alchemy等),以及會議出席和思想領導力。預算為100萬美元以上(隨收入擴增)。

\subsubsection{財務情境}

為了向投資者提供透明度,我們模擬了三種情境。

\paragraph{保守情境(高機率)}

表\ref{tab:conservative}呈現保守財務情境。

\begin{table}[htbp]
\centering
\caption{保守財務情境}
\label{tab:conservative}
\begin{tabular}{lrrr}
\hline
\textbf{Metric} & \textbf{Year 1} & \textbf{Year 2} & \textbf{Year 3} \\
\hline
Paying Users & 200 & 1,000 & 3,000 \\
ARPU (\$/month) & \$40 & \$60 & \$80 \\
MRR & \$8K & \$60K & \$240K \\
Annual Revenue & \$96K & \$720K & \$2.9M \\
Operating Costs & \$600K & \$900K & \$1.5M \\
Net Income & -\$504K & -\$180K & +\$1.4M \\
Cumulative Cash & -\$500K & -\$680K & +\$720K \\
\hline
\end{tabular}
\end{table}

\paragraph{Base Case Scenario (Medium probability)}

Table~\ref{tab:basecase} presents the base case financial scenario.

\begin{table}[htbp]
\centering
\caption{Base Case Financial Scenario}
\label{tab:basecase}
\begin{tabular}{lrrr}
\hline
\textbf{Metric} & \textbf{Year 1} & \textbf{Year 2} & \textbf{Year 3} \\
\hline
Paying Users & 500 & 2,500 & 8,000 \\
ARPU (\$/month) & \$50 & \$75 & \$100 \\
MRR & \$25K & \$188K & \$800K \\
Annual Revenue & \$300K & \$2.25M & \$9.6M \\
Operating Costs & \$800K & \$1.5M & \$3M \\
Net Income & -\$500K & +\$750K & +\$6.6M \\
\hline
\end{tabular}
\end{table}

\paragraph{Optimistic Scenario (Lower probability)}

Table~\ref{tab:optimistic} presents the optimistic financial scenario.

\begin{table}[htbp]
\centering
\caption{Optimistic Financial Scenario}
\label{tab:optimistic}
\begin{tabular}{lrrr}
\hline
\textbf{Metric} & \textbf{Year 1} & \textbf{Year 2} & \textbf{Year 3} \\
\hline
Paying Users & 1,000 & 5,000 & 20,000 \\
ARPU (\$/month) & \$75 & \$100 & \$150 \\
MRR & \$75K & \$500K & \$3M \\
Annual Revenue & \$900K & \$6M & \$36M \\
Operating Costs & \$1M & \$2.5M & \$8M \\
Net Income & -\$100K & +\$3.5M & +\$28M \\
\hline
\end{tabular}
\end{table}

\paragraph{Key Assumptions}

Scenarios reflect different market penetration rates and pricing power. Operating costs scale with growth but benefit from economies of scale. The conservative scenario assumes minimal group-buying contribution. All scenarios assume primary revenue from SaaS and transaction fees. CPT incentive costs are included in operating costs.

\paragraph{Funding Requirements}

Seed/Angel funding of \$500K--1M will cover Year 1 losses and product development. Series A funding of \$3--5M is planned for Year 2, if base case trajectory is confirmed. Series B funding of \$10--20M is planned for Year 3+, for international expansion.

\paragraph{Break-even Analysis}

Conservative scenario reaches break-even in Month 30--36. Base Case reaches break-even in Month 18--24. Optimistic scenario reaches break-even in Month 12--18.

This range provides investors with realistic expectations while demonstrating scalability potential.
