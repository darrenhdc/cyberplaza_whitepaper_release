\chapter{CyberPlaza Token (CPT) 與代幣經濟學}
\subsection{CPT 代幣概述與實用性}

\subsubsection{支付系統}

平台使用USDC作為所有市場交易的主要支付貨幣。此做法消除了與自營穩定幣相關的監管風險,同時確保符合全球穩定幣框架的監管規定、提供熟悉的使用者體驗(USDC被廣泛採用且可信賴)、透明的美元計價、與現有去中心化金融(DeFi)基礎設施的無縫整合,以及沒有演算法穩定幣失敗的風險。

\subsubsection{CyberPlaza Token (CPT)}

CPT為平台原生治理與實用代幣,旨在協調所有利害關係人的誘因並捕捉平台價值成長。

\subsubsection{核心CPT實用性}

\paragraph{治理權利}

CPT持有者可對平台參數(費用結構、收益分配比率等)表決,提議並表決新功能、合作夥伴及策略方向,並參與財務庫管理與資本配置決策。投票權重基於質押CPT數量及鎖倉期間(veToken模型)。平台每季度舉行治理會議,並實施透明的提案流程。

\paragraph{透過質押共享收益}

持有者可質押CPT以獲得平台收益分配(以USDC支付)。平台收益的30%分配至質押獎勵池(針對永續性最佳化)。質押獎勵以週或月為單位分配(由治理決定)。較長的質押期可獲得獎勵乘數(4年鎖倉最高2.5倍)。目標APY為6–10%,依平台績效(更永續)及質押比例而定。無非永久性損失風險(單資產質押)。

\paragraph{使用優惠}

質押CPT可享有平台服務5–15%折扣(階層制度),存取高級功能(含進階分析、API存取及優先支援),高交易量使用者的交易費減免,新服務及測試版功能的搶先體驗,以及高需求運算資源的優先分配。

\paragraph{生態系誘因}

平台為所有使用者類別提供誘因。使用者可獲得消費金額1–3%的CPT回饋(現金回饋計畫)。服務提供者可獲得交易量2–5%的CPT獎金。流動性提供者可獲得2–4%的CPT年化收益率作為額外收益。推薦新使用者或SP至平台可獲得CPT。社區貢獻透過漏洞獎勵、內容創作及程式碼貢獻進行回饋。

\paragraph{通縮機制}

平台收益的20%用於從公開市場回購CPT。購買的CPT代幣將永久銷毀(發送至0x0地址),長期降低流通供應量,創造稀缺性。平台實施透明的季度銷毀活動,並進行鏈上驗證,預計5年內供應量減少30–40%。這將使所有CPT持有者獲益,不僅限於質押者。

\subsection{收益模型與分配機制}

\subsubsection{平台收益來源}

平台透過表~\ref{tab:revenue}所示的多種管道產生收益。

\begin{table}[htbp]
\centering
\caption{平台收益預測}
\label{tab:revenue}
\begin{tabular}{lccccr}
\hline
\textbf{Revenue Stream} & \textbf{Rate/Amount} & \textbf{Year 1} & \textbf{Year 2} & \textbf{Year 3} & \textbf{\% of Total} \\
\hline
\textbf{SaaS Subscriptions} & \$50--500/month & \$1.5M & \$4M & \$8--10M & \textbf{40--50\%} \\
Transaction Fees & 2--5\% of GMV & \$0.8M & \$2.5M & \$5--7M & \textbf{25--30\%} \\
API \& Data Services & Variable & \$0.3M & \$1.5M & \$3--4M & \textbf{15--20\%} \\
Certification Services & \$5K--50K per SP & \$0.3M & \$0.8M & \$1--2M & \textbf{5--8\%} \\
Group-Buying Margins & 5--10\% margins & \$0.2M & \$0.7M & \$1.5--2M & \textbf{5--10\%} \\
\hline
\textbf{Total Revenue} & --- & \textbf{\$3.1M} & \textbf{\$9.5M} & \textbf{\$19--25M} & \textbf{100\%} \\
\hline
\end{tabular}
\end{table}

收益模型優先考慮SaaS訂閱作為主要收益來源(40--50%),以確保可預測性;團購降為輔助(5--10%),考量早期規模較為實際;API服務則強調其為高邊際利潤、可擴展的收益來源(15--20%)。保守預測基於第3年0.01%的市場滲透率。

\subsubsection{SaaS訂閱階層}

平台提供如表~\ref{tab:saas}所示的分階訂閱方案。

\begin{table}[htbp]
\centering
\caption{SaaS訂閱階層(說明性質)}
\label{tab:saas}
\begin{tabularx}{\textwidth}{llXXr}
\hline
\textbf{Tier} & \textbf{Price/Month} & \textbf{Target Users} & \textbf{Features} & \textbf{Est. Users (Y3)} \\
\hline
Free & \$0 & Individuals & 2 cloud accounts, basic monitoring & 10,000+ \\
Starter & \$50 & Small teams & 5 accounts, cost tracking, 1\% CPT cashback & 2,000 \\
Professional & \$200 & Dev teams & 10 accounts, AI optimization, API, 3\% CPT & 500 \\
Enterprise & \$500--2000 & Companies & Unlimited, custom integration, 5\% CPT & 50--100 \\
\hline
\end{tabularx}
\end{table}

此階層模型提供可預測的經常性收益,同時仍允許免費升級的使用者取得方式。

\textbf{重要注意事項}:這些預測代表我們的目標情境。我們也建模了保守情境,第1年收益為\$500K--1M,以確保即使初期成長較慢仍具財務永續性。我們的商業模式不依賴立即實現大規模團購折扣。

\subsubsection{收益分配模型}

平台收益(100%)分配如下:質押獎勵池獲得30%(為永續性而降低),並以USDC按比例分配給CPT質押者。營運與開發獲得35%(為成長而增加),分配至工程與產品開發(15%)、行銷與業務開發(10%)以及基礎設施與安全(5%)。回購與銷毀獲得20%,此處CPT從去中心化交易所(DEX)購買並永久銷毀。團隊與基金會獲得10%,用於核心團隊報酬(5%)及基金會營運(5%)。緊急準備金獲得5%,作為新的波動緩衝。

\subsubsection{質押獎勵計算範例}

考慮第3年平台每月收益\$1,500,000的成熟平台情境。質押池配置(40%)提供\$600,000。若質押CPT總額為40,000,000(供應量的40%),而您的質押額為10,000 CPT(質押總額的0.025%),則您的每月獎勵為\$600,000 $\times$ 0.025\% = \$150 USDC,年化獎勵為\$150 $\times$ 12 = \$1,800 USDC。

若CPT price = \$2,則您的質押價值為\$20,000,APY為\$1,800 / \$20,000 = 9%。此外,額外好處包括平台治理投票權、服務折扣(5--15%)以及回購/銷毀帶來的價格上漲。

\subsubsection{USDC存款人流動性池收益}

將USDC存入借貸池的流動性提供者可獲得如表~\ref{tab:liquidity}所示的收益。

\begin{table}[htbp]
\centering
\caption{流動性提供者收益}
\label{tab:liquidity}
\begin{tabular}{llll}
\hline
\textbf{Component} & \textbf{APY} & \textbf{Paid In} & \textbf{Source} \\
\hline
Base Interest & 6--8\% & USDC & Platform operational profits \\
CPT Incentives & 2--4\% & CPT & Token emission (vesting) \\
\textbf{Total Expected} & \textbf{8--12\%} & \textbf{Mixed} & \textbf{Sustainable yields} \\
\hline
\end{tabular}
\end{table}

關鍵功能包括:存款用於團購營運(透過鏈上追蹤透明化),逐步提領系統防止擠兌情境,保險基金覆蓋高達10%的池TVL,智慧合約由領導廠商審計,實時APY更新基於池使用率。

\subsection{代幣分配與歸屬時程}

\subsubsection{總供應量與分配}

總供應量為100,000,000 CPT(固定,無通貨膨脹)。分配明細如表~\ref{tab:allocation}所示。
\begin{table}[htbp]
\centering
\caption{CPT代幣分配}
\label{tab:allocation}
\begin{tabularx}{\textwidth}{lrrr>{\raggedright\arraybackslash}X}
\hline
\textbf{Category} & \textbf{Allocation} & \textbf{Tokens} & \textbf{\%} & \textbf{Lock \& Vesting Terms} \\
\hline
\textbf{Community Incentives} & \textbf{Total} & \textbf{55,000,000} & \textbf{55\%} & \textbf{Performance-based release} \\
\quad - User Rewards & & 25,000,000 & 25\% & Released based on platform GMV milestones \\
\quad - SP Incentives & & 20,000,000 & 20\% & Released based on transaction volume \\
\quad - LP Rewards & & 10,000,000 & 10\% & 5-year emission, front-loaded \\
\textbf{Foundation} & & \textbf{17,500,000} & \textbf{17.5\%} & \textbf{10\% at TGE, 90\% linear vest 24 months} \\
\textbf{Private Sale} & & \textbf{12,500,000} & \textbf{12.5\%} & \textbf{6-month cliff, 18-month linear vest} \\
\textbf{Team} & & \textbf{15,000,000} & \textbf{15\%} & \textbf{12-month cliff, 36-month linear vest} \\
\hline
\textbf{Total} & & \textbf{100,000,000} & \textbf{100\%} & \\
\hline
\end{tabularx}
\end{table}

代幣分配強調社區導向,55%分配至社區誘因(社群優先策略),12.5%分配至投資者,15%分配至團隊以加強一致性。流動性提供者誘因整合至社區分配架構。

\subsubsection{歸屬細節}

\paragraph{社區誘因(55\%)}

使用者獎勵(25M CPT)根據平台GMV目標每月釋放。公式為:每月釋放 = Base amount $\times$ (實際GMV / 目標GMV)。分配期為5年,未領取代幣滾入下一期。

SP誘因(20M CPT)根據交易量每季度釋放。高品質SP(CSPs)可獲得獎金乘數。分配期為5年,可根據績效加速。

LP獎勵(10M CPT)採用前載式釋放:第1年(40%),第2年(30%),第3--5年(30%)。每周分配給活躍流動性提供者,長期存款可獲得獎金。歸屬為50%立即釋放,50%在6個月內歸屬。

\paragraph{團隊分配(15\%)}

團隊分配包含12個月的 cliff(第一年無代幣釋放)。cliff後為36個月線性歸屬。總歸屬期為4年。歸屬合約透明且可公開驗證。

\paragraph{基金會分配(17.5\%)}

10%在TGE釋放用於初始營運(多簽控制)。剩餘90%在24個月內線性歸屬。這些資金用於合作夥伴、審計、法律、行銷和補助金,每季度發布透明度報告。

\paragraph{私募(12.5\%)}

私募包含6個月的 cliff 期,cliff後為18個月線性歸屬。總歸屬期為2年。反傾銷機制限制每日最大賣出量為5%的成交量。

\subsection{流動性誘因與veToken質押模型}

\subsubsection{veToken機制(Vote-Escrowed CPT)}

我們採用受Curve Finance啟發的veToken模型,該模型已證實能協調長期誘因。使用者鎖定CPT以獲得veCPT(不可轉讓)。鎖倉期間決定veCPT乘數如表~\ref{tab:vetoken}所示。

\begin{table}[htbp]
\centering
\caption{veToken乘數(依鎖倉期間)}
\label{tab:vetoken}
\begin{tabular}{ll}
\hline
\textbf{Lock Duration} & \textbf{veCPT Multiplier} \\
\hline
1 week & 0.01x \\
1 month & 0.04x \\
3 months & 0.25x \\
6 months & 0.50x \\
1 year & 1.00x \\
2 years & 1.50x \\
4 years & 2.50x (maximum) \\
\hline
\end{tabular}
\end{table}

\subsubsection{veCPT的好處}

增強治理權:1 veCPT = 1票(相較於標準CPT:除非鎖定否則無投票權),較長鎖倉期等於在平台方向上有更強的發言權。

強化質押獎勵:1年鎖倉的基本APY為8--12%,4年鎖倉的最大加成2.5x,最大鎖倉的強化APY高達20--30%。

費用共享優先權:veCPT持有者優先獲得收益分配,veCPT餘額越高,費用池的份額越高。

獨家福利:最大服務折扣(15%),搶購資源的優先存取權,以及獨家治理提案權(需最低veCPT)。

\subsubsection{流動性挖礦計畫}

第1階段:上線誘因(第1–6個月):高CPT釋放量以引導流動性。Uniswap V3上的CPT/USDC池每日獲得2000 CPT。CPT單資產質押每日獲得1500 CPT。USDC借貸池獲得每日1000 CPT等值。

第2階段:成長期(第7–24個月):降低釋放量,著重永續收益。總釋放量約為每日2500 CPT,增加USDC借貸池的權重(激勵流動性)。

第3階段:成熟期(第25個月起):新釋放量最小化(約每日1000 CPT)。收益驅動的收益成為主要吸引力,回購與銷毀創造供應稀缺性。

\subsubsection{反巨鯨與公平發行機制}

平台實施多項保護機制,包括私募單次最大購買限額\$100K、歸屬確保TGE時無大量拋售、時間加權投票防止治理攻擊、逐步釋放防止挖礦後拋售,以及社區分配大於團隊+投資者(55% > 27.5%)。

\subsubsection{比較:傳統模型 vs. veCPT模型}

表~\ref{tab:comparison}比較了傳統質押與veCPT模型。

\begin{table}[htbp]
\centering
\caption{傳統質押 vs. veCPT模型}
\label{tab:comparison}
\begin{tabular}{lll}
\hline
\textbf{Metric} & \textbf{Traditional Staking} & \textbf{veCPT Model} \\
\hline
Minimum commitment & None & 1 week \\
Maximum rewards & Fixed APY & Up to 2.5x boost \\
Governance power & Linear (1 token = 1 vote) & Time-weighted \\
Long-term alignment & Low & High \\
Mercenary capital risk & High & Low \\
Price stability & Lower & Higher \\
\hline
\end{tabular}
\end{table}

為何此模型有效?其由Curve(\$CRV)證實,並自2020年起經過實戰測試。它協調長期持有者的誘因,減少短期挖礦者的賣壓,創造強大的治理參與度,並提供不依賴永久通貨膨脹的永續代幣經濟學。

\subsection{上市策略與保守情境}

\subsubsection{冷啟動策略}

成功推出雙邊市場需要謹慎的順序規劃。我們的方法分為三個階段。

\paragraph{第0階段:種子使用者(第1–3個月)}

目標為50–100位付費使用者。來源包括ClusterTech既有客戶群及Web3專案。誘因包括3個月免費試用、早期採用者終身50%折扣,以及初始CPT空投(總預算10萬CPT)。預算約為\$150K(行銷+誘因)。

\paragraph{第1階段:早期採用者(第3–12個月)}

目標為500–1000位付費使用者及10位企業客戶。策略包括推薦計畫(推薦者與被推薦者各獲得\$50優惠券)、透過技術部落格與YouTube教學的內容行銷、駭客松贊助(Web3社群),以及雲端轉售合作夥伴。預算約為\$500K(行銷+銷售)。

\paragraph{第2階段:成長期(第12–24個月)}

目標為2000–5000位使用者及50位企業客戶。策略包括CPT質押誘因完全啟動、策略合作夥伴(如Infura、Alchemy等),以及會議參與和意見領袖活動。預算為\$1M以上(隨收益擴大)。

\subsubsection{財務情境}

為向投資者提供透明度,我們建模了三種情境。

\paragraph{保守情境(高機率)}

表~\ref{tab:conservative}呈現保守財務情境。

\begin{table}[htbp]
\centering
\caption{保守財務情境}
\label{tab:conservative}
\begin{tabular}{lrrr}
\hline
\textbf{Metric} & \textbf{Year 1} & \textbf{Year 2} & \textbf{Year 3} \\
\hline
Paying Users & 200 & 1,000 & 3,000 \\
ARPU (\$/month) & \$40 & \$60 & \$80 \\
MRR & \$8K & \$60K & \$240K \\
Annual Revenue & \$96K & \$720K & \$2.9M \\
Operating Costs & \$600K & \$900K & \$1.5M \\
Net Income & -\$504K & -\$180K & +\$1.4M \\
Cumulative Cash & -\$500K & -\$680K & +\$720K \\
\hline
\end{tabular}
\end{table}

\paragraph{基本情境(中機率)}

表~\ref{tab:basecase}呈現基本財務情境。

\begin{table}[htbp]
\centering
\caption{基本財務情境}
\label{tab:basecase}
\begin{tabular}{lrrr}
\hline
\textbf{Metric} & \textbf{Year 1} & \textbf{Year 2} & \textbf{Year 3} \\
\hline
Paying Users & 500 & 2,500 & 8,000 \\
ARPU (\$/month) & \$50 & \$75 & \$100 \\
MRR & \$25K & \$188K & \$800K \\
Annual Revenue & \$300K & \$2.25M & \$9.6M \\
Operating Costs & \$800K & \$1.5M & \$3M \\
Net Income & -\$500K & +\$750K & +\$6.6M \\
\hline
\end{tabular}
\end{table}

\paragraph{樂觀情境(低機率)}

表~\ref{tab:optimistic}呈現樂觀財務情境。

\begin{table}[htbp]
\centering
\caption{樂觀財務情境}
\label{tab:optimistic}
\begin{tabular}{lrrr}
\hline
\textbf{Metric} & \textbf{Year 1} & \textbf{Year 2} & \textbf{Year 3} \\
\hline
Paying Users & 1,000 & 5,000 & 20,000 \\
ARPU (\$/month) & \$75 & \$100 & \$150 \\
MRR & \$75K & \$500K & \$3M \\
Annual Revenue & \$900K & \$6M & \$36M \\
Operating Costs & \$1M & \$2.5M & \$8M \\
Net Income & -\$100K & +\$3.5M & +\$28M \\
\hline
\end{tabular}
\end{table}

\paragraph{關鍵假設}

情境反映不同的市場滲透率與定價權。營運成本隨成長規模化,但受益於經濟規模。保守情境假設團購貢獻最小。所有情境假設主要收益來自SaaS與交易費。CPT誘因成本包含在營運成本中。

\paragraph{資金需求}

種子/天使投資\$500K--1M將涵蓋第1年損失與產品開發。若確認基本情境軌跡,計畫於第2年進行\$3--5M的A輪融資。計畫於第3年及以後進行\$10--20M的B輪融資,用於國際擴張。

\paragraph{損益兩平分析}

保守情境於第30--36個月達到損益兩平。基本情境於第18--24個月達到損益兩平。樂觀情境於第12--18個月達到損益兩平。此範圍為投資者提供現實期望,同時展現可擴展性潛力。
