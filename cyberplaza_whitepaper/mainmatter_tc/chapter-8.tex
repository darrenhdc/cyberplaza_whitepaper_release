\chapter{核心團隊、基金會與顧問}

\subsection*{核心團隊}

核心團隊負責建立、維護並推進平臺的技術基礎設施。開發與維護範圍涵蓋CHESS運算能力分配軟體、上架運算資源的品質評估系統、區塊鏈平臺架構、智慧合約實作以及配套技術基礎設施。

團隊專長涵蓋分散式高效能運算、公共雲端服務、異質運算架構、去中心化金融投資策略、人工智慧與大數據應用、金融科技解決方案、分散式系統軟體開發以及運算資源商業化。

\subsubsection*{核心團隊成員}

\textbf{Wai-Mo Suen博士}擁有25年高效能運算技術與現代運算業務營運的專業經驗。自2000年起擔任ClusterTech的創辦人與執行長,他已提供高效能運算(HPC)、雲端運算、人工智慧與大數據解決方案,並因在高效能運算業務與金融科技創新方面的成就獲得多項獎項。

\textbf{Harry Yu博士}專精於FPGA技術,為CTAccel的創辦人與執行長,該公司於2018年獲得Intel Capital投資。他的投資專長包含2年去中心化金融經驗,實現年化報酬率18\%與4.6的夏普比率。

\textbf{Eric Leung先生}擁有15年高效能運算(HPC)系統管理經驗,並輔以10年領導公共雲端服務供應商營運的經歷。

\textbf{GY Han先生}擁有15年高效能運算(HPC)系統管理軟體開發的專業經驗。

\textbf{Terence Leung先生}擁有近30年執法專長,專精於反洗錢與詐欺調查,並輔以豐富的遵循與風險管理經驗。他擔任量化與去中心化金融投資基金的顧問與財務主管已有5年。

\textbf{龐寶林先生(Pong Po Lam Paul)}創辦了Pegasus Fund Managers Ltd.,並聯合創辦了香港財務策劃師學會、亞洲金融科技師學會以及香港財務分析師及投資專家學會。他的公共服務經歷涵蓋金融發展局、強積金諮詢委員會、香港會計師公會以及證券及期貨事務監察委員會的職務。他持有認證財務策劃師(CFPCM)與認證金融科技師(CFT)資格。

\textbf{XXX先生}擁有豐富的資訊科技業務與行銷營運經驗。

\subsection*{基金會、投資人與服務提供者(SPs)}

\subsubsection*{基金會}

基金會負責管理專案開發、推廣與維護,以確保長期永續性。職責涵蓋代幣分配與管理、社群建立與參與、行銷與推廣計畫、專案治理監督以及技術與經濟生態系支援。

基金會成員包括核心團隊成員與顧問,擁有高效能與雲端運算資源供應、人工智慧與大數據基礎設施、金融投資策略、金融產品開發以及符合商業規範與反洗錢要求的專長。

\subsubsection*{投資人}

待確認。

\subsubsection*{服務提供者(SPs)}

於平臺上線時,已有5家認證服務提供者(CSPs)完成註冊,提供的運算資源包含xx個CPU核心(相當於???個X86核心),可提供??? FP64 TFLOPS運算能力,yy個GPU(相當於xxx個32位元張量運算TOPS),zz個FPGA(相當於??? FP32運算TFLOPS),以及??? PB的儲存容量。

資源增長預測目標為在上線後一年內,CPU擴充10倍、GPU擴展20倍、FPGA增長5倍、儲存容量增加10倍。
