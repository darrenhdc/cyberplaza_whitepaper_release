\chapter{核心團隊、基金會與顧問}

\subsection*{核心團隊}

核心團隊負責建置、維護並推進平台的技術基礎架構。開發與維護範圍涵蓋 CHESS 運算能力分配軟體、上架運算資源的品質評估系統、區塊鏈平台架構、智能合約實作以及支援性技術基礎架構。

團隊專長涵蓋分散式高效能運算、公有雲服務、異質運算架構、去中心化金融投資策略、人工智慧與大數據應用、金融科技解決方案、分散式系統軟體開發以及運算資源商業化。

\subsubsection*{核心團隊成員}

\textbf{Wai-Mo Suen 博士}擁有 25 年高效能運算技術與現代運算業務營運的專業經驗。自 2000 年起擔任 ClusterTech 的創辦人與執行長,他提供了 HPC、雲端運算、AI 以及大數據解決方案,並獲得多項表彰其在高效能運算業務與金融科技創新方面成就的獎項。

\textbf{Harry Yu 博士}專精於 FPGA 技術,擔任 CTAccel 的創辦人與執行長,該公司於 2018 年獲得 Intel Capital 的投資。他的投資專才包括兩年的去中心化金融經驗,達成 18\% 的年化投資報酬率(ROI)與 4.6 的夏普比率。

\textbf{Eric Leung 先生}擁有 15 年 HPC 系統管理經驗,並輔以十年領導公有雲服務供應商營運的經驗。

\textbf{GY Han 先生}擁有 15 年專精於 HPC 系統管理軟體開發的經驗。

\textbf{Terence Leung 先生}擁有近三十年專精於反洗錢與詐欺調查的執法專業經驗,並輔以廣泛的遵循與風險管理經驗。他曾擔任量化與 DeFi 投資基金的顧問與財務長長達五年。

\textbf{Pong Po Lam Paul 先生(龐寶林)}創辦了 Pegasus Fund Managers Ltd.,並聯合創辦了香港財務策劃師學會、亞洲金融科技師學會以及香港財務分析師及投資顧問學會。他的公共服務涵蓋財務服務發展委員會、強積金(MPF)諮詢委員會、香港會計師公會以及證券及期貨事務監察委員會的職務。他持有認證財務策劃師(CFPCM)與認證金融科技師(CFT)證照。

\textbf{XXX 先生}擁有豐富的 IT 業務與行銷營運經驗。

\subsection*{基金會、投資者與服務供應商(SPs)}

\subsubsection*{基金會}

基金會管理專案開發、推廣與維護,以確保長期永續性。職責涵蓋代幣分配與管理、社群建立與參與、行銷與推廣計劃、專案治理監督以及技術與經濟生態系統支援。

基金會成員包含核心團隊成員與顧問,他們具備高效能與雲端運算資源供應、AI 與大數據基礎架構、金融投資策略、金融產品開發以及遵循商業法規與反洗錢要求的專業經驗。

\subsubsection*{投資者}

尚待確認。

\subsubsection*{服務供應商(SPs)}

在平台上線時,已有五家認證服務供應商(CSPs)完成註冊,提供的運算資源包括 xx 顆 CPU 核心(相當於 ??? 顆 X86 核心),可提供 ??? FP64 TFLOPS 的運算能力、yy 顆 GPU(相當於 xxx TOPS 的 32 位元張量運算能力)、zz 顆 FPGA(相當於 ??? TFLOPS 的 FP32 運算能力),以及 ??? PB 的儲存容量。

資源成長預測目標為在平台上線後一年內,CPU 擴充 10 倍、GPU 擴展 20 倍、FPGA 成長 5 倍,以及儲存容量增加 10 倍。
