\chapter{科技與架構}



\section{專案治理基礎設施}

\subsection{概述}

CyberPlaza網路由CyberPlaza基金會與CyberPlaza社群組成。

CyberPlaza基金會是一個非營利去中心化組織,致力於CyberPlaza平台的順利營運、運算技術與應用的推廣和發展,以及支援平台上的去中心化社群建構與發展。基金會由CyberPlaza社群中的CPT持有者擁有和控制,由網路的核心成員(參見白皮書第8節)以及基金會根據需要不定期任命的顧問運作。基金會將成立CyberPlaza Labs,負責開發和研究新的運算資源技術與應用,以推動平台所需的技術創新和進步。

CyberPlaza社群是網路的社群部分,由流動性提供者、使用者和SP組成,共同參與基金會治理、開發和推廣。社群成員可透過參與治理、向基金會提出業務方向、技術發展相關提案,以及交流分享經驗,來推動平台的發展與成長。

CyberPlaza基金會與CyberPlaza社群之間的緊密連結,是實現網路願景與使命的關鍵。

\subsection{智慧合約模組}

我們將CPT部署為符合ERC20規範的智慧合約在Arbitrum(以太坊的Layer 2)上,選擇Arbitrum是因其低交易成本與高吞吐量。平台也會根據生態系擴展需求,橋接至其他鏈。

CPT代幣合約包含以下核心功能:標準ERC20功能(轉帳、授權等)、用於veToken機制的質押與鎖定功能、治理投票整合、獎勵分配機制、由治理控制的緊急暫停功能,以及用於未來增強的可升級代理模式。

\textbf{注意}:平台直接使用USDC作為支付工具,因此無需專屬穩定幣,也可避免相關監管風險。

\begin{verbatim}
// SPDX-License-Identifier: MIT
pragma solidity ^0.8.0;

import "@openzeppelin/contracts/token/ERC20/ERC20.sol";

contract CPTToken is ERC20 {
  struct LockInfo {
     uint256 amount;
     uint256 lockTimestamp;
     uint256 unlockTimestamp;
  }

   mapping (address => LockInfo[]) public locks;

   constructor(uint256 initialSupply) ERC20("CPT Token", "CPT") {
     _mint(msg.sender, initialSupply);
   }

   function lock(uint256 _amount, uint256 _lockTime) public {
     require(_amount <= balanceOf(msg.sender), "Not enough CPT to lock");
     require(_lockTime > 0, "Lock time must be positive");

       uint256 lockUntil = block.timestamp + _lockTime;
    
       LockInfo memory newLock = LockInfo({
           amount: _amount,
           lockTimestamp: block.timestamp,
           unlockTimestamp: lockUntil
       });
    
       locks[msg.sender].push(newLock);
    
       _burn(msg.sender, _amount);

   }

  function unlock(uint256 lockIndex) public {
    require(lockIndex < locks[msg.sender].length, 
            "No lock found at this index");
    require(block.timestamp >= locks[msg.sender][lockIndex].unlockTimestamp,
            "CPT still locked");

        uint256 amountToUnlock = locks[msg.sender][lockIndex].amount;
        locks[msg.sender][lockIndex] = 
            locks[msg.sender][locks[msg.sender].length - 1];
        locks[msg.sender].pop();
    
        _mint(msg.sender, amountToUnlock);
    }

   function calculateLockedAmount(address user, uint256 lockDuration) 
       public view returns (uint256) {
     uint256 totalLockedAmount = 0;

        for (uint256 i = 0; i < locks[user].length; i++) {
           if (block.timestamp - locks[user][i].lockTimestamp > lockDuration) {
               totalLockedAmount += locks[user][i].amount;
           }
        }
    
        return totalLockedAmount;
    }

}
\end{verbatim}

\subsection{代幣標準與小數處理}

CPT代幣遵循標準ERC20規格,具有18位小數,而USDC則以6位小數運作。平台使用SafeMath函式庫進行所有轉換操作,以防止溢位和下溢錯誤。價格預言機整合了小數正規化邏輯,最低交易門檻則可緩解灰塵攻擊向量。針對分數金額,協議實施了保守的四捨五入機制。

\subsection{投票託管代幣機制}

平台實施投票託管(ve)代幣模型,以調整長期利益相關者的激勵。使用者可將CPT鎖定一週至四年不等的時間,獲得不可轉讓的veCPT代幣,該代幣將決定治理權重與獎勵分配。

veCPT餘額遵循以下關係:
\begin{equation}
\text{veCPT} = \text{CPT}_{\text{locked}} \times \min\left(\frac{t_{\text{lock}}}{t_{\text{max}}}, 1\right) \times 2.5
\end{equation}
其中$t_{\text{lock}}$代表選擇的鎖定期間,$t_{\text{max}} = 4$年則定義了最大鎖定期間。2.5倍的乘數為四年承諾提供了最大的治理權重。

當鎖定期間即將到期時,veCPT餘額會線性衰減:
\begin{equation}
\text{veCPT}(t) = \text{CPT}_{\text{locked}} \times \frac{t_{\text{remaining}}}{t_{\text{max}}} \times 2.5
\end{equation}
此衰減機制透過鎖定期延長或代幣重新鎖定來激勵持續參與。

\subsubsection{獎勵分配}

平台以USDC收取的營收中,有30%會按週或月分配給質押獎勵池。個別獎勵會根據veCPT持有比例計算:
\begin{equation}
\text{Reward}_{\text{user}} = \text{Revenue}_{\text{pool}} \times \frac{V_{\text{user}}}{V_{\text{total}}}
\end{equation}
其中$V_{\text{user}}$代表使用者的veCPT餘額,$V_{\text{total}}$代表總veCPT供應量。有效APY會根據質押參與度和平台績效動態變化:
\begin{equation}
\text{APY} = \frac{\text{Annual Revenue Pool}}{\text{Total CPT Staked Value}} \times \frac{\text{veCPT Multiplier}}{\text{Average Multiplier}}
\end{equation}

\subsubsection{安全性與最佳化}

智慧合約遵循OpenZeppelin標準進行第三方審計,多重簽名治理控制參數修改。所有獎勵分配均在鏈上追蹤以確保透明度。安全功能包括外部呼叫的可重入防護、基於角色的存取控制、緊急暫停功能以及可升級代理模式。關鍵參數變更需執行48小時的時間鎖。

瓦斯最佳化採用基於Merkle樹的批次領取、veCPT餘額的懶惰評估、壓縮儲存變數,以及事件驅動的離鏈索引。這些技術在維持安全保證的同時降低了交易成本。

\subsection{預言機整合}

平台整合Chainlink去中心化預言機以進行價格發現和資料彙總。CPT/USD價格摘要彙整來自Uniswap V3時間加權平均價格和中心化交易所報價的資料。USDC/USD驗證使用Chainlink的經過驗證的摘要,偏差閾值為0.5%。預言機每5分鐘更新一次,或在價格波動超過1%時更新,並具備手動備份機制以確保備援。

針對運算資源定價,離鏈彙整器監控主要雲端提供商(AWS、Azure、GCP、阿里巴巴雲)的公開API,計算運算、儲存和頻寬的即時市場價格。彙整後的定價每日或在偏差超過5%時發布至鏈上預言機合約。

預言機安全性依賴至少七個獨立Chainlink節點的共識。系統會拒絕與中位數偏差超過10%或資料超過一小時的價格更新。當偵測到操縱企圖時,斷路器會自動暫停交易。

\subsection{治理架構}

重大平台營運需要透過Gnosis Safe實施多重簽名審核。財庫移轉金額超過10萬USDC時,需要9人當中的5人簽名;智慧合約升級則需要9人當中的7人審核,並執行48小時的時間鎖。參數調整採用9人當中的4人共識,緊急安全回應則使用5人當中的3人快速回應配置。

治理流程遵循結構化時間表:持有10萬veCPT以上的使用者可提交提案,隨後進入7天社群討論期和5天鏈上投票階段,其中1 veCPT等於1票。經批准的提案會在48小時延遲後執行。多重簽名理事會保留對惡意提案的否決權,並接受每季審查。

\subsection{跨鏈基礎設施}

平台實施LayerZero跨鏈協議以進行多鏈部署。Arbitrum作為主鏈,因其低交易成本和高吞吐量。以太坊主網支援需要Layer 1安全性的機構使用者,而Polygon整合為對成本敏感的使用者提供更低的交易成本。未來擴展包括Optimism(2024年第三季)和Base(2024年第四季),以實現更廣泛的生態系整合。

橋接安全性整合了多項防護措施:流動性上限限制每條鏈的橋接供應量為10%,速率限制將吞吐量約束為每小時1百萬CPT,緊急暫停機制針對異常做出回應,5%的保險基金為橋接價值提供擔保,以對抗潛在的漏洞利用。

\subsection{錢包基礎設施}

作為標準ERC20代幣,CPT支援所有符合規範的錢包,包括瀏覽器外掛(MetaMask、Rabby、Rainbow)、行動應用程式(Trust Wallet、Coinbase Wallet、imToken)、硬體裝置(Ledger、Trezor)和智慧合約錢包(Argent、Gnosis Safe)。未來計劃部署與Fireblocks和Copper.co的機構託管整合。

網路入口網站實施WalletConnect和Web3Modal協議以進行標準化錢包連接。在連接授權後,平台會查詢使用者餘額、質押狀態和veCPT持有量,以實現完整的功能存取。交易簽署遵循EIP-712標準的類型化結構化資料,呈現人類可讀的訊息,提高了對釣魚攻擊向量的安全性。


\section{市場運算基礎設施}

\subsection{系統架構}

平台實施三層架構。Web3介面層使用React.js和ethers.js框架管理錢包驗證(WalletConnect)、USDC付款處理和CPT獎勵分配。協調層協調CHESS叢集管理系統、作業排程、資源分配、效能監控和SP認證流程。運算資源層整合CSP叢集、公有雲API(AWS、Azure、GCP、阿里巴巴)、私有HPC中心和未來的邊緣運算節點。

交易流程包括:提交作業並存入USDC,智慧合約託管至完成,CHESS中介的資源匹配,在分配的SP基礎設施上執行,即時SLA合規監控,結果交付與自動付款結算,以及按比例分配CPT獎勵(使用者1-3%,SP 2-5%)。

\subsection{HPC基礎設施元件}

高效能運算基礎設施包含專用節點類型:運算節點使用多核心處理器和大量記憶體執行數值模擬和資料分析;視覺化節點使用GPU加速渲染大型資料集;I/O節點管理儲存和運算架構之間的資料傳輸;儲存節點提供高併發檔案系統;管理節點協調資源分配和作業排程。

網路架構使用高速互連技術(InfiniBand、乙太網路)進行節點間通訊。平行檔案系統支援大型資料集和中間結果的並行多節點讀寫操作。

\subsubsection{軟體堆疊}

監控和管理工具為管理員提供系統元件的即時健康和效能資料,包括CPU使用率、記憶體消耗和網路流量模式。叢集管理軟體協調整體系統運作,具備跨地理位置分散安裝的運算節點的佈建、監控和維護功能。

資源分配採用專用排程器,管理CPU時間、記憶體和其他運算資源,以最大化系統利用率效率。使用者介面包括命令列工具和網路入口網站,用於作業提交和管理。HPC應用中心彙整領域特定的應用程式和範本,讓使用者能夠直接下載和部署運算工具。整合帳單系統在各資源類型和帳單週期間實施透明定價策略,促進合理的資源利用和準確的成本會計。

\subsection{付款與結算基礎設施}

\subsubsection{託管機制}

作業提交啟動託管流程,使用者批准將USDC支出至平台的智慧合約。託管合約計算估計成本,包含資源類型(CPU/GPU/儲存)、持續時間預測、預言機衍生的市場定價和20%的緩衝以應對潛在超支。USDC在批准後轉入託管,並針對唯一的作業識別碼鎖定。

作業完成後,實際資源消耗決定最終結算。服務提供商直接以USDC收取95-98%的費用,平台則保留2-5%的交易費。多餘的託管資金會自動退還給使用者,CPT獎勵則按比例分配給使用者(1-3%)和SP(2-5%)。

\subsubsection{爭議解決協議}

SLA違規觸發逐步解決機制。作業在5分鐘內失敗可獲得自動全額退款。部分完成則根據實際交付比例退款。使用者可在72小時內提出爭議並附上支持證據。金額超過1萬USDC的案件會升級至平台治理仲裁,保險基金則覆蓋經驗證的最高10萬USDC的索賠。

\subsubsection{服務提供商認證}

服務提供商認證需要多階段驗證流程。初始註冊需要公司驗證文件、基礎設施規格、付款錢包位址和安全合規證書(SOC 2、ISO 27001)。技術驗證使用業界標準基準,包括High-Performance Linpack(HPL)、High-Performance Conjugate Gradient(HPCG)、STREAM記憶體頻寬、用於AI工作負載的MLPerf,以及網路延遲評估。安全審計驗證AES-256加密、網路隔離和DDoS防護功能。

獲准的候選人進入30天試用期,接受加強監控且作業並發限制為10個。順利完成後授予認證服務提供商(CSP)狀態,可存取機構客戶和團購參與權。CSP會出現在高級目錄中並附有驗證徽章。

持續合規要求每月99.5%的正常運行時間、5分鐘內的作業啟動,以及效能在宣稱基準的10%以內。每季重新認證以驗證持續能力。關鍵漏洞的安全補丁必須在48小時內部署。違規會觸發逐步處罰:首次違規警告並給予7天修正期,第二次違規暫停30天,第三次違規撤銷認證。

\subsection{技術堆疊}

平台前端開發使用React.js 18+和TypeScript,Web3整合使用ethers.js v6和WalletConnect v2,介面一致性使用Material-UI。後端架構使用Node.js/Express.js或Python FastAPI作為API服務,PostgreSQL用於關聯式儲存,Redis用於快取,RabbitMQ/Kafka用於非同步作業佇列,The Graph用於區塊鏈事件索引,Prometheus/Grafana用於可觀測性。DevOps基礎設施透過Docker將所有服務容器化,透過Kubernetes協調生產部署,透過GitHub Actions實施CI/CD,透過Cloudflare CDN分發內容,透過Nginx進行流量負載平衡。

入門級CSP需要100個以上CPU核心(Intel Xeon/AMD EPYC)、500 GB記憶體、10 TB NVMe SSD或50 TB HDD、10 Gbps網路上行,以及可選的4個以上NVIDIA A100/H100 GPU。企業級CSP可擴展至1萬個以上CPU核心、50 TB以上總記憶體、1 PB以上平行檔案系統儲存(Lustre/GPFS)、100 Gbps InfiniBand骨幹,以及100個以上高端GPU。

\subsection{平台使用者功能}

CPT入口網站服務三個主要群體:探索專案資訊的訪客、購買市場資源(公有雲、HPC提供商、硬體、軟體、儲存)的使用者,以及執行USDC存款和鑄造操作的流動性提供者。

公有雲消費者可在包括FQ、Amazon和華為雲在內的廠商之間進行選擇,定價以USDC計價並附帶促銷優惠。廠商選擇會將使用者重新導向至原生入口網站(例如AWS),在該處進行標準操作,付款則透過CPT平台託管路由。平台隨後以法定貨幣與廠商結算。

HPC資源消費者可比較廠商(CT叢集、區域提供商、華為、AWS)的價格、硬體規格、效能指標和區域頻寬。廠商選擇和作業提交在CHESS入口網站進行,需存入足夠的USDC,資金託管至完成後以法定貨幣結算。儲存採購遵循相同的工作流程。

軟體選項包含使用者提供的應用程式,或來自Ansys、HPC軟體廠商和CHESS應用中心的平台列出解決方案。廠商上架可容納硬體、儲存、軟體和輔助運算產品。當兩個元件均來自平台清單時,系統會驗證硬體-軟體相容性,確保執行相容性。架構可根據需求演進,容納未來的功能擴展。


\subsection{公有雲整合}

市場整合來自主要公有雲廠商的運算資源,包括AWS、Azure、Google Cloud和阿里巴巴雲。定價以USDC為單位顯示,並附帶活動優惠和可用性狀態。

\subsubsection{廠商整合模型}

平台採用三種整合方法。直接API整合使用經銷商憑據透過廠商API(AWS EC2、Azure Resource Manager、GCP Compute Engine)進行即時佈建,支援自動實例生命週期管理。優惠券代碼系統透過預先生成的代碼解決容量限制,防止超賣,有基於價值(100美元通用積分)或資源特定(1000個GPU小時、10 TB儲存)的格式。託管服務提供商模型將CyberPlaza定位為MSP,擁有批量定價協議,管理廠商帳戶並提供整合帳單。

即時價格比較顯示運算、儲存和網路成本以及總擁有成本計算。團購折扣突出顯示與直接採購相比的潛在節省。

\subsubsection{作業提交工作流程}

HPC作業提交流程包括:透過篩選的CSP清單(CPU類型、GPU可用性、區域、定價)進行資源選擇;透過應用中心範本或自訂程式碼進行作業配置,並指定需求(節點、核心、記憶體、執行時間、GPU)和I/O位置;成本估計包含USDC明細和預計CPT獎勵;付款授權將USDC轉入託管並附帶應急緩衝;透過CHESS排程器分配執行,並進行即時狀態監控;完成結算交付結果,並自動進行付款分配、多餘退款和CPT獎勵發放。

進階功能包括支援100個以上作業的批次提交和參數掃描、定義循序執行的工作流程依賴、用於容錯的檢查點/重新啟動、可獲得50-70%折扣的預先佔用容量的點數實例投標,以及用於動態資源調整的自動擴展。

服務提供商透過中央儀表板管理運作,包括資源分配、作業監督、財務追蹤(USDC營收、CPT累積)和效能分析(客戶滿意度、使用率指標)。

\subsection{多叢集管理系統}

CHESS(叢集高效能執行和排程系統)平台提供跨地理位置分散的運算資源的統一管理。系統透過具備基於角色的存取控制的中央網路入口網站,整合監控、排程和資源分配。

\subsubsection{核心功能}

平台透過網路介面和SSH協議支援全面的資料管理,實現檔案操作,包括上傳、下載、壓縮和解壓縮。節點管理透過批次命令控制電源狀態、遠端存取(VNC、命令列),並支援異質硬體配置(CPU、GPU、FPGA)。資源配額對儲存和運算分配實施管理政策,在達到閾值時自動生成警示。

高可用性架構透過冗餘管理節點和資料庫複製消除單點故障。系統協調多個地理位置分散的叢集,在子叢集之間統一使用者角色傳播。


\subsection{效能監控基礎設施}

高效能和雲端運算系統整合了大量透過高速網路互聯的硬體資源,形成低延遲、高容量的配置。有效的叢集管理需要監控和管理工具,提供資源配置、即時效能追蹤、故障偵測與警示以及使用狀態視覺化。

\subsubsection{CHESS監控功能}

CHESS監控系統透過彙整儀表板提供全面的叢集監控,顯示CPU和記憶體使用率、負載狀態、儲存狀態,以及乙太網路和InfiniBand架構的網路吞吐量。自訂時間間隔選擇支援歷史趨勢分析和效能追蹤。儀表板顯示提供可自訂的大螢幕呈現,對儲存使用率、作業排程和網路統計資料進行動態指標更新。

多叢集監控擴展至地理位置分散的安裝,具有適應性螢幕配置和解析度最佳化。機架視覺化呈現物理拓撲,整合電源管理和VNC遠端存取控制。單節點監控捕捉細粒度的CPU、記憶體、儲存、負載和網路指標,同時提供故障診斷和復原建議。GPU監控追蹤裝置特定的使用率、記憶體利用率、溫度和頻寬。作業監控分析即時執行狀態和佇列組成,提供詳細的CPU利用率、記憶體消耗和節點負載統計資料。叢集警示實施可配置的閾值,並透過電子郵件和系統通知路由。

效能指標根據使用者定義的間隔收集,捕捉CPU、記憶體、磁碟和網路資料。物理拓撲視覺化包含機架和節點配置,以及基於閾值的故障警示。


\subsubsection*{排程器與資源管理}

高效的排程和資源管理在多叢集系統中至關重要。CHESS提供靈活的排程策略,包括FIFO、搶先和回填策略。系統支援具有服務品質(QoS)配置的資源預留、涵蓋串列、平行和GPU工作負載的進階作業提交,以及用於負載平衡最佳化的佇列管理。

\subsubsection{作業提交與管理}

使用者透過命令列介面、網路GUI或常見工作流程的應用程式範本提交作業。管理員配置資源配額、優先級和提交政策,以管理系統存取和使用。



% \subsection*{6.2.5 Pricing Module}

% This module will focus on calculating costs for resource usage and presenting pricing details to users. It will integrate with job scheduling and monitoring systems for real-time cost tracking.


% \subsubsection*{6.2.5 User Interfaces and Operational Portals}

\subsubsection{使用者管理}

平台支援自助註冊和管理員配置的帳戶,整合LDAP驗證以進行中央管理。基於角色的存取控制實施預設角色(管理員、部門管理員、使用者),並提供靈活的權限分配,以管理系統存取和功能。

\subsubsection{通知與訊息}

使用者會收到關於帳單和使用情況的自動警示,以及管理公告。


\subsection{應用中心}

應用中心透過可瀏覽的庫提供預安裝的HPC應用程式(Ansys、MATLAB、TensorFlow)。使用者透過具有互動式參數配置的圖形範本提交作業。輸出管理包含日誌檢視、錯誤分析、效能指標追蹤,以及整合的視覺化工具(用於AI應用的TensorBoard)。

\subsection{硬體效能評估}

硬體效能評估模組執行基準測試,測量CPU和GPU效能以及網路吞吐量和延遲。資源效率分析根據工作負載特性最佳化分配策略。故障復原指標評估硬體在故障情境下的可靠性和復原效能。

\subsection{安全架構與合規}

\subsubsection{多層安全模型}

平台在三層中實施深層防禦安全。智慧合約安全使用Certora或同等工具進行形式化驗證,由CertiK、Trail of Bits或OpenZeppelin進行年度第三方審計,提供高達50萬美元獎勵的漏洞賞金計畫,具有48小時時間鎖的可升級透明代理模式,以及用於緊急漏洞回應的斷路器。

平台安全包含透過OAuth 2.0和JWT驗證進行的API保護,速率限制為每分鐘100次請求,SP存取的IP白名單,以及90天的API金鑰輪換。資料加密實施TLS 1.3進行傳輸保護,AES-256進行靜態資料保護,敏感工作負載的端對端加密,以及用於金鑰管理的硬體安全模組。基礎設施安全部署Cloudflare DDoS保護、具有OWASP規則集的網路應用程式防火牆、每季滲透測試,以及用於事件監控的SIEM系統。

資料隱私和合規措施透過帳戶刪除權、資料可攜性、隱私設計原則和歐盟資料居留選項來滿足GDPR要求。KYC/AML程序對每月超過1萬USDC的交易實施基本驗證,對CSP認證實施增強驗證,對可疑活動進行交易監控,並遵守FATF旅行規則。資料隔離採用容器化或基於VM的作業執行、網路分段、完成後自動資料擦除,以及跨使用者洩漏預防。

\subsubsection{事件回應}

持續的安全作業中心監控異常活動,包括異常提款、智慧合約漏洞利用和API濫用。事件分類遵循四級嚴重性模型(重大、高、中、低),評估目標為15分鐘。重大事件會立即觸發合約暫停,並在一小時內發送多重簽名通知。重大事件的公開揭露在24小時內進行,後續檢討報告在7天內發布。復原程序透過治理管道部署修補程式,並從保險基金補償受影響的使用者。

\subsubsection{監管合規}

平台爭取在第1年獲得SOC 2 Type II認證(用於資料安全和可用性),在第2年獲得ISO 27001認證(用於資訊安全管理)。Cloud Security Alliance STAR認證驗證CSP安全狀態。PCI DSS合規性正在考慮用於未來的支付方式擴展。

\subsection{擴展性與效能最佳化}

\subsubsection{水平擴展架構}

平台透過分散式資料庫架構進行水平擴展,在各區域部署PostgreSQL讀取複本,根據ID雜湊進行使用者資料分片,使用Redis叢集處理熱資料(工作階段、定價),並透過Cloudflare CDN傳遞靜態資產。

微服務架構將功能分解為獨立可擴展的服務:使用者服務(驗證、設定檔)、作業服務(提交、排程、監控)、付款服務(USDC託管、結算、CPT獎勵)、SP服務(上架、認證、評級)、定價服務(預言機彙整)和通知服務(電子郵件、推送、鏈上事件)。每項服務根據需求自主擴展。

負載平衡在美國、歐盟和亞洲地區實施地理分布,使用Kubernetes水平Pod自動調整器進行動態容量調整,Hystrix斷路器防止連鎖故障,以及RabbitMQ佇列用於非同步作業處理。

\subsubsection{效能目標}

\begin{center}
\begin{tabular}{|l|c|c|}
\hline
\textbf{指標} & \textbf{目標(第1年)} & \textbf{目標(第3年)} \\
\hline
API回應時間 & <200ms (p95) & <100ms (p95) \\
作業提交時間 & <5秒 & <2秒 \\
付款結算時間 & <30秒 & <10秒 \\
頁面載入時間 & <2秒 & <1秒 \\
平台正常運行時間 & 99.5\% & 99.9\% \\
並行使用者數 & 10,000 & 100,000 \\
每日交易數 & 50,000 & 1,000,000 \\
\hline
\end{tabular}
\end{center}

\subsubsection{區塊鏈擴展性}

Arbitrum Layer 2部署為主要運作提供低於0.10美元的交易費和4萬TPS的吞吐量。批次交易處理將獎勵分配分組,以攤銷瓦斯成本。The Graph協議處理離鏈事件索引。未來開發包括針對高頻微支付場景的狀態通道。

瓦斯最佳化技術透過Merkle證明為基礎的獎勵領取(節省80%)、veCPT餘額的懶惰評估、壓縮儲存變數編碼,以及在功能等效時優先使用事件日誌而非狀態變數,來降低交易成本。

\subsection{災難復原}

\subsubsection{備份基礎設施}

資料庫備份每天執行一次(完整),每六小時執行一次(增量),並進行持續交易日誌複製。系統在冷儲存歸檔前維持30天的保留期。智慧合約狀態利用區塊鏈固有的不可變性,輔以封存節點部署和每季的去中心化儲存快照(IPFS/Arweave)。使用者作業結果備份至指定儲存端點,平台中繼資料保留90天,並具備根據GDPR合規要求的按需匯出功能。

\subsubsection{復原目標}

表~\ref{tab:recovery-targets}指定了元件級的復原時間(RTO)和復原點(RPO)目標。

\begin{table}[htbp]
\centering
\caption{復原時間與復原點目標}
\label{tab:recovery-targets}
\begin{tabular}{lcc}
\hline
\textbf{元件} & \textbf{RTO} & \textbf{RPO} \\
\hline
智慧合約 & 不適用 & 0 \\
網路入口網站 & 1小時 & 6小時 \\
資料庫 & 2小時 & 1小時 \\
作業排程器 & 30分鐘 & 15分鐘 \\
\hline
\end{tabular}
\end{table}

美國和歐盟地區的主動-主動部署在主要地區不可用5分鐘後,啟用自動DNS故障轉移。即時跨地區資料同步維持一致性,並具備操作干預的手動覆寫功能。

\subsection{發展路線圖}

近期開發(6-12個月)優先處理iOS和Android的行動應用程式,增強API服務(RESTful、GraphQL)以進行第三方整合,基於機器學習的成本最佳化,以及額外的區塊鏈橋部署(Polygon、Optimism)。

中期目標(1-2年)透過邊緣運算支援IoT部署、機密運算整合(Intel SGX、AMD SEV)用於敏感工作負載、去中心化儲存協議(Filecoin、Arweave)、專用AI/ML資源市場,以及探索性量子運算合作夥伴關係,來擴展平台功能。

長期願景(2-5年)包括全面轉向DAO治理、開放去中心化運算協議開發、零知識證明實施以增強隱私、透過IBC或同等協議的跨鏈互操作性,以及基於NFT的實體運算資源代幣化。

\subsection{總結}

本章詳細介紹了整合Web3區塊鏈基礎設施與成熟HPC系統的技術架構。混合設計將去中心化激勵機制(CPT代幣、投票託管治理)與成熟的CHESS叢集管理平台連接起來。安全架構透過智慧合約審計、基礎設施強化和監管合規路徑(SOC 2、ISO 27001)實施多層保護。系統可從數千名擴展至數十萬名並行使用者,同時維持低於200ms的API回應時間。

相較於現有的去中心化運算專案(Golem、iExec、Render),CyberPlaza透過成熟的基礎設施(20年以上的CHESS平台歷史)、企業合規導向、超越點對點架構的多雲整合、預先整合的應用生態系,以及結合去中心化存取與專業SP認證的混合市場進行差異化。這種定位滿足了企業運算需求,同時實現了Web3經濟參與。
