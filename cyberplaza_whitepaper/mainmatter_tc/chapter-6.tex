\chapter{技術與架構}



\section{專案治理基礎架構}

\subsection{整體說明}

CyberPlaza Network 由 CyberPlaza 基金會與 CyberPlaza 社群組成。

CyberPlaza 基金會是一個非營利去中心化組織,致力於 CyberPlaza 平台的順利運營、計算技術與應用的推廣與發展,以及支持平台上的去中心化社群建設與發展。基金會由 CyberPlaza 社群中的 CPT 持有者擁有並控制。基金會由網路核心成員(參見白皮書第8節)以及基金會根據需要不定期任命的顧問運作。基金會將成立 CyberPlaza Labs,負責開發與研究新的計算資源技術與應用,以推動平台所需的技術創新與進步。

CyberPlaza 社群是網路的社群部分,由流動性提供者、使用者與服務提供者(SP)組成,共同參與基金會治理、發展與推廣。社群成員可透過參與治理、向基金會提出業務方向與技術發展提案,以及交流與分享經驗,推動平台的發展與成長。

CyberPlaza 基金會與 CyberPlaza 社群之間的緊密聯繫,是實現網路願景與使命的關鍵。

\subsection{智慧合約模組}

我們將把 CPT 部署為可相容於 ERC20 的智慧合約至 Arbitrum(以太坊第2層),選擇此區塊鏈的原因是其低交易成本與高吞吐量。平台也將根據生態系擴展的需要,橋接到其他鏈。

CPT 代幣合約包含以下關鍵功能:標準 ERC20 功能(轉帳、授權等)、用於 veToken 機制的質押與鎖定功能、治理投票整合、獎勵分配機制、緊急暫停功能(由治理控制),以及可升級代理模式以利未來增強功能。

\textbf{注意}:本平台直接使用 USDC 進行支付,無需專屬穩定幣,也能避免相關監管風險。

\begin{verbatim}
// SPDX-License-Identifier: MIT
pragma solidity ^0.8.0;

import "@openzeppelin/contracts/token/ERC20/ERC20.sol";

contract CPTToken is ERC20 {
  struct LockInfo {
     uint256 amount;
     uint256 lockTimestamp;
     uint256 unlockTimestamp;
  }

   mapping (address => LockInfo[]) public locks;

   constructor(uint256 initialSupply) ERC20("CPT Token", "CPT") {
     _mint(msg.sender, initialSupply);
   }

   function lock(uint256 _amount, uint256 _lockTime) public {
     require(_amount <= balanceOf(msg.sender), "Not enough CPT to lock");
     require(_lockTime > 0, "Lock time must be positive");

       uint256 lockUntil = block.timestamp + _lockTime;
    
       LockInfo memory newLock = LockInfo({
           amount: _amount,
           lockTimestamp: block.timestamp,
           unlockTimestamp: lockUntil
       });
    
       locks[msg.sender].push(newLock);
    
       _burn(msg.sender, _amount);

   }

  function unlock(uint256 lockIndex) public {
    require(lockIndex < locks[msg.sender].length, 
            "No lock found at this index");
    require(block.timestamp >= locks[msg.sender][lockIndex].unlockTimestamp,
            "CPT still locked");

        uint256 amountToUnlock = locks[msg.sender][lockIndex].amount;
        locks[msg.sender][lockIndex] = 
            locks[msg.sender][locks[msg.sender].length - 1];
        locks[msg.sender].pop();
    
        _mint(msg.sender, amountToUnlock);
    }

   function calculateLockedAmount(address user, uint256 lockDuration) 
       public view returns (uint256) {
     uint256 totalLockedAmount = 0;

        for (uint256 i = 0; i < locks[user].length; i++) {
           if (block.timestamp - locks[user][i].lockTimestamp > lockDuration) {
               totalLockedAmount += locks[user][i].amount;
           }
        }
    
        return totalLockedAmount;
    }

}
\end{verbatim}

\subsection{代幣標準與小數點處理}

CPT 代幣遵循標準 ERC20 規範,使用18位小數,而 USDC 則使用6位小數。平台所有轉換操作均採用 SafeMath 函式庫,以防止溢位與不足溢位錯誤。價格預言機整合了小數點正規化邏輯,最低交易門檻則可減輕灰塵攻擊向量。對於分數金額,協議實施保守的捨入機制。

\subsection{投票質押代幣機制}

平台實施投票質押(ve)代幣模型,以協調長期利害關係人的誘因。使用者可將 CPT 鎖定1週至4年不等的期間,獲得不可轉讓的 veCPT 代幣,該代幣將決定治理權重與獎勵分配。

veCPT 餘額遵循以下關係:
\begin{equation}
\text{veCPT} = \text{CPT}_{\text{locked}} \times \min\left(\frac{t_{\text{lock}}}{t_{\text{max}}}, 1\right) \times 2.5
\end{equation}
其中 $t_{\text{lock}}$ 代表選擇的鎖定期間,$t_{\text{max}} = 4$ 年為最大鎖定期間。2.5倍的乘數為4年承諾提供最大治理權重。

當鎖定期間即將到期時,veCPT 餘額將線性遞減:
\begin{equation}
\text{veCPT}(t) = \text{CPT}_{\text{locked}} \times \frac{t_{\text{remaining}}}{t_{\text{max}}} \times 2.5
\end{equation}
此遞減機制透過鎖定延長或代幣重新鎖定,鼓勵持續參與。

\subsubsection{獎勵分配}

平台以 USDC 收取的收入中,30\% 將分配至質押獎勵池,分配頻率為每周或每月。個別獎勵將根據 veCPT 持有量按比例計算:
\begin{equation}
\text{Reward}_{\text{user}} = \text{Revenue}_{\text{pool}} \times \frac{V_{\text{user}}}{V_{\text{total}}}
\end{equation}
其中 $V_{\text{user}}$ 代表使用者的 veCPT 餘額,$V_{\text{total}}$ 代表 veCPT 總供應量。有效年收益率(APY)會根據質押參與度與平台績效動態變化:
\begin{equation}
\text{APY} = \frac{\text{Annual Revenue Pool}}{\text{Total CPT Staked Value}} \times \frac{\text{veCPT Multiplier}}{\text{Average Multiplier}}
\end{equation}

\subsubsection{安全性與最佳化}

智慧合約將遵循 OpenZeppelin 標準接受第三方審計,參數修改由多重簽章治理控制。所有獎勵分配均在鏈上追蹤,以確保透明度。安全功能包括外部呼叫的可重入防護、基於角色的存取控制、緊急暫停功能,以及可升級代理模式。關鍵參數變更實施48小時時間鎖。

瓦斯最佳化採用基於 Merkle 樹的批次領取、veCPT 餘額的延遲評估、打包儲存變數,以及基於事件的鏈外索引。這些技術可降低交易成本,同時維持安全性保證。

\subsection{預言機整合}

平台整合 Chainlink 去中心化預言機,用於價格發現與資料彙總。CPT/USD 價格摘要彙整來自 Uniswap V3 時間加權平均價格與中心化交易所報價的資料。USDC/USD 驗證使用 Chainlink 的驗證摘要,偏差閾值為0.5\%。預言機每5分鐘更新一次,或在價格波動超過1\%時更新,並具備手動備份機制以確保備援。

對於計算資源定價,鏈外彙整者監控主要雲端服務提供商(AWS、Azure、GCP、Alibaba Cloud)的公開 API,計算運算、儲存與頻寬的即時市場價格。彙整後的定價每天或在偏差超過5\%時發布至鏈上預言機合約。

預言機安全性仰賴至少7個獨立 Chainlink 節點的共識。系統將拒絕偏離中位數超過10\%或資料超過1小時的價格更新。斷路器會在偵測到操縱企圖時自動暫停交易。

\subsection{治理架構}

關鍵平台運作需透過 Gnosis Safe 實施的多重簽章批准。超過10萬 USDC 的財庫移動需要9簽5,智慧合約升級需要9簽7並搭配48小時時間鎖。參數調整採用9簽4共識,緊急安全回應則使用5簽3的快速回應配置。

治理流程遵循結構化時程:持有10萬以上 veCPT 的持有者可提出提案,接著進行7天社群討論期與5天鏈上投票階段,其中1 veCPT 等於1票。經批准的提案將在48小時延遲後執行。多重簽章理事會對惡意提案保留否決權,並接受每季度審查。

\subsection{跨鏈基礎架構}

平台實施 LayerZero 跨鏈協議以支援多鏈部署。Arbitrum 作為主鏈,因其低交易成本與高吞吐量。以太坊主網支援針對需要第1層安全性的機構使用者,而 Polygon 整合則為成本敏感型使用者提供更低的交易成本。未來擴展將包含 Optimism(2024年第3季)與 Base(2024年第4季),以實現更廣泛的生態系整合。

跨鏈橋安全性整合多項防護措施:流動性上限將每條鏈的跨鏈供應限制在10\%、速率限制將吞吐量約束為每小時100萬 CPT、緊急暫停機制可回應異常情況,以及5\%的保險基金為跨鏈價值提供擔保,以防潛在漏洞。

\subsection{錢包基礎架構}

作為標準 ERC20 代幣,CPT 支援所有相容錢包,包括瀏覽器擴充功能(MetaMask、Rabby、Rainbow)、行動應用程式(Trust Wallet、Coinbase Wallet、imToken)、硬體裝置(Ledger、Trezor),以及智慧合約錢包(Argent、Gnosis Safe)。未來計畫部署與 Fireblocks 和 Copper.co 的機構託管整合。

網頁入口實施 WalletConnect 與 Web3Modal 協議,以實現標準化錢包連線。在連線授權後,平台將查詢使用者餘額、質押部位與 veCPT 持有量,以實現完整功能存取。交易簽名遵循 EIP-712 標準的具型別結構化資料,呈現人類可讀的訊息,以提升防範網路釣魚向量的安全性。


\section{市場計算基礎架構}

\subsection{系統架構}

平台實施三層架構。Web3 介面層透過 React.js 與 ethers.js 框架管理錢包認證(WalletConnect)、USDC 支付處理與 CPT 獎勵分配。協調層協調 CHESS 叢集管理系統、工作排程、資源分配、效能監控與服務提供者認證流程。計算資源層彙整 CSP 叢集、公開雲端 API(AWS、Azure、GCP、Alibaba)、私有高效能運算(HPC)中心,以及未來的邊緣計算節點。

交易流程如下:提交工作並存入 USDC、智慧合約代管至完成、CHESS 調介的資源匹配、在分配的服務提供者基礎架構上執行、即時 SLA 合規性監控、結果交付並自動結算支付,以及按比例分配 CPT 獎勵(使用者1-3\%,服務提供者2-5\%)。

\subsection{高效能運算基礎架構元件}

高效能運算基礎架構包含專門的節點類型:運算節點使用多核處理器與大量記憶體執行數值模擬與資料分析;視覺化節點使用 GPU 加速渲染大型資料集;I/O 節點管理儲存與運算架構之間的資料傳輸;儲存節點提供高並發檔案系統;管理節點協調資源分配與工作排程。

網路架構採用高速互連技術(InfiniBand、乙太網路)進行節點間通訊。平行檔案系統可針對大型資料集與中間結果進行並發多節點讀寫操作。

\subsubsection{軟體堆疊}

監控與管理工具為管理員提供跨系統元件的即時健康與效能資料,包括 CPU 使用率、記憶體消耗與網路流量模式。叢集管理軟體協調整體系統運作,具備針對地理位置分散的安裝中運算節點的佈建、監控與維護能力。

資源分配採用專門的排程器管理 CPU 時間、記憶體與其他運算資源,以最大化系統使用率效率。使用者介面涵蓋命令列工具與網頁入口,用於工作提交與管理。HPC 應用中心彙整領域特定應用程式與範本,讓使用者可直接下載並部署運算工具。整合計費系統針對不同資源類型與計費週期實施透明定價策略,促進資源的合理使用與準確成本會計。

\subsection{支付與結算基礎架構}

\subsubsection{託管機制}

工作提交會啟動託管流程,使用者在此流程中批准將 USDC 支付至平台的智慧合約。託管合約會計算估計成本,其中包含資源類型(CPU/GPU/儲存)、持續時間預測、預言機衍生的市場定價,以及20\%的緩衝以應對潛在超支。USDC 在批准後轉移至託管,並針對唯一工作識別碼鎖定。

工作完成後,實際資源消耗將決定最終結算。服務提供者直接以 USDC 收到95-98\%的費用,平台則保留2-5\%的交易費。多餘的託管資金會自動退還給使用者,CPT 獎勵則按比例分配給使用者(1-3\%)與服務提供者(2-5\%)。

\subsubsection{爭議解決協議}

違反 SLA 會觸發階段式解決機制。在5分鐘內失敗的工作符合自動全額退款的條件。部分完成的工作將根據實際交付產生比例退款。使用者可在72小時內提出爭議並附上支持證據。價值超過1萬 USDC 的案例將升級至平台治理仲裁,而保險基金將補償最高10萬 USDC 的驗證索賠。

\subsubsection{服務提供者認證}

服務提供者認證需經過多階段驗證流程。初始註冊需要公司驗證文件、基礎架構規格、支付錢包位址,以及安全合規證書(SOC 2、ISO 27001)。技術驗證採用業界標準基準,包括 High-Performance Linpack(HPL)、High-Performance Conjugate Gradient(HPCG)、STREAM 記憶體頻寬、用於 AI 工作負載的 MLPerf,以及網路延遲評估。安全審計驗證 AES-256 加密、網路隔離與 DDoS 防護能力。

經批准的候選人進入30天試用期,期間會加強監控,且工作併發限制為10。順利完成後將授予認證服務提供者(CSP)資格,可存取機構客戶與團購參與權。CSP 將出現在高級目錄中,並附有驗證徽章。

持續合規要求每月99.5\%的上線時間、5分鐘內的工作啟動,以及效能在宣稱基準的10\%以內。每季重新認證以驗證持續能力。針對重大弱點的安全修補必須在48小時內部署。違規將觸發階段式處罰:第一次違規警告並要求7天內補救、第二次違規暫停30天、第三次違規撤銷認證。

\subsection{技術堆疊}

平台針對前端開發採用 React.js 18+ 搭配 TypeScript,針對 Web3 整合採用 ethers.js v6 與 WalletConnect v2,並使用 Material-UI 確保介面一致性。後端架構針對 API 服務採用 Node.js/Express.js 或 Python FastAPI,針對關聯式持久化採用 PostgreSQL,針對快取採用 Redis,針對非同步工作佇列採用 RabbitMQ/Kafka,針對區塊鏈事件索引採用 The Graph,以及針對可觀測性採用 Prometheus/Grafana。DevOps 基礎架構透過 Docker 容器化所有服務,透過 Kubernetes 協調生產部署,透過 GitHub Actions 實施 CI/CD,透過 Cloudflare CDN 散佈內容,並透過 Nginx 負載平衡流量。

入門級 CSP 需要100個以上的 CPU 核心(Intel Xeon/AMD EPYC)、500 GB RAM、10 TB NVMe SSD 或50 TB HDD、10 Gbps 網路上行連結,以及可選的4個以上 NVIDIA A100/H100 GPU。企業級 CSP 可擴展至1萬個以上的 CPU 核心、50 TB 以上的總 RAM、1 PB 以上的平行檔案系統儲存(Lustre/GPFS)、100 Gbps InfiniBand 骨幹,以及100個以上的高階 GPU。

\subsection{平台使用者功能}

CPT 入口服務三個主要族群:探索專案資訊的訪客、購買市場資源(公開雲端、HPC 提供者、硬體、軟體、儲存)的使用者,以及執行 USDC 存款與鑄造操作的流動性提供者。

公開雲端消費者可選擇廠商(包括 FQ、Amazon 與華為雲),定價以 USDC 計算,並附帶促銷優惠。廠商選擇會將使用者重新導向至原生入口(例如 AWS),標準操作在該處進行,支付則透過 CPT 平台託管路由。平台隨後以法定貨幣與廠商結算。

HPC 資源消費者可比較廠商(CT 叢集、區域提供者、華為、AWS)的價格、硬體規格、效能指標與區域頻寬。廠商選擇與工作提交透過 CHESS 入口進行,需存入足夠的 USDC,資金託管至完成後以法定貨幣結算。儲存採購遵循相同的工作流程。

軟體選項涵蓋使用者提供的應用程式,或平台列出的 Ansys、HPC 軟體廠商與 CHESS 應用中心的解決方案。廠商上架可容納硬體、儲存、軟體與輔助運算產品。當兩個元件均來自平台清單時,系統會驗證硬體-軟體相容性,以確保執行相容性。架構可根據需求演進,容納未來的功能擴展。


\subsection{公開雲端整合}

市場彙整來自主要公開雲端廠商(AWS、Azure、Google Cloud、Alibaba Cloud)的計算資源。定價以 USDC 顯示,並附帶活動促銷與可用性狀態。

\subsubsection{廠商整合模型}

平台採用三種整合方法。直接 API 整合利用轉銷商憑證,透過廠商 API(AWS EC2、Azure Resource Manager、GCP Compute Engine)進行即時佈建,可自動管理執行個體生命週期。優惠券代碼系統透過預先生成的代碼解決容量限制,防止超賣,提供基於價值(100美元通用點數)或基於資源(1000 GPU 小時、10 TB 儲存)的格式。受管理服務提供者(MSP)模型將 CyberPlaza 定位為 MSP,擁有大量定價協議,管理廠商帳戶並提供整合計費。

即時價格比較顯示運算、儲存與網路成本,以及總擁有成本計算。團購折扣強調相較於直接採購的潛在節省。

\subsubsection{工作提交工作流程}

HPC 工作提交流程涵蓋:透過篩選的 CSP 清單(CPU 類型、GPU 可用性、區域、定價)選擇資源;透過應用中心範本或自訂程式碼設定工作,並指定需求(節點、核心、記憶體、執行時間、GPU)與 I/O 位置;成本估算,包含 USDC 細目與預測 CPT 獎勵;支付授權,將 USDC 轉移至託管並附帶應變緩衝;透過 CHESS 排程器分配執行,並進行即時狀態監控;完成結算,交付結果並自動分配支付、退還多餘金額,以及發行 CPT 獎勵。

進階功能包括支援100個以上工作的批次提交與參數掃描、定義順序執行的工作流程依賴性、用於容錯的檢查點/重新啟動、搶購型執行個體出價(可獲得可搶購容量50-70\%的折扣),以及用於動態資源調整的自動擴展。

服務提供者透過中央儀表板管理運作,涵蓋資源分配、工作監督、財務追蹤(USDC 收入、CPT 累積)與效能分析(客戶滿意度、使用率指標)。

\subsection{多叢集管理系統}

CHESS(Cluster High-performance Execution and Scheduling System)平台針對地理位置分散的計算資源提供統一管理。系統透過具備基於角色的存取控制的中央網頁入口,整合監控、排程與資源分配。

\subsubsection{核心功能}

平台透過網頁介面與 SSH 協議支援全面的資料管理,可進行檔案操作,包括上傳、下載、壓縮與解壓縮。節點管理透過批次命令控制電源狀態、遠端存取(VNC、shell),並支援異質硬體配置(CPU、GPU、FPGA)。資源配額針對儲存與運算分配執行管理政策,在達到閾值時自動產生警示。

高可用性架構透過備援管理節點與資料庫複製,消除單點故障。系統協調多個地理位置分散的叢集,並在子叢集之間統一傳播使用者角色。


\subsection{效能監控基礎架構}

高效能與雲端計算系統彙整大量硬體資源,透過高速網路互連,形成低延遲、高容量的配置。有效的叢集管理需要監控與管理工具,提供資源配置、即時效能追蹤、故障偵測與警示,以及使用狀態視覺化。

\subsubsection{CHESS 監控功能}

CHESS 監控系統透過整合儀表板提供全面的叢集監控,顯示乙太網路與 InfiniBand 架構的 CPU 與記憶體使用率、負載狀態、儲存狀態與網路吞吐量。自訂時間間隔選擇可進行歷史趨勢分析與效能追蹤。儀表板顯示提供可自訂的大螢幕展示,針對儲存使用、工作排程與網路統計數據進行動態指標更新。

多叢集監控延伸至地理位置分散的安裝,具備適應性螢幕配置與解析度最佳化。機櫃視覺化呈現實體拓撲,整合電源管理與 VNC 遠端存取控制。單節點監控擷取細緻的 CPU、記憶體、儲存、負載與網路指標,同時提供故障診斷與復原建議。GPU 監控追蹤裝置特定使用率、記憶體利用率、溫度與頻寬。工作監控分析即時執行狀態與佇列組成,包含詳細的 CPU 使用率、記憶體消耗與節點負載統計數據。叢集警示實施可配置閾值,透過電子郵件與系統通知路由。

效能指標以使用者定義的間隔收集,擷取 CPU、記憶體、磁碟與網路資料。實體拓撲視覺化涵蓋機櫃與節點佈局,並具備基於閾值的故障警示。


\subsubsection*{排程器與資源管理}

高效的排程與資源管理在多叢集系統中至關重要。CHESS 提供靈活的排程政策,包括 FIFO、搶佔與回填策略。系統支援帶有服務品質(QoS)配置的資源預訂、涵蓋串列、平行與 GPU 工作負載的進階工作提交,以及用於負載平衡最佳化的佇列管理。

\subsubsection{工作提交與管理}

使用者透過命令列介面、網頁型 GUI 或針對常見工作流程的應用範本提交工作。管理員配置資源配額、優先順序層級與提交政策,以管理系統存取與使用率。



% \subsection*{6.2.5 Pricing Module}

% This module will focus on calculating costs for resource usage and presenting pricing details to users. It will integrate with job scheduling and monitoring systems for real-time cost tracking.


% \subsubsection*{6.2.5 User Interfaces and Operational Portals}

\subsubsection{使用者管理}

平台支援自行註冊與管理員設定的帳戶,整合 LDAP 認證以進行中央管理。基於角色的存取控制實施預設角色(管理員、部門管理員、使用者),並具備彈性權限分配,以管理系統存取與功能。

\subsubsection{通知與訊息}

使用者會收到關於計費與使用訊息的自動警示,以及管理公告。


\subsection{應用中心}

應用中心透過可瀏覽的圖書館提供預安裝的 HPC 應用程式(Ansys、MATLAB、TensorFlow)。使用者透過具備互動式參數配置的圖形化範本提交工作。輸出管理涵蓋日誌檢視、錯誤分析、效能指標追蹤,以及整合的視覺化工具(用於 AI 應用程式的 TensorBoard)。

\subsection{硬體效能評估}

硬體效能評估模組執行基準測試,測量 CPU 與 GPU 效能,以及網路吞吐量與延遲。資源效率分析根據工作負載特性最佳化分配策略。故障復原指標評估硬體可靠性與故障情況下的復原效能。

\subsection{安全架構與合規性}

\subsubsection{多層級安全模型}

平台在三個層級實施深層防禦安全。智慧合約安全採用 Certora 或同等工具進行正式驗證、由 CertiK、Trail of Bits 或 OpenZeppelin 進行年度第三方審計、針對重大弱點提供高達50萬美元獎金的漏洞回報計畫、具備48小時時間鎖的可升級透明代理模式,以及用於緊急漏洞回應的斷路器。

平台安全涵蓋透過 OAuth 2.0 與 JWT 認證的 API 保護,搭配每分鐘100次請求的速率限制、針對服務提供者存取的 IP 白名單,以及90天 API 金鑰輪替。資料加密實施 TLS 1.3 用於傳輸保護、AES-256 用於靜態資料保護、針對敏感工作負載的端對端加密,以及用於金鑰管理的硬體安全模組。基礎架構安全部署 Cloudflare DDoS 保護、帶有 OWASP 規則集的 Web 應用程式防火牆、每季滲透測試,以及用於事件監控的 SIEM 系統。

資料隱私與合規措施透過帳戶刪除權限、資料可攜性、隱私設計原則與 EU 資料駐留選項,處理 GDPR 要求。KYC/AML 程序針對每月超過1萬 USDC 的交易實施基本驗證、針對 CSP 認證的增強驗證、針對可疑活動的交易監控,以及 FATF 旅行規則合規性。資料隔離採用容器化或基於 VM 的工作執行、網路分段、完成後自動資料清除,以及跨使用者洩漏預防。

\subsubsection{事件回應}

持續安全操作中心監控異常活動,包括異常提領、智慧合約漏洞與 API 濫用。事件分類遵循四層嚴重性模型(重大、高、中、低),評估目標為15分鐘。重大事件觸發立即合約暫停,並在1小時內發送多重簽章通知。針對重大事件的公開揭露在24小時內進行,事後檢討報告則在7天內發布。復原程序透過治理管道部署修補程式,並從保險基金補償受影響的使用者。

\subsubsection{監管合規性}

平台追求 SOC 2 Type II 認證(第1年目標)以確保資料安全性與可用性,以及 ISO 27001 認證(第2年目標)以確保資訊安全管理。Cloud Security Alliance STAR 認證驗證 CSP 安全姿態。PCI DSS 合規性正在考慮中,以利未來支付方式擴展。

\subsection{可擴展性與效能最佳化}

\subsubsection{水平擴展架構}

平台透過分散式資料庫架構進行水平擴展,在各區域部署 PostgreSQL 唯讀副本、按 ID 雜湊進行使用者資料分片、針對熱資料(工作階段、定價)進行 Redis 叢集化,以及透過 Cloudflare CDN 進行靜態資產交付。

微服務架構將功能分解為可獨立擴展的服務:使用者服務(認證、設定檔)、工作服務(提交、排程、監控)、支付服務(USDC 託管、結算、CPT 獎勵)、服務提供者服務(上架、認證、評級)、定價服務(預言機彙整),以及通知服務(電子郵件、推播、鏈上事件)。每項服務根據需求自動擴展。

負載平衡在美國、歐盟與亞洲地區實施地理位置分散,透過 Kubernetes Horizontal Pod Autoscaler 進行動態容量調整、Hystrix 斷路器防止連鎖故障,以及 RabbitMQ 佇列進行非同步工作處理。

\subsubsection{效能目標}

\begin{center}
\begin{tabular}{|l|c|c|}
\hline
\textbf{指標} & \textbf{第一年目標} & \textbf{第三年目標} \\
\hline
API 回應時間 & <200ms (p95) & <100ms (p95) \\
工作提交時間 & <5秒 & <2秒 \\
支付結算 & <30秒 & <10秒 \\
頁面載入時間 & <2秒 & <1秒 \\
平台上線時間 & 99.5\% & 99.9\% \\
併發使用者 & 10,000 & 100,000 \\
每日交易數 & 50,000 & 1,000,000 \\
\hline
\end{tabular}
\end{center}

\subsubsection{區塊鏈可擴展性}

Arbitrum 第2層部署為主要運作提供低於0.10美元的交易費與4萬 TPS 的吞吐量。批次交易處理將獎勵分配分組,以分攤瓦斯成本。The Graph 協議負責鏈外事件索引。未來開發將包含用於高頻微支付場景的狀態通道。

瓦斯最佳化技術透過基於 Merkle 證明的獎勵領取(節省80\%)、veCPT 餘額的延遲評估、打包儲存變數編碼,以及在功能等效的情況下優先使用事件日誌而非狀態變數,來降低交易成本。

\subsection{災難復原}

\subsubsection{備份基礎架構}

資料庫備份每日執行一次完整備份,每六小時執行一次增量備份,並持續進行交易日誌複製。系統在冷儲存歸檔前維持30天的保留期。智慧合約狀態利用區塊鏈的固有不可變性,輔以歸檔節點部署與每季度分散式儲存快照(IPFS/Arweave)。使用者工作結果備份至指定的儲存端點,平台中繼資料保留90天,並具備符合 GDPR 規範的隨需匯出功能。

\subsubsection{復原目標}

表~\ref{tab:recovery-targets} 規格說明元件層級的復原時間(RTO)與復原點(RPO)目標。

\begin{table}[htbp]
\centering
\caption{復原時間與復原點目標}
\label{tab:recovery-targets}
\begin{tabular}{lcc}
\hline
\textbf{元件} & \textbf{RTO} & \textbf{RPO} \\
\hline
智慧合約 & N/A & 0 \\
網頁入口 & 1小時 & 6小時 \\
資料庫 & 2小時 & 1小時 \\
工作排程器 & 30分鐘 & 15分鐘 \\
\hline
\end{tabular}
\end{table}

美國與歐盟地區的雙活部署可在主要地區中斷5分鐘後自動進行 DNS 容錯移轉。即時跨地區資料同步維持一致性,並具備操作干預的手動覆寫功能。

\subsection{開發藍圖}

短期開發(6-12個月)優先開發 iOS 與 Android 的行動應用程式、針對第三方整合的增強 API 產品(RESTful、GraphQL)、基於機器學習的成本最佳化,以及額外的區塊鏈橋部署(Polygon、Optimism)。

中期目標(1-2年)透過支援 IoT 部署的邊緣計算、針對敏感工作負載的機密計算整合(Intel SGX、AMD SEV)、分散式儲存協議(Filecoin、Arweave)、專門的 AI/ML 資源市場,以及探索性量子計算夥伴關係,擴展平台功能。

長期願景(2-5年)包含全面轉移至 DAO 治理、開發開放去中心化運算協議、實施零知識證明以增強隱私、透過 IBC 或同等協議實現跨鏈互通性,以及基於 NFT 的實體計算資源代幣化。

\subsection{總結}

本章詳細介紹了整合 Web3 區塊鏈基礎架構與成熟 HPC 系統的技術架構。混合設計將去中心化誘因機制(CPT 代幣、投票質押治理)與經過驗證的 CHESS 叢集管理平台相連接。安全架構透過智慧合約審計、基礎架構強化與監管合規途徑(SOC 2、ISO 27001)實施多層級保護。系統可從數千名擴展至數十萬名併發使用者,同時維持低於200ms 的 API 回應時間。

相較於現有的去中心化運算專案(Golem、iExec、Render),CyberPlaza 透過成熟的基礎架構(20年以上的 CHESS 平台歷史)、企業合規導向、超越點對點架構的多雲端整合、預先整合的應用生態系,以及結合去中心化存取與專業服務提供者認證的混合市場進行差異化。此定位既能滿足企業運算需求,又能實現 Web3 經濟參與。
