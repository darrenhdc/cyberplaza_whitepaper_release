\chapter{技術與架構}

\section{專案治理基礎設施}

\subsection{概述}

CyberPlaza 網路由 CyberPlaza 基金會與 CyberPlaza 社群組成。

CyberPlaza 基金會是一個非營利的去中心化組織,致力於 CyberPlaza 平台的順利運作、計算技術與應用的推廣與發展,以及支援平台上的去中心化社群建設與發展。基金會由 CyberPlaza 社群中的 CPT 持有者所有並控制。基金會由網路核心成員(請見白皮書第8節)以及基金會依需要隨時任命的顧問管理。基金會將成立 CyberPlaza Labs,負責開發與研究新的計算資源技術與應用,以推動平台所需的技術創新與進步。

CyberPlaza 社群是網路的社群部門,由流動性提供者、使用者與服務提供者(SP)組成,共同參與基金會治理、開發與推廣。社群成員可透過參與治理、向基金會提出業務方向與技術發展的提案,以及交流分享經驗,推動平台的發展與成長。

CyberPlaza 基金會與 CyberPlaza 社群之間的緊密聯繫,對於實現網路的願景與使命至關重要。

\subsection{智慧合約模組}

我們將在 Arbitrum(以太坊第2層網路)上部署符合 ERC20 標準的 CPT 智慧合約,選擇 Arbitrum 的原因在於其低交易成本與高吞吐量。平台也會依生態系擴張的需要,與其他鏈進行跨鏈橋接。

CPT 代幣合約包含以下關鍵功能:標準 ERC20 功能(轉帳、授權等)、用於 veToken 機制的質押與鎖倉功能、治理投票整合、獎勵分配機制、緊急暫停功能(由治理控制),以及用於未來強化的可升級代理模式。

\textbf{注意}: 平台直接使用 USDC 進行支付,排除了專屬穩定幣的需求與相關監管風險。

\begin{verbatim}
// SPDX-License-Identifier: MIT
pragma solidity ^0.8.0;

import "@openzeppelin/contracts/token/ERC20/ERC20.sol";

contract CPTToken is ERC20 {
  struct LockInfo {
     uint256 amount;
     uint256 lockTimestamp;
     uint256 unlockTimestamp;
  }

   mapping (address => LockInfo[]) public locks;

   constructor(uint256 initialSupply) ERC20("CPT Token", "CPT") {
     _mint(msg.sender, initialSupply);
   }

   function lock(uint256 _amount, uint256 _lockTime) public {
     require(_amount <= balanceOf(msg.sender), "Not enough CPT to lock");
     require(_lockTime > 0, "Lock time must be positive");

       uint256 lockUntil = block.timestamp + _lockTime;
    
       LockInfo memory newLock = LockInfo({
           amount: _amount,
           lockTimestamp: block.timestamp,
           unlockTimestamp: lockUntil
       });
    
       locks[msg.sender].push(newLock);
    
       _burn(msg.sender, _amount);

   }

  function unlock(uint256 lockIndex) public {
    require(lockIndex < locks[msg.sender].length, 
            "No lock found at this index");
    require(block.timestamp >= locks[msg.sender][lockIndex].unlockTimestamp,
            "CPT still locked");

        uint256 amountToUnlock = locks[msg.sender][lockIndex].amount;
        locks[msg.sender][lockIndex] = 
            locks[msg.sender][locks[msg.sender].length - 1];
        locks[msg.sender].pop();
    
        _mint(msg.sender, amountToUnlock);
    }

   function calculateLockedAmount(address user, uint256 lockDuration) 
       public view returns (uint256) {
     uint256 totalLockedAmount = 0;

        for (uint256 i = 0; i < locks[user].length; i++) {
           if (block.timestamp - locks[user][i].lockTimestamp > lockDuration) {
               totalLockedAmount += locks[user][i].amount;
           }
        }
    
        return totalLockedAmount;
    }

}
\end{verbatim}

\subsection{代幣標準與小數點處理}

CPT 代幣遵循標準 ERC20 規格,使用 18 位小數,而 USDC 則使用 6 位小數。平台使用 SafeMath 函式庫進行所有轉換操作,以防止溢位與溢位錯誤。價格預言機整合了小數點正規化邏輯,而最低交易門檻則可減輕灰塵攻擊向量。對於分數金額,協議採用保守的四捨五入機制。

\subsection{投票鎖倉代幣機制}

平台實施投票鎖倉(ve)代幣模型,以調整長期利害關係人的誘因。使用者將 CPT 鎖倉 1 週至 4 年不等,並獲得不可轉讓的 veCPT 代幣,該代幣同時決定治理權重與獎勵分配。

veCPT 餘額遵循以下關係:
\begin{equation}
\text{veCPT} = \text{CPT}_{\text{locked}} \times \min\left(\frac{t_{\text{lock}}}{t_{\text{max}}}, 1\right) \times 2.5
\end{equation}
其中 $t_{\text{lock}}$ 代表選擇的鎖倉期間,$t_{\text{max}} = 4$ 年定義了最大鎖倉期間。2.5 的乘數為 4 年承諾提供最大的治理權重。

當鎖倉期即將到期時,veCPT 餘額會線性遞減:
\begin{equation}
\text{veCPT}(t) = \text{CPT}_{\text{locked}} \times \frac{t_{\text{remaining}}}{t_{\text{max}}} \times 2.5
\end{equation}
這種遞減機制透過鎖倉延長或代幣重新鎖倉,誘導持續參與。

\subsubsection{獎勵分配}

平台以 USDC 收取的收入中,30% 會分配至質押獎勵池,分配頻率為每週或每月。個人獎勵按 veCPT 持有比例計算:
\begin{equation}
\text{Reward}_{\text{user}} = \text{Revenue}_{\text{pool}} \times \frac{V_{\text{user}}}{V_{\text{total}}}
\end{equation}
其中 $V_{\text{user}}$ 代表使用者的 veCPT 餘額,$V_{\text{total}}$ 代表 veCPT 的總供應量。實際年利率(APY)會根據質押參與度與平台績效動態變化:
\begin{equation}
\text{APY} = \frac{\text{Annual Revenue Pool}}{\text{Total CPT Staked Value}} \times \frac{\text{veCPT Multiplier}}{\text{Average Multiplier}}
\end{equation}

\subsubsection{安全性與最佳化}

智慧合約會遵循 OpenZeppelin 標準接受第三方審計,由多重簽章治理控制參數修改。所有獎勵分配都會在鏈上追蹤,以確保透明度。安全性功能包括外部呼叫的可重入防護、基於角色的存取控制、緊急暫停功能,以及可升級代理模式。關鍵參數變更需要 48 小時的時間鎖。

瓦斯費最佳化採用基於 Merkle 樹的批次領取、veCPT 餘額的惰性評估、封包儲存變數,以及事件驅動的鏈下索引。這些技術在維持安全性保障的同時降低了交易成本。

\subsection{預言機整合}

平台整合了 Chainlink 去中心化預言機,用於價格發現與數據彙總。CPT/USD 價格供稿彙整了 Uniswap V3 時間加權平均價格與中心化交易所報價的數據。USDC/USD 驗證使用 Chainlink 的經認證供稿,偏差門檻為 0.5%。預言機每 5 分鐘更新一次,或在價格變動超過 1% 時更新,並具備手動備援機制。

至於計算資源定價,鏈下彙總器會監控主要雲端服務提供商(AWS、Azure、GCP、阿里巴巴雲)的公開 API,計算運算、儲存與頻寬的即時市場價格。彙總後的價格會每天或在偏差超過 5% 時,發佈至鏈上預言機合約。

預言機安全性依賴至少 7 個獨立 Chainlink 節點的共識。系統會拒絕與中位數偏差超過 10% 或數據過時超過 1 小時的價格更新。斷路器會在偵測到操縱企圖時自動暫停交易。

\subsection{治理架構}

平台的關鍵運作需要透過 Gnosis Safe 實施的多重簽章批准。超過 10 萬 USDC 的財庫轉移需要 9 簽 5 過,而智慧合約升級則需要 9 簽 7 過,並搭配 48 小時的時間鎖。參數調整採用 9 簽 4 過的共識,而緊急安全回應則使用 5 簽 3 過的快速回應配置。

治理流程遵循結構化時程:持有 10 萬以上 veCPT 的持有者可提交提案,隨後是為期 7 天的社群討論期與 5 天的鏈上投票階段,其中 1 veCPT 等於 1 票。已通過的提案會在延遲 48 小時後執行。多重簽章理事會對惡意提案擁有否決權,並須接受每季審查。

\subsection{跨鏈基礎設施}

平台實施 LayerZero 全鏈協議,用於多鏈部署。Arbitrum 作為主鏈,因其低交易成本與高吞吐量。以太坊主網支援針對需要第 1 層安全性的機構使用者,而 Polygon 整合則為對成本敏感的使用者提供降低的交易成本。未來的擴張將包括 Optimism(2024 年第 3 季)與 Base(2024 年第 4 季),以實現更廣泛的生態系整合。

跨鏈橋安全性整合了多項防護措施:流動性上限將每條鏈的跨鏈供應限制為 10%,速率限制將吞吐量限制為每小時 100 萬 CPT,緊急暫停機制可回應異常,以及 5% 的保險基金為跨鏈價值提供抵押,以對抗潛在的漏洞利用。

\subsection{錢包基礎設施}

作為標準 ERC20 代幣,CPT 支援所有相容的錢包,包括瀏覽器擴充功能(MetaMask、Rabby、Rainbow)、行動應用程式(Trust Wallet、Coinbase Wallet、imToken)、硬體裝置(Ledger、Trezor),以及智慧合約錢包(Argent、Gnosis Safe)。未來計劃部署與 Fireblocks 和 Copper.co 的機構託管整合。

網頁入口實施 WalletConnect 與 Web3Modal 協議,以實現標準化的錢包連接。在連接授權後,平台會查詢使用者的餘額、質押位置與 veCPT 持有量,以啟用完整的功能存取。交易簽名遵循 EIP-712 標準的類型化結構化數據,呈現人類可讀的訊息,以提升對抗釣魚攻擊向量的安全性。

\section{市場計算基礎設施}

\subsection{系統架構}

平台實施三層架構。Web3 介面層透過 React.js 與 ethers.js 框架,管理錢包認證(WalletConnect)、USDC 支付處理,以及 CPT 獎勵分配。協調層協調 CHESS 叢集管理系統、工作排程、資源分配、效能監控,以及服務提供者(SP)認證流程。計算資源層彙整了 CSP 叢集、公開雲端 API(AWS、Azure、GCP、阿里巴巴雲)、私有高效能計算(HPC)中心,以及未來的邊緣計算節點。

交易流程如下:提交工作並存入 USDC、智慧合約託管至完成、CHESS 中介的資源匹配、在分配的 SP 基礎設施上執行、即時 SLA 合規性監控、結果交付與自動支付結算,以及按比例分配 CPT 獎勵(使用者 1-3%、SP 2-5%)。

\subsection{HPC 基礎設施元件}

高效能計算(HPC)基礎設施包含專用節點類型:運算節點使用多核處理器與大量記憶體執行數值模擬與數據分析;視覺化節點使用 GPU 加速渲染大型資料集;I/O 節點管理儲存與運算架構之間的數據傳輸;儲存節點提供高併發檔案系統;管理節點協調資源分配與工作排程。

網路架構使用高速互連技術(InfiniBand、乙太網路)進行節點間通訊。平行檔案系統可針對大型資料集與中間結果進行併發多節點讀寫操作。

\subsubsection{軟體堆疊}

監控與管理工具為管理員提供了跨系統元件的即時健康與效能數據,包括 CPU 使用率、記憶體消耗與網路流量模式。叢集管理軟體協調整體系統運作,具備跨地理分佈安裝的運算節點的佈建、監控與維護功能。

資源分配採用專用排程器,管理 CPU 時間、記憶體與其他計算資源,以最大化系統使用率效率。使用者介面涵蓋命令列工具與網頁入口,用於工作提交與管理。HPC 應用程式中心彙整了領域特定的應用程式與範本,讓使用者能夠直接下載並部署計算工具。整合的計費系統在各種資源類型與計費週期之間實施透明的定價策略,促進合理的資源利用與準確的成本核算。

\subsection{支付與結算基礎設施}

\subsubsection{託管機制}

工作提交會啟動託管流程,其中使用者批准將 USDC 消費至平台的智慧合約。託管合約會計算估計成本,包含資源類型(CPU/GPU/儲存)、持續時間預測、預言機導出的市場定價,以及 20% 的緩衝區用於潛在的超支。USDC 會在批准後轉移至託管,針對唯一的工作識別碼進行鎖定。

工作完成後,實際的資源消耗將決定最終結算。服務提供者直接以 USDC 收取 95-98% 的費用,而平台則保留 2-5% 的交易費。多餘的託管資金會自動退還給使用者,而 CPT 獎勵則按比例分配給使用者(1-3%)與 SP(2-5%)。

\subsubsection{爭議解決協議}

SLA 違規會觸發分級解決機制。在 5 分鐘內失敗的工作有資格獲得自動全額退款。部分完成的工作會根據實際交付產生按比例退款。使用者可在 72 小時內提交爭議,並附上支持證據。價值超過 1 萬 USDC 的案件會升級至平台治理仲裁,而保險基金則負責賠償最高 10 萬 USDC 的經驗證索賠。

\subsubsection{服務提供者認證}

服務提供者認證需要經過多階段驗證流程。初始註冊需要公司驗證文件、基礎設施規格、支付錢包地址,以及安全合規憑證(SOC 2、ISO 27001)。技術驗證採用業界標準基準,包括 High-Performance Linpack(HPL)、High-Performance Conjugate Gradient(HPCG)、STREAM 記憶體頻寬、用於 AI 工作負載的 MLPerf,以及網路延遲評估。安全審計驗證 AES-256 加密、網路隔離與 DDoS 防護功能。

獲得批准的候選人進入為期 30 天的試用期,具有增強的監控與 10 個工作的併發限制。順利完成試用期即可獲得認證服務提供者(CSP)資格,能夠存取機構客戶與參與團購。CSP 會出現在高級目錄中,並配有已驗證徽章。

持續合規要求每月 99.5% 的可用性、小於 5 分鐘的工作啟動時間,以及效能在宣稱基準的 10% 以內。每季重新認證可驗證持續的能力。關鍵漏洞的安全修補程式必須在 48 小時內部署。違規會觸發分級處罰:第一次違規發出警告,並給予 7 天的補救期;第二次違規處以 30 天的暫停;第三次違規撤銷認證。

\subsection{技術堆疊}

平台針對前端開發使用 React.js 18+ 搭配 TypeScript,針對 Web3 整合使用 ethers.js v6 與 WalletConnect v2,並使用 Material-UI 確保介面一致性。後端架構針對 API 服務使用 Node.js/Express.js 或 Python FastAPI,針對關聯式持續性使用 PostgreSQL,針對快取使用 Redis,針對非同步工作佇列使用 RabbitMQ/Kafka,針對區塊鏈事件索引使用 The Graph,以及針對可觀測性使用 Prometheus/Grafana。DevOps 基礎設施透過 Docker 容器化所有服務,透過 Kubernetes 協調生產部署,透過 GitHub Actions 實施 CI/CD,透過 Cloudflare CDN 分發內容,以及透過 Nginx 進行流量負載平衡。

入門級 CSP 需要 100 個以上的 CPU 核心(Intel Xeon/AMD EPYC)、500 GB 記憶體、10 TB NVMe SSD 或 50 TB HDD、10 Gbps 網路上行連結,以及選擇性的 4 個以上 NVIDIA A100/H100 GPU。企業級 CSP 可擴展至 1 萬個以上的 CPU 核心、50 TB 以上的總記憶體、1 PB 以上的平行檔案系統儲存(Lustre/GPFS)、100 Gbps InfiniBand 骨幹,以及 100 個以上的高階 GPU。

\subsection{平台使用者功能}

CPT 入口網站服務三個主要群體:探索專案資訊的訪客、購買市場資源(公開雲端、HPC 提供者、硬體、軟體、儲存)的使用者,以及執行 USDC 存款與鑄造操作的流動性提供者。

公開雲端消費者可在 FQ、Amazon 與華為雲等廠商之間進行選擇,定價以 USDC 計算,並附有促銷優惠。廠商選擇會將使用者重新導向至原生入口網站(例如 AWS),其中標準操作會透過 CPT 平台託管進行付款路由。平台隨後以法定貨幣與廠商結算。

HPC 資源消費者會在廠商(CT 叢集、區域提供者、華為、AWS)之間,針對價格點、硬體規格、效能指標與區域頻寬進行比較。廠商選擇與工作提交會在 CHESS 入口網站進行,需先存入足夠的 USDC,資金會託管至完成,隨後進行法定貨幣結算。儲存採購遵循相同的工作流程。

軟體選項包括使用者提供的應用程式,或來自 Ansys、HPC 軟體廠商與 CHESS 應用程式中心的平台列出解決方案。廠商上架可容納硬體、儲存、軟體與輔助計算產品。當兩個元件都來自平台清單時,系統會驗證硬體與軟體的相容性,確保執行相容性。架構可容納未來隨需求演進的功能擴展。

\subsection{公開雲端整合}

市場彙整了來自主要公開雲端廠商(AWS、Azure、Google Cloud、阿里巴巴雲)的計算資源。定價以 USDC 顯示,並附帶主動促銷與可用性狀態。

\subsubsection{廠商整合模式}

平台採用三種整合方法。直接 API 整合利用經銷商憑證,透過廠商 API(AWS EC2、Azure Resource Manager、GCP Compute Engine)進行即時佈建,實現自動執行個體生命週期管理。優惠券代碼系統透過預先生成的代碼解決容量限制,防止超賣,可分為價值型(100 美元通用優惠金)或資源特定型(1000 GPU 小時、10 TB 儲存)。受託服務提供者(MSP)模式將 CyberPlaza 定位為具備大量定價協議的 MSP,管理廠商帳戶並提供整合計費。

即時價格比較會顯示運算、儲存與網路成本,以及總擁有權成本計算。團購折扣會強調相較於直接採購的潛在節省。

\subsubsection{工作提交工作流程}

HPC 工作提交流程包含:透過篩選的 CSP 清單(CPU 類型、GPU 可用性、區域、定價)進行資源選擇;透過應用程式中心範本或自訂程式碼進行工作配置,並指定需求(節點、核心、記憶體、執行時間、GPU)與 I/O 位置;成本估計,包含 USDC 細分與預計 CPT 獎勵;付款授權,將 USDC 轉移至託管並附帶應急緩衝區;透過 CHESS 排程器分配執行,並進行即時狀態監控;完成結算,交付結果並自動分配付款、退還多餘金額,以及發放 CPT 獎勵。

進階功能包括批次提交(支援 100 個以上的工作與參數掃描)、工作流程相依性(定義順序執行)、檢查點/重新啟動(用於容錯)、現貨執行個體出價(可預訂容量享有 50-70% 的折扣),以及自動擴縮容(用於動態資源調整)。

服務提供者透過集中式儀表板管理運作,包含資源分配、工作監督、財務追蹤(USDC 收入、CPT 累積),以及效能分析(客戶滿意度、使用率指標)。

\subsection{多叢集管理系統}

The CHESS (Cluster High-performance Execution and Scheduling System) platform provides unified management across geographically distributed computing resources. The system integrates monitoring, scheduling, and resource allocation through a centralized web portal with role-based access control.

\subsubsection{核心功能}

平台透過網頁介面與 SSH 協議支援全面的數據管理,實現檔案操作,包括上傳、下載、壓縮與解壓縮。節點管理透過批次命令進行,控制電源狀態、遠端存取(VNC、Shell),並支援異質硬體配置(CPU、GPU、FPGA)。資源配額針對儲存與運算分配執行行政政策,並在超過門檻時自動生成警示。

高可用性架構透過備援管理節點與資料庫複製,消除單點故障。系統協調多個地理分佈的叢集,並在子叢集之間進行統一的使用者角色傳播。

\subsection{效能監控基礎設施}

高效能與雲端計算系統彙整了大量硬體資源,透過高速網路互連,形成低延遲、高容量的配置。有效的叢集管理需要監控與管理工具,提供資源配置、即時效能追蹤、帶有警示的故障偵測,以及使用狀態視覺化。

\subsubsection{CHESS 監控功能}

CHESS 監控系統透過彙整儀表板提供全面的叢集監督,顯示跨乙太網路與 InfiniBand 架構的 CPU 與記憶體使用率、負載狀態、儲存狀態與網路吞吐量。自訂時間間隔選擇可實現歷史趨勢分析與效能追蹤。儀表板顯示提供可自訂的大螢幕呈現,針對儲存使用率、工作排程與網路統計資料進行動態指標更新。

多叢集監控延伸至地理分佈的安裝,具備適應性螢幕配置與解析度最佳化。機櫃視覺化呈現物理拓撲,並整合電源管理與 VNC 遠端存取控制。單節點監控擷取細緻的 CPU、記憶體、儲存、負載與網路指標,同時提供故障診斷與復原建議。GPU 監控追蹤裝置特定的使用率、記憶體利用率、溫度與頻寬。工作監控分析即時執行狀態與佇列組成,並包含詳細的 CPU 利用率、記憶體消耗與節點負載統計資料。叢集警示實現可配置的門檻,並透過電子郵件與系統通知進行路由。

效能指標會以使用者定義的間隔收集,擷取 CPU、記憶體、磁碟與網路數據。物理拓撲視覺化涵蓋機櫃與節點配置,並具備基於門檻的故障警示。

\subsubsection*{排程器與資源管理}

在多叢集系統中,高效的排程與資源管理至關重要。CHESS 提供彈性的排程政策,包括 FIFO、搶占與回填策略。系統支援具有服務品質(QoS)配置的資源預留、涵蓋串列、平行與 GPU 工作負載的進階工作提交,以及用於負載平衡最佳化的佇列管理。

\subsubsection{工作提交與管理}

使用者透過命令列介面、基於網路的 GUI,或常用工作流程的應用程式範本提交工作。管理員配置資源配額、優先權級別與提交政策,以管理系統存取與使用率。

% \subsection*{6.2.5 Pricing Module}

% This module will focus on calculating costs for resource usage and presenting pricing details to users. It will integrate with job scheduling and monitoring systems for real-time cost tracking.


% \subsubsection*{6.2.5 User Interfaces and Operational Portals}

\subsubsection{使用者管理}

平台支援自助註冊與管理員配置的帳戶,並整合 LDAP 認證以進行集中管理。基於角色的存取控制實施預設角色(管理員、部門管理員、使用者),並具備靈活的權限分配,以管理系統存取與功能。

\subsubsection{通知與訊息}

使用者會收到針對計費與使用訊息的自動警示,以及行政公告。

\subsection{應用程式中心}

應用程式中心透過可瀏覽的庫提供預安裝的 HPC 應用程式(Ansys、MATLAB、TensorFlow)。使用者透過圖形範本提交工作,並進行互動式參數配置。輸出管理涵蓋日誌檢視、錯誤分析、效能指標追蹤,以及整合的視覺化工具(用於 AI 應用程式的 TensorBoard)。

\subsection{硬體效能評估}

硬體效能評估模組執行基準測試,測量 CPU 與 GPU 效能,以及網路吞吐量與延遲。資源效率分析根據工作負載特性最佳化分配策略。故障復原指標評估硬體在故障情境下的可靠性與復原效能。

\subsection{安全架構與合規性}

\subsubsection{多層安全模型}

平台在三層中實施深度防禦安全性。智慧合約安全性使用 Certora 或同等工具進行形式化驗證,由 CertiK、Trail of Bits 或 OpenZeppelin 進行年度第三方審計,針對關鍵漏洞提供高達 50 萬美元的錯誤回報計劃,具備 48 小時時間鎖的可升級透明代理模式,以及用於緊急漏洞利用回應的斷路器。

平台安全性涵蓋透過 OAuth 2.0 與 JWT 認證的 API 保護,速率限制為每分鐘 100 次請求、針對 SP 存取的 IP 白名單,以及 90 天的 API 金鑰輪換。數據加密為傳輸保護實施 TLS 1.3,為靜態數據實施 AES-256,為敏感工作負載實施端到端加密,以及為金鑰管理實施硬體安全模組(HSM)。基礎設施安全性部署了 Cloudflare DDoS 保護、具備 OWASP 規則集的 Web 應用程式防火牆、每季滲透測試,以及用於事件監控的 SIEM 系統。

數據隱私與合規性措施透過帳戶刪除權、數據可攜性、設計隱私原則,以及歐盟數據駐留選項,來滿足 GDPR 要求。KYC/AML 程序為每月超過 1 萬 USDC 的交易實施基本驗證,為 CSP 認證實施增強型驗證,針對可疑活動進行交易監控,以及遵循 FATF 旅行規則。數據隔離採用容器化或基於 VM 的工作執行、網路分段、完成後自動數據擦除,以及跨使用者洩漏預防。

\subsubsection{事件回應}

持續的安全作業中心監控異常活動,包括不尋常的提款、智慧合約漏洞利用,以及 API 濫用。事件分類遵循四層嚴重性模型(重大、高、中、低),目標是在 15 分鐘內完成評估。重大事件會觸發立即合約暫停,並在 1 小時內發送多重簽章通知。針對重大事件,會在 24 小時內進行公開披露,而事後檢討報告則會在 7 天內發布。復原程序透過治理管道部署修補程式,並從保險基金中補償受影響的使用者。

\subsubsection{監管合規性}

平台追求 SOC 2 Type II 認證,以確保數據安全性與可用性(第 1 年目標),以及 ISO 27001 認證,以確保資訊安全管理(第 2 年目標)。雲端安全聯盟(CSA)的 STAR 認證可驗證 CSP 的安全狀態。PCI DSS 合規性目前正在考慮中,以用於未來的支付方式擴展。

\subsection{可擴展性與效能最佳化}

\subsubsection{水平擴展架構}

平台透過分散式資料庫架構進行水平擴展,在各區域中使用 PostgreSQL 唯讀複本,透過 ID 雜湊進行使用者數據分片,使用 Redis 叢集儲存熱數據(工作階段、定價),以及透過 Cloudflare CDN 進行靜態資產交付。

微服務架構將功能分解為獨立的可擴展服務:使用者服務(認證、個人檔案)、工作服務(提交、排程、監控)、支付服務(USDC 託管、結算、CPT 獎勵)、SP 服務(上架、認證、評級)、定價服務(預言機彙整),以及通知服務(電子郵件、推播、鏈上事件)。每個服務都會根據需求自主擴展。

負載平衡在美國、歐盟與亞洲地區實施地理分佈,使用 Kubernetes Horizontal Pod Autoscaler 進行動態容量調整,使用 Hystrix 斷路器防止連鎖故障,以及使用 RabbitMQ 佇列進行非同步工作處理。

\subsubsection{效能目標}

\begin{center}
\begin{tabular}{|l|c|c|}
\hline
\textbf{指標} & \textbf{目標(第1年)} & \textbf{目標(第3年)} \\
\hline
API 回應時間 & <200ms (p95) & <100ms (p95) \\
工作提交時間 & <5 seconds & <2 seconds \\
支付結算 & <30 seconds & <10 seconds \\
頁面載入時間 & <2 seconds & <1 second \\
平台可用性 & 99.5\% & 99.9\% \\
同時線上使用者 & 10,000 & 100,000 \\
每日交易數 & 50,000 & 1,000,000 \\
\hline
\end{tabular}
\end{center}

\subsubsection{區塊鏈可擴展性}

Arbitrum 第 2 層部署為主要操作提供低於 0.10 美元的交易費與 40,000 TPS 的吞吐量。批次交易處理會將獎勵分配分組,以攤銷瓦斯費成本。The Graph 協議負責鏈下事件索引。未來的開發將包括用於高頻微支付場景的狀態通道。

瓦斯費最佳化技術透過基於 Merkle 證明的獎勵領取(節省 80%)、veCPT 餘額的惰性評估、封包儲存變數編碼,以及在功能等效的情況下優先使用事件日誌而非狀態變數,來降低交易成本。

\subsection{災難復原}

\subsubsection{備份基礎設施}

資料庫備份每天執行一次(完整),每六小時執行一次(增量),並進行連續交易日誌複製。系統在冷儲存歸檔前會維持 30 天的保留期。智慧合約狀態利用區塊鏈的固有不可篡改性,並輔以歸檔節點部署與每季的去中心化儲存快照(IPFS/Arweave)。使用者工作結果會備份至指定的儲存端點,平台中繼資料會保留 90 天,並具備用於 GDPR 合規性的隨選匯出功能。

\subsubsection{復原目標}

表~\ref{tab:recovery-targets} 指定了元件層級的復原時間(RTO)與復原點(RPO)目標。

\begin{table}[htbp]
\centering
\caption{復原時間與復原點目標}
\label{tab:recovery-targets}
\begin{tabular}{lcc}
\hline
\textbf{元件} & \textbf{復原時間目標(RTO)} & \textbf{復原點目標(RPO)} \\
\hline
智慧合約 & N/A & 0 \\
網頁入口 & 1 hour & 6 hours \\
資料庫 & 2 hours & 1 hour \\
工作排程器 & 30 minutes & 15 minutes \\
\hline
\end{tabular}
\end{table}

在美國與歐盟地區實施的主動-主動部署,可在主區域不可用 5 分鐘後自動進行 DNS 容錯移轉。即時跨區域數據同步維持一致性,並具備手動覆蓋功能,用於操作干預。

\subsection{開發路線圖}

短期開發(6-12 個月)優先考慮 iOS 與 Android 的行動應用程式、針對第三方整合的增強型 API 產品(RESTful、GraphQL)、基於機器學習的成本最佳化,以及額外的區塊鏈跨鏈橋部署(Polygon、Optimism)。

中期目標(1-2 年)透過針對物聯網部署的邊緣計算支援、針對敏感工作負載的機密計算整合(Intel SGX、AMD SEV)、去中心化儲存協議(Filecoin、Arweave)、專用 AI/ML 資源市場,以及探索性的量子計算合作,來擴展平台功能。

長期願景(2-5 年)包括全面轉為 DAO 治理、開放去中心化計算協議開發、用於隱私增強的零知識證明實施、透過 IBC 或同等協議的跨鏈互通性,以及基於 NFT 的實體計算資源代幣化。

\subsection{總結}

本章詳細介紹了整合 Web3 區塊鏈基礎設施與成熟 HPC 系統的技術架構。混合設計將去中心化誘因機制(CPT 代幣、投票鎖倉治理)與成熟的 CHESS 叢集管理平台相連接。安全架構透過智慧合約審計、基礎設施強化,以及監管合規途徑(SOC 2、ISO 27001)實施多層保護。系統可從數千名擴展至數十萬名同時線上使用者,同時維持低於 200ms 的 API 回應時間。

與現有的去中心化計算專案(Golem、iExec、Render)相比,CyberPlaza 的差異在於成熟的基礎設施(20 年以上的 CHESS 平台歷史)、企業合規導向、超越點對點架構的多雲端整合、預先整合的應用程式生態系,以及結合去中心化存取與專業 SP 認證的混合市場。這種定位滿足了企業計算需求,同時實現了 Web3 經濟參與。
