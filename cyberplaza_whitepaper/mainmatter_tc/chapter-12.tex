\chapter{FAQ}

\section{常見問題}

\begin{enumerate}
\item \textbf{身為算力淘寶平台的使用者,我能獲得什麼?}

\textbf{答案}: 您可從平台列出的眾多供應商中選擇適合您使用的運算資源,包括 CPU、GPU、FPGA 運算能力、儲存空間、應用軟體及服務(例如:在特定硬體平台上最佳化您的軟體,或將您的雲端應用程式從一家雲端供應商遷移至另一家)。您可做出明智的服務選擇,因為運算資源的效能由平台評估並公佈,服務供應商的 SLA 由平台擔保,且 AWS、Azure、GCP 及眾多運算中心與資料中心等資源以類似淘寶/京東的折扣價格提供。此外,透過使用平台,您可共享平台的所有權,並透過獲得的 CPT 分享平台的部分利潤(一個您部分擁有的淘寶)。

\item \textbf{使用者對象是誰?一般大眾可能並非運算資源的主要使用者,另一方面,許多機構客戶可能無法參與代幣經濟學。}

\textbf{答案}: 目前全球一般大眾在公有雲上的運算使用價值超過 400 億美元,這確實僅占機構客戶使用量的一小部分。對於無法參與代幣經濟學的機構客戶,他們可透過平台的通路夥伴(請參閱白皮書的合作夥伴章節)以一般 B2B 方式購買運算服務,支付法定貨幣。

\item \textbf{有些機構服務供應商(例如 AWS 或中國境內的超級運算中心)可能無法在提供服務時接受代幣,其運算資源如何透過本平台供使用者使用?}

\textbf{答案}: 平台使用「儲備基金」以法定貨幣購買這些服務供應商的服務。透過團購(拼多多),平台可提供折扣服務。

\item \textbf{為何像 AWS 這樣的雲端供應商會屈服於團購壓力?}

\textbf{答案}: 本平台將成為 AWS 寶貴的銷售通路,協助其接觸 Web 3 和 DeFi 社群。此外,在平台上眾多服務供應商的競爭壓力下,加上足夠大規模的團購協議以及來自「儲備池」的預付款,折扣對所有雲端和運算資源供應商而言都是合理的。

\item \textbf{假設運作完美,算力淘寶平台的業務規模會有多大?}

\textbf{答案}: AWS 的年度營收在 2019 年為 350 億美元、2020 年為 450 億美元、2021 年為 620 億美元、2022 年為 814 億美元(根據顧能 (Gartner) 的數據,其中約 93\% 由機構客戶使用,7\% 由個人使用者使用)。若我們假設全球商業運算業務總值為 AWS 的 7 倍(2022 年全球為 5520 億美元,即 AWS 的 7 倍,根據聯合市場研究 (Allied Market Research)),2024 年全球總營收將超過 1 兆美元。若算力淘寶平台能占據總市場的 0.1\%,則年營收將超過 10 億美元,並會迅速成長。

\item \textbf{身為流動性提供者或 CPT 持有者,我能獲得哪些好處?}

\textbf{身為流動性提供者(存入 USDC)}: 參與者可從平台營運利潤中獲得 5–7\% 的 USDC 年化收益率 (APY),額外加碼 2–3\% 的 CPT 代幣年化收益率(附鎖倉機制),總預期年化收益率達 8–12\% APY,維持 USDC 流動性(需提前通知方可提領),並在獲得可持續收益的同時支援平台成長。

\textbf{身為 CPT 持有者/質押者}: 參與者可質押 CPT 以獲得 8–12\% 的年化收益率 (APY)(4 年鎖倉加碼可達 15–20\%),獲得平台 40\% 利潤的 USDC 收益分配,受惠於通縮型回購與銷毀機制(占營收的 20\%),獲得治理權(對平台發展方向投票),質押期間可享平台服務 5–15\% 折扣,使用進階功能與優先支援,並優先參與新產品發布。

\textbf{為何這些收益具可持續性}: 不同於演算法穩定幣或龐氏騙局,我們的收益來自真實交易手續費(占平台活動的 2–5\%)、團購利差(大宗採購的 10–20\%)、加值服務(認證、訂閱、API),以及透明可審計的收益來源。

\item \textbf{為何資金(無論是鑄幣者或投資者)會選擇加入算力淘寶平台而非其他 Web 3 專案?}

\textbf{答案}: 詳情請參閱「與其他『資產代幣化專案』的競爭分析」及「與其他 Web 3 運算資源專案的競爭分析」頁面。簡而言之:與其他資產相比,算力的價值成長更為迅速;且我們的團隊具備建立算力淘寶平台的獨特資格。

\item \textbf{有些潛在使用者或服務供應商可能無法參與代幣經濟學,他們該如何參與?}

\textbf{答案}: 這些客戶可在平台上找到合適產品後,透過平台的合作夥伴(請參閱第 10 節所列的代理)購買。服務供應商也可透過合作夥伴在平台上列出其產品。合作夥伴與使用者/供應商之間的交易可透過法定貨幣進行,無需涉及代幣。

\item \textbf{有些消費者認為淘寶和拼多多上的產品品質較低,平台如何防範此問題?}

\textbf{答案}: 月列出價格超過 \$10,000 USDC 的所有產品均需經平台認證。如上述第 4 節所述,平台要求服務供應商以標準效能測試(包括 High-Performance Linpack、High-Performance Conjugate Gradient、STREAM Sustainable Bandwidth、HPC Challenge、MLPerf、ResNet-50 Image Classification、BERT Language Processing、CUDA Benchmark Suite、SPECviewperf graphics performance、DeepBench 等)列出其服務的效能。平台將定期驗證服務供應商聲稱的效能,並將效能指標與服務價格一併列出供使用者選擇。

\item \textbf{為何像 AWS 或華為雲這樣的公司願意透過本平台銷售其服務?}

\textbf{答案}: 雲端運算公司目前向經銷商提供折扣以銷售其服務,經銷商則僱用銷售團隊銷售服務。本平台在某種程度上擔任這些廠商的經銷商角色,差異在於透過 Web 3 架構,廠商可接觸 Web 3 和 DeFi 社群。

\end{enumerate}
