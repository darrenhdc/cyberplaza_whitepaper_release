\chapter{FAQ}

\section{Frequently Asked Questions}

\begin{enumerate}
\item \textbf{身為算力淘寶平台的使用者,我能獲得什麼?}

\textbf{Answer}: 您可以在平台上列示的眾多供應商中選擇適合您使用的運算資源,包括CPU、GPU、FPGA運算力、儲存空間、應用軟體以及服務(例如,於特定硬體平台上優化您的軟體,或將您的雲端應用從一家雲端供應商遷移至另一家)。您可以充分了解後選擇服務,因為運算資源的效能已由平台評估並公開,服務供應商(SP)的服務水準協議(SLA)由平台保證,且以優惠價格(類似淘寶/京東)提供AWS、Azure、GCP以及眾多計算中心與資料中心的資源。進一步來說,透過使用平台,您將共享平台的所有權,因此可透過獲得的CPT(平台代幣)分享平台利潤(一個您部分擁有的淘寶)。

\item \textbf{誰是使用者?一般大眾可能不是運算資源的主要使用者。另一方面,許多機構客戶可能無法參與代幣經濟。}

\textbf{Answer}: 全球一般大眾目前在公共雲端使用價值超過400億美元的運算資源,這確實僅佔機構客戶使用量的一小部分。對於無法參與代幣經濟的機構客戶,他們可以透過平台的通路合作夥伴以一般B2B方式購買運算服務(請見白皮書的合作夥伴章節),使用法定貨幣付款。

\item \textbf{一些機構服務供應商,例如AWS或中國的超級計算中心,可能無法在提供服務時接受代幣。如何透過平台讓使用者使用他們的資源?}

\textbf{Answer}: 平台使用「準備金」以法定貨幣向這些服務供應商購買服務。透過團購(拼多多),平台可以提供優惠價格的服務。

\item \textbf{為什麼像AWS這樣的雲端供應商會屈服於團購的壓力?}

\textbf{Answer}: 我們的平台對AWS而言是寶貴的銷售通路,可協助其接觸Web 3與DeFi社群。此外,由於平台上眾多服務供應商之間的競爭壓力,以及搭配來自「準備金池」的部分預付款項之足夠規模團購交易,提供折扣對所有雲端與運算資源供應商而言都是完全合理的。

\item \textbf{假設運營完美,算力淘寶平台可能有多少業務?}

\textbf{Answer}: AWS的年度營收分別為2019年350億美元、2020年450億美元、2021年620億美元、2022年814億美元(根據Gartner數據,其中約93\%由機構客戶消費,7\%由個人使用者消費)。如果我們假設全球商業運算業務總價值是AWS的7倍(2022年全球總額為5520億美元,即AWS的7倍,來自Allied Market Research),則2024年全球總營收將超過1兆美元。如果算力淘寶平台能佔據總市場的0.1\%,則每年營收將超過10億美元,並將迅速增長。

\item \textbf{身為流動性提供者或CPT持有者,我能獲得什麼好處?}

\textbf{身為流動性提供者(存入USDC)}: 參與者可從平台營運利潤中獲得5–7\%的USDC年化收益率(APY),額外獲得2–3\%的CPT代幣年化收益率(帶有鎖倉期),總預期年化收益率達8–12\%,維持USDC流動性(可在給予一定通知期後提領),並在獲得永續收益的同時支持平台成長。

\textbf{身為CPT持有者/質押者}: 參與者可質押CPT以獲得8–12\%的年化收益率(透過4年鎖倉與增益機制最高可達15–20\%),從平台40\%的利潤中獲得USDC收益分配,受益於通縮的回購與銷毀機制(20\%的營收),獲得治理權(對平台發展方向進行表決),質押時可獲得平台服務5–15\%的折扣,使用高級功能與優先支援,以及早期參與新產品發布。

\textbf{為何這些收益具有永續性}: 不同於演算法穩定幣或龐氏騙局,我們的收益來自真實交易費用(市場活動的2–5\%)、團購利潤(批量採購的10–20\%)、加值服務(認證、訂閱、API),以及透明且可審計的收益來源。

\item \textbf{為什麼資金(無論是鑄幣者還是投資者)願意加入算力淘寶平台,而非其他Web 3專案?}

\textbf{Answer}: 詳情請見「與其他「資產代幣化專案」的競爭分析」以及「與其他Web 3運算資源專案的競爭分析」頁面。簡而言之:與其他資產相比,算力的價值增長更為迅速;且我們的團隊具備建立算力淘寶平台的特別資格。

\item \textbf{一些潛在使用者或服務供應商可能無法參與代幣經濟。他們如何參與?}

\textbf{Answer}: 這些客戶在平台上找到合適的產品後,可以透過平台的合作夥伴購買(請見第10章所列的代理)。服務供應商也可以透過合作夥伴在平台上列出他們的產品。合作夥伴與使用者/供應商之間的交易可以透過法定貨幣進行,無需涉及代幣。

\item \textbf{一些消費者認為淘寶和拼多多上的產品品質較差。平台如何防止這種情況?}

\textbf{Answer}: 每月列出價格超過10,000 USDC的所有產品都必須經過平台認證。如上述第4章所述,平台要求服務供應商透過標準效能測試(包括High-Performance Linpack、High-Performance Conjugate Gradient、STREAM Sustainable Bandwidth、HPC Challenge、MLPerf、ResNet-50影像分類、BERT語言處理、CUDA Benchmark Suite、SPECviewperf繪圖效能、DeepBench等)列出其服務的效能。平台將定期驗證服務供應商聲稱的效能,並將效能指標與服務價格一同列出,供使用者選擇。

\item \textbf{為什麼像AWS或華為雲這樣的公司願意在平台上銷售他們的服務?}

\textbf{Answer}: 雲端運算公司目前會向經銷商提供折扣以銷售他們的服務。經銷商僱用銷售團隊來銷售服務。從某種意義上來說,平台充當了這些供應商的經銷商,不同之處在於,透過Web 3架構,供應商可以接觸到Web 3與DeFi社群。

\end{enumerate}
