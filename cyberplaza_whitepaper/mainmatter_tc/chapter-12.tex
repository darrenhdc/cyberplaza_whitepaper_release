\chapter{FAQ}

\section{常見問題}

\begin{enumerate}
\item \textbf{身為「算力淘寶平台」的使用者,我可以獲得哪些服務?}

\textbf{回答}: 您可從平台上列出的眾多供應商中選擇所需的運算資源,包括CPU、GPU、FPGA運算能力、儲存空間、應用軟體及服務(例如:在特定硬體平台上最佳化您的軟體,或將您的雲端應用程式從一家雲端供應商遷移至另一家)。您可做出知情的服務選擇,因為運算資源的效能會由平台評估並公佈,服務供應商(SPs)的服務等級協議(SLA)也由平台擔保,且使用AWS、Azure、GCP以及眾多運算中心與資料中心等資源時,可享受折扣價格(類似淘寶/京東的模式)。此外,透過使用平台,您將分享平台的所有權,並透過所獲得的CPT,享有平台利潤的一部分(相當於您部分擁有的淘寶)。

\item \textbf{平台的使用者對象為何?一般大眾可能並非運算資源的主要使用者,另一方面,許多機構客戶可能無法參與代幣經濟。}

\textbf{回答}: 目前全球大眾在公共雲端使用的運算資源價值超過\$40B,確實僅佔機構客戶使用量的一小部分。對於無法參與代幣經濟的機構客戶,他們可以透過平台的通路合作夥伴(請參閱白皮書的合作夥伴章節)以一般B2B方式購買運算服務,並使用法定貨幣支付。

\item \textbf{部分機構服務供應商(例如:AWS或中國境內的超級運算中心)可能無法在提供服務時接受代幣。這些供應商的資源如何能透過平台供使用者使用?}

\textbf{回答}: 平台會使用「準備基金」以法定貨幣向這些服務供應商購買服務。透過團購(拼多多)模式,平台可提供折扣服務。

\item \textbf{為何像AWS這樣的雲端供應商願意接受團購的折扣要求?}

\textbf{回答}: 本平台將成為AWS寶貴的銷售通路,協助其接觸Web 3與DeFi社群。此外,由於平台上眾多服務供應商(SPs)的競爭壓力,加上透過「準備資金池」提供部分預付款的足夠規模團購(團購)交易,提供折扣對所有雲端與運算資源供應商而言都是合理的。

\item \textbf{假設運作完美,算力淘寶平台的業務規模可能有多大?}

\textbf{回答}: AWS的年度營收分別為35B(2019)、45B(2020)、62B(2021)、81.4B(2022)(其中根據Gartner的資料,約93\%由機構客戶使用,7\%由個人使用者使用)。若我們假設全球商業運算業務總價值為AWS的7倍(2022年全球總額為552B,即AWS的7倍,來源:Allied Market Research),則2024年全球總營收將超過1兆美元。若算力淘寶平台能搶佔0.1\%的市場份額,每年營收將超過1B,並會快速成長。

\item \textbf{身為流動性提供者(Liquidity Provider)或CPT持有者,我可以獲得哪些好處?}

\textbf{身為流動性提供者(存入USDC)}: 參與者可從平台營運利潤中獲得5--7\% APY的USDC報酬,額外獲得2--3\% APY的CPT代幣報酬(附帶vesting條件),預期總年化報酬率為8--12\% APY,維持USDC流動性(可在給予通知期後提領),並在獲得穩定收益的同時支持平台成長。

\textbf{身為CPT持有者/質押者}: 參與者可質押CPT以獲得8--12\% APY(若鎖倉4年並使用加成機制,最高可達15--20\%),從平台40\%的利潤中獲得USDC收益分配,享有通縮型回購與銷毀機制(20\%的營收)帶來的好處,獲得治理權(可對平台發展方向投票),質押期間可享平台服務5--15\%的折扣,使用進階功能與優先支援服務,並可提前參與新產品發布。

\textbf{為何這些報酬率具可持續性}: 不同於演算法穩定幣或龐氏騙局,我們的收益來自真實的交易手續費(佔市場活動的2--5\%)、團購利潤(大宗採購的10--20\%)、加值服務(認證、訂閱、API),以及透明、可審計的收益來源。

\item \textbf{為何資金(無論是鑄幣者或投資者)會選擇加入算力淘寶平台,而非其他Web 3專案?}

\textbf{回答}: 詳情請參閱「與其他`資產代幣化專案'的競爭分析」及「與其他Web 3運算資源專案的競爭分析」頁面。簡而言之:與其他資產相比,算力的價值成長更為迅速;且我們的團隊具備建立算力淘寶平台的獨特資格。

\item \textbf{部分潛在使用者或服務供應商可能無法參與代幣經濟。他們如何參與平台?}

\textbf{回答}: 這些客戶可在平台上找到合適的產品後,透過平台的合作夥伴購買(請參閱第10節中列出的代理商)。服務供應商也可透過合作夥伴在平台上列出其產品。合作夥伴與使用者/供應商之間的交易可透過法定貨幣進行,無須涉及代幣。

\item \textbf{部分消費者認為淘寶和拼多多上的產品品質較差。平台如何防範這種情況?}

\textbf{回答}: 所有每月列出價格超過\$10,000 USDC的產品必須經過平台認證。如前所述的第4節,平台要求服務供應商依據標準效能測試列出其服務的效能(包括High-Performance Linpack、High-Performance Conjugate Gradient、STREAM Sustainable Bandwidth、HPC Challenge、MLPerf、ResNet-50 Image Classification、BERT Language Processing、CUDA Benchmark Suite、SPECviewperf graphics performance、DeepBench等)。平台將定期驗證服務供應商聲稱的效能,並將效能指標與服務價格一同列出,供使用者選擇。

\item \textbf{為何像AWS或華為雲這類公司願意在平台上銷售其服務?}

\textbf{回答}: 目前雲端運算公司會向經銷商提供折扣以銷售其服務,經銷商則僱用銷售團隊來銷售這些服務。本平台在某種意義上充當這些供應商的經銷商,與傳統經銷商不同的是,透過Web 3架構,供應商可接觸到Web 3與DeFi社群。

\end{enumerate}
