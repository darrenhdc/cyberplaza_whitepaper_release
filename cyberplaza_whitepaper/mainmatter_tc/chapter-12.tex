%% TEST TRANSLATION (繁體) - placeholder

\chapter{FAQ}

\section{Frequently Asked Questions}

\begin{enumerate}
\item \textbf{What can I get as a user on the 算力淘寶平台?}

\textbf{Answer}: You can choose for your usage the computational resources from many providers listed on the Platform, including CPU, GPU, FPGA computing power, storage, application software and services (e.g., to optimize your software on a particular hardware platform or migrate your cloud application from one cloud vendor to another). You can make a well informed choice of services, as the performances of the computing resources are evaluated and broadcasted by the Platform, the SLA's of the SPs are guaranteed by the Platform, and at a discounted price (like 淘寶/京東), for using resources like AWS, Azure, GCP, and many computing centers and data centers. Further, through using it, you share the ownership of our Platform, and hence part of the profit of the Platform with the CPTs you are awarded (a 淘寶 you partly own).

\item \textbf{Who are the users? The general public may not be the major users of computing resources. On the other hand, many institutional clients may not be able to participate in tokenomics.}

\textbf{Answer}: The general public is currently using over \$40B worth of computation on public cloud worldwide, which is indeed a small fraction of that of the institutional clients. For the institutional clients who cannot participate in tokenomics they can buy their computational usage in the usual B2B manner from the channel partners of the Platform (see the Partnership Section of the White Paper), paying legal tender.

\item \textbf{Some institutional service providers, e.g., AWS or a supercomputing center in China, may not be able to accept tokens in providing services. How can their resources be used by our Users through the Platform?}

\textbf{Answer}: The Platform uses the ``Reserve Fund'' to buy the services of these service providers with legal tender. Through Group-Buying (拼多多), the Platform can offer the services with discount.

\item \textbf{Why would a cloud vendor like AWS yield to the pressure of Group-Buying?}

\textbf{Answer}: Our Platform will be a valuable sales channel to AWS providing access to the web 3 and DeFi community. Further, with the pressure of competition between many SPs on the Platform, and for a large enough group-buying (團購) deal with some amount of pre-payment from the ``Reserve Pool'', a discount makes perfect sense to all cloud and computing resource vendors.

\item \textbf{How much business may the 算力淘寶平台 have, assuming perfect operation?}

\textbf{Answer}: The annual revenues of AWS was 35B (2019), 45B (2020), 62B (2021), 81.4B (2022) (of which approximately 93\% consumed by institutional clients, and 7\% by individual users, according to Gartner). If we take the worldwide total commercial computational business value to be 7 times that of AWS (in 2022, it was 552B worldwide, i.e., 7 times of AWS (Allied Market Research)), the total worldwide revenue would exceed 1 Trillion in 2024. If the 算力淘寶平台 can capture 0.1\% of the total market, it would be over 1B per year and would be increasing rapidly.

\item \textbf{What benefits can I get from participating as a Liquidity Provider or CPT holder?}

\textbf{As a Liquidity Provider (depositing USDC)}: Participants earn 5--7\% APY in USDC from platform operational profits, receive additional 2--3\% APY in CPT tokens (with vesting), achieve total expected return of 8--12\% APY, maintain USDC liquidity (can withdraw with some notice period), and support platform growth while earning sustainable yields.

\textbf{As a CPT Holder/Staker}: Participants can stake CPT to earn 8--12\% APY (up to 15--20\% with 4-year lock and boost), receive USDC revenue distributions from 40\% of platform profits, benefit from deflationary buyback \& burn mechanism (20\% of revenue), receive governance rights (vote on platform direction), get 5--15\% discounts on platform services when staking, access premium features and priority support, and gain early participation in new product launches.

\textbf{Why These Returns Are Sustainable}: Unlike algorithmic stablecoins or ponzi schemes, our yields come from real transaction fees (2--5\% of marketplace activity), group-buying margins (10--20\% from bulk purchasing), value-added services (certifications, subscriptions, APIs), and transparent, auditable revenue streams.

\item \textbf{Why would money (either minter or investor) like to join the Platform (算力淘寶平台), instead of to other Web 3 projects?}

\textbf{Answer}: For details, see the page on ``Competitive Analysis against other `asset tokenization projects'\,'', and ``Competitive Analysis against other Web 3 computing resource projects''. In short: Comparing to other assets, the value of 算力 is growing more rapidly; and our team is particularly well qualified to establishing a 算力淘寶平台.

\item \textbf{Some potential users or service providers may not be able to participate in tokenomics. How can they participate?}

\textbf{Answer}: These clients after finding suitable products on our Platform can buy them through the Partners of the Platform (see Agents listed in Section 10). Service Providers can also list their products on the Platform through the Partners. The transactions between the Partners and the Users/providers can be carried out through legal tenders without involving tokens.

\item \textbf{Some consumers have the impression that products on 淘寶 and Pinduoduo are of lower quality. How can the Platform safeguard against this?}

\textbf{Answer}: All products with listed price more than \$10,000 USDC monthly must be Certified by the Platform. As described in Sec. 4 above, the Platform requires the Service Providers to list the performance of their services in terms of standard performance tests (including High-Performance Linpack, High-Performance Conjugate Gradient, STREAM Sustainable Bandwidth, HPC Challenge, MLPerf, ResNet-50 Image Classification, BERT Language Processing, CUDA Benchmark Suite, SPECviewperf graphics performance, DeepBench etc.). The Platform will verify the performance claimed by the service provider periodically, and list the performance index along with the price of the service for Users to choose from.

\item \textbf{Why do companies like AWS or Huawei Cloud want to sell their services on the Platform?}

\textbf{Answer}: The cloud computing companies currently offer discounts to distributors to sell their services. The distributors employ sell teams to sell the services. The Platform in a sense serves as a distributor to these vendors, except that with the Web 3 setup, the vendors gain access to the web 3 and DeFi community.

\end{enumerate}
