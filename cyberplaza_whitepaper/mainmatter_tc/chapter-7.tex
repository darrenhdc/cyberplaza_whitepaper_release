\chapter{發展路線圖}
\section{路線圖與募資計劃}

\subsection{專案路線圖}

此專案開發採用自2026 Q1起實施的階段式方法。初始階段活動包括募資活動、核心團隊擴編、官方網站部署、白皮書發布,以及在Twitter與Discord平台上建立社區。

\textbf{2026 Q1} 展開協議架構與代幣經濟機制的α測試,驗證核心功能與經濟模型參數。

\textbf{2026 Q2} 釋出供公眾參與的測試網,讓社區可在分散式基礎架構上進行測試與回饋蒐集。

\textbf{2026 Q3} 推出開放公眾存取的主網,同時進行Initial DEX Offering (IDO),標誌著全面營運部署與代幣發放的啟動。

\subsection{募資計劃}

本募資策略採用三階段代幣分配方法。2026 Q1的初始發行將分配5\%的CPT代幣,目標集資\$4M USD,讓早期支持者能以優惠估值參與。比較市場分析將本專案與成熟的分散式運算網路進行對比,其中值得注意的是Golem以低於8,000個核心基礎架構及每月\$30K的使用率,達到\$200M的估值。

2026 Q2與2026 Q3的後續募資回合將各自以當前市場估值分配5\%的CPT供應量,配合達成的里程碑與已顯現的網路成長進行集資。漸進式定價反映了平台成熟度與運算資源擴充。
