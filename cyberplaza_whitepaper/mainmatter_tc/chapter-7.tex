\chapter{發展路線圖}
\section{發展路線圖與募資計畫}

\subsection{專案發展路線圖}

本專案發展採階段式進程,始於2026年第1季。初始階段活動涵蓋募資專案、核心團隊擴編、官方網站部署、白皮書發布,以及於Twitter和Discord平台建立社群。

\textbf{2026年第1季} 啟動協議架構與代幣經濟機制的Alpha測試,驗證核心功能與經濟模型參數。

\textbf{2026年第2季} 發布可供公眾參與的測試網,讓社群能在分散式基礎設施上進行測試並收集回饋。

\textbf{2026年第3季} 推出可供公眾使用的主網,並同步舉行去中心化交易所首次發行(IDO),標誌著完整營運部署與代幣發行的開始。

\subsection{募資計畫}

募資策略採用三階段代幣分配方式。2026年第1季的首次發行將分配5\%的CPT代幣,目標募得\$4M USD資金,為早期支持者提供優惠估值的認購機會。比較市場分析將本專案與既有的分散式運算網路對照,尤以Golem為例,其估值達\$200M,核心基礎設施不到8,000個,每月使用率為\$30K。

接下來於2026年第2季與2026年第3季的募資回合,將分別以當時市場估值分配5\%的CPT供應量,使募資能與達成的里程碑及已證實的網路成長相符。漸進式定價反映平台的成熟度與運算資源的擴張。
