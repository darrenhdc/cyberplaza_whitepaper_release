\chapter{執行摘要}

運算技術在現代生活中扮演著日益關鍵的角色,且此趨勢預期在可預見的未來將持續發展。Web3 CyberPlaza Network 專案旨在讓個人與機構皆能以開放且具包容性的方式從此趨勢中獲益。

本專案推出 CyberPlaza 平台,為一可稱為「計算資源版淘寶」(算力淘寶平台)的去中心化交易市場。此平台匹配使用者與服務提供商(SP)的需求,涵蓋高效能運算、智慧運算與雲端運算領域。使用者可於單一平台存取多元計算能力、儲存空間、軟體應用程式、資料與運算服務,以具成本效益的方式滿足其特定需求。另一方面,服務提供商(SP)可獲得無限制的全球銷售管道。服務提供商與使用者皆可透過持有 CyberPlaza 代幣(CPTs\footnote{CPT 是物理學中的守恆量;所有物理定律均不得違反 CPT 守恆,與能量守恆類似。CPT(Carriage Paid To,運費付至指定地點)也是國際貿易術語,意為賣方將支付貨品運送至消費者的費用。本專案代幣名稱承載了這兩種隱喻。})共享平台的成功,CPTs 代表了此「淘寶式」平台的治理股權。

\textbf{付款與結算}:平台上的交易以 USDC(一種被廣泛採用且受監管的穩定幣)進行結算,確保監管合規性與使用者熟悉度。此方式消除了與專屬穩定幣相關的複雜性及監管風險,同時維持透明的美元計價定價。

\textbf{營收模式}:平台透過多種管道產生收益:SaaS 訂閱(40–50\%),包含平台存取的月付/年付訂閱;交易費用(25–30\%),即計算資源購買的 2–5\% 費用;API 與資料服務(15–20\%),提供高級 API 存取與分析;以及團購(5–10\%),從批量購買中獲取利潤作為補充收益。這些收益透過透明的質押獎勵機制分配給 CPT 代幣持有者,讓參與者可透過提供平台流動性與治理貢獻,獲得永續的年報酬率(目標 6–10\% APY)。

\textbf{去中心化流動性池}:為支援平台的團購營運並確保具競爭力的價格,本專案實施去中心化借貸池,參與者可存入 USDC 以獲得利息(5–7\% APY)加上 CPT 獎勵(2–3\% APY),同時為平台提供營運資金。此模式以更透明、可審計且去中心化的方式取代傳統的預備金。

CyberPlaza 平台並不直接擁有其上列出的計算資源。相反地,為確保持續供應與具競爭力的價格,平台利用社群提供的流動性,透過「團購」模式(靈感來自拼多多的商業模式,即算力拼多多)取得計算資源。

\textbf{團購說明}:儘管團購為本策略的一部分,但它是\textbf{補充收益來源}(佔總收益的 5–10\%),而非主要業務模式。本平台的主要價值來自 SaaS 訂閱與智慧雲端管理工具。團購折扣將隨平台擴張而推行,但本平台的核心價值主張並不依賴於從雲端提供商取得大量批量折扣。

此團購模式減少了消費者與當前壟斷所有計算資源的工業巨擘之間的不平衡。這使平台能夠批量取得計算資源並提供給使用者,打造一個充滿活力且繁榮的交易市場。拼多多模式業務的收益透過質押獎勵機制分配給 CPT 質押者,其中 30\% 的平台收益分配至質押池(從 40\% 調整以提升永續性)、35\% 用於營運與成長、20\% 用於買回與銷毀、10\% 用於團隊、5\% 用於緊急預備金。

本專案的核心團隊在與專案相關的領域擁有豐富經驗,包括分散式高效能運算、公有雲服務、異質運算、AI 與大數據應用、分散式系統軟體開發、DeFi 投資、商業犯罪預防,以及計算資源的商業與行銷。

總體而言,CyberPlaza Network Web3 專案旨在提供一個去中心化的計算資源交易市場,讓使用者與服務提供商皆能獲益,同時透過 CPT 代幣持有與質押獎勵讓所有參與者共享平台的成功。憑藉強大的核心團隊以及受淘寶與拼多多等成功平台啟發的創新商業模式,本專案為運算產業的所有參與者創造了一個繁榮、合規且永續的生態系統。
