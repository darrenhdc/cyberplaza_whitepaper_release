\chapter{執行摘要}

運算在現代生活中扮演著日益重要的角色,且此趨勢預計在可預見的未來仍將持續。Web3 CyberPlaza Network Project旨在讓個人與機構都能以開放且具包容性的方式從此趨勢中獲益。

本專案引進CyberPlaza平台,一個可被描述為「算力淘寶平台」的去中心化市場平台。此平台匹配用戶與服務提供者(SPs)的需求,涵蓋高效能運算、智慧運算與雲端運算領域。用戶可於單一平台取得多樣化的算力、儲存空間、軟體應用、數據及運算服務,以具成本效益的方式滿足其特定需求。另一方面,服務提供者可獲得無限制的全球銷售管道。服務提供者與用戶兩者皆可透過持有CyberPlaza代幣(CPTs\footnote{CPT為物理學中的守恆量,所有物理定律皆不得違反CPT守恆,如同能量守恆一般。CPT(Carriage Paid To,貨費付訖)亦是一種國際貿易術語,意為賣方將支付貨品運送至消費者的費用。我方代幣之名稱即承載了這兩種隱喻。})共享平台的成功,該代幣代表此「淘寶」平台的治理股份。

\textbf{付款與結算}:平台上的交易以USDC結算,USDC為一種被廣泛採用且受監管的穩定幣,可確保符合規範且讓用戶熟悉。此方式消除了與專屬穩定幣相關的複雜性與監管風險,同時維持透明的美元計價。

\textbf{營收模式}:平台透過多種管道產生營收:SaaS訂閱(40--50\%),包含平台存取的月付/年付訂閱;交易費用(25--30\%),包含運算資源購買的2--5\%費用;API與數據服務(15--20\%),提供進階API存取與分析;以及團購(5--10\%),透過大量採購獲取利潤作為輔助營收。這些營收透過透明的質押獎勵機制分配給CPT代幣持有者,使參與者能透過對平台流動性與治理的貢獻,獲得永續的收益(目標6--10\% APY)。

\textbf{去中心化流動性池}:為支持平台的團購營運並確保具競爭力的價格,本專案實施一個去中心化借貸池,參與者可存入USDC以獲得利息(5--7\% APY)以及CPT獎勵(2--3\% APY),同時為平台提供營運資金。此模式以更透明、可稽核且去中心化的方式取代傳統準備金。

CyberPlaza平台不直接擁有其上列出的運算資源。相反地,為確保持續供應與具競爭力的價格,平台利用社群提供的流動性,透過靈感來自拼多多商業模式的「團購」模式(團購)來取得運算資源。

\textbf{團購注意事項}:儘管團購為我方策略的一部分,但其為\textbf{輔助營收來源}(占總營收的5--10\%)而非主要商業模式。我方的主要價值來自SaaS訂閱與智慧雲端管理工具。團購折扣將在平台擴張時推行,但我方核心價值主張不依賴於從雲端服務提供者取得大量批量折扣。

此團購方式減少了消費者與目前控制所有運算資源的產業巨頭之間的不平衡。這讓平台得以大量取得運算資源並提供給用戶,創建一個充滿活力且蓬勃發展的市場。拼多多業務的收益透過質押獎勵機制分配給CPT質押者,平台營收的30\%分配至質押池(為提升永續性,已從40\%調降),35\%用於營運與成長,20\%用於回購與銷毀,10\%用於團隊,5\%用於緊急儲備金。

本專案的核心團隊在與專案相關的領域擁有豐富經驗,包括分散式高效能運算、公共雲端服務、異質運算、AI與大數據應用、分散式系統軟體開發、DeFi投資、商業犯罪預防,以及運算資源的商業與行銷。

總之,CyberPlaza Network Web3專案旨在提供一個去中心化的運算資源市場平台,讓用戶與服務提供者皆能獲益,同時透過CPT代幣持有與質押獎勵,讓所有參與者共享平台的成功。憑藉堅強的核心團隊,以及受到淘寶與拼多多等成功平台啟發的創新商業模式,本專案為運算產業的所有參與者創建了一個蓬勃發展、符合規範且永續的生態系統。
