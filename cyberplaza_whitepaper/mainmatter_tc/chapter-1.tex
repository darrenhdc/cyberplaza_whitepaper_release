\chapter{執行摘要}

運算科技在現代生活中扮演著日益重要的角色,且預期此趨勢在可預見的未來仍將持續。Web3 CyberPlaza Network 專案旨在讓個人與機構均能以開放且包容的方式從此趨勢中獲利。

專案推出 CyberPlaza 平台,一個去中心化市集,可被稱為「運算資源版淘寶」(算力淘寶平台)。此平台媒合使用者與服務供應商(SPs)之需求,涵蓋高效能運算、智慧運算與雲端運算領域。使用者可在單一平台獲得多元的運算能力、儲存空間、軟體應用、資料與運算服務,以具成本效益的方式滿足其特定需求。另一方面,服務供應商(SPs)可獲得無限制的全球銷售管道。服務供應商(SPs)與使用者透過持有 CyberPlaza 代幣(CPTs\footnote{CPT 是物理學中的守恆量;所有物理定律皆不得違反 CPT 守恆,類似於能量守恆。CPT(Carriage Paid To,運費付至指定地)亦為國際貿易術語,指賣方將支付貨品遞送至消費者的費用。我方代幣名稱蘊含上述兩種隱喻。})共享平台成功,CPTs 代表「淘寶版」平台的治理股權。

\textbf{支付與結算}:平台上的交易以 USDC 結算,USDC 為廣受採用且受規範的穩定幣,確保符合規範且使用者熟悉。此方式消除了與專屬穩定幣相關的複雜性與規範風險,同時維持透明的美元計價定價機制。

\textbf{營收模式}:平台透過多個管道產生營收:SaaS 訂閱(40--50\%),包含平台存取的月付/年付訂閱;交易費用(25--30\%),包含運算資源購買的 2--5\% 費用;API \& 資料服務(15--20\%),提供高級 API 存取與分析;以及團購(5--10\%),從大量採購中獲取利差作為輔助營收。這些營收透過透明的質押獎勵機制分配給 CPT 代幣持有者,使參與者能透過貢獻平台流動性與治理獲得可持續收益(目標 6--10\% APY)。

\textbf{去中心化流動性池}:為支援平台的團購營運並確保具競爭力的定價,專案實施了去中心化借貸池,參與者可存入 USDC 以獲得利息(5--7\% APY)加上 CPT 獎勵(2--3\% APY),同時為平台提供營運資金。此模式為平台營運資金籌措提供了透明、可審計且去中心化的方式。

CyberPlaza 平台並不直接擁有其上列出的運算資源。取而代之的是,為確保持續供應與具競爭力的定價,平台運用社群提供的流動性,透過「團購」模式(團購)取得運算資源,靈感源自拼多多(Pinduoduo,算力拼多多)的商業模式。

\textbf{團購注意事項}:儘管團購為我方策略的一部分,但它是 \textbf{輔助營收管道}(5--10\% 占總營收)而非主要商業模式。我方的主要價值來自 SaaS 訂閱與智慧雲端管理工具。團購折扣將隨平台規模擴大而推動,但我方核心價值主張並不仰賴從雲端供應商取得大量折扣。

此團購方式降低了消費者與當前壟斷所有運算資源的產業巨頭之間的失衡。這使平台能大量取得運算資源並提供給使用者,創造動態且蓬勃發展的市集。拼多多模式業務的收益透過質押獎勵機制分配給 CPT 質押者,平台營收的 30\%(為提升永續性從 40\% 調降)配置於質押池,35\% 用於營運與成長,20\% 用於回購 \& 銷毀,10\% 用於團隊,5\% 用於緊急準備金。

專案核心團隊在與專案相關的領域擁有豐富經驗,包括分散式高效能運算、公共雲服務、異質運算、AI 與大數據應用、分散式系統軟體開發、DeFi 投資、商業犯罪預防,以及運算資源的商業與行銷。

總體而言,CyberPlaza Network Web3 專案旨在提供一個去中心化的運算資源市集,讓使用者與服務供應商均能獲利,同時透過 CPT 代幣持有與質押獎勵讓所有參與者共享平台成功。憑藉強大的核心團隊以及受淘寶與拼多多等成功平台啟發的創新商業模式,專案為運算產業的所有參與者創造了一個蓬勃發展、符合規範且永續的生態系。
