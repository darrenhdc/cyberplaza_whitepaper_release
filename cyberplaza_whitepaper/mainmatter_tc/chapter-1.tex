\chapter{執行摘要}

運算在現代生活中扮演著日趨重要的角色,且預期此趨勢在可預見的未來仍會持續。Web3 CyberPlaza 網路計畫旨在讓個人和機構皆能以開放且具包容性的方式從此趨勢中獲益。

本計畫推出CyberPlaza平台,一個可稱為「算力淘寶平台」的去中心化市集。此平台媒合用戶與服務提供者(SPs)的需求,涵蓋高效能運算、智慧運算與雲端運算領域。用戶可在單一平台存取多樣化的運算算力、儲存空間、軟體應用、資料與運算服務,以具成本效益的方式滿足其特定需求。另一方面,服務提供者(SPs)可獲得無限制的全球銷售管道。SPs與用戶均可透過持有代表「淘寶」平台治理份額的CyberPlaza代幣(CPTs\footnote{CPT是物理學中的守恆量;所有物理定律都不得違反CPT守恆,類似於能量守恆。CPT(Carriage Paid To)也是國際貿易術語,意指賣方將負責支付貨物運送至消費者的費用。本代幣名稱包含這兩層隱喻。})分享平台的成功。

\textbf{付款與結算}:平台上的交易以USDC(一種廣泛採用且受監管的穩定幣)進行結算,確保符合監管規定且用戶熟悉。此做法消除了與專屬穩定幣相關的複雜性和監管風險,同時維持透明的美元計價定價機制。

\textbf{營收模式}:平台透過多種管道產生營收:SaaS訂閱(40--50\%),包含平台存取的月付/年付訂閱費;交易費(25--30\%),即運算資源購買的2--5\%費用;API與資料服務(15--20\%),提供高級API存取與分析服務;以及團購(5--10\%),透過大量採購獲利作為輔助營收。這些營收透過透明的質押獎勵機制分配給CPT代幣持有者,讓參與者可透過貢獻平台流動性與治理權來獲得可持續的收益(目標年收益率6--10\% APY)。

\textbf{去中心化流動性池}:為支援平台的團購營運並確保具競爭力的定價,本計畫實施去中心化借貸池,參與者可存入USDC以獲得利息(5--7\% APY)加CPT獎勵(2--3\% APY),同時為平台提供營運資金。此模式以更透明、可審計且去中心化的方式取代傳統儲備金。

CyberPlaza平台並不直接擁有其上列出的運算資源。相反地,為確保穩定供應與具競爭力的定價,平台藉由社群提供的流動性,透過受拼多多(算力拼多多)商業模式啟發的「團購」模式獲取運算資源。

\textbf{團購注意事項}:儘管團購是我們策略的一部分,但它是\textbf{輔助營收管道}(占總營收的5--10\%),而非主要商業模式。我們的主要價值來自SaaS訂閱與智慧雲端管理工具。隨著平台擴張,我們將追求團購折扣,但我們的核心價值主張並不依賴於從雲端服務提供商獲得大量批量折扣。

團購模式減少了消費者與提供者(即當前控制所有運算資源的產業巨頭)之間的不平衡。這使平台能夠批量獲取運算資源並提供給用戶,創造一個充滿活力且蓬勃發展的市集。拼多多業務的收益透過質押獎勵機制分配給CPT質押者,平台營收的30\%分配給質押池(為提升永續性從40\%調降),35\%用於營運與成長,20\%用於回購與銷毀,10\%用於團隊,5\%用於緊急儲備金。

本計畫的核心團隊具備與計畫相關領域的豐富經驗,包括分散式高效能運算、公有雲服務、異質運算、AI與大數據應用、分散式系統軟體開發、DeFi投資、商業犯罪預防以及運算資源的商業與行銷。

總體而言,Web3 CyberPlaza 網路計畫旨在為運算資源提供一個去中心化市集,讓用戶與服務提供者皆能獲益,同時讓所有參與者透過持有CPT代幣與質押獎勵分享平台的成功。憑藉強大的核心團隊以及受淘寶與拼多多等成功平台啟發的創新商業模式,本計畫為運算產業的所有參與者創造了一個蓬勃發展、符合監管且具永續性的生態系。
