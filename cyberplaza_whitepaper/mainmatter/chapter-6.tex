\chapter{Technology and Architecture}



\section{Project Governing Infrastructures}

\subsection{General Description}

CyberPlaza Network consists of CyberPlaza Foundation and CyberPlaza Community.

CyberPlaza Foundation is a non-profit decentralized organization dedicated to the successful operation of the CyberPlaza Platform, the promotion and development of computing technologies and applications, and supporting decentralized community building and development on the platform. The Foundation is owned and controlled by the CPT holders in the CyberPlaza Community. The Foundation is run by the Core members of the Network (See Sec. 8 of the White Paper) and Advisors appointed by the Foundation from time to time as needed. The Foundation will establish CyberPlaza Labs responsible for developing and researching new computing resource technologies and applications to promote technological innovation and progress needed by the platform.

CyberPlaza Community is the community section of the Network, composed of the Liquidity Providers, Users, and SPs, who participate in Foundation governance, development, and promotion together. Community members can promote the development and growth of the Platform by participating in governance, making proposals to the Foundation on business directions, technology developments and in exchanging and sharing experiences.

The tight connection between the CyberPlaza Foundation and the CyberPlaza Community is essential in achieving the Vision and Mission of the Network.

\subsection{Smart Contract Modules}

We will deploy CPT as an ERC20-compliant smart contract on Arbitrum (Layer 2 of Ethereum), chosen for its low transaction costs and high throughput. The platform will also bridge to other chains as needed for ecosystem expansion.

The CPT token contract includes the following key features: Standard ERC20 functionality (transfer, approve, etc.), staking and lock functions for veToken mechanism, governance voting integration, reward distribution mechanisms, emergency pause functionality (governance-controlled), and upgradeable proxy pattern for future enhancements.

\textbf{Note}: The platform uses USDC directly for payments, eliminating the need for a proprietary stablecoin and associated regulatory risks.

\begin{verbatim}
// SPDX-License-Identifier: MIT
pragma solidity ^0.8.0;

import "@openzeppelin/contracts/token/ERC20/ERC20.sol";

contract CPTToken is ERC20 {
  struct LockInfo {
     uint256 amount;
     uint256 lockTimestamp;
     uint256 unlockTimestamp;
  }

   mapping (address => LockInfo[]) public locks;

   constructor(uint256 initialSupply) ERC20("CPT Token", "CPT") {
     _mint(msg.sender, initialSupply);
   }

   function lock(uint256 _amount, uint256 _lockTime) public {
     require(_amount <= balanceOf(msg.sender), "Not enough CPT to lock");
     require(_lockTime > 0, "Lock time must be positive");

       uint256 lockUntil = block.timestamp + _lockTime;
    
       LockInfo memory newLock = LockInfo({
           amount: _amount,
           lockTimestamp: block.timestamp,
           unlockTimestamp: lockUntil
       });
    
       locks[msg.sender].push(newLock);
    
       _burn(msg.sender, _amount);

   }

  function unlock(uint256 lockIndex) public {
    require(lockIndex < locks[msg.sender].length, 
            "No lock found at this index");
    require(block.timestamp >= locks[msg.sender][lockIndex].unlockTimestamp,
            "CPT still locked");

        uint256 amountToUnlock = locks[msg.sender][lockIndex].amount;
        locks[msg.sender][lockIndex] = 
            locks[msg.sender][locks[msg.sender].length - 1];
        locks[msg.sender].pop();
    
        _mint(msg.sender, amountToUnlock);
    }

   function calculateLockedAmount(address user, uint256 lockDuration) 
       public view returns (uint256) {
     uint256 totalLockedAmount = 0;

        for (uint256 i = 0; i < locks[user].length; i++) {
           if (block.timestamp - locks[user][i].lockTimestamp > lockDuration) {
               totalLockedAmount += locks[user][i].amount;
           }
        }
    
        return totalLockedAmount;
    }

}
\end{verbatim}

\subsection{Token Standard and Decimal Handling}

The CPT token adheres to the standard ERC20 specification with 18 decimals, while USDC operates with 6 decimals. The platform employs SafeMath libraries for all conversion operations to prevent overflow and underflow errors. Price oracles incorporate decimal normalization logic, and minimum transaction thresholds mitigate dust attack vectors. For fractional amounts, the protocol implements conservative rounding mechanisms.

\subsection{Vote-Escrowed Token Mechanism}

The platform implements a vote-escrowed (ve) token model to align long-term stakeholder incentives. Users lock CPT for periods ranging from one week to four years, receiving non-transferable veCPT tokens that determine both governance weight and reward allocation.

The veCPT balance follows the relationship:
\begin{equation}
\text{veCPT} = \text{CPT}_{\text{locked}} \times \min\left(\frac{t_{\text{lock}}}{t_{\text{max}}}, 1\right) \times 2.5
\end{equation}
where $t_{\text{lock}}$ represents the chosen lock duration and $t_{\text{max}} = 4$ years defines the maximum lock period. The multiplier of 2.5 provides maximum governance weight for four-year commitments.

As the lock period approaches expiration, veCPT balance decays linearly:
\begin{equation}
\text{veCPT}(t) = \text{CPT}_{\text{locked}} \times \frac{t_{\text{remaining}}}{t_{\text{max}}} \times 2.5
\end{equation}
This decay mechanism incentivizes continuous participation through lock extensions or token relocking.

\subsubsection{Reward Distribution}

Platform revenue collected in USDC is distributed with 30\% allocated to the staking rewards pool on a weekly or monthly basis. Individual rewards are calculated proportionally to veCPT holdings:
\begin{equation}
\text{Reward}_{\text{user}} = \text{Revenue}_{\text{pool}} \times \frac{V_{\text{user}}}{V_{\text{total}}}
\end{equation}
where $V_{\text{user}}$ represents the user's veCPT balance and $V_{\text{total}}$ denotes total veCPT supply. The effective APY varies dynamically based on staking participation and platform performance:
\begin{equation}
\text{APY} = \frac{\text{Annual Revenue Pool}}{\text{Total CPT Staked Value}} \times \frac{\text{veCPT Multiplier}}{\text{Average Multiplier}}
\end{equation}

\subsubsection{Security and Optimization}

Smart contracts undergo third-party audits following OpenZeppelin standards, with multi-signature governance controlling parameter modifications. All reward distributions are tracked on-chain for transparency. Security features include reentrancy guards on external calls, role-based access control, emergency pause functionality, and upgradeable proxy patterns. Critical parameter changes enforce a 48-hour timelock.

Gas optimization employs Merkle tree-based batch claiming, lazy evaluation of veCPT balances, packed storage variables, and event-driven off-chain indexing. These techniques reduce transaction costs while maintaining security guarantees.

\subsection{Oracle Integration}

The platform integrates Chainlink decentralized oracles for price discovery and data aggregation. CPT/USD price feeds aggregate data from Uniswap V3 time-weighted average prices and centralized exchange quotations. USDC/USD verification employs Chainlink's verified feed with a 0.5\% deviation threshold. Oracles update at 5-minute intervals or upon 1\% price movements, with manual fallback mechanisms for redundancy.

For computing resource pricing, off-chain aggregators monitor public APIs of major cloud providers (AWS, Azure, GCP, Alibaba Cloud), calculating real-time market rates for compute, storage, and bandwidth. Aggregated pricing publishes to on-chain oracle contracts daily or when deviations exceed 5\%.

Oracle security relies on consensus from at least seven independent Chainlink nodes. The system rejects price updates deviating beyond 10\% from the median or data older than one hour. Circuit breakers automatically halt trading upon detecting manipulation attempts.

\subsection{Governance Architecture}

Critical platform operations require multi-signature approval through Gnosis Safe implementation. Treasury movements exceeding 100K USDC require 5-of-9 signatures, while smart contract upgrades mandate 7-of-9 approval with a 48-hour timelock. Parameter adjustments operate under 4-of-9 consensus, and emergency security responses utilize a 3-of-5 rapid response configuration.

The governance process follows a structured timeline: holders of 100K+ veCPT may submit proposals, followed by a 7-day community discussion period and 5-day on-chain voting phase where 1 veCPT equals 1 vote. Approved proposals execute after a 48-hour delay. The multi-signature council retains veto authority over malicious proposals, subject to quarterly review.

\subsection{Cross-Chain Infrastructure}

The platform implements LayerZero omnichain protocol for multi-chain deployment. Arbitrum serves as the primary chain for its low transaction costs and high throughput. Ethereum mainnet support targets institutional users requiring Layer-1 security, while Polygon integration offers reduced transaction costs for cost-sensitive users. Future expansion includes Optimism (Q3 2024) and Base (Q4 2024) for broader ecosystem integration.

Bridge security incorporates multiple safeguards: liquidity caps restrict bridged supply to 10\% per chain, rate limiting constrains throughput to 1M CPT per hour, emergency pause mechanisms respond to anomalies, and a 5\% insurance fund collateralizes bridged value against potential exploits.

\subsection{Wallet Infrastructure}

As a standard ERC20 token, CPT supports all compliant wallets including browser extensions (MetaMask, Rabby, Rainbow), mobile applications (Trust Wallet, Coinbase Wallet, imToken), hardware devices (Ledger, Trezor), and smart contract wallets (Argent, Gnosis Safe). Institutional custody integration with Fireblocks and Copper.co is planned for future deployment.

The web portal implements WalletConnect and Web3Modal protocols for standardized wallet connectivity. Upon connection authorization, the platform queries user balances, staking positions, and veCPT holdings to enable full feature access. Transaction signing follows EIP-712 standards for typed structured data, presenting human-readable messages that improve security against phishing vectors.


\section{Marketplace Computing Infrastructure}

\subsection{System Architecture}

The platform implements a three-tier architecture. The Web3 interface layer manages wallet authentication (WalletConnect), USDC payment processing, and CPT reward distribution through React.js and ethers.js frameworks. The orchestration layer coordinates the CHESS cluster management system, job scheduling, resource allocation, performance monitoring, and SP certification processes. The computing resource layer aggregates CSP clusters, public cloud APIs (AWS, Azure, GCP, Alibaba), private HPC centers, and future edge computing nodes.

Transaction flow proceeds through job submission with USDC deposit, smart contract escrow until completion, CHESS-mediated resource matching, execution on allocated SP infrastructure, real-time SLA compliance monitoring, result delivery with automated payment settlement, and proportional CPT reward distribution (1-3\% users, 2-5\% SPs).

\subsection{HPC Infrastructure Components}

High-Performance Computing infrastructure comprises specialized node types: compute nodes execute numerical simulations and data analysis with multi-core processors and substantial memory; visualization nodes render large datasets using GPU acceleration; I/O nodes manage data transfer between storage and compute fabric; storage nodes provide high-concurrency file systems; management nodes coordinate resource allocation and job scheduling.

Network fabric employs high-speed interconnect technologies (InfiniBand, Ethernet) for inter-node communication. Parallel file systems enable concurrent multi-node read/write operations for large datasets and intermediate results.

\subsubsection{Software Stack}

Monitoring and management tools provide administrators real-time health and performance data across system components including CPU utilization, memory consumption, and network traffic patterns. Cluster management software coordinates overall system operations with provisioning, monitoring, and maintenance capabilities for compute nodes across geographically distributed installations.

Resource allocation employs specialized schedulers managing CPU time, memory, and other computational resources to maximize system utilization efficiency. User interfaces span command-line tools and web portals for job submission and management. The HPC Application Center aggregates domain-specific applications and templates, enabling users to download and deploy computational tools directly. Integrated billing systems implement transparent pricing strategies across resource types and billing cycles, facilitating rational resource utilization and accurate cost accounting.

\subsection{Payment and Settlement Infrastructure}

\subsubsection{Escrow Mechanism}

Job submission initiates an escrow process wherein users approve USDC spending to the platform's smart contract. The escrow contract calculates estimated costs incorporating resource type (CPU/GPU/Storage), duration projections, oracle-derived market pricing, and a 20\% buffer for potential overruns. USDC transfers to escrow upon approval, locked against the unique job identifier.

Upon job completion, actual resource consumption determines final settlement. Service Providers receive 95-98\% of the fee directly in USDC, while the platform retains a 2-5\% transaction fee. Excess escrowed funds return to users automatically, and CPT rewards distribute proportionally to both users (1-3\%) and SPs (2-5\%).

\subsubsection{Dispute Resolution Protocol}

SLA violations trigger graduated resolution mechanisms. Jobs failing within five minutes qualify for automatic full refunds. Partial completion generates pro-rated refunds based on actual delivery. Users may file disputes within a 72-hour window with supporting evidence. Cases exceeding 10K USDC value escalate to platform governance arbitration, while the insurance fund covers validated claims up to 100K USDC.

\subsubsection{Service Provider Certification}

Service Provider certification requires a multi-stage verification process. Initial registration demands company verification documents, infrastructure specifications, payment wallet addresses, and security compliance certificates (SOC 2, ISO 27001). Technical verification employs industry-standard benchmarks including High-Performance Linpack (HPL), High-Performance Conjugate Gradient (HPCG), STREAM memory bandwidth, MLPerf for AI workloads, and network latency assessments. Security audits verify AES-256 encryption, network isolation, and DDoS protection capabilities.

Approved candidates enter a 30-day probationary period with enhanced monitoring and a 10-job concurrency limit. Successful completion grants Certified Service Provider (CSP) status, enabling access to institutional clients and group-buying participation. CSPs appear in the premium directory with verified badges.

Ongoing compliance mandates 99.5\% monthly uptime, sub-5-minute job initiation, and performance within 10\% of advertised benchmarks. Quarterly re-certification validates continued capability. Security patches for critical vulnerabilities must deploy within 48 hours. Violations trigger graduated penalties: first offense warnings with 7-day remediation, second offense 30-day suspension, third offense certification revocation.

\subsection{Technical Stack}

The platform employs React.js 18+ with TypeScript for frontend development, ethers.js v6 and WalletConnect v2 for Web3 integration, and Material-UI for interface consistency. Backend architecture utilizes Node.js/Express.js or Python FastAPI for API services, PostgreSQL for relational persistence, Redis for caching, RabbitMQ/Kafka for asynchronous job queuing, The Graph for blockchain event indexing, and Prometheus/Grafana for observability. DevOps infrastructure containerizes all services via Docker, orchestrates production deployment through Kubernetes, implements CI/CD via GitHub Actions, distributes content through Cloudflare CDN, and load balances traffic via Nginx.

Entry-level CSPs require 100+ CPU cores (Intel Xeon/AMD EPYC), 500 GB RAM, 10 TB NVMe SSD or 50 TB HDD, 10 Gbps network uplink, and optionally 4+ NVIDIA A100/H100 GPUs. Enterprise CSPs scale to 10,000+ CPU cores, 50 TB+ aggregate RAM, 1 PB+ parallel file system storage (Lustre/GPFS), 100 Gbps InfiniBand backbone, and 100+ high-end GPUs.

\subsection{Platform User Functions}

The CPT portal serves three primary constituencies: visitors exploring project information, users procuring marketplace resources (public cloud, HPC providers, hardware, software, storage), and Liquidity Providers executing USDC deposits and minting operations.

Public cloud consumers select between vendors including FQ, Amazon, and Huawei Cloud with pricing denominated in USDC alongside promotional offerings. Vendor selection redirects users to native portals (e.g., AWS) where standard operations proceed with payment routing through CPT platform escrow. The platform subsequently settles with vendors in legal tender.

HPC resource consumers compare vendors (CT clusters, regional providers, Huawei, AWS) across price points, hardware specifications, performance metrics, and regional bandwidth. Vendor selection and job submission occur through CHESS portal following adequate USDC deposit, with funds escrowed until completion whereupon fiat settlement proceeds. Storage procurement follows identical workflows.

Software options encompass user-provided applications or platform-listed solutions from Ansys, HPC software vendors, and the CHESS application center. Vendor onboarding accommodates hardware, storage, software, and ancillary computing products. The system validates hardware-software compatibility when both components originate from platform listings, ensuring execution compatibility. The architecture accommodates future functional extensions as requirements evolve.


\subsection{Public Cloud Integration}

The marketplace aggregates computing resources from major public cloud vendors including AWS, Azure, Google Cloud, and Alibaba Cloud. Pricing displays in USDC denomination with active promotions and availability status.

\subsubsection{Vendor Integration Models}

The platform employs three integration methodologies. Direct API integration utilizes reseller credentials for real-time provisioning through vendor APIs (AWS EC2, Azure Resource Manager, GCP Compute Engine), enabling automatic instance lifecycle management. The coupon code system addresses capacity constraints through pre-generated codes preventing overselling, available in value-based (\$100 universal credits) or resource-specific (1000 GPU hours, 10 TB storage) formats. The Managed Service Provider model positions CyberPlaza as an MSP with bulk pricing agreements, managing vendor accounts and providing consolidated billing.

Real-time pricing comparison displays compute, storage, and network costs with total ownership calculations. Group-buying discounts highlight potential savings relative to direct procurement.

\subsubsection{Job Submission Workflow}

The HPC job submission process encompasses resource selection through filtered CSP listings (CPU type, GPU availability, region, pricing), job configuration via Application Center templates or custom code with specified requirements (nodes, cores, memory, runtime, GPUs) and I/O locations, cost estimation with USDC breakdown and projected CPT rewards, payment authorization transferring USDC to escrow with contingency buffer, execution through CHESS scheduler allocation with real-time status monitoring, and completion settlement delivering results with automated payment distribution, excess refunds, and CPT reward issuance.

Advanced capabilities include batch submission supporting 100+ jobs with parameter sweeps, workflow dependencies defining sequential execution, checkpoint/restart for fault tolerance, spot instance bidding yielding 50-70\% discounts on preemptible capacity, and auto-scaling for dynamic resource adjustment.

Service Providers manage operations through a centralized dashboard encompassing resource allocation, job oversight, financial tracking (USDC revenue, CPT accumulation), and performance analytics (customer satisfaction, utilization metrics).

\subsection{Multi-Cluster Management System}

The CHESS (Cluster High-performance Execution and Scheduling System) platform provides unified management across geographically distributed computing resources. The system integrates monitoring, scheduling, and resource allocation through a centralized web portal with role-based access control.

\subsubsection{Core Capabilities}

The platform supports comprehensive data management through web interfaces and SSH protocols, enabling file operations including upload, download, compression, and extraction. Node management operates through batch commands controlling power states, remote access (VNC, shell), and supporting heterogeneous hardware configurations (CPU, GPU, FPGA). Resource quotas enforce administrative policies on storage and compute allocation with automated alert generation upon threshold violations.

High availability architecture eliminates single points of failure through redundant management nodes and database replication. The system coordinates multiple geographically distributed clusters with unified user role propagation across sub-clusters.


\subsection{Performance Monitoring Infrastructure}

High-performance and cloud computing systems aggregate substantial hardware resources interconnected via high-speed networks forming low-latency, high-capacity configurations. Effective cluster management necessitates monitoring and management tools providing resource configuration, real-time performance tracking, fault detection with alerting, and usage state visualization.

\subsubsection{CHESS Monitoring Capabilities}

The CHESS monitoring system provides comprehensive cluster oversight through aggregated dashboards displaying CPU and memory usage, load status, storage state, and network throughput across Ethernet and InfiniBand fabrics. Custom time interval selection enables historical trend analysis and performance tracking. Dashboard displays offer customizable large-screen presentations with dynamic metric refresh for storage usage, job scheduling, and network statistics.

Multi-cluster monitoring extends across geographically distributed installations with adaptive screen layouts and resolution optimization. Rack visualization renders physical topology with integrated power management and VNC remote access controls. Single-node monitoring captures granular CPU, memory, storage, load, and network metrics while providing fault diagnostics and recovery recommendations. GPU monitoring tracks device-specific usage rates, memory utilization, temperature, and bandwidth. Job monitoring analyzes real-time execution status and queue composition with detailed CPU utilization, memory consumption, and node load statistics. Cluster alerts implement configurable thresholds with email and system notification routing.

Performance metrics collect at user-defined intervals capturing CPU, memory, disk, and network data. Physical topology visualization encompasses rack and node arrangements with threshold-based fault alerting.


\subsubsection*{Schedulers and Resource Management}

Efficient scheduling and resource management are critical in multi-cluster systems. CHESS offers flexible scheduling policies including FIFO, preemption, and backfilling strategies. The system supports resource reservations with Quality of Service (QoS) configurations, advanced job submissions spanning serial, parallel, and GPU workloads, and queue management for load balancing optimization.

\subsubsection{Job Submission and Management}

Users submit jobs via command-line interfaces, web-based GUIs, or application templates for common workflows. Administrators configure resource quotas, priority levels, and submission policies to govern system access and utilization.



% \subsection*{6.2.5 Pricing Module}

% This module will focus on calculating costs for resource usage and presenting pricing details to users. It will integrate with job scheduling and monitoring systems for real-time cost tracking.


% \subsubsection*{6.2.5 User Interfaces and Operational Portals}

\subsubsection{User Management}

The platform supports self-registration and administrator-configured accounts with LDAP authentication integration for centralized management. Role-based access control implements default roles (admin, department admin, user) with flexible privilege assignment governing system access and capabilities.

\subsubsection{Notifications and Messaging}

Users receive automated alerts for billing and usage messages alongside administrative announcements.


\subsection{Application Center}

The Application Center provides access to pre-installed HPC applications (Ansys, MATLAB, TensorFlow) through a browsable library. Users submit jobs via graphical templates with interactive parameter configuration. Output management encompasses log viewing, error analysis, performance metric tracking, and integrated visualization tools (TensorBoard for AI applications).

\subsection{Hardware Performance Evaluation}

The hardware performance evaluation module executes benchmarking tests measuring CPU and GPU performance alongside network throughput and latency. Resource efficiency analysis optimizes allocation strategies based on workload characteristics. Fault recovery metrics assess hardware reliability and recovery performance under failure scenarios.

\subsection{Security Architecture and Compliance}

\subsubsection{Multi-Layer Security Model}

The platform implements defense-in-depth security across three layers. Smart contract security employs formal verification using Certora or equivalent tools, annual third-party audits by CertiK, Trail of Bits, or OpenZeppelin, bug bounty programs offering up to \$500K for critical vulnerabilities, upgradeable transparent proxy patterns with 48-hour timelocks, and circuit breakers for emergency exploit response.

Platform security encompasses API protection through OAuth 2.0 and JWT authentication with 100 requests/minute rate limiting, IP whitelisting for SP access, and 90-day API key rotation. Data encryption implements TLS 1.3 for transit protection, AES-256 for data at rest, end-to-end encryption for sensitive workloads, and Hardware Security Modules for key management. Infrastructure security deploys Cloudflare DDoS protection, Web Application Firewalls with OWASP rulesets, quarterly penetration testing, and SIEM systems for event monitoring.

Data privacy and compliance measures address GDPR requirements through account deletion rights, data portability, privacy-by-design principles, and EU data residency options. KYC/AML procedures implement basic verification for transactions exceeding 10K USDC monthly, enhanced verification for CSP certification, transaction monitoring for suspicious activity, and FATF Travel Rule compliance. Data isolation employs containerized or VM-based job execution, network segmentation, automatic post-completion data wiping, and cross-user leakage prevention.

\subsubsection{Incident Response}

A continuous security operations center monitors for anomalous activity including unusual withdrawals, smart contract exploits, and API abuse. Incident classification follows a four-tier severity model (Critical, High, Medium, Low) with 15-minute assessment targets. Critical incidents trigger immediate contract pausing and multi-signature notification within one hour. Public disclosure occurs within 24 hours for critical events, while post-mortem reports publish within 7 days. Recovery procedures deploy patches through governance channels and compensate affected users from the insurance fund.

\subsubsection{Regulatory Compliance}

The platform pursues SOC 2 Type II certification for data security and availability (Year 1 target) and ISO 27001 for information security management (Year 2 target). Cloud Security Alliance STAR certification validates CSP security posture. PCI DSS compliance remains under consideration for future payment method expansion.

\subsection{Scalability and Performance Optimization}

\subsubsection{Horizontal Scaling Architecture}

The platform scales horizontally through distributed database architecture employing PostgreSQL read replicas across regions, user data sharding by ID hash, Redis clustering for hot data (sessions, pricing), and Cloudflare CDN for static asset delivery.

Microservices architecture decomposes functionality into independently scalable services: User Service (authentication, profiles), Job Service (submission, scheduling, monitoring), Payment Service (USDC escrow, settlement, CPT rewards), SP Service (onboarding, certification, rating), Pricing Service (oracle aggregation), and Notification Service (email, push, on-chain events). Each service scales autonomously based on demand.

Load balancing implements geographic distribution across US, EU, and Asia regions with Kubernetes Horizontal Pod Autoscaler for dynamic capacity adjustment, Hystrix circuit breakers preventing cascading failures, and RabbitMQ queuing for asynchronous job processing.

\subsubsection{Performance Targets}

\begin{center}
\begin{tabular}{|l|c|c|}
\hline
\textbf{Metric} & \textbf{Target (Year 1)} & \textbf{Target (Year 3)} \\
\hline
API Response Time & <200ms (p95) & <100ms (p95) \\
Job Submission Time & <5 seconds & <2 seconds \\
Payment Settlement & <30 seconds & <10 seconds \\
Page Load Time & <2 seconds & <1 second \\
Platform Uptime & 99.5\% & 99.9\% \\
Concurrent Users & 10,000 & 100,000 \\
Daily Transactions & 50,000 & 1,000,000 \\
\hline
\end{tabular}
\end{center}

\subsubsection{Blockchain Scalability}

Arbitrum Layer 2 deployment provides sub-\$0.10 transaction fees and 40,000 TPS throughput for primary operations. Batch transaction processing groups reward distributions to amortize gas costs. The Graph protocol handles off-chain event indexing. Future development includes state channels for high-frequency micro-payment scenarios.

Gas optimization techniques reduce transaction costs through Merkle proof-based reward claims (80\% savings), lazy veCPT balance evaluation, packed storage variable encoding, and preferential use of event logs over state variables where functionally equivalent.

\subsection{Disaster Recovery}

\subsubsection{Backup Infrastructure}

Database backups execute daily (full) and six-hourly (incremental) with continuous transaction log replication. The system maintains 30-day retention before cold storage archival. Smart contract state leverages blockchain's inherent immutability, supplemented by archive node deployment and quarterly decentralized storage snapshots (IPFS/Arweave). User job results backup to designated storage endpoints, with platform metadata retention for 90 days and on-demand export capability for GDPR compliance.

\subsubsection{Recovery Objectives}

Table~\ref{tab:recovery-targets} specifies component-level recovery time (RTO) and recovery point (RPO) objectives.

\begin{table}[htbp]
\centering
\caption{Recovery Time and Point Objectives}
\label{tab:recovery-targets}
\begin{tabular}{lcc}
\hline
\textbf{Component} & \textbf{RTO} & \textbf{RPO} \\
\hline
Smart Contracts & N/A & 0 \\
Web Portal & 1 hour & 6 hours \\
Database & 2 hours & 1 hour \\
Job Scheduler & 30 minutes & 15 minutes \\
\hline
\end{tabular}
\end{table}

Active-active deployment across US and EU regions enables automatic DNS failover upon 5-minute primary region unavailability. Real-time inter-region data synchronization maintains consistency, with manual override capability for operational intervention.

\subsection{Development Roadmap}

Near-term development (6-12 months) prioritizes mobile applications for iOS and Android, enhanced API offerings (RESTful, GraphQL) for third-party integration, machine learning-based cost optimization, and additional blockchain bridge deployment (Polygon, Optimism).

Medium-term objectives (1-2 years) expand platform capabilities through edge computing support for IoT deployments, confidential computing integration (Intel SGX, AMD SEV) for sensitive workloads, decentralized storage protocols (Filecoin, Arweave), specialized AI/ML resource marketplaces, and exploratory quantum computing partnerships.

Long-term vision (2-5 years) encompasses full DAO governance transition, open decentralized compute protocol development, zero-knowledge proof implementation for privacy enhancement, cross-chain interoperability via IBC or equivalent protocols, and NFT-based tokenization of physical computing resources.

\subsection{Summary}

This chapter detailed the technical architecture integrating Web3 blockchain infrastructure with established HPC systems. The hybrid design bridges decentralized incentive mechanisms (CPT token, vote-escrowed governance) with the proven CHESS cluster management platform. Security architecture implements multi-layer protection through smart contract audits, infrastructure hardening, and regulatory compliance pathways (SOC 2, ISO 27001). The system scales from thousands to hundreds of thousands of concurrent users while maintaining sub-200ms API response times.

Relative to existing decentralized computing projects (Golem, iExec, Render), CyberPlaza differentiates through mature infrastructure (20+ year CHESS platform history), enterprise compliance orientation, multi-cloud integration beyond peer-to-peer architectures, pre-integrated application ecosystem, and hybrid marketplace combining decentralized access with professional SP certification. This positioning addresses enterprise computing requirements while enabling Web3 economic participation.


