\chapter{执行摘要}

计算在现代生活中扮演着日益重要的角色,且这一趋势预计将在可预见的未来持续下去。Web3 CyberPlaza 网络项目旨在让个人和机构都能以开放、包容的方式从这一趋势中受益。

该项目推出了 CyberPlaza 平台,这是一个去中心化市场,可被描述为“算力淘宝平台”。该平台匹配用户与服务提供商(SP)的需求,涵盖高性能计算、智能计算和云计算领域。用户能够在单一平台获取多样化的算力、存储、软件应用、数据及计算服务,以高成本效益的方式满足其特定需求。而服务提供商则获得了无限制的全球销售渠道。服务提供商和用户都可通过持有 CyberPlaza 代币(CPT\footnote{CPT 是物理学中的守恒量;所有物理定律都不得违反 CPT 守恒,类似于能量守恒。CPT(Carriage Paid To,运费付至)也是一项国际贸易术语,指卖方将支付货物向消费者的交付费用。我们的代币名称承载了这两种隐喻。})分享平台的成功,CPT 代表着该“淘宝”平台的治理份额。

\textbf{支付与结算}:平台上的交易使用 USDC 结算,USDC 是一种被广泛采用且受监管的稳定币,可确保监管合规性和用户熟悉度。这种方式消除了与专有稳定币相关的复杂性和监管风险,同时保持了透明的美元计价定价。

\textbf{收入模型}:平台通过多种渠道产生收入:SaaS 订阅(40–50\%),包括平台访问的月度/年度订阅;交易手续费(25–30\%),对计算资源购买收取 2–5\% 的手续费;API 与数据服务(15–20\%),提供高级 API 访问和分析服务;以及团购(5–10\%),通过批量采购产生的利润作为补充收入。这些收入通过透明的质押奖励机制分配给 CPT 代币持有者,使参与者能够通过为平台提供流动性和治理支持来获得可持续收益(目标年化收益率 6–10\%)。

\textbf{去中心化流动性池}:为支持平台的团购运营并确保具有竞争力的定价,该项目实施了一个去中心化借贷池,参与者可在其中存入 USDC 以赚取利息(年化收益率 5–7\%)和 CPT 奖励(年化收益率 2–3\%),同时为平台提供运营资金。这种模式为平台运营资金提供了一种透明、可审计的去中心化方式。

CyberPlaza 平台并不直接拥有其上面列出的计算资源。相反,为确保持续供应和具有竞争力的定价,平台借鉴拼多多的商业模式,利用社区提供的流动性,通过“团购”模型(团购)来获取计算资源。

\textbf{关于团购的说明}:虽然团购是我们策略的一部分,但它是一种\textbf{补充收入来源}(占总收入的 5–10\%),而非主要商业模式。我们的主要价值来自 SaaS 订阅和智能云管理工具。平台规模扩大后将寻求团购折扣,但我们的核心价值主张并不依赖于从云提供商处获得大额批量折扣。

团购方式减少了消费者与当前控制所有计算资源的行业巨头之间的失衡。这使平台能够批量获取计算资源并将其提供给用户,从而创建一个充满活力且蓬勃发展的市场。拼多多模式业务的收益通过质押奖励机制分配给 CPT 质押者,平台收入的 30\% 分配给质押池(从 40\% 降至 30\% 以提高可持续性),35\% 用于运营与增长,20\% 用于回购销毁,10\% 用于团队,5\% 用于应急储备。

该项目的核心团队在与项目相关的领域拥有丰富经验,包括分布式高性能计算、公共云服务、异构计算、人工智能与大数据应用、分布式系统软件开发、DeFi 投资、商业犯罪预防,以及计算资源的业务与营销。

总体而言,CyberPlaza Network Web 3 项目旨在为计算资源提供一个去中心化市场,使用户和服务提供商都能受益,同时让所有参与者通过 CPT 代币持有和质押奖励分享平台的成功。凭借强大的核心团队以及受淘宝和拼多多等成功平台启发的创新商业模式,该项目为计算行业的所有参与者创造了一个蓬勃发展、合规且可持续的生态系统。
