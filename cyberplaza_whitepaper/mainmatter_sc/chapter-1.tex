\chapter{执行摘要}

计算在现代生活中发挥着日益关键的作用,且这一趋势预计将在可预见的未来持续。Web3 CyberPlaza网络项目旨在让个人和机构都能以开放、包容的方式从这一趋势中受益。

本项目推出CyberPlaza平台——一个可被描述为「算力淘宝平台」的去中心化市场。该平台匹配用户与服务提供商(SPs)的需求,覆盖高性能计算、智能计算和云计算领域。用户可在一处获取多样化的算力、存储、软件应用、数据及计算服务,以高性价比满足其特定需求。另一方面,服务提供商可获得无限制的全球销售渠道。服务提供商和用户均可通过持有CyberPlaza代币(CPTs\footnote{CPT是物理学中的守恒量,所有物理定律都不得违反CPT守恒,类似于能量守恒;CPT(Carriage Paid To,运费付至)也是国际贸易术语,意为卖家将支付商品运送给消费者的费用。我们的代币名称承载了这两层隐喻。})共享平台的成功,CPTs代表该「淘宝」平台的治理份额。

\textbf{支付与结算}:平台上的交易使用USDC结算——USDC是一种被广泛采用且受监管的稳定币,可确保合规性和用户熟悉度。这种方式消除了与专有稳定币相关的复杂性和监管风险,同时维持了透明的美元计价。

\textbf{营收模型}:平台通过多种渠道产生营收:SaaS订阅(40--50\%),包括平台访问的月度/年度订阅;交易手续费(25--30\%),即算力资源购买的2--5\%手续费;API与数据服务(15--20\%),提供高级API访问和分析服务;以及团购(5--10\%),通过批量采购产生的利润作为补充营收。这些营收通过透明的质押奖励机制分配给CPT代币持有者,使参与者能够通过为平台提供流动性和治理贡献来获得可持续收益(目标6--10\% APY)。

\textbf{去中心化流动性池}:为支持平台的团购运营并确保有竞争力的定价,本项目实施了一个去中心化借贷池,参与者可存入USDC以赚取利息(5--7\% APY)和CPT激励(2--3\% APY),同时为平台提供运营资金。这种模式提供了透明、可审计且去中心化的平台运营资金方式。

CyberPlaza平台并不直接拥有其上列出的算力资源。相反,为确保持续供应和有竞争力的定价,平台借鉴拼多多的商业模式,利用社区提供的流动性,通过「团购」模式锁定算力资源(算力拼多多)。

\textbf{团购说明}:尽管团购是我们策略的一部分,但它是\textbf{补充营收渠道}(占总营收的5--10\%),而非主要商业模式。我们的核心价值来自SaaS订阅和智能云管理工具。团购折扣将随着平台规模扩大而推进,但我们的核心价值主张并不依赖于从云提供商处获得大额批量折扣。

这种团购模式减少了消费者与当前控制所有算力资源的行业巨头之间的力量不平衡。这使平台能够批量锁定算力资源并提供给用户,打造一个动态且蓬勃发展的市场。拼多多模式业务的收益通过质押奖励机制分配给CPT质押者,平台营收的30\%(为提高可持续性从40\%下调)分配至质押池,35\%用于运营和增长,20\%用于回购销毁,10\%用于团队,5\%用于应急储备。

本项目核心团队在与项目相关的领域拥有丰富经验,包括分布式高性能计算、公共云服务、异构计算、AI与大数据应用、分布式系统软件开发、DeFi投资、商业犯罪预防,以及算力资源的商业和营销。

总体而言,Web3 CyberPlaza网络项目旨在为算力资源提供一个去中心化市场,让用户和服务提供商都能受益,同时让所有参与者通过CPT代币持有和质押奖励共享平台成功。凭借强大的核心团队以及受淘宝、拼多多等成功平台启发的创新商业模式,本项目为计算行业的所有参与者打造了一个蓬勃发展、合规且可持续的生态系统。
