\chapter{执行摘要}

计算在现代生活中扮演着日益关键的角色,且这一趋势预计在可预见的未来仍将持续。Web3 算力广场网络项目旨在让个人和机构都能以开放、包容的方式从这一趋势中获益。

该项目推出了算力广场平台,这是一个去中心化的市场,可被称为 ``算力淘宝平台''(算力淘宝平台)。该平台匹配用户与服务提供商(SPs)的需求,涵盖高性能计算、智能计算和云计算领域。用户可在一处获得多样化的算力、存储、软件应用、数据及计算服务,以高性价比满足其特定需求。另一方面,服务提供商获得了不受限制的全球销售渠道。服务提供商和用户均通过持有算力广场代币(CPTs\footnote{CPT 是物理学中的守恒量;所有物理定律均不得违反 CPT 守恒,与能量守恒类似。CPT(Carriage Paid To,运费付至)也是一项国际贸易术语,指卖方将承担货物运送给消费者的费用。我们的代币名称承载了这两种隐喻。})—— 即该 ``淘宝'' 平台的治理份额——共享平台的成功。

\textbf{支付与结算}:平台上的交易使用 USDC 结算,USDC 是一种被广泛采用且受监管的稳定币,确保了合规性和用户熟悉度。这种方式消除了与专有稳定币相关的复杂性和监管风险,同时保持了透明的美元计价。

\textbf{收入模式}:平台通过多种渠道产生收入:SaaS 订阅(40--50\%),包括平台访问的月度/年度订阅;交易费(25--30\%),涵盖算力资源购买的 2--5\% 费用;API 与数据服务(15--20\%),提供高级 API 访问和分析服务;团购(5--10\%),通过批量采购的利润作为补充收入。这些收入通过透明的质押奖励机制分配给 CPT 代币持有者,使参与者能够通过为平台提供流动性和治理贡献来获得可持续收益(目标年化收益率 6--10\% APY)。

\textbf{去中心化流动性池}:为支持平台的团购运营并确保有竞争力的定价,项目采用了一个去中心化借贷池,参与者可存入 USDC 以赚取利息(5--7\% APY)加 CPT 奖励(2--3\% APY),同时为平台提供运营资金。这种模式以更透明、可审计且去中心化的方式取代了传统储备金。

算力广场平台并不直接拥有其平台上列出的算力资源。相反,为确保持续供应和有竞争力的定价,平台借鉴拼多多的商业模式(算力拼多多),利用社区提供的流动性,通过 ``团购'' 模式锁定算力资源。

\textbf{团购说明}:虽然团购是我们策略的一部分,但它是一个 \textbf{补充收入流}(占总收入的 5--10\%),而非主要业务模式。我们的核心价值来自 SaaS 订阅和智能云管理工具。随着平台规模扩大,我们会推行团购折扣,但我们的核心价值主张并不依赖于从云提供商处获得大量批量折扣。

这种团购方式减少了消费者与当前控制所有算力资源的行业巨头之间的不平衡。这使平台能够批量锁定算力资源并提供给用户,创建一个动态且繁荣的市场。拼多多样式业务的收益通过质押奖励机制分配给 CPT 质押者,平台收入的 30\%(为提高可持续性已从 40\% 下调)分配给质押池,35\% 用于运营和增长,20\% 用于回购销毁,10\% 用于团队,5\% 用于应急储备。

该项目的核心团队在与项目相关的领域拥有丰富经验,包括分布式高性能计算、公共云服务、异构计算、人工智能与大数据应用、分布式系统软件开发、DeFi 投资、商业犯罪预防,以及算力资源的商务与营销。

总体而言,Web3 算力广场网络项目旨在为算力资源提供一个去中心化的市场,让用户和服务提供商都能受益,同时通过 CPT 代币所有权和质押奖励让所有参与者分享平台的成功。凭借强大的核心团队和受淘宝、拼多多等成功平台启发的创新商业模式,该项目为计算行业的所有参与者创建了一个繁荣、合规且可持续的生态系统。
