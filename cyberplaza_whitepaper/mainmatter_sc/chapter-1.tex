%% TEST TRANSLATION - AutoTranslated (placeholder)

\chapter{Executive Summary}

Computing is playing an increasingly crucial role in modern life, and this trend is expected to continue in the foreseeable future. The Web3 CyberPlaza Network Project aims to enable both individuals and institutions to benefit from this trend in an open and inclusive manner.

The Project introduces the CyberPlaza Platform, a decentralized marketplace that can be described as a ``Taobao for Computing Resources'' (算力淘寶平台). This platform matches the needs of users and service providers (SPs), covering the areas of High Performance Computing, Intelligent Computing, and Cloud Computing. Users gain access to a diverse range of computing power, storage, software applications, data and computing services in one place, meeting their specific needs cost-effectively. SPs, on the other hand, gain a global sales channel without limitations. Both SPs and users share in the success of the platform through ownership of CyberPlaza Tokens (CPTs\footnote{CPT is a conserved quantity in physics; all physical laws must not violate the conservation of CPT, similar to the conservation of energy. CPT (Carriage Paid To) is also an international trade term that means the seller will pay for the delivery of goods to the consumers. The name of our token carries both metaphors.}), representing governing shares of the ``Taobao'' Platform.

\textbf{Payment and Settlement}: Transactions on the platform are settled using USDC, a widely-adopted and regulated stablecoin, ensuring regulatory compliance and user familiarity. This approach eliminates the complexities and regulatory risks associated with proprietary stablecoins while maintaining transparent, dollar-denominated pricing.

\textbf{Revenue Model}: The platform generates revenue through multiple streams: SaaS Subscriptions (40--50\%), which include monthly/annual subscriptions for platform access; Transaction Fees (25--30\%), comprising 2--5\% fee on computing resource purchases; API \& Data Services (15--20\%), offering premium API access and analytics; and Group-Buying (5--10\%), generating margin from bulk purchasing as supplementary revenue. These revenues are distributed to CPT token holders through a transparent staking-rewards mechanism, enabling participants to earn sustainable yields (target 6--10\% APY) by contributing to platform liquidity and governance.

\textbf{Decentralized Liquidity Pool}: To support the platform's group-buying operations and ensure competitive pricing, the project implements a decentralized lending pool where participants can deposit USDC to earn interest (5--7\% APY) plus CPT incentives (2--3\% APY), while providing the platform with operational capital. This model replaces traditional reserve funds with a more transparent, auditable, and decentralized approach.

The CyberPlaza Platform does not directly own the computing resources listed on it. Instead, to ensure a continuous supply and competitive pricing, the platform leverages community-provided liquidity to secure computing resources through a ``group buying'' model (團購), inspired by the business model of Pinduoduo (算力拼多多). 

\textbf{Note on Group-Buying}: While group-buying is part of our strategy, it is a \textbf{supplementary revenue stream} (5--10\% of total revenue) rather than the primary business model. Our main value comes from SaaS subscriptions and intelligent cloud management tools. Group-buying discounts will be pursued as the platform scales, but we do not depend on obtaining large bulk discounts from cloud providers for our core value proposition.

The group approach reduces the imbalance between a consumer and the providers, i.e., the industrial giants currently controlling all computing resources. This allows the platform to secure computing resources in bulk and offer them to users, creating a dynamic and thriving marketplace. The proceeds from the Pinduoduo business are distributed to CPT stakers through the staking rewards mechanism, with 30\% of platform revenue allocated to the staking pool (reduced from 40\% for better sustainability), 35\% for operations and growth, 20\% for buyback \& burn, 10\% for team, and 5\% for emergency reserves.

The Core Team of the Project possesses extensive experience across areas relevant to the project, including distributed high-performance computing, public cloud services, heterogeneous computing, AI and Big Data applications, distributed system software development, DeFi investment, commercial crime prevention, and business and marketing of computing resources.

Overall, the CyberPlaza Network Web 3 Project aims to provide a decentralized marketplace for computing resources, benefiting both Users and Service Providers, while enabling all participants to share in platform success through CPT token ownership and staking rewards. With a strong Core Team and an innovative business model inspired by successful platforms like Taobao and Pinduoduo, the project creates a thriving, compliant, and sustainable ecosystem for all participants in the computing industry.
