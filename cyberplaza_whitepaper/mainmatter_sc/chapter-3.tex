\chapter{运行概述}
\subsection{我们旨在解决的挑战}

\begin{enumerate}
\item \textbf{中心化控制}:在现代社会,计算——尤其是云计算、高性能计算和人工智能(AI)领域——的重要性毋庸置疑。然而,这些关键资源主要由大型企业控制,限制了大多数用户的收益。我们认为解决方案在于去中心化市场,该市场能使计算资源的访问民主化,营造更开放、更具包容性的环境。在这样的系统中,用户不仅是消费者,也是贡献者,他们能影响计算发展的轨迹,并在计算的未来中拥有利益。

\item \textbf{低效性}:当前的计算资源分配模式常导致不平衡,造成资源利用率不足或饱和度过高。我们的项目旨在创建一个平台,高效匹配算力需求与可用资源,从而优化利用率并减少浪费。

\item \textbf{高成本}:目前,大多数用户面临不必要的高昂计算成本。我们的愿景是建立一个市场平台,以有竞争力的价格直接提供各类算力、存储解决方案、软件应用、数据及服务。这不仅能降低整体成本,还能扩大用户群体。

\item \textbf{缺乏透明度}:现有的计算资源分配系统在定价、可用性和服务质量方面缺乏透明度。我们旨在构建一个开放、公正的平台,让用户能根据资源、提供商和定价的可靠信息做出明智决策。

\item \textbf{缺乏用户赋权}:对我们大多数人来说,执行需要计算的想法是一个繁琐的过程,通常需要依赖第三方服务。例如,人们不得不依赖政府机构模拟得出的电视天气预报,或为了创建自己的数字孪生而将个人数据委托给中心化实体。我们的项目旨在打造一个去中心化市场,提供所有必要的计算资源,让用户能在保持完全控制的同时执行任何所需的计算。
\end{enumerate}

对于现代社会这一重要发展方向,我们需要解决构建去中心化综合生态系统的挑战,以实现计算资源更易获取、更高效的分配和利用。

\subsection{我们的解决方案概述}

\begin{enumerate}
\item 我们正在推出一个作为开放民主组织运营的平台。该平台类似于计算资源市场,让人联想到淘宝等平台(即“算力淘宝平台”)。该平台的所有权由CyberPlaza Token(CPT)的所有持有者分散持有,他们是我们平台的“股东”。

\item \textbf{支付系统}:我们的平台所有交易均使用USDC,确保合规性、价格透明度和熟悉的用户体验。这消除了与专有稳定币相关的风险,并与全球监管框架保持一致。

\item 在平台上,服务提供商(SP)列出其计算资源——包括算力、存储、软件应用、数据和服务——供用户根据需求选择。作为服务的回报,SP直接收到USDC支付,并根据交易量获得CPT代币激励。

\item 平台本身不拥有列出的计算资源。不过,它可以通过“团购”采购计算资源转售给用户。该模式类似于拼多多的商业模式,使用去中心化流动性池,社区成员可存入USDC以获得收益,同时支持平台运营。

\item 我们的平台是开放且包容的。任何人担任四种角色(平台“股东”、流动性提供者、SP和用户)中的任何一种或全部都没有限制。这种灵活性使参与者能以最适合其需求和能力的方式参与平台。
\end{enumerate}
