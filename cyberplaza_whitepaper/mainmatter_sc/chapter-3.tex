\chapter{操作概述}
\subsection{我们旨在解决的挑战}

\begin{enumerate}
\item \textbf{中心化控制}:计算的重要性日益凸显,尤其是在云计算、高性能计算和人工智能(AI)领域,这在我们现代社会是不可否认的。然而,这些关键资源主要由大型企业控制,限制了大多数用户的优势。我们认为,解决方案在于去中心化市场,该市场能够民主化地访问计算资源,营造更加开放和包容的环境。在这样的系统中,用户不仅是消费者,也是贡献者,他们可以影响计算发展的轨迹,并在计算的未来中拥有利益。

\item \textbf{低效率}:当前的计算资源分配模型通常导致失衡,造成资源利用率不足或过饱和。我们的项目旨在创建一个平台,将计算能力的需求与可用资源高效匹配,从而优化利用率并减少浪费。

\item \textbf{高成本}:目前,大多数用户面临不必要的高额计算成本。我们的愿景是建立一个市场平台,以具有竞争力的价格直接提供广泛的计算能力、存储解决方案、软件应用、数据和服务。这不仅降低了总体成本,还扩大了用户基础。

\item \textbf{缺乏透明度}:现有的计算资源分配系统在定价、可用性和服务质量方面缺乏透明度。我们旨在构建一个开放公正的平台,使用户能够根据有关资源、提供商和定价的可靠信息做出知情决策。

\item \textbf{缺乏用户赋权}:对于我们大多数人来说,执行需要计算的想法可能是一个繁琐的过程,通常需要依赖第三方服务。例如,人们不得不依赖基于政府机构进行的模拟的电视天气预报,或者为了为自己创建数字孪生,不得不将个人数据委托给中心化实体。我们的项目旨在打造一个去中心化市场,提供所有必要的计算资源,使用户能够在保持完全控制的同时执行任何他们想要的计算。
\end{enumerate}

对于现代社会这一重要的发展方向,我们需要解决建立去中心化综合生态系统的挑战,以实现计算资源的更便捷、高效分配和利用。

\subsection{我们的解决方案概述}

\begin{enumerate}
\item 我们正在推出一个作为开放民主组织运营的平台。该平台类似于计算资源的市场,让人想起淘宝等平台(即“算力淘宝平台”)。此设置的所有权分配给所有CyberPlaza Token(CPT)持有者,他们是我们平台的“股东”。

\item \textbf{支付系统}:我们的平台使用USDC进行所有交易,确保监管合规性、价格透明度和熟悉的用户体验。这消除了与专有稳定币相关的风险,并与全球监管框架保持一致。

\item 在平台上,服务提供商(SP)列出其计算资源——包括计算能力、存储、软件应用、数据和服务——供用户根据需要选择。作为服务的交换,SP直接收到USDC付款,以及基于其交易量的CPT代币激励。

\item 平台本身不拥有列出的计算资源。但是,它可以通过“团购”采购计算资源,再转售给用户。该模式类似于拼多多的商业模式,使用去中心化流动性池,社区成员可以在其中存入USDC以获得回报,同时支持平台运营。

\item 我们的平台是开放且包容的。任何人担任四个角色中的任何一个或全部都没有限制:平台“股东”、流动性提供者、SP和用户。这种灵活性使参与者能够以最适合其需求和能力的方式与平台互动。
\end{enumerate}
