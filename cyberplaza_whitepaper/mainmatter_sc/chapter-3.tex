\chapter{运营概述}
\subsection{我们旨在解决的挑战}

\begin{enumerate}
\item \textbf{集中式控制}:在现代社会,计算的重要性日益提升,尤其是在云计算、高性能计算和人工智能(AI)领域,这一点毋庸置疑。然而,这些关键资源主要由大型企业控制,将优势局限于大多数用户之外。我们认为,解决方案在于一个去中心化的市场,该市场使计算资源的访问民主化,营造一个更开放、更具包容性的环境。在这样的系统中,用户不仅是消费者,还是能够影响计算发展轨迹并在计算未来拥有权益的贡献者。

\item \textbf{低效问题}:当前的计算资源分配模式往往导致失衡,造成资源利用率不足或过度饱和。我们的项目旨在创建一个平台,将算力需求与可用资源高效匹配,从而优化利用率并减少浪费。

\item \textbf{高昂成本}:目前,大多数用户面临不必要的高昂计算成本。我们的愿景是建立一个市场平台,以有竞争力的价格提供对广泛算力、存储方案、软件应用、数据和服务的直接访问。这不仅降低了整体成本,还扩大了用户群体。

\item \textbf{透明度不足}:现有的计算资源分配系统在定价、可用性和服务质量方面缺乏透明度。我们旨在构建一个开放公正的平台,让用户能够基于关于资源、提供商和定价的可靠信息做出明智决策。

\item \textbf{用户赋能不足}:对我们大多数人来说,执行需要计算的想法可能是一个繁琐的过程,往往需要依赖第三方服务。例如,人们必须依赖政府机构运行模拟得出的电视天气预报,或者必须将个人数据委托给集中式实体才能为自己创建数字孪生。我们的项目旨在打造一个去中心化市场,提供所有必要的计算资源,让用户能够在保持完全控制权的同时执行任何所需的计算。
\end{enumerate}

对于现代社会这一重要的发展方向,我们需要解决建立去中心化综合生态系统的挑战,以实现计算资源更易获取、更高效的分配和利用。

\subsection{我们的解决方案概述}

\begin{enumerate}
\item 我们正在推出一个作为开放民主组织运作的平台。该平台类似于计算资源的市场,让人联想到淘宝等平台(即“算力淘宝平台”)。该设置的所有权由CyberPlaza Token(CPT)的所有持有者分散持有,他们是我们平台的“股东”。

\item \textbf{支付系统}:我们的平台将USDC用于所有交易,确保合规性、价格透明度和熟悉的用户体验。这消除了与专有稳定币相关的风险,并与全球监管框架保持一致。

\item 在平台上,服务提供商(SPs)列出他们的计算资源——包括算力、存储、软件应用、数据和服务——供用户根据需求选择。作为服务的回报,SPs直接收到USDC付款,并根据交易量获得CPT代币激励。

\item 平台本身并不拥有列出的计算资源。然而,它可以通过“团购”采购计算资源转售给用户。该模式类似于拼多多(Pinduoduo)的商业策略,使用去中心化流动性池,社区成员可以存入USDC以获得回报,同时支持平台运营。

\item 我们的平台是开放且包容的。任何人担任四个角色中的任意一个或全部都没有限制:平台“股东”、流动性提供商、SP和用户。这种灵活性使参与者能够以最适合自己需求和能力的方式参与平台。
\end{enumerate}
