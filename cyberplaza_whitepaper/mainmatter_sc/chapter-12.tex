\chapter{FAQ}

\section{常见问题}

\begin{enumerate}
\item \textbf{作为算力淘宝平台的用户,我能获得什么?}

\textbf{回答}:您可以从平台上列出的众多提供商处选择供您使用的计算资源,包括CPU、GPU、FPGA算力、存储、应用软件和服务(例如,在特定硬件平台上优化您的软件,或把您的云应用从一家云厂商迁移到另一家)。您可以基于充分的信息选择服务,因为计算资源的性能由平台评估并公示,SP的SLA由平台保障,并且以折扣价(如淘宝/京东)提供AWS、Azure、GCP以及众多计算中心和数据中心的资源使用权限。此外,通过使用平台,您可凭借获得的CPT共享平台所有权,进而共享平台的部分收益(一个您部分拥有的淘宝)。

\item \textbf{用户群体是谁?普通公众可能不是计算资源的主要用户。另一方面,许多机构客户可能无法参与代币经济学。}

\textbf{回答}:目前,全球普通公众在公有云上的计算使用价值已超过400亿美元,这确实仅占机构客户的一小部分。对于无法参与代币经济学的机构客户,他们可以通过平台的渠道合作伙伴以常规B2B方式购买计算资源使用权限(见白皮书的合作部分),并以法定货币支付。

\item \textbf{一些机构服务提供商,例如AWS或中国的某超级计算中心,可能无法在提供服务时接受代币。那么平台用户如何使用这些提供商的资源?}

\textbf{回答}:平台使用「储备基金」以法定货币购买这些服务提供商的服务。通过团购(拼多多),平台可以提供折扣服务。

\item \textbf{为什么像AWS这样的云厂商会屈服于团购的压力?}

\textbf{回答}:我们的平台将成为AWS的宝贵销售渠道,为其提供触达Web 3和DeFi社区的机会。此外,由于平台上众多SP之间的竞争压力,以及来自「储备池」的一定金额预付款的足够大的团购交易,为所有云厂商和计算资源供应商提供折扣是完全合理的。

\item \textbf{假设运营完美,算力淘宝平台的业务规模会有多大?}

\textbf{回答}:根据Gartner的数据,AWS的年营收在2019年为350亿美元、2020年为450亿美元、2021年为620亿美元、2022年为814亿美元(其中约93%由机构客户使用,7%由个人用户使用)。如果我们将全球商业计算业务总价值视为AWS的7倍(根据Allied Market Research的数据,2022年全球商业计算业务总价值为5520亿美元,即AWS的7倍),那么全球总营收将在2024年超过1万亿美元。如果算力淘宝平台能占据0.1%的市场份额,其年营收将超过10亿美元,并将迅速增长。

\item \textbf{作为流动性提供者或CPT持有者参与平台,我能获得什么好处?}

\textbf{作为流动性提供者(存入USDC)}:参与者可从平台运营收益中获得以USDC计的5--7\% APY,还可获得以CPT代币计的额外2--3\% APY(带锁仓条件),预期总收益为8--12\% APY,保持USDC流动性(可在提前通知期后提取),并在支持平台成长的同时获得可持续收益。

\textbf{作为CPT持有者/质押者}:参与者可质押CPT以获得8--12\% APY(通过4年锁仓和增益可高达15--20\%),获得平台收益40\%的USDC收益分配,受益于通缩型回购销毁机制(占营收的20\%),获得治理权(对平台方向进行投票),质押期间可享受平台服务5--15\%的折扣,使用高级功能并获得优先支持,还可提前参与新产品上线。

\textbf{为何这些收益可持续}:与算法稳定币或庞氏骗局不同,我们的收益来自真实的交易手续费(占市场活动的2--5\%)、团购利差(批量采购带来的10--20\%)、增值服务(认证、订阅、API),以及透明、可审计的营收流。

\item \textbf{为什么资金(无论是铸币者还是投资者)愿意加入算力淘宝平台,而非其他Web 3项目?}

\textbf{回答}:详情请见「与其他『资产代币化项目』的竞争分析」以及「与其他Web 3计算资源项目的竞争分析」页面。简言之:与其他资产相比,算力的价值增长更为迅速;且我们的团队具备建立算力淘宝平台的特别资质。

\item \textbf{一些潜在用户或服务提供商可能无法参与代币经济学。他们如何参与平台?}

\textbf{回答}:这些客户在平台上找到合适的产品后,可以通过平台的合作伙伴购买(见第10节列出的代理商)。服务提供商也可以通过合作伙伴在平台上上架他们的产品。合作伙伴与用户/提供商之间的交易可通过法定货币进行,无需涉及代币。

\item \textbf{一些消费者认为淘宝和拼多多上的产品质量较差。平台如何防范这一点?}

\textbf{回答}:所有月挂牌价超过10,000 USDC的产品必须通过平台认证。正如上文第4节所述,平台要求服务提供商根据标准性能测试(包括高性能Linpack、高性能共轭梯度、STREAM可持续带宽、HPC挑战、MLPerf、ResNet-50图像分类、BERT语言处理、CUDA基准测试套件、SPECviewperf图形性能、DeepBench等)列出其服务的性能。平台将定期验证服务提供商声称的性能,并将性能指标与服务价格一同列出,供用户选择。

\item \textbf{为什么像AWS或华为云这样的公司希望在平台上销售其服务?}

\textbf{回答}:目前,云计算公司会为销售其服务的分销商提供折扣。这些分销商雇佣销售团队来销售服务。从某种意义上说,平台充当了这些厂商的分销商,不同的是,借助Web 3架构,厂商可以触达Web 3和DeFi社区。

\end{enumerate}
