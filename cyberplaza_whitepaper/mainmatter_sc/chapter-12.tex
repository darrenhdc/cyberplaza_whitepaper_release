\chapter{FAQ}

\section{常见问题}

\begin{enumerate}
\item \textbf{作为算力淘宝平台的用户,我能获得什么?}

\textbf{Answer}: 你可以从平台上列出的众多提供商处选择供自身使用的计算资源,包括CPU、GPU、FPGA算力、存储、应用软件及服务(例如,在特定硬件平台上优化你的软件,或将你的云应用从一家云供应商迁移至另一家)。你可做出明智的服务选择,因为计算资源的性能由平台评估并公示,服务提供商(SP)的服务等级协议(SLA)由平台保障,且能以折扣价(如淘宝/京东)使用AWS、Azure、GCP等以及众多计算中心和数据中心的资源。此外,通过使用平台,你将与获得的CPT通证一同分享平台的所有权,以及平台的部分利润(一个由你部分拥有的淘宝)。

\item \textbf{用户群体是谁?普通大众或许并非计算资源的主要用户。另一方面,许多机构客户可能无法参与通证经济。}

\textbf{Answer}: 目前,全球普通大众在公有云上的计算使用量价值超过400亿美元,这确实仅占机构客户使用量的一小部分。对于无法参与通证经济的机构客户,他们可通过平台的渠道合作伙伴(见白皮书的合作伙伴章节)以常规B2B方式购买计算服务,支付法定货币。

\item \textbf{部分机构服务提供商(例如AWS或中国的超级计算中心)可能无法在提供服务时接受通证。那么他们的资源如何通过平台供我们的用户使用?}

\textbf{Answer}: 平台将使用“储备基金”以法定货币向这些服务提供商购买服务。通过团购(拼多多),平台可提供折扣服务。

\item \textbf{为什么像AWS这样的云供应商会接受团购带来的价格压力?}

\textbf{Answer}: 我们的平台将成为AWS的宝贵销售渠道,使其能够触达Web 3与DeFi社区。此外,在平台上众多服务提供商(SP)的竞争压力下,再加上足够规模的团购交易并从“储备池”获得一定金额的预付款,折扣对所有云和计算资源供应商而言都是完全合理的。

\item \textbf{假设运营完美,算力淘宝平台的业务规模可能达到多少?}

\textbf{Answer}: AWS的年收入在2019年为350亿美元,2020年为450亿美元,2021年为620亿美元,2022年为814亿美元(根据Gartner的数据,其中约93%由机构客户消费,7%由个人用户消费)。若将全球商业计算业务的总价值视为AWS的7倍(2022年全球商业计算业务总价值为5520亿美元,即AWS的7倍——Allied Market Research),那么到2024年,全球总收入将超过1万亿美元。若算力淘宝平台能占据总市场的0.1%,则年业务规模将超过10亿美元,并将快速增长。

\item \textbf{作为流动性提供者或CPT持有者,我能获得哪些收益?}

\textbf{作为流动性提供者(存入USDC)}: 参与者可从平台运营利润中获得5--7\%的USDC年化收益率(APY),额外获得2--3\%的CPT通证年化收益率(设有锁仓期),预期总年化收益率为8--12\%,可保持USDC流动性(可在提前通知后提取),并在获得可持续收益的同时支持平台增长。

\textbf{作为CPT持有者/质押者}: 参与者可质押CPT以获得8--12\%的年化收益率(锁定4年并使用增强功能可高达15--20\%),从平台利润的40\%中获得USDC收入分配,受益于通缩性回购与销毁机制(收入的20\%),获得治理权(对平台方向进行投票),质押时可享受平台服务5--15\%的折扣,访问高级功能并获得优先支持,以及提前参与新产品发布。

\textbf{为什么这些收益具有可持续性}: 与算法稳定币或庞氏骗局不同,我们的收益来自真实的交易费用(市场活动的2--5\%)、团购利润(批量采购的10--20\%)、增值服务(认证、订阅、API)以及透明、可审计的收入流。

\item \textbf{为什么资金(无论是铸币者还是投资者)愿意加入算力淘宝平台,而非其他Web 3项目?}

\textbf{Answer}: 详情请见“与其他‘资产通证化项目’的竞争分析”和“与其他Web 3计算资源项目的竞争分析”页面。简而言之:与其他资产相比,算力的价值增长更为迅速;且我们的团队在搭建算力淘宝平台方面具备得天独厚的资质。

\item \textbf{部分潜在用户或服务提供商可能无法参与通证经济。他们如何参与平台?}

\textbf{Answer}: 这些客户在平台上找到合适产品后,可通过平台的合作伙伴(见第10节列出的代理)进行购买。服务提供商也可通过合作伙伴在平台上上架其产品。合作伙伴与用户/提供商之间的交易可通过法定货币进行,无需涉及通证。

\item \textbf{部分消费者认为淘宝和拼多多上的产品质量较低。平台如何防范这种情况?}

\textbf{Answer}: 所有月度挂牌价格超过10,000 USDC的产品均需通过平台认证。如上文第4节所述,平台要求服务提供商根据标准性能测试(包括高性能Linpack、高性能共轭梯度、STREAM可持续带宽、HPC挑战、MLPerf、ResNet-50图像分类、BERT语言处理、CUDA基准测试套件、SPECviewperf图形性能、DeepBench等)列出其服务的性能。平台将定期验证服务提供商宣称的性能,并将性能指标与服务价格一同列出,供用户选择。

\item \textbf{为什么像AWS或华为云这样的公司愿意在平台上销售其服务?}

\textbf{Answer}: 云计算公司目前会向分销商提供折扣以销售其服务。分销商会雇佣销售团队来推广这些服务。从某种意义上说,平台是这些供应商的分销商,只是借助Web 3的设置,供应商可获得触达Web 3与DeFi社区的机会。

\end{enumerate}
