\chapter{核心团队、基金会与顾问}

\subsection*{核心团队}

核心团队承担平台技术基础设施的搭建、维护与推进责任。开发与维护范围涵盖CHESS算力分发软件、上架算力资源的质量评估系统、区块链平台架构、智能合约实现及配套技术基础设施。

团队专长横跨分布式高性能计算、公有云服务、异构计算架构、去中心化金融投资策略、人工智能与大数据应用、金融科技解决方案、分布式系统软件开发及算力资源商业化。

\subsubsection*{核心团队成员}

\textbf{Dr. Wai-Mo Suen} 拥有25年高性能计算技术与现代计算业务运营经验。自2000年起担任ClusterTech创始人兼首席执行官,曾交付高性能计算(HPC)、云计算、人工智能(AI)及大数据解决方案,并多次荣获高性能计算业务与金融科技创新领域的奖项。

\textbf{Dr. Harry Yu} 专注于FPGA技术,是CTAccel创始人兼首席执行官,该公司于2018年获得英特尔资本投资。他具备投资敏锐度,拥有2年去中心化金融(DeFi)经验,实现了18\%的年化收益率与4.6的夏普比率。

\textbf{Mr. Eric Leung} 拥有15年高性能计算(HPC)系统管理经验,且曾在一家公有云服务提供商担任运营负责人10年。

\textbf{Mr. GY Han} 拥有15年高性能计算(HPC)系统管理软件开发的专业经验。

\textbf{Mr. Terence Leung} 拥有近30年执法专业经验,专注于反洗钱与欺诈调查,且具备丰富的合规与风险管理经验。他曾担任量化投资与去中心化金融(DeFi)投资基金的顾问与财务控制器5年。

\textbf{Mr. Pong Po Lam Paul (龐寶林)} 创立了Pegasus Fund Managers Ltd.,并联合创立了香港财务策划师学会、亚洲金融科技师学会以及香港金融分析师及专业评论员协会。他的公职经历包括在金融发展局、强积金咨询委员会、香港会计师公会及证券及期货事务监察委员会担任职务。他持有认证财务策划师(CFPCM)及认证金融科技师(CFT)资质。

\textbf{Mr. XXX} 具备丰富的IT业务与营销运营经验。

\subsection*{基金会、投资者与服务提供商(SP)}

\subsubsection*{基金会}

基金会负责项目的开发、推广与维护,以确保长期可持续性。其职责涵盖通证分配与管理、社区建设与参与、营销与推广举措、项目治理监督及技术与经济生态系统支持。

基金会成员包括核心团队成员与顾问,他们具备高性能计算与云计算资源供应、人工智能与大数据基础设施、金融投资策略、金融产品开发及商业法规与反洗钱要求合规方面的专长。

\subsubsection*{投资者}

待定。

\subsubsection*{服务提供商(SP)}

平台上线时,已有5家认证服务提供商(CSP)完成注册,贡献的算力资源包括xx个CPU核心(相当于???个X86核心),提供??? FP64 TFLOPS,yy个GPU(32位张量运算性能相当于xxx TOPS),zz个FPGA(FP32运算性能相当于??? TFLOPS),以及??? PB的存储容量。

资源增长预测目标为:上线后1年内,CPU扩展10倍,GPU扩容20倍,FPGA增长5倍,存储容量增加10倍。
