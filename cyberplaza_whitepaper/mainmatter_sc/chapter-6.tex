\chapter{技术与架构}

\section{项目治理基础设施}

\subsection{概述}

CyberPlaza网络由CyberPlaza基金会和CyberPlaza社区组成。

CyberPlaza基金会是一家非营利性去中心化组织,致力于CyberPlaza平台的成功运营、计算技术与应用的推广和发展,以及支持平台上的去中心化社区建设与发展。该基金会由CyberPlaza社区的CPT持有者拥有和控制。基金会由网络核心成员(见白皮书第8节)以及基金会根据需要不时任命的顾问运营。基金会将设立CyberPlaza实验室,负责开发和研究新的计算资源技术与应用,以推动平台所需的技术创新与进步。

CyberPlaza社区是网络的社区部分,由流动性提供者、用户和服务提供者(SP)组成,共同参与基金会治理、开发和推广。社区成员可通过参与治理、就业务方向和技术发展向基金会提出建议以及交流分享经验来推动平台的发展与成长。

CyberPlaza基金会与CyberPlaza社区之间的紧密联系对于实现网络的愿景和使命至关重要。

\subsection{智能合约模块}

我们将把CPT部署为符合ERC20标准的智能合约,运行在Arbitrum(以太坊的第2层)上,选择该网络是因其交易成本低且吞吐量高。平台还将根据生态系统扩展的需要与其他链桥接。

CPT代币合约包含以下关键功能:标准ERC20功能(转账、授权等)、支持veToken机制的质押和锁定功能、治理投票集成、奖励分配机制、应急暂停功能(由治理控制),以及用于未来增强的可升级代理模式。

\textbf{注意}:平台直接使用USDC进行支付,消除了对专有稳定币的需求及相关监管风险。

\begin{verbatim}
// SPDX-License-Identifier: MIT
pragma solidity ^0.8.0;

import "@openzeppelin/contracts/token/ERC20/ERC20.sol";

contract CPTToken is ERC20 {
  struct LockInfo {
     uint256 amount;
     uint256 lockTimestamp;
     uint256 unlockTimestamp;
  }

   mapping (address => LockInfo[]) public locks;

   constructor(uint256 initialSupply) ERC20("CPT Token", "CPT") {
     _mint(msg.sender, initialSupply);
   }

   function lock(uint256 _amount, uint256 _lockTime) public {
     require(_amount <= balanceOf(msg.sender), "Not enough CPT to lock");
     require(_lockTime > 0, "Lock time must be positive");

       uint256 lockUntil = block.timestamp + _lockTime;
    
       LockInfo memory newLock = LockInfo({
           amount: _amount,
           lockTimestamp: block.timestamp,
           unlockTimestamp: lockUntil
       });
    
       locks[msg.sender].push(newLock);
    
       _burn(msg.sender, _amount);

   }

  function unlock(uint256 lockIndex) public {
    require(lockIndex < locks[msg.sender].length, 
            "No lock found at this index");
    require(block.timestamp >= locks[msg.sender][lockIndex].unlockTimestamp,
            "CPT still locked");

        uint256 amountToUnlock = locks[msg.sender][lockIndex].amount;
        locks[msg.sender][lockIndex] = 
            locks[msg.sender][locks[msg.sender].length - 1];
        locks[msg.sender].pop();
    
        _mint(msg.sender, amountToUnlock);
    }

   function calculateLockedAmount(address user, uint256 lockDuration) 
       public view returns (uint256) {
     uint256 totalLockedAmount = 0;

        for (uint256 i = 0; i < locks[user].length; i++) {
           if (block.timestamp - locks[user][i].lockTimestamp > lockDuration) {
               totalLockedAmount += locks[user][i].amount;
           }
        }
    
        return totalLockedAmount;
    }

}
\end{verbatim}

\subsection{代币标准与小数处理}

CPT代币遵循18位小数的标准ERC20规范,而USDC则采用6位小数。平台所有转换操作均使用SafeMath库,以防止溢出和下溢错误。价格预言机集成了小数归一化逻辑,最低交易阈值可缓解粉尘攻击向量。对于分数金额,协议采用保守的四舍五入机制。

\subsection{投票锁定代币机制}

平台采用投票锁定(ve)代币模型,以协调长期利益相关者的激励。用户将CPT锁定1周至4年不等,获得不可转让的veCPT代币,该代币决定治理权重和奖励分配。

veCPT余额遵循以下关系:
\begin{equation}
\text{veCPT} = \text{CPT}_{\text{locked}} \times \min\left(\frac{t_{\text{lock}}}{t_{\text{max}}}, 1\right) \times 2.5
\end{equation}
其中$t_{\text{lock}}$代表所选锁定时长,$t_{\text{max}} = 4$年为最长锁定周期。2.5倍乘数为4年锁定承诺提供最大治理权重。

随着锁定周期接近到期,veCPT余额线性衰减:
\begin{equation}
\text{veCPT}(t) = \text{CPT}_{\text{locked}} \times \frac{t_{\text{remaining}}}{t_{\text{max}}} \times 2.5
\end{equation}
这种衰减机制通过锁定延长或代币重锁激励持续参与。

\subsubsection{奖励分配}

平台以USDC收取的收入将有30%分配至质押奖励池,分配周期为每周或每月。个人奖励根据veCPT持有量按比例计算:
\begin{equation}
\text{Reward}_{\text{user}} = \text{Revenue}_{\text{pool}} \times \frac{V_{\text{user}}}{V_{\text{total}}}
\end{equation}
其中$V_{\text{user}}$代表用户的veCPT余额,$V_{\text{total}}$代表veCPT总供应量。有效年化收益率(APY)根据质押参与度和平台表现动态变化:
\begin{equation}
\text{APY} = \frac{\text{Annual Revenue Pool}}{\text{Total CPT Staked Value}} \times \frac{\text{veCPT Multiplier}}{\text{Average Multiplier}}
\end{equation}

\subsubsection{安全与优化}

智能合约遵循OpenZeppelin标准接受第三方审计,参数修改由多签治理控制。所有奖励分配均在链上追踪,确保透明度。安全功能包括外部调用的重入保护、基于角色的访问控制、应急暂停功能以及可升级代理模式。关键参数变更需执行48小时时间锁。

Gas优化采用基于默克尔树的批量申领、veCPT余额的延迟评估、紧凑存储变量以及事件驱动的链下索引。这些技术在保持安全保障的同时降低了交易成本。

\subsection{预言机集成}

平台集成Chainlink去中心化预言机用于价格发现和数据聚合。CPT/USD价格馈送聚合来自Uniswap V3时间加权平均价格和中心化交易所报价的数据。USDC/USD验证采用Chainlink的已验证馈送,偏差阈值为0.5%。预言机每5分钟或价格变动1%时更新,并配有手动后备机制以确保冗余。

对于计算资源定价,链下聚合器监控主要云服务提供商(AWS、Azure、GCP、阿里云)的公开API,计算计算、存储和带宽的实时市场费率。聚合定价每天或偏差超过5%时发布至链上预言机合约。

预言机安全依赖至少7个独立Chainlink节点的共识。系统拒绝偏离中位数超过10%或超过1小时的数据的价格更新。熔断机制在检测到操纵尝试时自动暂停交易。

\subsection{治理架构}

平台关键操作需通过Gnosis Safe实现的多签审批。超过10万USDC的国库资金移动需9签5过,智能合约升级需9签7过并执行48小时时间锁。参数调整需9签4过共识,应急安全响应采用5签3过的快速响应配置。

治理流程遵循结构化时间线:持有10万及以上veCPT的用户可提交提案,随后进入7天社区讨论期和5天链上投票期,其中1 veCPT相当于1票。获批提案在延迟48小时后执行。多签委员会对恶意提案保留否决权,该权力需接受季度审查。

\subsection{跨链基础设施}

平台采用LayerZero跨链协议实现多链部署。Arbitrum因其交易成本低、吞吐量高而作为主链。以太坊主网支持面向需要第1层安全的机构用户,而Polygon集成则为成本敏感型用户提供更低的交易成本。未来扩展包括Optimism(2024年第三季度)和Base(2024年第四季度),以实现更广泛的生态系统集成。

跨链桥安全包含多重保障:流动性上限将每条链的桥接供应量限制在10%,速率限制将吞吐量约束在每小时100万CPT,应急暂停机制应对异常情况,5%的保险基金为桥接价值提供抵押,以防范潜在漏洞。

\subsection{钱包基础设施}

作为标准ERC20代币,CPT支持所有兼容钱包,包括浏览器扩展(MetaMask、Rabby、Rainbow)、移动应用(Trust Wallet、Coinbase Wallet、imToken)、硬件设备(Ledger、Trezor)和智能合约钱包(Argent、Gnosis Safe)。未来计划与Fireblocks和Copper.co集成实现机构托管。

Web门户采用WalletConnect和Web3Modal协议实现标准化钱包连接。连接授权后,平台查询用户余额、质押头寸和veCPT持有量,以启用完整功能访问。交易签名遵循EIP-712类型化结构化数据标准,呈现人类可读的消息,提升抵御钓鱼攻击向量的安全性。

\section{市场计算基础设施}

\subsection{系统架构}

平台采用三层架构。Web3界面层通过React.js和ethers.js框架管理钱包认证(WalletConnect)、USDC支付处理和CPT奖励分配。编排层协调CHESS集群管理系统、作业调度、资源分配、性能监控和SP认证流程。计算资源层聚合CSP集群、公共云API(AWS、Azure、GCP、阿里云)、私有HPC中心以及未来的边缘计算节点。

交易流程包括:提交作业并存入USDC、智能合约托管至完成、CHESS介导的资源匹配、在分配的SP基础设施上执行、实时SLA合规监控、结果交付与自动支付结算,以及按比例分配CPT奖励(用户1-3%,SP 2-5%)。

\subsection{HPC基础设施组件}

高性能计算(HPC)基础设施包含专用节点类型:计算节点使用多核处理器和大容量内存执行数值模拟和数据分析;可视化节点使用GPU加速渲染大型数据集;I/O节点管理存储与计算网络结构之间的数据传输;存储节点提供高并发文件系统;管理节点协调资源分配和作业调度。

网络结构采用高速互连技术(InfiniBand、Ethernet)实现节点间通信。并行文件系统支持大型数据集和中间结果的并发多节点读写操作。

\subsubsection{软件栈}

监控和管理工具为管理员提供系统各组件的实时健康和性能数据,包括CPU利用率、内存占用和网络流量模式。集群管理软件协调整体系统运营,为地理分布的计算节点提供配置、监控和维护能力。

资源分配采用专用调度器管理CPU时间、内存和其他计算资源,以最大化系统利用效率。用户界面包括命令行工具和Web门户,用于作业提交和管理。HPC应用中心聚合特定领域的应用和模板,允许用户直接下载和部署计算工具。集成计费系统针对不同资源类型和计费周期实施透明的定价策略,促进合理的资源利用和准确的成本核算。

\subsection{支付与结算基础设施}

\subsubsection{托管机制}

提交作业将启动托管流程,用户授权向平台智能合约支付USDC。托管合约计算预估成本,包含资源类型(CPU/GPU/存储)、时长预测、预言机得出的市场定价,以及20%的潜在超支缓冲。USDC在授权后转入托管,针对唯一作业标识锁定。

作业完成后,实际资源消耗决定最终结算。服务提供者直接获得95-98%的USDC费用,平台保留2-5%的交易费。超额托管资金自动返还给用户,CPT奖励按比例分配给用户(1-3%)和SP(2-5%)。

\subsubsection{纠纷解决协议}

SLA违规会触发分级解决机制。5分钟内失败的作业符合自动全额退款条件。部分完成的作业将根据实际交付按比例退款。用户可在72小时窗口内提交纠纷并提供支持证据。价值超过1万USDC的案件将升级至平台治理仲裁,保险基金将覆盖最高10万USDC的已验证索赔。

\subsubsection{服务提供者认证}

服务提供者认证需经过多阶段验证流程。初始注册需要公司验证文件、基础设施规格、支付钱包地址和安全合规证书(SOC 2、ISO 27001)。技术验证采用行业标准基准,包括高性能Linpack(HPL)、高性能共轭梯度(HPCG)、STREAM内存带宽、AI工作负载的MLPerf,以及网络延迟评估。

安全审计验证AES-256加密、网络隔离和DDoS防护能力。

获批候选者进入30天试用期,期间将受到强化监控,且作业并发限制为10个。试用期顺利完成将授予认证服务提供者(CSP)身份,可访问机构客户并参与团购。CSP将出现在高级目录中,并带有已验证徽章。

持续合规要求月度可用时间达到99.5%、作业启动时间低于5分钟、性能在宣传基准的10%范围内。每季度重新认证以验证持续能力。关键漏洞的安全补丁必须在48小时内部署。违规将触发分级处罚:首次违规警告并要求7天内整改,第二次违规暂停30天,第三次违规吊销认证。

\subsection{技术栈}

平台前端开发采用React.js 18+结合TypeScript,Web3集成采用ethers.js v6和WalletConnect v2,界面一致性采用Material-UI。后端架构使用Node.js/Express.js或Python FastAPI提供API服务,PostgreSQL用于关系型持久化,Redis用于缓存,RabbitMQ/Kafka用于异步作业队列,The Graph用于区块链事件索引,Prometheus/Grafana用于可观测性。DevOps基础设施通过Docker将所有服务容器化,Kubernetes编排生产部署,GitHub Actions实现CI/CD,Cloudflare CDN分发内容,Nginx负载均衡流量。

入门级CSP需要100个以上CPU核心(Intel Xeon/AMD EPYC)、500 GB RAM、10 TB NVMe SSD或50 TB HDD、10 Gbps网络上行链路,以及可选的4个以上NVIDIA A100/H100 GPU。企业级CSP可扩展至10,000个以上CPU核心、50 TB以上总RAM、1 PB以上并行文件系统存储(Lustre/GPFS)、100 Gbps InfiniBand骨干网,以及100个以上高端GPU。

\subsection{平台用户功能}

CPT门户服务于三类主要用户:探索项目信息的访客、采购市场资源(公共云、HPC提供者、硬件、软件、存储)的用户,以及执行USDC存款和铸币操作的流动性提供者。

公共云消费者可在包括FQ、Amazon和华为云在内的供应商之间选择,价格以USDC计价,并提供促销活动。选择供应商后,用户将被重定向至原生门户(如AWS),标准操作在此进行,支付通过CPT平台托管路由。平台随后以法定货币与供应商结算。

HPC资源消费者可在价格点、硬件规格、性能指标和区域带宽方面比较供应商(CT集群、区域提供者、华为、AWS)。在存入足够USDC后,通过CHESS门户进行供应商选择和作业提交,资金将托管至完成,随后进行法定货币结算。存储采购遵循相同的工作流程。

软件选项包括用户提供的应用程序或平台列出的来自Ansys、HPC软件供应商和CHESS应用中心的解决方案。供应商入驻支持硬件、存储、软件和辅助计算产品。当硬件和软件均来自平台列表时,系统将验证两者的兼容性,确保执行兼容性。该架构可根据需求变化容纳未来的功能扩展。

\subsection{公共云集成}

市场聚合了来自AWS、Azure、Google Cloud和阿里云等主要公共云供应商的计算资源。价格以USDC计价显示,并带有活动促销和可用性状态。

\subsubsection{供应商集成模式}

平台采用三种集成方法。直接API集成利用经销商凭证通过供应商API(AWS EC2、Azure Resource Manager、GCP Compute Engine)进行实时配置,实现自动实例生命周期管理。优惠券代码系统通过预生成的代码防止超卖,以解决容量限制,提供基于价值的(100美元通用信用额)或基于资源的(1000 GPU小时、10 TB存储)格式。托管服务提供者(MSP)模式将CyberPlaza定位为拥有批量定价协议的MSP,管理供应商账户并提供合并计费。

实时价格比较显示计算、存储和网络成本以及总拥有成本计算。团购折扣突出显示相对于直接采购的潜在节省。

\subsubsection{作业提交工作流程}

HPC作业提交流程包括:通过筛选后的CSP列表(CPU类型、GPU可用性、区域、价格)选择资源;通过应用中心模板或带有指定要求(节点、核心、内存、运行时间、GPU)和I/O位置的自定义代码配置作业;包含USDC明细和预计CPT奖励的成本估算;将USDC转入托管并带有应急缓冲的支付授权;通过CHESS调度器分配执行并进行实时状态监控;完成结算,交付结果并自动分配支付、退还超额资金和发放CPT奖励。

高级功能包括支持100个以上作业和参数扫描的批量提交、定义顺序执行的工作流依赖、用于容错的checkpoint/重启、可在可抢占容量上获得50-70%折扣的竞价实例投标,以及用于动态资源调整的自动扩容。

服务提供者通过中央仪表板管理运营,该仪表板包含资源分配、作业监督、财务跟踪(USDC收入、CPT累计)和性能分析(客户满意度、利用指标)。

\subsection{多集群管理系统}

CHESS(集群高性能执行与调度系统)平台提供对地理分布的计算资源的统一管理。系统通过带有基于角色的访问控制的中央Web门户集成监控、调度和资源分配。

\subsubsection{核心能力}

平台通过Web界面和SSH协议支持全面的数据管理,可执行文件上传、下载、压缩和提取等操作。节点管理通过批量命令进行,控制电源状态、远程访问(VNC、shell),并支持异构硬件配置(CPU、GPU、FPGA)。资源配额对存储和计算分配执行管理策略,在阈值违规时自动生成警报。

高可用架构通过冗余管理节点和数据库复制消除单点故障。系统协调多个地理分布的集群,在子集群间统一传播用户角色。

\subsection{性能监控基础设施}

高性能和云计算系统聚合大量硬件资源,这些资源通过高速网络互连,形成低延迟、高容量的配置。有效的集群管理需要监控和管理工具提供资源配置、实时性能跟踪、带警报的故障检测以及使用状态可视化。

\subsubsection{CHESS监控能力}

CHESS监控系统通过聚合仪表板提供全面的集群监控,显示Ethernet和InfiniBand结构下的CPU和内存使用情况、负载状态、存储状态以及网络吞吐量。自定义时间间隔选择支持历史趋势分析和性能跟踪。仪表板显示提供可定制的大屏幕展示,为存储使用、作业调度和网络统计数据提供动态指标刷新。

多集群监控通过自适应屏幕布局和分辨率优化扩展到地理分布的安装。机架可视化呈现物理拓扑,并集成电源管理和VNC远程访问控制。单节点监控捕获细粒度的CPU、内存、存储、负载和网络指标,同时提供故障诊断和恢复建议。GPU监控跟踪设备特定的使用率、内存利用率、温度和带宽。作业监控分析实时执行状态和队列构成,并提供详细的CPU利用率、内存占用和节点负载统计数据。集群警报采用可配置的阈值,并通过电子邮件和系统通知路由。

性能指标以用户定义的间隔收集,捕获CPU、内存、磁盘和网络数据。物理拓扑可视化包括机架和节点布置,并带有基于阈值的故障警报。

\subsubsection*{调度器与资源管理}

高效的调度和资源管理在多集群系统中至关重要。CHESS提供灵活的调度策略,包括FIFO、抢占和回填策略。系统支持带有服务质量(QoS)配置的资源预留、涵盖串行、并行和GPU工作负载的高级作业提交,以及用于负载均衡优化的队列管理。

\subsubsection{作业提交与管理}

用户通过命令行界面、基于Web的GUI或常见工作流的应用模板提交作业。管理员配置资源配额、优先级和提交策略,以管理系统访问和利用。

% \subsection*{6.2.5 Pricing Module}

% This module will focus on calculating costs for resource usage and presenting pricing details to users. It will integrate with job scheduling and monitoring systems for real-time cost tracking.

% \subsubsection*{6.2.5 User Interfaces and Operational Portals}

\subsubsection{用户管理}

平台支持自注册和管理员配置的账户,并集成LDAP认证以实现集中管理。基于角色的访问控制实现了默认角色(管理员、部门管理员、用户),并通过灵活的权限分配管理系统访问和功能。

\subsubsection{通知与消息}

用户将收到关于计费和使用情况的自动警报以及管理公告。

\subsection{应用中心}

应用中心通过可浏览的库提供对预安装的HPC应用(Ansys、MATLAB、TensorFlow)的访问。用户通过带有交互式参数配置的图形模板提交作业。输出管理包括日志查看、错误分析、性能指标跟踪以及集成的可视化工具(AI应用的TensorBoard)。

\subsection{硬件性能评估}

硬件性能评估模块执行基准测试,测量CPU和GPU性能以及网络吞吐量和延迟。资源效率分析根据工作负载特性优化分配策略。故障恢复指标评估硬件在故障场景下的可靠性和恢复性能。

\subsection{安全架构与合规}

\subsubsection{多层安全模型}

平台在三个层面上实现纵深防御安全。智能合约安全采用Certora或等效工具进行形式化验证、CertiK、Trail of Bits或OpenZeppelin的年度第三方审计、为关键漏洞提供最高50万美元奖励的漏洞赏金计划、带有48小时时间锁的可升级透明代理模式,以及用于应急漏洞响应的熔断机制。

平台安全包括通过OAuth 2.0和JWT认证实现的API保护(限制为100请求/分钟)、SP访问的IP白名单以及90天API密钥轮换。数据加密采用TLS 1.3保护传输数据、AES-256保护静态数据、端到端加密保护敏感工作负载,以及硬件安全模块进行密钥管理。基础设施安全部署了Cloudflare DDoS保护、带有OWASP规则集的Web应用防火墙、季度渗透测试以及用于事件监控的SIEM系统。

数据隐私和合规措施通过账户删除权、数据可携带性、隐私设计原则和欧盟数据驻留选项满足GDPR要求。KYC/AML程序对月度超过1万USDC的交易实施基础验证、对CSP认证实施增强验证、对可疑活动实施交易监控,并遵守FATF旅行规则。数据隔离采用容器化或基于VM的作业执行、网络分段、完成后自动数据擦除以及跨用户泄漏预防。

\subsubsection{事件响应}

一个持续的安全运营中心监控异常活动,包括异常提款、智能合约漏洞利用和API滥用。事件分类遵循四级严重程度模型(关键、高、中、低),评估目标为15分钟。关键事件将触发立即合约暂停,并在1小时内发出多签通知。关键事件将在24小时内公开披露,事后报告将在7天内发布。恢复程序通过治理渠道部署补丁,并从保险基金向受影响用户提供赔偿。

\subsubsection{监管合规}

平台追求SOC 2 Type II数据安全与可用性认证(第1年目标)和ISO 27001信息安全管理认证(第2年目标)。云安全联盟STAR认证验证CSP的安全态势。PCI DSS合规性仍在考虑中,用于未来支付方式的扩展。

\subsection{可扩展性与性能优化}

\subsubsection{水平扩展架构}

平台通过分布式数据库架构实现水平扩展,在各区域部署PostgreSQL只读副本,按ID哈希对用户数据进行分片,使用Redis集群存储热点数据(会话、定价),并通过Cloudflare CDN交付静态资产。

微服务架构将功能分解为可独立扩展的服务:用户服务(认证、配置文件)、作业服务(提交、调度、监控)、支付服务(USDC托管、结算、CPT奖励)、SP服务(入驻、认证、评级)、定价服务(预言机聚合)和通知服务(邮件、推送、链上事件)。每个服务根据需求自主扩展。

负载均衡在US、EU和亚洲地区实现地理分布,使用Kubernetes Horizontal Pod Autoscaler进行动态容量调整,Hystrix熔断机制防止级联故障,RabbitMQ队列用于异步作业处理。

\subsubsection{性能目标}

\begin{center}
\begin{tabular}{|l|c|c|}
\hline
\textbf{指标} & \textbf{目标(第1年)} & \textbf{目标(第3年)} \\
\hline
API响应时间 & <200ms (p95) & <100ms (p95) \\
作业提交时间 & <5 seconds & <2 seconds \\
支付结算时间 & <30 seconds & <10 seconds \\
页面加载时间 & <2 seconds & <1 second \\
平台可用时间 & 99.5\% & 99.9\% \\
并发用户数 & 10,000 & 100,000 \\
日交易量 & 50,000 & 1,000,000 \\
\hline
\end{tabular}
\end{center}

\subsubsection{区块链可扩展性}

Arbitrum第2层部署为主要操作提供低于0.1美元的交易费用和40,000 TPS的吞吐量。批量交易处理将奖励分配分组,以分摊Gas成本。The Graph协议处理链下事件索引。未来开发包括用于高频微支付场景的状态通道。

Gas优化技术通过基于默克尔证明的奖励申领(节省80%)、veCPT余额的延迟评估、紧凑存储变量编码以及在功能等效的情况下优先使用事件日志而非状态变量来降低交易成本。

\subsection{灾难恢复}

\subsubsection{备份基础设施}

数据库备份每天执行一次全量备份,每6小时执行一次增量备份,并进行持续的事务日志复制。系统在冷存储归档前保留30天的备份。智能合约状态利用区块链的固有不可篡改性,并辅以归档节点部署和季度去中心化存储快照(IPFS/Arweave)。用户作业结果备份至指定存储端点,平台元数据保留90天,并具备按需导出能力以满足GDPR合规性。

\subsubsection{恢复目标}

表~\ref{tab:recovery-targets}指定了组件级别的恢复时间目标(RTO)和恢复点目标(RPO)。

\begin{table}[htbp]
\centering
\caption{恢复时间与恢复点目标}
\label{tab:recovery-targets}
\begin{tabular}{lcc}
\hline
\textbf{组件} & \textbf{恢复时间目标(RTO)} & \textbf{恢复点目标(RPO)} \\
\hline
智能合约 & N/A & 0 \\
Web门户 & 1 hour & 6 hours \\
数据库 & 2 hours & 1 hour \\
作业调度器 & 30 minutes & 15 minutes \\
\hline
\end{tabular}
\end{table}

在US和EU地区的双活部署可在主区域不可用5分钟后自动触发DNS故障转移。实时跨区域数据同步保持一致性,并具备手动覆盖能力以进行运营干预。

\subsection{发展路线图}

短期开发(6-12个月)优先开发iOS和Android移动应用、用于第三方集成的增强API产品(RESTful、GraphQL)、基于机器学习的成本优化,以及额外的区块链桥部署(Polygon、Optimism)。中期目标(1-2年)通过支持IoT部署的边缘计算、用于敏感工作负载的机密计算集成(Intel SGX、AMD SEV)、去中心化存储协议(Filecoin、Arweave)、专门的AI/ML资源市场以及探索性的量子计算合作伙伴关系来扩展平台能力。长期愿景(2-5年)包括全面过渡到DAO治理、开发开放式去中心化计算协议、实施零知识证明以增强隐私、通过IBC或等效协议实现跨链互操作性,以及基于NFT的物理计算资源代币化。

\subsection{总结}

本章详细介绍了将Web3区块链基础设施与成熟的HPC系统集成的技术架构。混合设计将去中心化激励机制(CPT代币、投票锁定治理)与经过验证的CHESS集群管理平台桥接起来。安全架构通过智能合约审计、基础设施加固和监管合规路径(SOC 2、ISO 27001)实现多层保护。系统可从数千用户扩展到数十万并发用户,同时保持API响应时间低于200ms。

与现有的去中心化计算项目(Golem、iExec、Render)相比,CyberPlaza的差异化在于成熟的基础设施(CHESS平台拥有20多年历史)、企业合规导向、超越点对点架构的多云集成、预集成的应用生态系统,以及结合去中心化访问与专业SP认证的混合市场。这种定位在满足企业计算需求的同时,实现了Web3经济参与。
