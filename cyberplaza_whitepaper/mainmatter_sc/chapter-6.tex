\chapter{技术与架构}

\section{项目治理基础设施}

\subsection{概述}

CyberPlaza网络由CyberPlaza基金会和CyberPlaza社区组成。

CyberPlaza基金会是一个非营利性去中心化组织,致力于CyberPlaza平台的成功运营、计算技术及应用的推广与发展,并支持平台上的去中心化社区建设与发展。基金会由CyberPlaza社区中的CPT持有者所有并控制。基金会由网络核心成员(见白皮书第8节)以及基金会根据需要不时任命的顾问运营。基金会将设立CyberPlaza实验室,负责开发和研究新的计算资源技术及应用,以推动平台所需的技术创新和进步。

CyberPlaza社区是网络的社区部分,由流动性提供者、用户和服务提供商(SP)组成,他们共同参与基金会治理、开发和推广。社区成员可通过参与治理、向基金会提出业务方向和技术发展建议、交流和分享经验来推动平台的发展与成长。

CyberPlaza基金会与CyberPlaza社区之间的紧密联系对于实现网络的愿景和使命至关重要。

\subsection{智能合约模块}

我们将把CPT部署为符合ERC20标准的智能合约,运行在Arbitrum(以太坊第二层)上,选择该网络的原因是其交易成本低、吞吐量高。平台还将根据生态系统扩展的需要桥接到其他链。

CPT代币合约包含以下关键功能:标准ERC20功能(转账、授权等)、用于veToken机制的质押和锁定功能、治理投票集成、奖励分配机制、应急暂停功能(由治理控制),以及用于未来增强的可升级代理模式。

\textbf{注意}:平台直接使用USDC进行支付,无需专有稳定币,避免了相关监管风险。

\begin{verbatim}
// SPDX-License-Identifier: MIT
pragma solidity ^0.8.0;

import "@openzeppelin/contracts/token/ERC20/ERC20.sol";

contract CPTToken is ERC20 {
  struct LockInfo {
     uint256 amount;
     uint256 lockTimestamp;
     uint256 unlockTimestamp;
  }

   mapping (address => LockInfo[]) public locks;

   constructor(uint256 initialSupply) ERC20("CPT Token", "CPT") {
     _mint(msg.sender, initialSupply);
   }

   function lock(uint256 _amount, uint256 _lockTime) public {
     require(_amount <= balanceOf(msg.sender), "Not enough CPT to lock");
     require(_lockTime > 0, "Lock time must be positive");

       uint256 lockUntil = block.timestamp + _lockTime;
    
       LockInfo memory newLock = LockInfo({
           amount: _amount,
           lockTimestamp: block.timestamp,
           unlockTimestamp: lockUntil
       });
    
       locks[msg.sender].push(newLock);
    
       _burn(msg.sender, _amount);

   }

  function unlock(uint256 lockIndex) public {
    require(lockIndex < locks[msg.sender].length, 
            "No lock found at this index");
    require(block.timestamp >= locks[msg.sender][lockIndex].unlockTimestamp,
            "CPT still locked");

        uint256 amountToUnlock = locks[msg.sender][lockIndex].amount;
        locks[msg.sender][lockIndex] = 
            locks[msg.sender][locks[msg.sender].length - 1];
        locks[msg.sender].pop();
    
        _mint(msg.sender, amountToUnlock);
    }

   function calculateLockedAmount(address user, uint256 lockDuration) 
       public view returns (uint256) {
     uint256 totalLockedAmount = 0;

        for (uint256 i = 0; i < locks[user].length; i++) {
           if (block.timestamp - locks[user][i].lockTimestamp > lockDuration) {
               totalLockedAmount += locks[user][i].amount;
           }
        }
    
        return totalLockedAmount;
    }

}
\end{verbatim}

\subsection{代币标准与小数处理}

CPT代币遵循标准ERC20规范,小数位数为18,而USDC的小数位数为6。平台在所有转换操作中采用SafeMath库,以防止溢出和下溢错误。价格预言机整合了小数归一化逻辑,最低交易阈值可减轻粉尘攻击风险。对于小数金额,协议采用保守的四舍五入机制。

\subsection{投票锁定代币机制}

平台采用投票锁定(ve)代币模型,以协调长期利益相关者的激励。用户将CPT锁定1周至4年不等的时间,获得不可转让的veCPT代币,该代币决定治理权重和奖励分配。

veCPT余额遵循以下关系:
\begin{equation}
\text{veCPT} = \text{CPT}_{\text{locked}} \times \min\left(\frac{t_{\text{lock}}}{t_{\text{max}}}, 1\right) \times 2.5
\end{equation}
其中$t_{\text{lock}}$表示所选锁定时长,$t_{\text{max}} = 4$年定义了最大锁定周期。2.5倍的乘数为4年锁仓提供了最大治理权重。

随着锁定期接近到期,veCPT余额线性衰减:
\begin{equation}
\text{veCPT}(t) = \text{CPT}_{\text{locked}} \times \frac{t_{\text{remaining}}}{t_{\text{max}}} \times 2.5
\end{equation}
这种衰减机制通过锁仓延期或代币重锁激励持续参与。

\subsubsection{奖励分配}

平台以USDC形式收取的收入将进行分配,其中30%分配给质押奖励池,频率为每周或每月。个人奖励根据veCPT持有量按比例计算:
\begin{equation}
\text{Reward}_{\text{user}} = \text{Revenue}_{\text{pool}} \times \frac{V_{\text{user}}}{V_{\text{total}}}
\end{equation}
其中$V_{\text{user}}$表示用户的veCPT余额,$V_{\text{total}}$表示veCPT总供应量。有效APY根据质押参与度和平台绩效动态变化:
\begin{equation}
\text{APY} = \frac{\text{Annual Revenue Pool}}{\text{Total CPT Staked Value}} \times \frac{\text{veCPT Multiplier}}{\text{Average Multiplier}}
\end{equation}

\subsubsection{安全与优化}

智能合约将遵循OpenZeppelin标准接受第三方审计,参数修改由多签治理控制。所有奖励分配在链上跟踪,确保透明度。安全功能包括外部调用的重入保护、基于角色的访问控制、应急暂停功能和可升级代理模式。关键参数变更执行48小时时间锁。

 gas优化采用基于默克尔树的批量申领、veCPT余额的惰性评估、打包存储变量以及事件驱动的链下索引。这些技术在降低交易成本的同时,保持了安全保障。

\subsection{预言机集成}

平台集成Chainlink去中心化预言机,用于价格发现和数据聚合。CPT/USD价格馈送聚合Uniswap V3时间加权平均价格和中心化交易所报价的数据。USDC/USD验证采用Chainlink的验证馈送,偏差阈值为0.5%。预言机每5分钟更新一次,或价格变动超过1%时更新,并配有手动回退机制以确保冗余。

对于计算资源定价,链下聚合器监控主要云提供商(AWS、Azure、GCP、阿里云)的公共API,计算计算、存储和带宽的实时市场价格。聚合定价每天或偏差超过5%时发布到链上预言机合约。

预言机安全依赖至少7个独立Chainlink节点的共识。系统拒绝与中位数偏差超过10%或超过1小时的价格更新。电路断路器在检测到操纵企图时自动停止交易。

\subsection{治理架构}

平台关键操作需要通过Gnosis Safe实现的多签审批。超过100K USDC的国库转移需要5-of-9签名,智能合约升级需要7-of-9审批并执行48小时时间锁。参数调整采用4-of-9共识,紧急安全响应采用3-of-5快速响应配置。

治理流程遵循结构化时间线:持有100K以上veCPT的用户可提交提案,随后是7天的社区讨论期和5天的链上投票期,其中1 veCPT等于1票。获批提案在48小时延迟后执行。多签委员会对恶意提案拥有否决权,该权利需每季度审查。

\subsection{跨链基础设施}

平台采用LayerZero全链协议实现多链部署。Arbitrum作为主链,因其交易成本低、吞吐量高。以太坊主网支持面向需要Layer-1安全性的机构用户,而Polygon集成为对成本敏感的用户提供更低的交易成本。未来扩展包括Optimism(2024年第三季度)和Base(2024年第四季度),以实现更广泛的生态系统集成。

桥接安全整合了多项保障措施:流动性上限将每条链的桥接供应量限制在10%;速率限制将吞吐量限制在每小时1M CPT;应急暂停机制应对异常情况;5%的保险基金为桥接价值提供抵押,以应对潜在漏洞。

\subsection{钱包基础设施}

作为标准ERC20代币,CPT支持所有兼容钱包,包括浏览器扩展(MetaMask、Rabby、Rainbow)、移动应用(Trust Wallet、Coinbase Wallet、imToken)、硬件设备(Ledger、Trezor)和智能合约钱包(Argent、Gnosis Safe)。计划在未来部署与Fireblocks和Copper.co的机构托管集成。

Web门户实现WalletConnect和Web3Modal协议,用于标准化钱包连接。在获得连接授权后,平台查询用户余额、质押头寸和veCPT持有量,以启用完整功能访问。交易签名遵循EIP-712标准的类型化结构化数据,呈现人类可读的消息,提高了抵御钓鱼攻击的安全性。

\section{市场计算基础设施}

\subsection{系统架构}

平台采用三层架构。Web3接口层通过React.js和ethers.js框架管理钱包认证(WalletConnect)、USDC支付处理和CPT奖励分配。编排层协调CHESS集群管理系统、作业调度、资源分配、性能监控和SP认证流程。计算资源层聚合CSP集群、公共云API(AWS、Azure、GCP、阿里云)、私有HPC中心和未来的边缘计算节点。

交易流程包括:提交作业并存入USDC;智能合约托管资金直至完成;CHESS介导的资源匹配;在分配的SP基础设施上执行;实时SLA合规性监控;结果交付并自动结算支付;以及CPT奖励的按比例分配(用户1-3%,SP 2-5%)。

\subsection{HPC基础设施组件}

高性能计算(HPC)基础设施包含专用节点类型:计算节点使用多核处理器和大容量内存执行数值模拟和数据分析;可视化节点使用GPU加速渲染大型数据集;I/O节点管理存储和计算架构之间的数据传输;存储节点提供高并发文件系统;管理节点协调资源分配和作业调度。

网络架构采用高速互连技术(InfiniBand、以太网)实现节点间通信。并行文件系统支持大型数据集和中间结果的并发多节点读写操作。

\subsubsection{软件栈}

监控和管理工具为管理员提供系统组件的实时健康和性能数据,包括CPU利用率、内存消耗和网络流量模式。集群管理软件协调整个系统的操作,为地理上分散的安装中的计算节点提供配置、监控和维护能力。

资源分配采用专用调度器管理CPU时间、内存和其他计算资源,以最大限度地提高系统利用率效率。用户界面包括命令行工具和Web门户,用于作业提交和管理。HPC应用中心聚合领域特定的应用和模板,使用户能够直接下载和部署计算工具。集成的计费系统对不同资源类型和计费周期实施透明的定价策略,促进合理的资源利用和准确的成本核算。

\subsection{支付与结算基础设施}

\subsubsection{托管机制}

作业提交启动托管流程,用户批准将USDC支付至平台的智能合约。托管合约计算预估成本,包括资源类型(CPU/GPU/存储)、时长预估、预言机提供的市场定价以及20%的潜在超支缓冲。USDC在批准后转入托管,并针对唯一作业标识符锁定。

作业完成后,实际资源消耗决定最终结算。服务提供商直接获得95-98%的费用(以USDC形式),平台保留2-5%的交易费。多余的托管资金自动返还给用户,CPT奖励按比例分配给用户(1-3%)和SP(2-5%)。

\subsubsection{争议解决协议}

SLA违规触发分级解决机制。5分钟内失败的作业有资格获得自动全额退款。部分完成的作业根据实际交付情况产生按比例退款。用户可在72小时内提交争议并提供支持证据。超过10K USDC价值的案件升级到平台治理仲裁,保险基金覆盖经验证的最高100K USDC的索赔。

\subsubsection{服务提供商认证}

服务提供商认证需要多阶段验证流程。初始注册需要公司验证文件、基础设施规格、支付钱包地址和安全合规证书(SOC 2、ISO 27001)。技术验证采用行业标准基准,包括高性能Linpack(HPL)、高性能共轭梯度(HPCG)、STREAM内存带宽、针对AI工作负载的MLPerf,以及网络延迟评估。安全审计验证AES-256加密、网络隔离和DDoS保护能力。

获批候选人进入30天试用期,期间加强监控,作业并发限制为10个。试用期成功完成后获得认证服务提供商(CSP)身份,可访问机构客户并参与团购。CSP在高级目录中显示,并带有已验证徽章。

持续合规要求月度 uptime 达到99.5%、作业启动时间低于5分钟、性能在宣传基准的10%以内。每季度重新认证以验证持续能力。关键漏洞的安全补丁必须在48小时内部署。违规触发分级处罚:首次违规警告并给予7天整改期;第二次违规暂停30天;第三次违规吊销认证。

\subsection{技术栈}

平台采用React.js 18+结合TypeScript进行前端开发,使用ethers.js v6和WalletConnect v2实现Web3集成,通过Material-UI保证界面一致性。后端架构使用Node.js/Express.js或Python FastAPI提供API服务,PostgreSQL用于关系型持久化,Redis用于缓存,RabbitMQ/Kafka用于异步作业队列,The Graph用于区块链事件索引,Prometheus/Grafana用于可观测性。DevOps基础设施通过Docker容器化所有服务,通过Kubernetes编排生产部署,通过GitHub Actions实现CI/CD,通过Cloudflare CDN分发内容,通过Nginx实现流量负载均衡。

入门级CSP需要100个以上CPU核心(Intel Xeon/AMD EPYC)、500 GB内存、10 TB NVMe SSD或50 TB HDD、10 Gbps网络上行链路,以及可选的4个以上NVIDIA A100/H100 GPU。企业级CSP可扩展至10,000个以上CPU核心、50 TB以上总内存、1 PB以上并行文件系统存储(Lustre/GPFS)、100 Gbps InfiniBand骨干网和100个以上高端GPU。

\subsection{平台用户功能}

CPT门户服务于三个主要群体:探索项目信息的访客、采购市场资源(公共云、HPC提供商、硬件、软件、存储)的用户,以及执行USDC存款和铸造操作的流动性提供者。

公共云消费者可在FQ、亚马逊和华为云等供应商之间选择,定价以USDC计价并附带促销优惠。选择供应商后,用户将被重定向到原生门户(如AWS),标准操作通过CPT平台托管进行支付路由。平台随后以法定货币与供应商结算。

HPC资源消费者可在CT集群、区域提供商、华为、AWS等供应商之间比较价格、硬件规格、性能指标和区域带宽。选择供应商和提交作业通过CHESS门户进行,需存入足够的USDC,资金托管直至完成,随后进行法定结算。存储采购遵循相同的工作流程。

软件选项包括用户提供的应用或平台列出的Ansys、HPC软件供应商和CHESS应用中心的解决方案。供应商入驻可容纳硬件、存储、软件和辅助计算产品。当两个组件均来自平台列表时,系统验证硬件-软件兼容性,确保执行兼容性。架构可根据需求演变适应未来的功能扩展。

\subsection{公共云集成}

市场聚合了来自主要公共云供应商(AWS、Azure、Google Cloud、阿里云)的计算资源。定价以USDC计价,并显示当前促销和可用状态。

\subsubsection{供应商集成模式}

平台采用三种集成方法。直接API集成利用转售商凭证,通过供应商API(AWS EC2、Azure Resource Manager、GCP Compute Engine)实现实时配置,支持自动实例生命周期管理。优惠券代码系统通过预生成代码解决容量限制,防止超卖,提供基于价值(100美元通用 credits)或基于资源(1000 GPU小时、10 TB存储)的格式。托管服务提供商模式将CyberPlaza定位为MSP,拥有批量定价协议,管理供应商账户并提供统一账单。

实时价格比较显示计算、存储和网络成本以及总拥有成本计算。团购折扣突出与直接采购相比的潜在节省。

\subsubsection{作业提交流程}

HPC作业提交流程包括:通过过滤的CSP列表(CPU类型、GPU可用性、区域、定价)选择资源;通过应用中心模板或自定义代码配置作业,指定要求(节点、核心、内存、运行时间、GPU)和I/O位置;成本估算,包括USDC明细和预计CPT奖励;支付授权,将USDC转入托管并附带应急缓冲;通过CHESS调度器分配执行,实时状态监控;完成结算,交付结果并自动分配支付、退还多余资金和发放CPT奖励。

高级功能包括:支持100个以上作业的批量提交,带有参数扫描;定义顺序执行的工作流依赖;用于容错的 checkpoint/重启;现货实例竞标,可在抢占式容量上获得50-70%的折扣;以及用于动态资源调整的自动缩放。

服务提供商通过集中式仪表板管理操作,包括资源分配、作业监督、财务跟踪(USDC收入、CPT积累)和性能分析(客户满意度、利用率指标)。

\subsection{多集群管理系统}

CHESS(集群高性能执行与调度系统)平台为地理上分散的计算资源提供统一管理。系统通过支持基于角色的访问控制的集中式Web门户集成监控、调度和资源分配功能。

\subsubsection{核心能力}

平台通过Web界面和SSH协议支持全面的数据管理,启用文件操作,包括上传、下载、压缩和提取。节点管理通过批量命令控制电源状态、远程访问(VNC、shell),并支持异构硬件配置(CPU、GPU、FPGA)。资源配额对存储和计算分配执行行政政策,在阈值违规时自动生成警报。

高可用性架构通过冗余管理节点和数据库复制消除单点故障。系统协调多个地理上分散的集群,在子集群之间统一用户角色传播。

\subsection{性能监控基础设施}

高性能和云计算系统聚合了大量硬件资源,通过高速网络互联,形成低延迟、高容量的配置。有效的集群管理需要监控和管理工具提供资源配置、实时性能跟踪、故障检测与告警,以及使用状态可视化。

\subsubsection{CHESS监控功能}

CHESS监控系统通过聚合仪表板提供全面的集群监控,显示CPU和内存使用情况、负载状态、存储状态以及以太网和InfiniBand架构的网络吞吐量。自定义时间间隔选择支持历史趋势分析和性能跟踪。仪表板显示提供可定制的大屏幕展示,动态刷新存储使用情况、作业调度和网络统计数据。

多集群监控扩展到地理上分散的安装,具有自适应屏幕布局和分辨率优化。机架可视化呈现物理拓扑,集成电源管理和VNC远程访问控制。单节点监控捕获细粒度的CPU、内存、存储、负载和网络指标,同时提供故障诊断和恢复建议。GPU监控跟踪设备特定的使用率、内存利用率、温度和带宽。作业监控分析实时执行状态和队列组成,提供详细的CPU利用率、内存消耗和节点负载统计数据。集群告警实现可配置阈值,并通过电子邮件和系统通知路由。

性能指标按用户定义的时间间隔收集,捕获CPU、内存、磁盘和网络数据。物理拓扑可视化包括机架和节点布置,以及基于阈值的故障告警。

\subsubsection*{调度器与资源管理}

高效的调度和资源管理在多集群系统中至关重要。CHESS提供灵活的调度策略,包括FIFO、抢占和回填策略。系统支持带有服务质量(QoS)配置的资源预留、跨串行、并行和GPU工作负载的高级作业提交,以及用于负载均衡优化的队列管理。

\subsubsection{作业提交与管理}

用户通过命令行界面、基于Web的GUI或常见工作流的应用模板提交作业。管理员配置资源配额、优先级和提交政策,以管理系统访问和利用率。

% \subsection*{6.2.5 Pricing Module}

% This module will focus on calculating costs for resource usage and presenting pricing details to users. It will integrate with job scheduling and monitoring systems for real-time cost tracking.

% \subsubsection*{6.2.5 User Interfaces and Operational Portals}

\subsubsection{用户管理}

平台支持自注册和管理员配置的账户,并集成LDAP认证以实现集中管理。基于角色的访问控制实现默认角色(管理员、部门管理员、用户),并通过灵活的权限分配管理系统访问和功能。

\subsubsection{通知与消息}

用户会收到关于账单和使用情况的自动告警,以及行政公告。

\subsection{应用中心}

应用中心通过可浏览的库提供对预安装HPC应用(Ansys、MATLAB、TensorFlow)的访问。用户通过带有交互式参数配置的图形模板提交作业。输出管理包括日志查看、错误分析、性能指标跟踪和集成可视化工具(用于AI应用的TensorBoard)。

\subsection{硬件性能评估}

硬件性能评估模块执行基准测试,测量CPU和GPU性能以及网络吞吐量和延迟。资源效率分析根据工作负载特征优化分配策略。故障恢复指标评估硬件在故障场景下的可靠性和恢复性能。

\subsection{安全架构与合规性}

\subsubsection{多层安全模型}

平台在三层实现纵深防御安全。智能合约安全采用Certora或等效工具进行形式化验证,每年由CertiK、Trail of Bits或OpenZeppelin进行第三方审计,提供最高50万美元的关键漏洞漏洞 bounty 计划,带有48小时时间锁的可升级透明代理模式,以及用于应急漏洞响应的电路断路器。

平台安全包括:通过OAuth 2.0和JWT认证保护API,速率限制为每分钟100次请求;SP访问的IP白名单;90天API密钥轮换。数据加密实现:传输保护使用TLS 1.3,静态数据使用AES-256,敏感工作负载使用端到端加密,密钥管理使用硬件安全模块。基础设施安全部署Cloudflare DDoS保护、带有OWASP规则集的Web应用防火墙、季度渗透测试,以及用于事件监控的SIEM系统。

数据隐私和合规措施通过以下方式满足GDPR要求:账户删除权利、数据可移植性、隐私设计原则,以及欧盟数据驻留选项。KYC/AML程序:每月超过10K USDC的交易实施基本验证;CSP认证实施增强验证;对可疑活动进行交易监控;遵守FATF旅行规则。数据隔离采用容器化或基于VM的作业执行、网络分段、完成后自动数据擦除,以及跨用户泄漏预防。

\subsubsection{事件响应}

持续的安全运营中心监控异常活动,包括异常提款、智能合约漏洞和API滥用。事件分类遵循四级严重性模型(临界、高、中、低),评估目标为15分钟。临界事件触发立即合约暂停,并在1小时内通知多签。临界事件在24小时内公开披露,事后报告在7天内发布。恢复程序通过治理渠道部署补丁,并从保险基金补偿受影响用户。

\subsubsection{监管合规}

平台追求SOC 2 Type II认证,以确保数据安全和可用性(第1年目标),并追求ISO 27001信息安全管理认证(第2年目标)。云安全联盟STAR认证验证CSP安全态势。PCI DSS合规性正在考虑用于未来支付方式扩展。

\subsection{可扩展性与性能优化}

\subsubsection{水平扩展架构}

平台通过分布式数据库架构实现水平扩展:跨区域的PostgreSQL读副本;按ID哈希对用户数据进行分片;用于热数据(会话、定价)的Redis集群;用于静态资产交付的Cloudflare CDN。

微服务架构将功能分解为可独立扩展的服务:用户服务(认证、配置文件)、作业服务(提交、调度、监控)、支付服务(USDC托管、结算、CPT奖励)、SP服务(入驻、认证、评级)、定价服务(预言机聚合)和通知服务(电子邮件、推送、链上事件)。每个服务根据需求自主扩展。

负载均衡在美、欧、亚地区实现地理分布,使用Kubernetes Horizontal Pod Autoscaler进行动态容量调整,使用Hystrix电路断路器防止级联故障,使用RabbitMQ队列处理异步作业。

\subsubsection{性能目标}

\begin{center}
\begin{tabular}{|l|c|c|}
\hline
\textbf{指标} & \textbf{目标(第1年)} & \textbf{目标(第3年)} \\
\hline
API响应时间 & <200ms (p95) & <100ms (p95) \\
作业提交时间 & <5秒 & <2秒 \\
支付结算时间 & <30秒 & <10秒 \\
页面加载时间 & <2秒 & <1秒 \\
平台 uptime & 99.5\% & 99.9\% \\
并发用户数 & 10,000 & 100,000 \\
每日交易数 & 50,000 & 1,000,000 \\
\hline
\end{tabular}
\end{center}

\subsubsection{区块链可扩展性}

Arbitrum Layer 2部署为主要操作提供低于0.10美元的交易费和40,000 TPS的吞吐量。批量交易处理将奖励分配分组,以分摊gas成本。The Graph协议处理链下事件索引。未来开发包括用于高频微支付场景的状态通道。

 gas优化技术通过以下方式降低交易成本:基于默克尔证明的奖励申领(节省80%)、veCPT余额的惰性评估、打包存储变量编码,以及在功能等效的情况下优先使用事件日志而非状态变量。

\subsection{灾难恢复}

\subsubsection{备份基础设施}

数据库备份每天执行一次完全备份,每六小时执行一次增量备份,并持续进行交易日志复制。系统在冷存储归档前保留30天。智能合约状态利用区块链固有的不可变性,并辅以归档节点部署和每季度的去中心化存储快照(IPFS/Arweave)。用户作业结果备份到指定存储端点,平台元数据保留90天,并提供GDPR合规的按需导出功能。

\subsubsection{恢复目标}

表~\ref{tab:recovery-targets}规定了组件级恢复时间目标(RTO)和恢复点目标(RPO)。

\begin{table}[htbp]
\centering
\caption{恢复时间与点目标}
\label{tab:recovery-targets}
\begin{tabular}{lcc}
\hline
\textbf{组件} & \textbf{RTO} & \textbf{RPO} \\
\hline
智能合约 & N/A & 0 \\
Web门户 & 1小时 & 6小时 \\
数据库 & 2小时 & 1小时 \\
作业调度器 & 30分钟 & 15分钟 \\
\hline
\end{tabular}
\end{table}

美国和欧盟地区的主-主部署在主区域不可用5分钟后启用自动DNS故障转移。实时跨区域数据同步保持一致性,并具有操作干预的手动覆盖能力。

\subsection{发展路线图}

近期开发(6-12个月)优先考虑iOS和Android移动应用、用于第三方集成的增强API(RESTful、GraphQL)、基于机器学习的成本优化,以及额外的区块链桥部署(Polygon、Optimism)。

中期目标(1-2年)通过以下方式扩展平台功能:为IoT部署提供边缘计算支持、为敏感工作负载集成机密计算(Intel SGX、AMD SEV)、去中心化存储协议(Filecoin、Arweave)、专门的AI/ML资源市场,以及探索性量子计算合作伙伴关系。

长期愿景(2-5年)包括:全面过渡到DAO治理、开发开放式去中心化计算协议、实现零知识证明以增强隐私、通过IBC或等效协议实现跨链互操作性,以及基于NFT的物理计算资源代币化。

\subsection{总结}

本章详细介绍了将Web3区块链基础设施与成熟HPC系统集成的技术架构。混合设计将去中心化激励机制(CPT代币、投票锁定治理)与成熟的CHESS集群管理平台相结合。安全架构通过智能合约审计、基础设施加固和合规路径(SOC 2、ISO 27001)实现多层保护。系统可从数千名并发用户扩展到数十万名并发用户,同时保持API响应时间低于200ms。

与现有去中心化计算项目(Golem、iExec、Render)相比,CyberPlaza的差异化在于成熟的基础设施(20+年CHESS平台历史)、企业合规导向、超越点对点架构的多云集成、预集成的应用生态系统,以及结合去中心化访问与专业SP认证的混合市场。这种定位满足了企业计算需求,同时支持Web3经济参与。
