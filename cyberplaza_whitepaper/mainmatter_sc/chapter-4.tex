\chapter{运营中的角色描述}

\section{运营的4种角色}

平台生态系统由4种不同角色组成:平台角色、服务提供商(SP)角色、流动性提供者角色和用户角色。任何人担任或退出1种或全部4种角色均不受限制。

\subsection{平台角色}

\subsubsection{所有权与参与}

平台是由CyberPlaza Token(CPT)的所有持有者拥有的开放民主组织。任何人可通过以下方式参与项目以获取CPT:(i) 为平台提供服务;(ii) 成为流动性提供者;(iii) 成为服务提供商(SP);(iv) 成为用户;或(v) 在二级市场购买CPT。

\subsubsection{平台功能}

平台作为分销商、匹配方和担保人,确保用户、服务提供商(SP)与流动性提供者之间的信任并促进交易。平台维护一份认证服务提供商(CSP)列表和一份「普通」SP列表。CSP是指服务价值超过特定阈值(当前定义为「未来10天内将达成的销售额中,每月提供价值10,000+ USDC的服务」)的提供商。「普通」SP是指服务价值低于该阈值的提供商。平台第一阶段将仅纳入CSP,后续再引入普通SP。

平台将评估CSP,考量因素包括其过往记录、声誉及CSP的绩效指标,并将评估结果在平台上公示,以便用户做出明智决策。「普通」SP不接受评估,用户自行选择使用。平台作为可信中介的角色增加了一层问责机制,并提高了SP按照其服务等级协议(SLA)履行承诺的可能性。平台赚取一部分交易手续费(用户支付价格与SP获得价格之间的差额)。

\subsubsection{储备基金管理}

平台负责运营由存入USDC代币的流动性提供者设立的USDC「储备基金」,该基金是Web3项目的货币,用于市场交易。储备基金将通过多种方式为USDC代币持有者生成利息,从而使USDC代币的铸造成为一项高收益投资。这些方式包括通过团购以折扣价获取计算资源并转售给用户,即使用储备基金开展算力拼多多(Pinduoduo)业务。团购将面向全球主流云服务提供商,包括AWS、Azure、Google Cloud、Alibaba Cloud等,以及美国、欧洲和中国等地的超级计算中心。平台还通过投资盈利性计算资产(如比特币挖矿设施)获取利润,并通过高流动性的去中心化金融或传统金融投资获取利息。

\subsubsection{平台参与与未来扩展}

平台可根据需要参与其他角色(流动性提供者、SP、用户),以启动流动性并确保服务质量。如果CPT持有者通过治理机制投票通过,平台未来可将淘宝平台(Taobao)和拼多多(Pinduoduo)的运营扩展至算力(计算资源)之外。

\subsection{服务提供商(SP)角色}

\subsubsection{注册与服务上架}

SP在平台上注册服务,向用户提供计算能力(核心时长、存储、带宽、应用软件、数据和服务等)。SP在平台上列出其不同时段的计算资源可用性(例如未来24小时内的1,000个Intel Core i7核心时长、未来一个月内的10,000个核心时长)和价格列表,供用户使用/预订。SP还将发布其提供的资源的各类基准测试结果(按平台要求)及其服务等级协议(SLA)。

\subsubsection{支付与激励}

当SP的服务被用户选择并使用时,SP将直接获得USDC支付。此外,SP将获得与交易规模成正比的CPT代币激励(交易价值的2--5%,以CPT等价物计算)。质押CPT代币的SP还可享受降低的平台手续费和提升的市场曝光度。

\subsubsection{质量保障}

平台的评估系统验证CSP的质量和可靠性,确保所有列出的主要SP(CSP)均可信赖。平台通过声誉系统、用户评价和绩效指标,为CSP建立基于绩效的排名系统。该评估系统让用户在选择SP进行大量使用时做出明智决策,降低选择不可靠或不合适SP的概率。

\subsubsection{灵活的服务配置}

用户可为一个任务选择组合的SP,例如,计算的主要部分使用CSP,而数据分析的最后环节使用「普通」SP(例如用户自己提供的笔记本电脑)。平台将对所选CSP进行评估,但不对非认证SP进行评估。

\subsection{流动性提供者角色}

\subsubsection{概述}

流动性提供者是将USDC存入平台去中心化借贷池的参与者,以支持团购和平台运营的运营资金。该角色实现了透明且去中心化的平台运营资金模式。

\subsubsection{流动性提供机制}

流动性提供机制的运作方式如下:参与者将USDC存入经审计的智能合约,并获得代表其存款的rUSDC代币(收据代币)。平台将池内资金用于团购运营和营运资金。参与者可提取存款,但受池内流动性情况限制。

\subsubsection{可参与性}

任何人(包括SP、用户和外部投资者)均可通过将USDC存入池中成为流动性提供者。最低存款额的设计兼顾了可及性和有效贡献。

\subsubsection{收益与权益}

流动性提供者通过多种机制获取收益。他们将从平台运营利润中获得年化收益率(APY)为6--8%的USDC利息收益,以及额外的、以CPT代币支付的年化2--4%的CPT代币激励(带锁定期),合计预期年化总收益率为8--12%。除财务收益外,他们还可通过积累CPT获得治理权(拥有投票权),以及平台权益(包括降低手续费、优先访问权和新产品优先体验)。风险保障通过智能合约审计、保险基金(覆盖率10%)和透明追踪实现。

\subsection{用户角色}

\subsubsection{获取计算资源}

用户可通过简单流程在平台上获取计算资源:(i) 将USDC存入其平台钱包;(ii) 在市场中浏览并选择服务提供商;(iii) 通过平台门户提交任务并支付USDC。

\subsubsection{有竞争力的定价}

作为计算资源淘宝平台(算力淘宝平台),用户可获取最适合自身的计算资源,且价格具有竞争力——由于团购福利,通常比直接从云提供商购买低10--30%。

\subsubsection{支付保障与透明性}

平台实施全面的支付保障和透明性措施。智能合约托管将暂扣USDC支付,直至服务交付确认;若SP未达到服务等级协议(SLA)要求,将自动退款。系统确保定价透明、无隐藏费用,提供实时性能监控与报告,并通过平台治理建立争议解决机制。

\subsubsection{用户激励计划}

用户可通过多种机制参与平台获取CyberPlaza Token(CPT)。消费奖励向用户提供消费金额的1--3%的CPT代币。推荐奖励允许用户通过邀请新用户或SP加入平台赚取CPT。忠诚度等级为持续使用平台的用户提供更高奖励,质量反馈机制允许用户通过提供详细的服务评价赚取CPT。

持有和质押CPT的权益十分可观。使用折扣允许用户质押CPT以获得5--15%的服务折扣。收益分成允许质押的CPT获得平台收益分配。治理权允许用户对平台参数和功能优先级进行投票。高级功能允许用户访问高级工具、分析和API服务。
