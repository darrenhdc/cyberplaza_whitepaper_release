\chapter{运营中的角色描述}

\section{运营的4种角色}

平台生态系统由四个不同的角色组成:平台角色、服务提供商(SP)角色、流动性提供商角色和用户角色。任何人都可以承担或离开这4个角色中的任何一个或全部,没有限制。

\subsection{平台的角色}

\subsubsection{所有权与参与}

平台是一个由所有CyberPlaza Token(CPT)持有者拥有的开放且民主的组织。任何人都可以通过以下方式参与项目获得CPT:(i) 为平台提供服务,(ii) 作为流动性提供商,(iii) 作为服务提供商SP,(iv) 作为用户,或(v) 在二级市场购买CPT。

\subsubsection{平台功能}

平台充当分销商、匹配者和担保人的角色,确保用户、服务提供商(SP)与流动性提供商之间的信任并促进交易。平台维护认证SP(CSP)列表和“普通”SP列表。CSP是指提供的服务价值超过一定水平的SP(目前定义为“在未来10天内,每月提供价值10,000美元以上USDC的销售服务”)。“普通”SP是指提供的服务低于该阈值的SP。平台在第一阶段将仅以CSP开始,稍后再引入普通SP。

平台会评估CSP,考虑其过往记录、声誉和CSP的绩效指标,并将评估结果列在平台上,以便用户做出知情决策。“普通”SP不接受评估,用户自行选择使用。平台作为可信中介的角色增加了一层问责制,并提高了SP根据其SLA履行承诺的可能性。平台从交易费用中赚取一部分(用户支付的价格与SP获得的价格之间的差额)。

\subsubsection{储备基金管理}

平台负责运营由存入USDC代币的流动性提供商设立的USDC“储备基金”,该基金是Web 3项目的货币,用于市场交易。储备基金将通过各种方式为USDC代币持有者产生利息,从而使USDC代币的铸造成为高收益投资。这些方式包括通过团购(團購)以折扣价获取计算资源,再转售给用户,即利用储备基金开展算力拼多多(Pinduoduo)业务。团购将来自全球主要云服务,包括AWS、Azure、Google Cloud、Alibaba Cloud等,以及美国、欧洲和中国等地的超级计算中心。平台还通过投资产生收益的计算资产(如比特币挖矿设施)获得利润,并通过高流动性的去中心化金融或传统金融投资获得利息。

\subsubsection{平台参与和未来扩展}

平台也可以根据需要参与其他角色(流动性提供商、SP和用户),以引导流动性并确保服务质量。如果CPT持有者通过治理机制投票支持,平台未来可以将淘宝平台(Taobao)和拼多多(Pinduoduo)的运营扩展到算力(computing resources)之外。

\subsection{服务提供商(SP)的角色}

\subsubsection{注册和服务上架}

SP在平台上注册服务,向用户提供计算能力(核心小时、存储、带宽、应用软件、数据和服务等)。SP在平台上列出其不同时期的计算资源可用性(例如未来24小时内1,000核小时的Intel Core i7,未来一个月内10,000核小时)和价格表,供用户使用/预订。SP还将发布其提供的资源的各种基准(按照平台要求)以及其SLA。

\subsubsection{付款与激励}

当SP的服务被用户选择和使用时,SP直接接收USDC付款。此外,他们还会获得与交易规模成比例的CPT代币激励(交易价值的2--5\%,以CPT等价物计算)。质押CPT代币的SP还可以获得平台费用减免和在市场上的更高可见性。

\subsubsection{质量保证}

平台的评估系统验证CSP的质量和可靠性,确保所有列出的主要SP(CSP)都是可信的。通过利用声誉系统、用户评论和绩效指标,平台为CSP建立了基于绩效的排名系统。评估系统让用户在选择SP进行任何大量使用时做出知情决策,减少选择不可靠或不合适SP的机会。

\subsubsection{灵活的服务配置}

用户可以选择结合多个SP来完成一项工作,例如,主要部分的计算使用CSP,而最后阶段的数据分析使用“普通”SP(例如,用户自己提供的笔记本电脑)。平台为所选的CSP提供评估,但不为未认证的SP提供评估。

\subsection{流动性提供商的角色}

\subsubsection{概述}

流动性提供商是将USDC存入平台去中心化借贷池以支持团购和平台运营营运资金的参与者。该角色以更透明和去中心化的模式取代了之前的“启用者”概念。

\subsubsection{流动性提供机制}

流动性提供机制的运作方式如下:参与者将USDC存入经过审计的智能合约,并收到代表其存款的rUSDC代币(收据代币)。平台将集合资金用于团购运营和营运资金。参与者可以在池流动性可用的情况下提取存款。

\subsubsection{可及性}

任何人都可以通过将USDC存入池中成为流动性提供商,包括SP、用户和外部投资者。最低存款额的设计兼顾了可及性和有意义的贡献。

\subsubsection{回报与利益}

流动性提供商通过多种机制获得回报。他们从平台运营利润中获得6--8\%的年化收益率(APY)的USDC利息收入,并获得CPT代币激励(额外2--4\%的年化收益率,以CPT代币计算,带有锁定期),综合总预期收益率为8--12\%的年化收益率。除了财务回报外,他们还通过CPT积累获得治理权,拥有投票权,以及平台福利,包括费用减免、优先访问和早期产品发布。风险保护通过智能合约审计、保险基金(10\%覆盖率)和透明跟踪来确保。

\subsection{用户的角色}

\subsubsection{获取计算资源}

用户可以通过简单的流程在平台上获取计算资源:(i) 将USDC存入他们的平台钱包,(ii) 从市场浏览并选择服务提供商,以及(iii) 通过平台门户提交作业并支付USDC。

\subsubsection{具有竞争力的定价}

通过将平台作为算力淘宝平台,用户可以以具有竞争力的价格访问最适合他们的计算资源,由于团购优惠,价格通常比直接从云提供商购买低10--30\%。

\subsubsection{支付保护与透明度}

平台实施了全面的支付保护和透明度措施。智能合约托管持有USDC付款,直到服务交付得到确认,如果SP未能满足SLA要求,则自动退款。该系统确保价格透明,无隐藏费用,实时性能监控和报告,以及通过平台治理的争议解决机制。

\subsubsection{用户激励计划}

用户通过平台参与,可通过多种赚取机制获得CyberPlaza Tokens(CPT)。消费奖励为用户提供消费金额的1--3\%的CPT代币。推荐奖励允许用户通过将新用户或SP带到平台来赚取CPT。忠诚等级为持续的平台使用提供更高的奖励,质量反馈机制使用户能够通过提供详细的服务评论来赚取CPT。

持有和质押CPT的好处是显著的。使用折扣允许用户质押CPT以获得5--15\%的服务折扣。收益分享使质押的CPT能够获得平台收益分配。治理权允许对平台参数和功能优先级进行投票。高级功能提供对高级工具、分析和API服务的访问。
