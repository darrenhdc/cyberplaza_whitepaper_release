\chapter{运营中的角色描述}

\section{运营中的4个角色}

平台生态系统由四个不同的角色组成:平台角色、服务提供商(SP)角色、流动性提供商角色和用户角色。任何人担任或离开这4个角色中的任何一个或所有角色都不受限制。

\subsection{平台的角色}

\subsubsection{所有权与参与}

该平台是一个开放、民主的组织,由CyberPlaza代币(CPT)的所有持有者共同拥有。任何人都可以通过以下方式参与项目以获得CPT:(i) 为平台提供服务,(ii) 作为流动性提供商,(iii) 作为服务提供商(SP),(iv) 作为用户,或(v) 在二级市场购买CPT。

\subsubsection{平台功能}

平台充当分发者、匹配者和担保人,确保用户、服务提供商(SP)和流动性提供商之间的信任并促进交易。平台维护一份认证服务提供商(CSP)名单和一份“普通”服务提供商名单。CSP是指提供的服务价值超过一定水平的提供商(当前定义为“在未来10天内为销售提供每月价值10,000美元以上USDC的服务”)。“普通”SP是指服务价值低于该阈值的提供商。平台在第一阶段将仅以CSP启动,后续再引入普通SP。

平台将评估CSP,考虑其过往记录、声誉和CSP的绩效指标等因素,并将评估结果在平台上列出,以便用户做出知情决策。“普通”SP不接受评估,用户自行选择使用。平台作为可信中介的角色增加了一层问责制,并提高了SP根据其服务水平协议(SLA)履行承诺的可能性。平台从交易费用中抽取一部分收益(用户支付的价格与SP获得的价格之间的差额)。

\subsubsection{储备基金管理}

平台负责运营由存入USDC代币的流动性提供商设立的USDC“储备基金”,该基金是Web 3项目的货币,用于市场中的交易。储备基金将通过各种方式为USDC代币持有者产生利息,从而使USDC代币的铸造成为一种高收益投资。这些方式包括通过团购以折扣价获取计算资源,再转售给用户,即利用储备基金开展算力拼多多业务。团购将来自全球主要云服务,包括AWS、Azure、Google Cloud、阿里云等,以及美国、欧洲和中国等地的超级计算中心。平台还通过投资能产生收入的计算资产(如比特币挖矿设施)获得利润,并通过高流动性的去中心化金融或传统金融投资获得利息。

\subsubsection{平台参与与未来扩展}

平台可能还会根据需要参与其他角色(流动性提供商、SP和用户),以启动流动性并确保服务质量。如果CPT持有者通过治理机制投票赞成,平台未来可能将淘宝平台和拼多多运营扩展到算力之外。

\subsection{服务提供商(SP)的角色}

\subsubsection{注册与服务上线}

SP在平台上注册其服务,向用户提供算力(核心小时、存储、带宽、应用程序软件、数据和服务等)。SP在平台上列出其不同时间段的计算资源可用性(例如未来24小时内1,000个Intel Core i7核心小时,未来一个月内10,000个核心小时)和价格列表,供用户使用/预订。SP还将发布其提供的资源的各种基准测试(平台要求)以及其SLA。

\subsubsection{支付与激励}

当SP的服务被用户选择和使用时,SP会直接收到USDC支付。此外,他们还会获得与交易 volume成正比的CPT代币激励(交易价值的2--5\%,以CPT等价物计算)。质押CPT代币的SP还可以获得更低的平台费用和在市场上更高的知名度。

\subsubsection{质量保证}

平台的评估系统验证CSP的质量和可靠性,确保所有列出的主要SP(CSP)都是值得信赖的。通过利用声誉系统、用户评论和绩效指标,平台为CSP建立了基于 merit的排名系统。该评估系统让用户在选择SP进行任何大量使用时能做出知情决策,减少选择不可靠或不合适SP的机会。

\subsubsection{灵活的服务配置}

用户可以选择组合使用多个SP来完成一项工作,例如,计算的主要部分使用CSP,而数据分析的最后阶段使用“普通”SP(例如,用户自己提供的笔记本电脑)。平台对选择的CSP提供评估,但不对非认证SP提供评估。

\subsection{流动性提供商的角色}

\subsubsection{概述}

流动性提供商是将USDC存入平台去中心化借贷池的参与者,以支持团购和平台运营的运营资本。该角色以更透明、更去中心化的模式取代了之前的“赋能者”概念。

\subsubsection{流动性提供的运作方式}

流动性提供机制的运作方式如下:参与者将USDC存入经过审计的智能合约,并获得代表其存款的rUSDC代币(收据代币)。平台利用池中的资金进行团购运营和营运资金。参与者可以在池流动性可用的情况下提取存款。

\subsubsection{可访问性}

任何人都可以通过向池中存入USDC成为流动性提供商,包括SP、用户和外部投资者。最低存款额的设计既易于获取,又能确保有意义的贡献。

\subsubsection{回报与福利}

流动性提供商通过多种机制获得回报。他们从平台运营利润中获得6--8\% APY的利息收益(以USDC支付),以及CPT代币激励(额外2--4\% APY的CPT代币,带有归属期),总预期收益率为8--12\% APY(合并计算)。除了财务回报外,他们还通过积累CPT获得治理权(提供投票权),以及平台福利,包括降低的费用、优先访问权和早期产品发布。智能合约审计、保险基金(10\%覆盖率)和透明追踪确保了风险保护。

\subsection{用户的角色}

\subsubsection{获取计算资源}

用户可以通过以下简单流程在平台上获取计算资源:(i) 将USDC存入他们的平台钱包,(ii) 从市场中浏览并选择服务提供商,以及(iii) 通过平台门户提交作业并使用USDC支付。

\subsubsection{具有竞争力的定价}

通过将平台用作算力淘宝平台,用户可以以极具竞争力的价格获得最适合自己的计算资源,由于团购优势,价格通常比直接从云提供商购买低10--30\%。

\subsubsection{支付保护与透明度}

平台实施了全面的支付保护和透明度措施。智能合约托管将USDC支付保留至服务交付确认,如果SP未能满足SLA要求,则自动退款。该系统确保透明的定价(无隐藏费用)、实时的性能监控和报告,以及通过平台治理的争议解决机制。

\subsubsection{用户激励计划}

用户通过平台参与度通过多种赚取机制获得CyberPlaza代币(CPT)。消费奖励为用户提供消费金额的1--3\%的CPT代币。推荐奖励允许用户通过为平台带来新用户或SP来赚取CPT。忠诚度等级为持续平台使用提供更高的奖励,质量反馈机制使用户能够通过提供详细的服务评论来赚取CPT。

持有和质押CPT的好处是巨大的。使用折扣允许用户质押CPT以获得5--15\%的服务折扣。收益共享使质押的CPT能够获得平台收益分配。治理权允许对平台参数和功能优先级进行投票。高级功能提供对高级工具、分析和API服务的访问。
