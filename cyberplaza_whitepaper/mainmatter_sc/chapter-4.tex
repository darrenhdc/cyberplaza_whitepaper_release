\chapter{运营中的角色描述}

\section{运营的4种角色}

平台生态系统由四种不同的角色组成:平台角色、服务提供商(SP)角色、流动性提供商角色和用户角色。任何人担任或退出全部或部分这4种角色均不受限制。

\subsection{平台的角色}

\subsubsection{所有权与参与}

本平台是由CyberPlaza Token(CPT)的所有持有者拥有的开放民主组织。任何人都可以通过以下方式参与项目以获取CPT:(i) 向平台提供服务,(ii) 担任流动性提供商,(iii) 担任服务提供商(SP),(iv) 担任用户,或(v) 在二级市场购买CPT。

\subsubsection{平台功能}

本平台充当分销商、匹配者和担保人的角色,确保用户、服务提供商(SP)与流动性提供商之间的信任并促进交易。平台维护一份认证服务提供商(CSP)列表和一份“普通”SP列表。CSP是指提供的服务价值超过一定水平的提供商(目前定义为“在未来10天内提供每月价值\$10,000+ USDC的销售服务”)。“普通”SP是指提供的服务低于该阈值的提供商。平台第一阶段将仅以CSP启动,后续再引入普通SP。

平台将对CSP进行评估,评估因素包括其过往记录、声誉和绩效指标,并将评估结果在平台上公布,以便用户做出明智决策。“普通”SP不接受评估,用户自行选择使用。平台作为可信中介,增加了一层问责机制,并提高了SP根据其SLA履行承诺的可能性。平台从交易费中获得一部分收入(用户支付的价格与SP获得的价格之间的差额)。

\subsubsection{储备基金管理}

平台负责运营由存入USDC代币的流动性提供商设立的“储备基金”(以USDC计价),该基金是Web 3项目的货币,用于市场交易。储备基金将通过各种方式为USDC代币持有者产生利息,从而使USDC代币的铸造成为高收益投资。这些方式包括通过团购以折扣价获取计算资源,再转售给用户,即利用储备基金开展“算力拼多多”业务。团购将来自全球主要云服务商,包括AWS、Azure、Google Cloud、阿里云等,以及美国、欧洲和中国等地的超级计算中心。平台还通过投资能产生收入的计算资产(如比特币挖矿设施)获得利润,并通过高流动性的去中心化金融或传统金融投资获得利息。

\subsubsection{平台参与与未来扩展}

平台可能会根据需要参与其他角色(流动性提供商、SP和用户),以引导流动性并确保服务质量。如果CPT持有者通过治理机制投票支持,平台未来可能将“淘宝平台”和“拼多多模式”扩展到算力之外的领域。

\subsection{服务提供商(SP)的角色}

\subsubsection{注册与服务发布}

SP在平台上注册其服务,为用户提供算力(核心小时、存储、带宽、应用软件、数据及服务等)。SP在平台上列出其不同时期的计算资源可用性(例如未来24小时内的1000个Intel Core i7核心小时,未来一个月内的10000个核心小时)和价格表,供用户使用/预订。SP还将发布其提供的资源的各种基准(按平台要求)及其SLA。

\subsubsection{支付与激励}

当SP的服务被用户选择并使用时,SP将直接收到USDC付款。此外,他们还会获得与交易规模成比例的CPT代币激励(交易价值的2–5\%,以CPT等价物计算)。质押CPT代币的SP还可以获得更低的平台费用和更高的市场可见性。

\subsubsection{质量保证}

平台的评估系统验证CSP的质量和可靠性,确保所有列出的主要SP(CSP)都是可信的。通过利用声誉系统、用户评价和绩效指标,平台为CSP建立了基于 merit 的排名系统。评估系统让用户在选择SP进行任何实质性使用时能够做出明智决策,降低选择不可靠或不合适SP的机会。

\subsubsection{灵活的服务配置}

用户可以选择结合多个SP来完成一项任务,例如,计算的主要部分使用CSP,而数据分析的最后阶段使用“普通”SP(例如用户自己提供的笔记本电脑)。平台为所选CSP提供评估,但不为非认证SP提供评估。

\subsection{流动性提供商的角色}

\subsubsection{概述}

流动性提供商是将USDC存入平台去中心化借贷池以支持团购和平台运营运营资金的参与者。该角色以更透明、去中心化的模式取代了之前的“赋能者”概念。

\subsubsection{流动性提供的工作原理}

流动性提供机制的运作方式如下:参与者将USDC存入经过审计的智能合约,并获得代表其存款的rUSDC代币(收据代币)。平台将池内资金用于团购运营和营运资金。参与者可以在池内流动性可用的情况下提取存款。

\subsubsection{可及性}

任何人都可以通过将USDC存入池中成为流动性提供商,包括SP、用户和外部投资者。最低存款额的设计兼顾了可及性和有意义的贡献。

\subsubsection{收益与权益}

流动性提供商通过多种机制获得收益。他们从平台运营利润中获得以USDC支付的6–8\% APY的利息收益,并获得额外2–4\% APY的CPT代币激励(带锁定期),综合总预期收益率为8–12\% APY。除财务收益外,他们通过积累CPT获得治理权(提供投票权),并获得平台权益,包括降低费用、优先访问和早期产品发布。风险保护通过智能合约审计、保险基金(10\%覆盖率)和透明跟踪来确保。

\subsection{用户的角色}

\subsubsection{获取计算资源}

用户可以通过简单的流程在平台上获取计算资源:(i) 将USDC存入其平台钱包,(ii) 从市场浏览并选择服务提供商,(iii) 通过平台门户提交作业并使用USDC支付。

\subsubsection{有竞争力的定价}

通过将平台作为“算力淘宝平台”,用户可以以有竞争力的价格获取最适合他们的计算资源,由于团购优势,价格通常比直接从云服务商购买低10–30\%。

\subsubsection{支付保护与透明度}

平台实施了全面的支付保护和透明度措施。智能合约托管会持有USDC付款,直到服务交付得到确认,如果SP未满足SLA要求,则自动退款。该系统确保定价透明,无隐藏费用,实时绩效监控和报告,以及通过平台治理的争议解决机制。

\subsubsection{用户激励计划}

用户通过平台参与,可通过多种赚取机制获得CyberPlaza Tokens(CPT)。消费奖励为用户提供消费金额1–3\%的CPT代币。推荐奖励允许用户因推荐新用户或SP到平台而赚取CPT。忠诚等级为持续平台使用提供更高奖励,质量反馈机制允许用户因提供详细服务评价而赚取CPT。

持有和质押CPT的好处是巨大的。使用折扣允许用户质押CPT以获得5–15\%的服务折扣。收入分成使质押的CPT能够获得平台收入分配。治理权允许对平台参数和功能优先级进行投票。高级功能提供对高级工具、分析和API服务的访问。
