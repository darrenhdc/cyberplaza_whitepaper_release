\chapter{核心團隊、基金會與顧問}

\subsection*{核心團隊}

核心團隊負責平台技術基礎架構的建立、維護與推進。開發與維護範圍涵蓋CHESS運算能力分配軟體、上架運算資源的品質評估系統、區塊鏈平台架構、智慧合約實作以及配套技術基礎架構。

團隊專業領域涵蓋分散式高效能運算、公有雲服務、異質運算架構、去中心化金融投資策略、人工智慧與大數據應用、金融科技解決方案、分散式系統軟體開發以及運算資源商業化。

\subsubsection*{核心團隊成員}

\textbf{Dr. Wai-Mo Suen} 擁有25年高效能運算技術與現代運算業務營運經驗。自2000年起擔任ClusterTech創辦人與執行長,他已提供高效能運算(HPC)、雲端運算、人工智慧(AI)與大數據解決方案,並因在高效能運算業務與金融科技創新領域的成就獲得多項獎項。

\textbf{Dr. Harry Yu} 專精於FPGA技術,為CTAccel創辦人與執行長,該公司於2018年獲得Intel Capital投資。他的投資敏銳度體現在2年的去中心化金融經驗中,期間實現年化報酬率18\%、夏普比率4.6。

\textbf{Mr. Eric Leung} 擁有15年高效能運算(HPC)系統管理經驗,並輔以10年擔任公有雲服務供應商營運領導的經歷。

\textbf{Mr. GY Han} 具備15年高效能運算(HPC)系統管理軟體開發的專業經驗。

\textbf{Mr. Terence Leung} 擁有近30年執法專業知識,專精於反洗錢與詐欺調查,並具備豐富的遵規與風險管理經驗。他擔任量化與DeFi投資基金的顧問與財務長已有5年。

\textbf{Mr. Pong Po Lam Paul (龐寶林)} 創立Pegasus Fund Managers Ltd.,並共同創立香港財務策劃師學會、亞洲金融科技師學會以及香港金融分析師及專業評論員協會。他的公共服務歷練包括擔任金融發展局、強積金諮詢委員會、香港會計師公會以及證券及期貨事務監察委員會的職務。他持有認證財務策劃師(CFPCM)與認證金融科技師(CFT)資格。

\textbf{Mr. XXX} 具備豐富的資訊科技(IT)業務與行銷營運經驗。

\subsection*{基金會、投資者與服務供應商(SPs)}

\subsubsection*{基金會}

基金會負責專案的開發、推廣與維護,以確保長期永續性。職責涵蓋代幣分配與管理、社群建立與參與、行銷與推廣計劃、專案治理監督以及技術與經濟生態系支持。

基金會成員包含核心團隊成員與顧問,他們具備高效能與雲端運算資源供應、人工智慧與大數據基礎架構、金融投資策略、金融產品開發以及商業法規與反洗錢要求遵規方面的專業知識。

\subsubsection*{投資者}

待確認。

\subsubsection*{服務供應商(SPs)}

平台上線時,已有5家認證服務供應商(CSPs)註冊,提供的運算資源包含xx個CPU核心(相當於???個X86核心)、??? FP64 TFLOPS的運算能力、yy個GPU(相當於xxx TOPS的32位元張量運算)、zz個FPGA(相當於??? TFLOPS的FP32運算)以及??? PB的儲存容量。

資源成長預測目標為上線後1年內,CPU擴充10倍、GPU擴充20倍、FPGA成長5倍、儲存容量增加10倍。
