\chapter{核心團隊、基金會與顧問}

\subsection*{核心團隊}

核心團隊負責建立、維護並推進平台的技術基礎架構。開發與維護範圍涵蓋CHESS運算能力分配軟體、上架運算資源的品質評估系統、區塊鏈平台架構、智能合約實作以及配套技術基礎架構。

團隊專長涵蓋分散式高效能運算、公共雲服務、異質運算架構、去中心化金融投資策略、人工智慧與大數據應用、金融科技解決方案、分散式系統軟體開發以及運算資源商業化。

\subsubsection*{核心團隊成員}

\textbf{Dr. Wai-Mo Suen} 擁有25年高效能運算技術與現代運算業務營運的專業經驗。自2000年起擔任ClusterTech創辦人與執行長,提供高效能運算、雲端運算、人工智慧與大數據解決方案,並多次獲得表彰高效能運算業務與金融科技創新成果的獎項。

\textbf{Dr. Harry Yu} 專精於FPGA技術,為CTAccel的創辦人與執行長,該公司於2018年獲得Intel Capital投資。他的投資專才包括兩年去中心化金融經驗,實現18%年化報酬率與4.6 Sharpe ratio。

\textbf{Mr. Eric Leung} 擁有15年高效能運算系統管理經驗,並輔以10年領導公共雲服務提供者營運的經歷。

\textbf{Mr. GY Han} 具備15年高效能運算系統管理軟體開發的專門經驗。

\textbf{Mr. Terence Leung} 擁有近30年執法專長,專精於反洗錢與詐欺調查,並輔以豐富的合規與風險管理經驗。他擔任量化與DeFi投資基金的顧問與財務總監已達五年。

\textbf{Mr. Pong Po Lam Paul (龐寶林)} 創辦Pegasus Fund Managers Ltd.,並共同創辦香港財務策劃師學會(The Institute of Financial Planners of Hong Kong)、亞洲金融科技學會(The Institute of Financial Technologists of Asia)以及香港金融分析師及專業評論員學會(HK Institute of Financial Analysts \& Professional Commentators)。他的公共服務職務涵蓋金融發展局、強積金諮詢委員會(MPF Advisory Committee)、香港會計師公會以及證券及期貨事務監察委員會的職位。他持有認證財務策劃師(CFPCM)與認證金融科技師(CFT)資格。

\textbf{Mr. XXX} 具備豐富的資訊科技業務與行銷營運經驗。

\subsection*{基金會、投資者與服務提供者(SPs)}

\subsubsection*{基金會}

基金會管理專案的開發、推廣與維護,以確保長期可持續性。職責涵蓋代幣分配與管理、社群建置與參與、行銷與推廣計畫、專案治理監督以及技術與經濟生態系支援。

基金會成員包括核心團隊成員與顧問,具備高效能與雲端運算資源供應、人工智慧與大數據基礎架構、金融投資策略、金融產品開發以及商業法規合規與反洗錢要求的專業知識。

\subsubsection*{投資者}

尚待確認。

\subsubsection*{服務提供者(SPs)}

平台推出時,已有五家認證服務提供者(CSPs)完成註冊,提供的運算資源包括xx個CPU核心(相當於???個X86核心),可提供??? FP64 TFLOPS運算能力;yy個GPU(相當於32位元張量運算的xxx TOPS);zz個FPGA(相當於??? FP32運算的TFLOPS);以及??? PB的儲存容量。

資源成長預測目標為:在推出後一年內,CPU擴充10倍、GPU擴展20倍、FPGA成長5倍、儲存容量增加10倍。
