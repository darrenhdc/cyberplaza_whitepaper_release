\chapter{營運概覽}
\subsection{我們欲解決的挑戰}

\begin{enumerate}
\item \textbf{中心化控制}:運算(特別是雲端運算、高效能運算及人工智慧(AI)領域)在現代社會的重要性毋庸置疑。然而,這些關鍵資源主要由大型企業控制,多數使用者無法充分享受其優勢。我們認為,解決之道在於去中心化市場,該市場將普及化存取運算資源的機會,培養更開放、更具包容性的環境。在此系統中,使用者不僅是消費者,更是能影響運算發展軌跡並與運算未來利害相關的貢獻者。

\item \textbf{效率不彰}:當前的運算資源分配模式常導致失衡,造成資源使用率不足或過度飽和。本計畫旨在建立一個能有效匹配算力需求與可用資源的平台,從而優化資源使用率、減少浪費。

\item \textbf{成本高昂}:目前多數使用者面臨不必要的高昂運算成本。我們的願景是建立一個市場平台,提供直接存取多元算力、儲存方案、軟體應用、數據及服務的管道,且價格具競爭力。這不僅降低整體成本,也擴大使用者群。

\item \textbf{缺乏透明度}:現有的運算資源分配系統在價格、可用性及服務品質方面缺乏透明度。我們旨在建立一個開放且公正的平台,讓使用者能基於關於資源、供應商及價格的可靠資訊,做出知情決策。

\item \textbf{缺乏使用者自主權}:對大多數人而言,執行需要運算的想法可能是一個繁複的過程,通常需要依賴第三方服務。例如,必須仰賴政府機構進行模擬後提供的電視天氣預報,或必須將個人數據委託給中心化機構以創建自己的數位分身。本計畫旨在建立一個提供所有必要運算資源的去中心化市場,讓使用者能執行任何運算並維持完全控制權。
\end{enumerate}

對於現代社會這項重要的發展方向,我們需要解決建立去中心化綜合生態系的挑戰,以實現運算資源更易取得、更有效率的分配與使用。

\subsection{我們的解決方案大綱}

\begin{enumerate}
\item 我們即將推出一個以開放且民主組織形式運作的平台。該平台是一個運算資源市場,類似於淘寶的模式(亦即「算力淘寶平台」)。此架構的所有權由CyberPlaza Token(CPT)的所有持有者共享,這些持有者是我們平台的「股東」。

\item \textbf{支付系統}:我們的平台使用USDC進行所有交易,確保符合監管規定、價格透明度及熟悉的使用者體驗。這消除了與專有穩定幣相關的風險,並符合全球監管架構。

\item 在平台上,服務供應商(SPs)會上架其運算資源──包括算力、儲存、軟體應用、數據及服務──供使用者依其需求選擇。作為服務回報,服務供應商直接收到USDC付款,並根據交易額獲得CPT代幣獎勵。

\item 平台本身並不擁有上架的運算資源。然而,它可以透過「團購」(團購)採購運算資源,再轉售給使用者。此模式類似於拼多多(Pinduoduo)的商業策略,利用去中心化流動性池,社群成員可存入USDC以賺取收益,同時支持平台營運。

\item 我們的平台開放且具包容性。任何人擔任四種角色──平台「股東」、流動性提供者、SP及使用者──中的任一或全部均無限制。這種彈性讓參與者能以最符合其需求及能力的方式參與平台。
\end{enumerate}
