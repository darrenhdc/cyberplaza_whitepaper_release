\chapter{作業概觀}
\subsection{我們旨在解決的挑戰}

\begin{enumerate}
\item \textbf{集中式控制}:在現代社會中,計算的重要性日益提升,尤其在雲端運算、高效能運算及人工智慧(AI)領域,這一點毋庸置疑。然而,這些關鍵資源主要由大型企業控制,將優勢限制在多數使用者以外。我們認為解決方案在於去中心化市場,該市場可民主化存取計算資源,培育更開放、更具包容性的環境。在此類系統中,使用者不僅是消費者,更是能影響運算發展軌跡並在運算未來中擁有利益的貢獻者。

\item \textbf{低效}:當前的計算資源分配模型常導致失衡,造成資源利用率不足或過度飽和。我們的專案旨在打造一個平台,有效匹配計算能力需求與可用資源,從而最佳化利用率並最小化浪費。

\item \textbf{高成本}:目前,大多數使用者面臨不必要的高計算成本。我們的願景是建立一個市場平台,以具競爭力的價格直接提供廣泛的計算能力、儲存解決方案、軟體應用、資料及服務。這不僅降低整體成本,也擴大使用者基礎。

\item \textbf{缺乏透明度}:現有的計算資源分配系統在定價、可用性及服務品質方面缺乏透明度。我們旨在打造一個開放且公正的平台,讓使用者能夠憑藉關於資源、提供者及定價的可靠資訊,做出明智的決策。

\item \textbf{缺乏使用者賦權}:對我們大多數人而言,執行需要計算的想法可能是一個繁瑣的過程,經常需要依賴第三方服務。例如,人們必須依賴政府機構執行模擬所產生的電視天氣預報,或必須將個人資料委託給集中式實體,才能為自己建立數位分身。我們的專案旨在打造一個去中心化市場,提供所有必要的計算資源,讓使用者能夠執行任何想要的計算,同時保持完全控制權。
\end{enumerate}

對於現代社會這一重要的發展方向,我們需要解決建立去中心化綜合生態系統的挑戰,以實現更易存取、更高效的計算資源分配與利用。

\subsection{我們的解決方案大綱}

\begin{enumerate}
\item 我們正在推出一個作為開放且民主組織運作的平台。該平台類似於計算資源的市場,令人聯想到淘寶(Taobao)等平台(即「算力淘寶平台」)。此架構的所有權由CyberPlaza代幣(CPT)的所有持有者分散持有,這些持有者是我們平台的「股東」。

\item \textbf{付款系統}:我們的平台使用USDC進行所有交易,確保法規遵循、定價透明及熟悉的使用者體驗。這消除了與專屬穩定幣相關的風險,並符合全球監管架構。

\item 在平台上,服務提供者(SPs)列出其計算資源──包括計算能力、儲存、軟體應用、資料及服務──供使用者根據自身需求選擇。作為服務的回報,SPs直接收到USDC付款,以及基於其交易量的CPT代幣獎勵。

\item 平台本身並不擁有列出的計算資源。然而,它可以透過「團購」採購計算資源,再轉售給使用者。此模型類似於拼多多(Pinduoduo)的業務策略,使用去中心化流動性池,社群成員可存入USDC以獲得回報,同時支援平台運作。

\item 我們的平台是開放且具包容性的。任何人擔任四個角色中的任何一個或所有角色都沒有限制:平台「股東」、流動性提供者、SP及使用者。這種彈性使參與者能夠以最適合其需求和能力的方式與平台互動。
\end{enumerate}
