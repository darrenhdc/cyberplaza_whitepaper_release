\chapter{營運角色說明}

\section{營運的4種角色}

平台生態系統由四種不同的角色組成:平台角色、服務提供者(SP)角色、流動性提供者角色及使用者角色。任何人皆可擔任或離開任一或所有這4種角色,無任何限制。

\subsection{平台角色}

\subsubsection{所有權與參與方式}

平台為開放且民主的組織,所有CyberPlaza代幣(CPT)持有者皆為其所有者。任何人可透過以下方式參與專案以獲得CPT:(i) 向平台提供服務,(ii) 擔任流動性提供者,(iii) 擔任服務提供者(SP),(iv) 擔任使用者,或(v) 在二級市場購買CPT。

\subsubsection{平台功能}

平台作為分配者、撮合者及擔保人,確保使用者、服務提供者(SP)與流動性提供者之間的信賴,並促進交易進行。平台維護認證服務提供者(CSP)名單及「一般」服務提供者名單。CSP為服務價值超過特定門檻(目前定義為「未來10天內每月銷售價值超過10,000美元USDC的服務」)者。「一般」服務提供者為服務價值低於該門檻者。平台初期僅提供CSP,後續將引入一般SP。

平台將評估CSP,考量因素包括其交易記錄、聲譽及CSP的績效指標,並將評估結果列於平台上,讓使用者能做出明智決定。「一般」服務提供者不經評估,使用者自行選擇使用。平台作為可信賴的仲介,增加了服務水準協定(SLA)下SP履行承諾的層層問責及可能性。平台的收入來自交易費的一部分(即使用者支付價格與SP獲得價格之間的差額)。

\subsubsection{準備金管理}

平台負責管理由存入USDC代幣的流動性提供者設立的「USDC準備金」,該準備金為Web 3專案的貨幣,用於市場中的交易。準備金將透過各種方式為USDC代幣持有者產生利息,因此使USDC代幣的鑄造成為高收益投資。這些方式包括透過團購以折扣取得計算資源,再轉售給使用者,即利用準備金開展算力拼多多(Pinduoduo)業務。團購將向全球主要雲服務廠商進行,包括AWS、Azure、Google Cloud、阿里巴巴雲等,以及美國、歐洲及中國等地的超級電腦中心。平台也透過投資能產生收入的計算資產(如比特幣礦場)獲取利潤,並透過去中心化金融或傳統金融的高流動性投資獲得利息。

\subsubsection{平台參與及未來擴展}

平台也可視需要參與其他角色(流動性提供者、SP及使用者),以引導流動性並確保服務品質。若CPT持有者透過治理機制投票通過,平台未來可將淘寶平台及拼多多營運模式擴展至算力之外。

\subsection{服務提供者(SP)角色}

\subsubsection{註冊與服務上架}

SP在平台上註冊其服務,以向使用者提供計算資源(核心時數、儲存空間、頻寬、應用軟體、資料及服務等)。SP在平台上列出其不同時段的計算資源可用情況(例如未來24小時1,000個Intel Core i7核心時數、未來一個月10,000個核心時數)及價格表,供使用者使用/預訂。SP也將張貼其提供資源的各種基準測試(依平台要求)及其SLA。

\subsubsection{付款與獎勵}

當SP的服務被使用者選擇並使用時,SP將直接收到USDC付款。此外,他們將獲得與其交易額成正比的CPT代幣獎勵(交易價值的2–5\%,以CPT等價計算)。質押CPT代幣的SP也可獲得平台費用折扣及市場上的曝光率提升。

\subsubsection{品質保證}

平台的評估系統驗證CSP的品質與可靠性,確保所有列出的主要SP(CSP)皆可信賴。透過利用聲譽系統、使用者評價及績效指標,平台為CSP建立了基於績效的排名系統。評估系統讓使用者在選擇SP進行任何大量使用時能做出明智決定,降低選擇不可靠或不適合SP的機率。

\subsubsection{彈性服務配置}

使用者可選擇組合多個SP的服務來執行工作,例如主要計算部分使用CSP,而最後的資料分析部分使用「一般」SP(例如使用者自己提供的筆記型電腦)。平台為所選擇的CSP提供評估,但不評估非認證SP。

\subsection{流動性提供者角色}

\subsubsection{概觀}

流動性提供者為將USDC存入平台去中心化借貸池,以支援團購和平台營運資金的參與者。此角色以更透明、去中心化的模式取代了先前的「促成者」概念。

\subsubsection{流動性提供機制運作方式}

流動性提供機制的運作方式如下:參與者將USDC存入經過審計的智能合約,並獲得代表其存款的rUSDC代幣(收據代幣)。平台將集資用於團購營運和營運資金。參與者可在池有足夠流動性時提取存款。

\subsubsection{可及性}

任何人皆可透過將USDC存入池成為流動性提供者,包括SP、使用者及外部投資者。最低存款門檻設計為易於取得,同時確保有意義的貢獻。

\subsubsection{回報與好處}

流動性提供者透過多種機制獲得回報。他們從平台營運利潤中獲得6–8\%年收益率(APY)的USDC利息收入,以及額外2–4\%年收益率(APY)的CPT代幣獎勵(有歸屬期),合計預期總收益率為8–12\%年收益率(APY)。除財務回報外,他們透過累積CPT獲得治理權,擁有投票權,以及平台福利,包括費用折扣、優先存取及早期產品發布。風險保護透過智能合約審計、保險基金(10\%覆蓋)及透明追蹤來確保。

\subsection{使用者角色}

\subsubsection{存取計算資源}

使用者可透過簡單流程在平台上存取計算資源:(i) 將USDC存入其平台錢包,(ii) 從市場瀏覽並選擇服務提供者,以及(iii) 透過平台入口提交工作並以USDC支付。

\subsubsection{具競爭力的價格}

透過利用平台作為算力淘寶平台,使用者可以具競爭力的價格取得最適合其需求的計算資源,由於團購優惠,價格通常比直接向雲服務廠商購買低10–30\%。

\subsubsection{付款保障與透明度}

平台實施全面的付款保障與透明度措施。智能合約託管USDC付款,直到服務交付確認,若SP未達SLA要求則自動退款。系統確保透明定價且無隱藏費用、即時效能監控與報告,以及透過平台治理的爭議解決機制。

\subsubsection{使用者獎勵計劃}

使用者透過多種獲取機制,在平台參與中獲得CyberPlaza代幣(CPT)。消費回饋為使用者提供消費金額的1–3\%作為CPT代幣。推薦獎勵允許使用者因介紹新使用者或SP加入平台而獲得CPT。忠誠等級為持續使用平台的使用者提供更高獎勵,品質回饋機制讓使用者因提供詳細的服務評價而獲得CPT。

持有與質押CPT的好處相當可觀。使用折扣允許使用者質押CPT以獲得5–15\%的服務折扣。收入分潤使質押的CPT能獲得平台收入分配。治理權允許對平台參數和功能優先級進行投票。高級功能提供對進階工具、分析及API服務的存取權。
