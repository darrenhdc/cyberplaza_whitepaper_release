\chapter{營運中的角色說明}

\section{營運的4種角色}

平台生態系統由四種不同角色組成:平台角色、服務提供者(SP)角色、流動性提供者角色以及使用者角色。任何人擔任或離開任一或所有這4種角色均無限制。

\subsection{平台的角色}

\subsubsection{所有權與參與}

本平台是由所有CyberPlaza Token(CPT)持有者擁有的開放且民主的組織。任何人可透過以下方式參與專案以取得CPT:(i) 向平台提供服務;(ii) 擔任流動性提供者;(iii) 擔任服務提供者(SP);(iv) 擔任使用者;或(v) 在次級市場購買CPT。

\subsubsection{平台功能}

平台擔任分銷者、媒合者及擔保者的角色,確保信任並促進使用者、服務提供者(SP)與流動性提供者之間的交易。平台維護一份認證服務提供者(CSP)清單,以及一份「一般」服務提供者清單。CSP是指所提供的服務價值超過特定門檻者(目前定義為「未來10天內的銷售中,每月提供價值10,000美元以上USDC的服務」)。「一般」服務提供者是指所提供的服務低於該門檻者。平台在第一階段僅會以CSP開始營運,稍後再引進一般SP。

平台將評估CSP,考量因素包括其往績、聲譽及CSP的績效指標,並將評估結果列示於平台上,讓使用者能做出明智的決策。「一般」服務提供者不會受到評估,使用者需自行選擇使用。平台作為可信賴的仲介,增加了一層問責性,並提升SP根據其SLA履行承諾的可能性。平台從交易費中獲利(即使用者支付的價格與SP獲得的價格之間的差額)。

\subsubsection{儲備基金管理}

平台負責運作由存入USDC代幣的流動性提供者所設立、以USDC計價的「儲備基金」;USDC是Web3專案的貨幣,用於市場中的交易。儲備基金將透過各種方式為USDC代幣持有者產生利息,因此使USDC代幣的鑄造成為一項高收益投資。這些方式包括透過團購以折扣取得運算資源,再轉售給使用者,即利用儲備基金開展算力拼多多(Pinduoduo)業務。團購將來自全球主要雲端服務(包括AWS、Azure、Google Cloud、Alibaba Cloud等)以及美國、歐洲及中國等地的超級運算中心。平台也透過投資可產生收入的運算資產(例如比特幣挖礦設施)獲利,並透過高流動性的去中心化金融或傳統金融投資取得利息。

\subsubsection{平台參與與未來擴展}

平台可根據需要參與其他角色(流動性提供者、SP及使用者),以啟動流動性並確保服務品質。若CPT持有者透過治理機制表決同意,平台未來可將淘寶平台(Taobao)與拼多多營運模式(Pinduoduo)擴展至算力(computing resources)以外的領域。

\subsection{服務提供者(SP)的角色}

\subsubsection{註冊與服務上架}

SP在平台上註冊其服務,以向使用者提供運算能力(核心時數、儲存空間、頻寬、應用軟體、數據與服務等)。SP在平台上列示其運算資源在不同期間的可用性(例如未來24小時的1,000個Intel Core i7核心時數、未來一個月的10,000個核心時數)及價目表,供使用者使用/預訂。SP也會張貼其提供之資源的各種效能基準(依平台要求)及其SLA。

\subsubsection{付款與獎勵}

當SP的服務被使用者選擇並使用時,SP將直接收到USDC付款。此外,他們將獲得與其交易額成正比的CPT代幣獎勵(交易價值的2--5\%,以CPT等值計算)。質押CPT代幣的SP也可享有降低的平台費用,以及在市場上更高的能見度。

\subsubsection{品質保證}

平台的評估系統會驗證CSP的品質與可靠性,確保所有列示的主要SP(CSP)均可信賴。透過運用聲譽系統、使用者評價及績效指標,平台為CSP建立基於績效的排名系統。該評估系統讓使用者在選擇SP進行大量使用時能做出明智的決策,降低選擇不可靠或不適合SP的機率。

\subsubsection{彈性服務配置}

使用者可選擇組合使用多個SP的服務,例如大部分運算使用CSP,而最後一階段的數據分析使用「一般」服務提供者(例如使用者自己提供的筆記型電腦)。平台會針對選擇的CSP提供評估,但不會針對非認證SP提供評估。

\subsection{流動性提供者的角色}

\subsubsection{概述}

流動性提供者是將USDC存入平台去中心化借貸池,以支持團購與平台營運之營運資金的參與者。此角色為平台營運的資金籌措提供透明且去中心化的模式。

\subsubsection{流動性提供機制運作方式}

流動性提供機制的運作方式如下:參與者將USDC存入經審計的智能合約,並收到代表其存款的rUSDC代幣(收據代幣)。平台將集合資金用於團購營運與營運資金。參與者可視資金池的流動性狀況提取存款。

\subsubsection{可及性}

任何人(包括SP、使用者及外部投資者)均可透過將USDC存入資金池而成為流動性提供者。最低存款額設計為易於取得,同時確保有意義的貢獻。

\subsubsection{報酬與福利}

流動性提供者透過多種機制獲得報酬。他們可從平台營運利潤中獲得以USDC支付的6--8\%年化利息收益,以及額外提供2--4\%年化CPT代幣獎勵(有歸屬期),合計預期總年化報酬率為8--12\%。除了財務報酬外,他們透過累積CPT獲得治理權(提供投票權),以及包括降低費用、優先存取權及提前體驗新產品在內的平台福利。風險保護透過智能合約審計、保險基金(10\%覆蓋率)及透明追蹤來確保。

\subsection{使用者的角色}

\subsubsection{存取運算資源}

使用者可透過以下簡單流程存取平台上的運算資源:(i) 將USDC存入其平台錢包;(ii) 從市場中瀏覽並選擇服務提供者;(iii) 透過平台入口提交任務並以USDC付款。

\subsubsection{具競爭力的定價}

透過將平台作為運算資源的淘寶——算力淘寶平台,使用者可取得最適合自己的運算資源,且價格具競爭力,由於團購優惠,通常比直接向雲端提供者購買便宜10--30\%。

\subsubsection{付款保護與透明度}

平台實施完善的付款保護與透明度措施。智能合約託管會持有USDC付款,直到服務交付確認;若SP未達SLA要求,則會自動退款。該系統確保定價透明且無隱藏費用、即時績效監控與報告,以及透過平台治理運作的爭議解決機制。

\subsubsection{使用者獎勵計畫}

使用者透過多種獲取機制,可透過平台參與獲得CyberPlaza Token(CPT)。消費獎勵為使用者提供消費金額的1--3\%(以CPT代幣計算)。推薦獎金讓使用者可透過介紹新使用者或SP到平台獲得CPT。忠誠等級為持續使用平台的使用者提供更高獎勵,而品質回饋機制讓使用者可透過提供詳細的服務評論獲得CPT。

持有與質押CPT的好處相當可觀。使用折扣讓使用者可透過質押CPT享有服務費用5--15\%的折扣。收益分享讓質押的CPT可獲得平台收益分配。治理權允許對平台參數與功能優先順序進行投票。Premium功能提供進階工具、分析與API服務的存取權。
