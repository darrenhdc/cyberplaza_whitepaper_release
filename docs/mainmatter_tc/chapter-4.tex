\chapter{營運中各角色之說明}

\section{營運的4種角色}

平台生態系統包含四種明確角色:平台角色、服務提供者(SP)角色、流動性提供者角色以及使用者角色。任何人擔任或離開其中任一或所有4種角色均無限制。

\subsection{平台的角色}

\subsubsection{所有權與參與}

平台為由所有CyberPlaza Token(CPT)持有者共同擁有的開放且民主的組織。任何人可透過以下方式參與專案以取得CPT:(i) 向平台提供服務、(ii) 擔任流動性提供者、(iii) 擔任服務提供者(SP)、(iv) 擔任使用者,或(v) 在二級市場購買CPT。

\subsubsection{平台功能}

平台擔任分銷商、媒合者與保證者的角色,確保信任並促進使用者、服務提供者(SP)與流動性提供者之間的交易。平台維護一份認證服務提供者(CSP)清單以及一份「一般」SP清單。CSP為提供的服務價值超過特定標準的提供者(目前定義為「未來10天內銷售的每月服務價值達10,000+ USDC」)。「一般」SP為提供的服務低於該門檻的提供者。平台第一階段將僅先以CSP開始,稍後再引進一般SP。

平台將評估CSP,考量因素包括其過去紀錄、聲譽以及CSP的績效指標,並將評估結果列於平台上,讓使用者能做出明智決策。「一般」SP不經評估,使用者自行選擇使用。平台作為可信賴的中介者,增加了一層問責制,並提高SP依其服務等級協議(SLA)履行承諾的可能性。平台從交易費用(使用者支付價格與SP獲得價格的差額)中抽取一部分作為收入。

\subsubsection{準備金管理}

平台負責管理由存入USDC代幣的流動性提供者所建立之USDC「準備金」的運作,該USDC為web3專案的貨幣,用於市場交易。準備金將透過多種方式為USDC代幣持有者產生利息,因此使USDC代幣的鑄造成為高收益投資。這些方式包括透過團購(group-buying)以折扣取得計算資源,再轉售給使用者,亦即使用準備金開展算力拼多多(Pinduoduo)業務。團購對象為全球主要雲端服務商,包括AWS、Azure、Google Cloud、Alibaba Cloud等,以及美國、歐洲與中國等國家和地區的超級電腦中心。平台也透過投資能產生收入的計算資產(例如比特幣挖礦設施)獲利,並透過高流動性的去中心化金融或傳統金融投資獲得利息。

\subsubsection{平台參與與未來擴展}

平台也可依需要參與其他角色(流動性提供者、SP與使用者),以啟動流動性並確保服務品質。若CPT持有者透過治理機制表決贊成,平台未來可將淘寶平台(Taobao)與拼多多營運(Pinduoduo)擴展至算力(computing resources)以外的領域。

\subsection{服務提供者(SP)的角色}

\subsubsection{註冊與服務上架}

SP在平台上註冊其服務,向使用者提供計算能力(core-hour、儲存、頻寬、應用軟體、資料與服務等)。SP於平台上列出其計算資源在不同時期的可用狀況(例如未來24小時的1,000個Intel Core i7核心小時、未來一個月的10,000個核心小時)以及價格表,供使用者使用/預訂。SP也將依平台要求張貼其所提供資源的各種基準測試,以及其SLA。

\subsubsection{付款與獎勵}

當SP的服務被使用者選擇並使用時,他們直接收到USDC付款。此外,他們可獲得與其交易量成正比的CPT代幣獎勵(交易價值的2–5%,以CPT等值計算)。質押CPT代幣的SP也可享受平台費用減免以及市場可見度提升。

\subsubsection{品質保證}

平台的評估系統驗證CSP的品質與可靠性,確保所有上架的主要SP(CSP)均值得信賴。平台透過運用聲譽系統、使用者評價與績效指標,為CSP建立基於優異表現的排名系統。該評估系統讓使用者在選擇SP進行任何大量使用時能做出明智決策,降低選擇不可靠或不適合SP的機率。

\subsubsection{彈性服務配置}

使用者可選擇為一項工作組合使用多個SP,例如主要計算部分使用CSP,而最後一段資料分析使用「一般」SP(例如使用者自己提供的筆記型電腦)。平台為所選擇的CSP提供評估,但不提供非認證SP的評估。

\subsection{流動性提供者的角色}

\subsubsection{概述}

流動性提供者為將USDC存入平台去中心化借貸池以支持團購與平台營運所需營運資金的參與者。此角色以更透明且去中心化的模型取代了先前的「Enabler」概念。

\subsubsection{流動性提供的運作方式}

流動性提供機制運作如下:參與者將USDC存入經過審計的智能合約,並收到代表其存款的rUSDC代幣(收據代幣)。平台將集合資金用於團購營運與營運資金。參與者可在池流動性充足的情況下提領存款。

\subsubsection{可及性}

任何人(包括SP、使用者與外部投資者)均可透過將USDC存入池而成為流動性提供者。最低存款額設計為容易達到,同時確保有意義的貢獻。

\subsubsection{報酬與優惠}

流動性提供者透過多種機制獲得報酬。他們從平台營運利潤中獲得以USDC支付的6–8\%年報酬率(APY)利息收入,以及額外的2–4\%年報酬率(APY)CPT代幣獎勵(帶有歸屬期),合計總預期報酬率為8–12\%年報酬率(APY)。除了財務報酬外,他們透過累積CPT獲得具有投票權的治理權利,以及包括費用減免、優先存取與產品提前推出等平台優惠。風險保護透過智能合約審計、保險基金(覆蓋10\%)與透明追蹤來確保。

\subsection{使用者的角色}

\subsubsection{取得計算資源}

使用者可透過簡單流程在平台上取得計算資源:(i) 將USDC存入其平台錢包、(ii) 從市場瀏覽並選擇服務提供者,以及(iii) 透過平台入口網站提交工作並以USDC付款。

\subsubsection{具競爭力的價格}

透過將平台作為計算資源淘寶(算力淘寶平台),使用者可以具競爭力的價格取得最適合他們的計算資源,由於團購優惠,價格通常比直接向雲端服務商購買低10–30\%。

\subsubsection{付款保護與透明度}

平台實施全面的付款保護與透明度措施。智能合約託管持有USDC付款,直到服務交付確認後才釋放,若SP未能達到SLA要求,則自動退款。該系統確保價格透明、無隱藏費用、即時績效監控與報告,以及透過平台治理的爭議解決機制。

\subsubsection{使用者獎勵計畫}

使用者透過多種獲取機制,於平台互動中賺取CyberPlaza Token(CPT)。消費獎勵提供使用者消費金額的1–3\%作為CPT代幣。推薦獎金允許使用者透過為平台帶來新使用者或SP來賺取CPT。忠誠度等級針對持續使用平台提供更高獎勵,而品質回饋機制則讓使用者透過提供詳細的服務評價來賺取CPT。

持有與質押CPT的好處相當豐碩。使用折扣允許使用者質押CPT以獲得服務費用5–15\%的折扣。收入分享使質押的CPT能獲得平台收入分配。治理權利允許對平台參數與功能優先順序進行投票。高級功能提供進階工具、分析與API服務的存取權限。
