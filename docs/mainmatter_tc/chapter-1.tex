\chapter{執行摘要}

計算在現代生活中扮演著日益關鍵的角色,此趨勢預計在可預見的未來將持續發展。Web3 CyberPlaza 網路專案旨在讓個人與機構皆能以開放且包容的方式從此趨勢中獲益。

本專案引進CyberPlaza平台,這是一個去中心化的市集,可被稱為``計算資源版淘寶''(算力淘寶平台)。此平台匹配使用者與服務提供者(SPs)的需求,涵蓋高效能計算、智慧計算及雲端計算領域。使用者可在單一平台上取得多元的算力、儲存空間、軟體應用、資料與計算服務,以具成本效益的方式滿足其特定需求。另一方面,SPs可取得無限制的全球銷售管道。SPs與使用者兩者皆可透過持有CyberPlaza Tokens(CPTs\footnote{CPT是物理學中的守恆量;所有物理定律不得違反CPT守恆,與能量守恆類似。CPT(Carriage Paid To,貨物運費付至)亦是國際貿易術語,指賣方將支付貨物運送至消費者的費用。本代幣名稱承載了這兩層隱喻。})共享平台的成功,CPTs代表此``淘寶''平台的治理股份。

\textbf{付款與結算}:平台上的交易以USDC進行結算,USDC是一種廣泛採用且受監管的穩定幣,可確保符合監管要求並讓使用者感到熟悉。此做法消除了與專屬穩定幣相關的複雜性與監管風險,同時維持透明的美元計價定價。

\textbf{營收模式}:平台透過多種管道產生營收:SaaS訂閱(40--50\%),包含平台使用權的月付/年付訂閱;交易費用(25--30\%),包含計算資源購買的2--5\%費用;API \& 資料服務(15--20\%),提供高級API存取權與分析服務;以及團購(5--10\%),透過大量採購產生利潤作為輔助營收。這些營收透過透明的質押獎勵機制分配給CPT代幣持有者,讓參與者可透過貢獻平台流動性與治理獲得可持續的收益(目標年化收益率6--10\% APY)。

\textbf{去中心化流動性池}:為支援平台的團購營運並確保具競爭力的定價,本專案實施去中心化借貸池,參與者可存入USDC以獲得利息(5--7\% APY)加上CPT獎勵(2--3\% APY),同時為平台提供營運資金。此模式以更透明、可稽核且去中心化的方式取代傳統準備金。

CyberPlaza平台並未直接擁有其上列出的計算資源。相反地,為確保持續供應與具競爭力的定價,平台運用社群提供的流動性,透過受拼多多商業模式啟發的``團購''模式(團購)取得計算資源。

\textbf{關於團購的說明}:雖然團購是我們策略的一部分,但它是一個\textbf{輔助營收管道}(佔總營收的5--10\%),而非核心商業模式。我們的核心價值來自SaaS訂閱與智慧雲端管理工具。隨著平台規模擴大,我們將尋求團購折扣,但我們的核心價值主張不依賴於從雲端服務提供者獲得大量批發折扣。

團購模式減少了消費者與提供者之間的不平衡,也就是目前控制所有計算資源的產業巨頭。這讓平台得以大量取得計算資源並提供給使用者,創造一個充滿活力且蓬勃發展的市集。拼多多式業務的收益透過質押獎勵機制分配給CPT質押者,平台營收的30\%分配給質押池(為提升可持續性,從40\%調降)、35\%用於營運與成長、20\%用於回購與銷毀、10\%用於團隊、5\%用於緊急儲備金。

本專案的核心團隊具備與專案相關領域的豐富經驗,包括分散式高效能計算、公有雲服務、異質計算、AI與大數據應用、分散式系統軟體開發、DeFi投資、商業犯罪預防,以及計算資源的業務與行銷。

總而言之,CyberPlaza Network Web 3專案旨在提供一個去中心化的計算資源市集,讓使用者與服務提供者皆能獲益,同時讓所有參與者可透過持有CPT代幣與質押獎勵共享平台的成功。憑藉強大的核心團隊,以及受淘寶與拼多多等成功平台啟發的創新商業模式,本專案為計算產業的所有參與者創造了一個蓬勃發展、符合規範且可持續的生態系統。
