\chapter{開發路線圖}
\section{路線圖與募資計畫}

\subsection{專案路線圖}

本專案開發採用階段式作法,自2026年第1季展開。初始階段活動涵蓋募資措施、核心團隊擴編、官方網站部署、白皮書發布,以及在Twitter與Discord平台上建立社群。

\textbf{2026年第1季} 展開協定架構與代幣經濟機制的Alpha測試,驗證核心功能與經濟模型參數。

\textbf{2026年第2季} 發布可供公眾參與的測試網,讓社群能在分散式基礎架構上進行測試與意見回饋收集。

\textbf{2026年第3季} 推出可供公眾存取的主網,並同步舉辦初始去中心化交易所發行(Initial DEX Offering, IDO),象徵完整營運部署與代幣分發的展開。

\subsection{募資計畫}

本募資策略採用三階段代幣分配方式。2026年第1季的初始發行將分配5\%的CPT代幣,目標募集\$4M USD資金,讓早期支持者能以優惠估值參與。比較市場分析將本專案與成熟的去中心化運算網路進行定位比較,其中特別提及Golem以低於8,000個核心基礎架構與每月\$30K使用率達到\$200M估值。

2026年第2季與第3季的後續募資回合,將各自以當前市場估值分配5\%的CPT供應量,讓資金募集與達成的里程碑及已證實的網路成長一致。逐步調整的定價反映了平台的成熟度與運算資源的擴張。
