\chapter{發展路線圖}
\section{路線圖與募資計畫}

\subsection{專案路線圖}

本專案開發採用始於2026年第1季的階段式方法。初始階段的活動包含募資舉措、核心團隊擴編、官方網站部署、白皮書發布,以及跨Twitter與Discord平台的社群建置。

\textbf{2026年第1季}展開協議架構與代幣經濟學機制的Alpha測試,驗證核心功能與經濟模型參數。

\textbf{2026年第2季}釋出開放公眾參與的測試網,讓社群可在分散式基礎設施上進行測試並收集回饋。

\textbf{2026年第3季}推出開放公眾存取的主網,同步辦理初始去中心化交易所發行(IDO),標誌著完整營運部署與代幣分配的展開。

\subsection{募資計畫}

本募資策略採用三階段代幣分配方法。2026年第1季的初始發行將分配5\%的CPT代幣,目標為\$4M USD的籌資額,以優惠估值提供早期支持者參與管道。比較市場分析將本專案與成熟的去中心化運算網路對比,特別是Golem的\$200M估值,其核心基礎設施數量低於8,000,月使用率為\$30K。

2026年第2季與第3季的後續募資回合,將分別以當時市場估值分配5\%的CPT供應量,使資金籌集與達成的里程碑及已證實的網路成長一致。階段性定價反映平台的成熟化與運算資源的擴充。
