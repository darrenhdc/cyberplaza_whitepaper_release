\chapter{常見問題集}

\section{常見問題}

\begin{enumerate}
\item \textbf{身為算力淘寶平台的使用者,我能獲得什麼?}

\textbf{答}: 您可從本平台列出的眾多供應商中選擇適合您使用的運算資源,包括CPU、GPU、FPGA算力、儲存空間、應用軟體及服務(例如在特定硬體平台上最佳化您的軟體,或是將您的雲端應用程式從一家雲端供應商遷移至另一家)。您可做出資訊充足的服務選擇,因為運算資源的效能由本平台評估並公佈,SP的SLA由本平台保證,且使用AWS、Azure、GCP以及眾多運算中心與資料中心等資源時可享折扣價格(如淘寶/京東)。此外,透過使用本平台,您將可分享平台所有權,並憑獲得的CPT享有平台利潤的一部分(即您部分擁有的淘寶)。

\item \textbf{使用者對象是誰?一般大眾可能並非運算資源的主要使用者。另一方面,許多機構客戶可能無法參與代幣經濟。}

\textbf{答}: 目前全球一般大眾在公共雲端上的運算使用價值已超過400億美元,而這確實僅是機構客戶使用量的一小部分。對於無法參與代幣經濟的機構客戶,他們可透過本平台的通路合作夥伴,以一般B2B模式購買運算使用量(請參閱白皮書的合作夥伴章節),並以法定貨幣支付。

\item \textbf{部分機構服務供應商,例如AWS或中國的超級運算中心,在提供服務時可能無法接受代幣。我們的使用者如何透過本平台使用他們的資源?}

\textbf{答}: 本平台會以法定貨幣透過「儲備基金」向這些服務供應商購買服務。透過團購(拼多多),本平台可提供優惠的服務。

\item \textbf{為什麼像AWS這樣的雲端供應商會屈服於團購的壓力?}

\textbf{答}: 本平台對AWS而言將是珍貴的銷售通路,可協助其接觸Web 3及DeFi社群。此外,由於本平台上眾多SP之間的競爭壓力,加上透過「儲備池」支付部分預付款的足夠大型團購交易,提供優惠對所有雲端及運算資源供應商而言都是完全合理的。

\item \textbf{假設運作完美,算力淘寶平台的業務規模可能有多大?}

\textbf{答}: AWS的年營收分別為2019年的350億美元、2020年的450億美元、2021年的620億美元、2022年的814億美元(根據Gartner的數據,其中約93\%由機構客戶消費,7\%由個人使用者消費)。如果我們假設全球商業運算業務總價值為AWS的7倍(2022年全球為5,520億美元,即AWS的7倍,來自Allied Market Research),那麼2024年全球總營收將超過1兆美元。如果算力淘寶平台能佔據總市場的0.1\%,那麼每年將超過10億美元,並將快速成長。

\item \textbf{身為流動性提供者或CPT持有者,我能獲得什麼好處?}

\textbf{身為流動性提供者(存入USDC)}: 參與者可從平台營運利潤中獲得5--7\%的USDC年化報酬率(APY),額外獲得2--3\%的CPT代幣年化報酬率(有鎖倉限制),預期總年化報酬率達8--12\%,維持USDC的流動性(可在給予通知期後提領),並在支持平台成長的同時獲得可持續的收益。

\textbf{身為CPT持有者/質押者}: 參與者可質押CPT以獲得8--12\%的年化報酬率(4年鎖倉並使用加成機制最高可達15--20\%),獲得平台40\%利潤的USDC收益分配,受益於通縮型回購與銷毀機制(20\%的營收用於此),享有治理權利(對平台發展方向投票),質押時可獲得平台服務5--15\%的折扣,享有高級功能與優先支援,以及參與新產品推出的早期權利。

\textbf{為何這些報酬是可持續的}: 不同於演算法穩定幣或龐氏騙局,我們的收益來自真實的交易費用(市場活動的2--5\%)、團購利潤(大量採購的10--20\%)、加值服務(認證、訂閱、API),以及透明、可審計的營收來源。

\item \textbf{為什麼資金(無論是鑄幣者或投資者)會願意加入算力淘寶平台,而非其他Web 3專案?}

\textbf{答}: 詳情請參閱「與其他資產代幣化專案的競爭分析」以及「與其他Web 3運算資源專案的競爭分析」頁面。簡而言之:與其他資產相比,算力的價值成長更為迅速;且我們的團隊具備建立算力淘寶平台的特別優異資格。

\item \textbf{部分潛在使用者或服務供應商可能無法參與代幣經濟。他們如何參與?}

\textbf{答}: 這些客戶在本平台找到合適的產品後,可透過本平台的合作夥伴購買(請參閱第10章節列出的代理商)。服務供應商也可透過合作夥伴在本平台上架產品。合作夥伴與使用者/供應商之間的交易可透過法定貨幣進行,無需涉及代幣。

\item \textbf{部分消費者認為淘寶和拼多多上的產品品質較低。本平台如何防範這種情況?}

\textbf{答}: 所有每月掛牌價超過\$10,000 USDC的產品都必須經過本平台認證。如以上第4節所述,本平台要求服務供應商以標準效能測試(包括High-Performance Linpack、High-Performance Conjugate Gradient、STREAM Sustainable Bandwidth、HPC Challenge、MLPerf、ResNet-50 Image Classification、BERT Language Processing、CUDA Benchmark Suite、SPECviewperf graphics performance、DeepBench等)列出其服務的效能。本平台將定期驗證服務供應商聲稱的效能,並將效能指標與服務價格一同列出,供使用者選擇。

\item \textbf{為什麼像AWS或華為雲這樣的公司想要在本平台上銷售他們的服務?}

\textbf{答}: 目前,雲端運算公司會向銷售其服務的分銷商提供優惠。分銷商會僱用銷售團隊來銷售這些服務。在某種意義上,本平台是這些供應商的分銷商,唯透過Web 3架構,供應商可接觸Web 3及DeFi社群。

\end{enumerate}
