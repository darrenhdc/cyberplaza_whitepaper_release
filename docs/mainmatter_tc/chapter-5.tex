\chapter{CyberPlaza代幣(CPT)與代幣經濟學}
\subsection{CPT代幣概覽與效用}

\subsubsection{支付系統}

平臺以USDC作為所有市場交易的主要支付貨幣。此方法消除了與專有穩定幣相關的監管風險,同時確保符合全球穩定幣框架的監管要求、熟悉的使用者體驗(USDC已被廣泛採用且受信任)、透明的美元計價定價、與現有DeFi基礎設施的無縫整合,以及無演算法穩定幣失敗的風險。

\subsubsection{CyberPlaza代幣(CPT)}

CPT是平臺的原生治理與效用代幣,旨在協調所有利害關係人的激勵機制,並捕捉平臺價值成長。

\subsubsection{CPT核心效用}

\paragraph{治理權利}

CPT持有者可對平臺參數(費用結構、收益分配比例等)進行投票,對新功能、合作夥伴和戰略方向提出並投票,並參與財庫管理和資本配置決策。投票權重基於質押的CPT數量和鎖倉期限(veToken模型)。平臺實施每季治理會議和透明的提案流程。

\paragraph{透過質押分潤}

持有者可質押CPT以賺取平臺收益分配(以USDC支付)。平臺收益的30\%分配至質押獎勵池(針對永續性優化)。質押獎勵以週或月為單位分配(由治理決定)。較長的質押期可獲得獎勵倍數(4年鎖倉最高可達2.5倍)。目標APY為6--10\%,取決於平臺表現(更具永續性)和質押比例。無非永久性損失風險(單資產質押)。

\paragraph{使用優惠}

質押CPT可享平臺服務5--15\%的折扣(分級制度)、存取高級功能(包括進階分析、API存取和優先支援)、高交易量使用者的交易費折扣、新服務和測試版功能的提前存取,以及高需求運算資源的優先分配。

\paragraph{生態系激勵}

平臺為所有使用者類別提供激勵。使用者可獲得消費金額的1--3\%作為CPT(現金回饋計畫)。服務提供者可獲得交易量的2--5\%作為CPT獎勵。流動性提供者可獲得2--4\%的CPT APY作為額外收益。推薦者可因帶來新使用者或SP而獲得CPT。社群貢獻透過漏洞獎金、內容創作和程式碼貢獻獲得獎勵。

\paragraph{通縮機制}

20\%的平臺收益用於從公開市場回購CPT。購買的CPT代幣將永久銷毀(發送至0x0地址),這會隨時間減少流通供應,創造稀缺性。平臺實施透明的每季銷毀活動,並透過鏈上驗證,預計5年內供應量將減少30--40\%。這有利於所有CPT持有者,不僅限於質押者。

\subsection{收益模型與分配機制}

\subsubsection{平臺收益來源}

平臺透過多個來源產生收益,如表~\ref{tab:revenue}所示。

\begin{table}[htbp]
\centering
\caption{平臺收益預測}
\label{tab:revenue}
\begin{tabular}{lccccr}
\hline
\textbf{收益來源} & \textbf{費率/金額} & \textbf{第1年} & \textbf{第2年} & \textbf{第3年} & \textbf{占總額百分比} \\
\hline
\textbf{SaaS訂閱} & \$50--500/月 & \$1.5M & \$4M & \$8--10M & \textbf{40--50\%} \\
交易費用 & GMV的2--5\% & \$0.8M & \$2.5M & \$5--7M & \textbf{25--30\%} \\
API與數據服務 & 變動 & \$0.3M & \$1.5M & \$3--4M & \textbf{15--20\%} \\
認證服務 & 每SP \$5K--50K & \$0.3M & \$0.8M & \$1--2M & \textbf{5--8\%} \\
團購利潤 & 5--10\%利潤 & \$0.2M & \$0.7M & \$1.5--2M & \textbf{5--10\%} \\
\hline
\textbf{總收益} & --- & \textbf{\$3.1M} & \textbf{\$9.5M} & \textbf{\$19--25M} & \textbf{100\%} \\
\hline
\end{tabular}
\end{table}

與原始模型的主要變化包括:SaaS訂閱現在作為主要收益來源(40--50\%)以確保可預測性,團購減少為輔助收益(5--10\%),考慮到早期規模這更為現實,以及強調API服務(15--20\%)作為高利潤、可擴展的收益來源。保守預測基於第3年0.01\%的市場滲透率。

\subsubsection{SaaS訂閱等級}

平臺提供分級訂閱方案,如表~\ref{tab:saas}所示。

\begin{table}[htbp]
\centering
\caption{SaaS訂閱等級(說明性)}
\label{tab:saas}
\begin{tabularx}{\textwidth}{llXXr}
\hline
\textbf{等級} & \textbf{價格/月} & \textbf{目標使用者} & \textbf{功能} & \textbf{預估使用者(第3年)} \\
\hline
免費 & \$0 & 個人 & 2個雲端帳號、基本監控 & 10,000+ \\
入門 & \$50 & 小型團隊 & 5個帳號、成本追蹤、1\% CPT現金回饋 & 2,000 \\
專業 & \$200 & 開發團隊 & 10個帳號、AI最佳化、API、3\% CPT & 500 \\
企業 & \$500--2000 & 企業 & 無限制、自訂整合、5\% CPT & 50--100 \\
\hline
\end{tabularx}
\end{table}

此分級模型提供可預測的重複收益,同時仍允許免費增值使用者取得。

\textbf{重要注意事項}:這些預測代表我們的目標情境。我們也模擬了保守情境,第1年收益為\$500K--1M,以確保即使初始增長較慢也能維持財務永續性。我們的商業模式不依賴立即實現大規模團購折扣。

\subsubsection{收益分配模型}

平臺收益(100\%)分配如下:質押獎勵池獲得30\%(為永續性而降低),並按比例以USDC分配給CPT質押者。營運與開發獲得35\%(為成長而增加),分配至工程與產品開發(15\%)、行銷與業務開發(10\%),以及基礎設施與安全(5\%)。回購與銷毀獲得20\%,其中CPT從DEX購買並永久銷毀。團隊與基金會獲得10\%,用於核心團隊薪酬(5\%)和基金會營運(5\%)。緊急準備金獲得5\%,作為新的波動緩衝。

\subsubsection{質押獎勵計算範例}

考慮第3年平臺成熟的情境,平臺每月收益為\$1,500,000。質押池分配(40\%)提供\$600,000。如果質押的CPT總量為40,000,000(供應量的40\%),而您的質押量為10,000 CPT(質押總量的0.025\%),則您的每月獎勵為\$600,000 $\times$ 0.025\% = \$150 USDC,年獎勵為\$150 $\times$ 12 = \$1,800 USDC。

如果CPT price = \$2,則您的質押價值為\$20,000,APY為\$1,800 / \$20,000 = 9\%。加上額外好處包括平臺治理的投票權、服務折扣(5--15\%),以及回購/銷毀帶來的價格上漲。

\subsubsection{USDC存款人的流動性池報酬}

將USDC存入借貸池的流動性提供者可獲得如表~\ref{tab:liquidity}所示的報酬。

\begin{table}[htbp]
\centering
\caption{流動性提供者報酬}
\label{tab:liquidity}
\begin{tabular}{llll}
\hline
\textbf{項目} & \textbf{APY} & \textbf{支付貨幣} & \textbf{來源} \\
\hline
基礎利率 & 6--8\% & USDC & 平臺營運利潤 \\
CPT激勵 & 2--4\% & CPT & 代幣發放(歸屬) \\
\textbf{預期總額} & \textbf{8--12\%} & \textbf{混合} & \textbf{永續收益} \\
\hline
\end{tabular}
\end{table}

主要功能包括存款用於團購營運(透過鏈上透明追蹤)、漸進式提款系統防止擠兌情境、保險基金覆蓋高達10\%的池TVL、智慧合約由領先公司審計,以及基於池使用率的即時APY更新。

\subsection{代幣分配與歸屬時程表}

\subsubsection{總供應量與分配}

總供應量為100,000,000 CPT(固定,無通貨膨脹)。分配細節如表~\ref{tab:allocation}所示。
\begin{table}[htbp]
\centering
\caption{CPT代幣分配}
\label{tab:allocation}
\begin{tabularx}{\textwidth}{lrrr>{\raggedright\arraybackslash}X}
\hline
\textbf{類別} & \textbf{分配總額} & \textbf{代幣數量} & \textbf{百分比} & \textbf{鎖倉與歸屬條款} \\
\hline
\textbf{社群激勵} & \textbf{總額} & \textbf{55,000,000} & \textbf{55\%} & \textbf{基於績效釋放} \\
\quad - 使用者獎勵 & & 25,000,000 & 25\% & 基於平臺GMV里程碑釋放 \\
\quad - SP激勵 & & 20,000,000 & 20\% & 基於交易量釋放 \\
\quad - LP獎勵 & & 10,000,000 & 10\% & 5年發放,前載式 \\
\textbf{基金會} & & \textbf{17,500,000} & \textbf{17.5\%} & \textbf{10\%於TGE釋放,90\%線性歸屬24個月} \\
\textbf{私募} & & \textbf{12,500,000} & \textbf{12.5\%} & \textbf{6個月鎖定期,18個月線性歸屬} \\
\textbf{團隊} & & \textbf{15,000,000} & \textbf{15\%} & \textbf{12個月鎖定期,36個月線性歸屬} \\
\hline
\textbf{總計} & & \textbf{100,000,000} & \textbf{100\%} & \\
\hline
\end{tabularx}
\end{table}

與原始模型的主要變化包括:社群分配從50\%增加到55\%(移除USDC持有者分配),投資者分配從15\%減少到12.5\%(社群優先方針),團隊分配從17.5\%減少到15\%(更強的一致性),以及取消「流動性提供者」類別(替換為流動性提供者激勵)。

\subsubsection{歸屬細節}

\paragraph{社群激勵(55\%)}

使用者獎勵(25M CPT)按月基於平臺GMV目標釋放。公式為:每月釋放 = 基礎金額 $\times$ (實際GMV / 目標GMV)。分配期為5年,未領取的代幣順延至下一期。

SP激勵(20M CPT)按季基於交易量釋放。品質較高的SP(CSPs)可獲得獎勵倍數。分配期為5年,可基於績效加速釋放。

LP獎勵(10M CPT)採取前載式發放:第1年(40\%)、第2年(30\%)、第3--5年(30\%)。每周分配給活躍流動性提供者,長期存款可獲得獎勵。歸屬為50\%立即釋放,50\%在6個月內線性歸屬。

\paragraph{團隊分配(15\%)}

團隊分配包括12個月的鎖定期(第一年無代幣釋放)。鎖定期後,進行36個月的線性歸屬。總歸屬期為4年。歸屬合約透明且可公開驗證。

\paragraph{基金會分配(17.5\%)}

10\%於TGE釋放用於初始營運(由多重簽名控制)。剩餘90\%在24個月內線性歸屬。這些資金用於合作夥伴關係、審計、法律、行銷和贈款,並每季度發布透明報告。

\paragraph{私募(12.5\%)}

私募包括6個月的鎖定期,鎖定期後進行18個月的線性歸屬。總歸屬期為2年。反拋售機制限制每日交易量的最高5\%。

\subsection{流動性激勵與veToken質押模型}

\subsubsection{veToken機制(投票質押CPT)}

我們實施受Curve Finance啟發的veToken模型,已證明可協調長期激勵機制。使用者鎖定CPT以獲得veCPT(不可轉讓)。鎖定期限決定veCPT倍數,如表~\ref{tab:vetoken}所示。

\begin{table}[htbp]
\centering
\caption{不同鎖定期限的veToken倍數}
\label{tab:vetoken}
\begin{tabular}{ll}
\hline
\textbf{鎖定期限} & \textbf{veCPT倍數} \\
\hline
1周 & 0.01x \\
1個月 & 0.04x \\
3個月 & 0.25x \\
6個月 & 0.50x \\
1年 & 1.00x \\
2年 & 1.50x \\
4年 & 2.50x(最大值) \\
\hline
\end{tabular}
\end{table}

\subsubsection{veCPT的好處}

增強的治理權力提供1 veCPT = 1票(相較於標準CPT:除非鎖定否則無投票權),鎖定期限越長,在平臺方向的發言權越強。

提升的質押獎勵包括基礎APY 8--12\%(1年鎖倉),最高2.5倍提升(4年鎖倉),最大鎖倉可達到20--30\%的提升APY。

分潤優先權表示veCPT持有者優先獲得收益分配,veCPT餘額越高,費用池的分配比例越高。

獨家好處包括最高15\%的服務折扣、優先存取超額訂閱資源,以及獨家治理提案權(需達到最低veCPT門檻)。

\subsubsection{流動性挖礦計畫}

階段1:上市激勵(第1–6個月)提供高CPT排放以引導流動性。Uniswap V3上的CPT/USDC池每天獲得2000 CPT。CPT單質押每天獲得1500 CPT。USDC借貸池每天獲得相當於1000 CPT的獎勵。

階段2:成長(第7–24個月)減少排放,專注於永續收益。總排放約為每天2500 CPT,增加USDC借貸池的權重(激勵流動性)。

階段3:成熟(第25個月後)新排放降至最低(約每天1000 CPT)。收益驅動的收益成為主要吸引力,回購與銷毀創造供應稀缺性。

\subsubsection{反巨鯨與公平上市機制}

平臺實施多項保護機制,包括私募單次購買限額最高\$100K、歸屬機制確保TGE無大量拋售、時間加權投票防止治理攻擊、漸進式排放防止挖礦拋售,以及社群分配大於團隊+投資者(55\% > 27.5\%)。

\subsubsection{比較:傳統模型與veCPT模型}

表~\ref{tab:comparison}比較了傳統質押與veCPT模型。

\begin{table}[htbp]
\centering
\caption{傳統質押與veCPT模型比較}
\label{tab:comparison}
\begin{tabular}{lll}
\hline
\textbf{指標} & \textbf{傳統質押} & \textbf{veCPT模型} \\
\hline
最低承諾 & 無 & 1周 \\
最大獎勵 & 固定APY & 最高2.5倍提升 \\
治理權力 & 線性(1代幣=1票) & 時間加權 \\
長期一致性 & 低 & 高 \\
投機資金風險 & 高 & 低 \\
價格穩定性 & 較低 & 較高 \\
\hline
\end{tabular}
\end{table}

為何此模型有效:Curve(\$CRV)已證明其有效性,並自2020年以來經過戰鬥測試。它協調了長期持有者的激勵機制,減少了短期挖礦者的拋售壓力,創造了強大的治理參與度,並提供了不依賴永久通貨膨脹的永續代幣經濟學。

\subsection{上市策略與保守情境}

\subsubsection{冷啟動策略}

成功推出雙邊市場需要謹慎的時序安排。我們的方法分為三個階段。

\paragraph{階段0:種子使用者(第1–3個月)}

目標為50–100位付費使用者。來源包括ClusterTech現有客戶群和Web3專案。激勵包括3個月免費試用、早期採用者終身50\%折扣,以及初始CPT空投(總預算100K CPT)。預算約為\$150K(行銷+激勵)。

\paragraph{階段1:早期採用者(第3–12個月)}

目標為500–1000位付費使用者和10家企業客戶。策略包括推薦計畫(推薦者和被推薦者均獲得\$50信用額)、透過技術部落格和YouTube教學進行內容行銷、黑客松贊助(Web3社群),以及雲端轉售合作夥伴關係。預算約為\$500K(行銷+銷售)。

\paragraph{階段2:成長(第12–24個月)}

目標為2000–5000位使用者和50家企業客戶。策略包括CPT質押激勵全面啟動、策略合作夥伴關係(Infura、Alchemy等),以及會議出席和思想領導。預算為\$1M+(隨收益擴大)。

\subsubsection{財務情境}

為向投資者提供透明度,我們模擬了三種情境。

\paragraph{保守情境(高機率)}

表~\ref{tab:conservative}呈現保守財務情境。

\begin{table}[htbp]
\centering
\caption{保守財務情境}
\label{tab:conservative}
\begin{tabular}{lrrr}
\hline
\textbf{指標} & \textbf{第1年} & \textbf{第2年} & \textbf{第3年} \\
\hline
付費使用者 & 200 & 1,000 & 3,000 \\
ARPU(\$/月) & \$40 & \$60 & \$80 \\
MRR & \$8K & \$60K & \$240K \\
年度收益 & \$96K & \$720K & \$2.9M \\
營運成本 & \$600K & \$900K & \$1.5M \\
淨收入 & -\$504K & -\$180K & +\$1.4M \\
累積現金 & -\$500K & -\$680K & +\$720K \\
\hline
\end{tabular}
\end{table}

\paragraph{基本情境(中機率)}

表~\ref{tab:basecase}呈現基本財務情境。

\begin{table}[htbp]
\centering
\caption{基本財務情境}
\label{tab:basecase}
\begin{tabular}{lrrr}
\hline
\textbf{指標} & \textbf{第1年} & \textbf{第2年} & \textbf{第3年} \\
\hline
付費使用者 & 500 & 2,500 & 8,000 \\
ARPU(\$/月) & \$50 & \$75 & \$100 \\
MRR & \$25K & \$188K & \$800K \\
年度收益 & \$300K & \$2.25M & \$9.6M \\
營運成本 & \$800K & \$1.5M & \$3M \\
淨收入 & -\$500K & +\$750K & +\$6.6M \\
\hline
\end{tabular}
\end{table}

\paragraph{樂觀情境(低機率)}

表~\ref{tab:optimistic}呈現樂觀財務情境。

\begin{table}[htbp]
\centering
\caption{樂觀財務情境}
\label{tab:optimistic}
\begin{tabular}{lrrr}
\hline
\textbf{指標} & \textbf{第1年} & \textbf{第2年} & \textbf{第3年} \\
\hline
付費使用者 & 1,000 & 5,000 & 20,000 \\
ARPU(\$/月) & \$75 & \$100 & \$150 \\
MRR & \$75K & \$500K & \$3M \\
年度收益 & \$900K & \$6M & \$36M \\
營運成本 & \$1M & \$2.5M & \$8M \\
淨收入 & -\$100K & +\$3.5M & +\$28M \\
\hline
\end{tabular}
\end{table}

\paragraph{關鍵假設}

情境反映不同的市場滲透率和定價能力。營運成本隨成長擴大,但受益於規模經濟。保守情境假設團購貢獻最小。所有情境假設主要收益來自SaaS和交易費用。CPT激勵成本包含在營運成本中。

\paragraph{融資需求}

種子/天使融資\$500K--1M將覆蓋第1年損失和產品開發。如果確認達到基本情境軌跡,第2年計畫進行\$3--5M的A輪融資。第3年後計畫進行\$10--20M的B輪融資,用於國際擴張。

\paragraph{損益平衡分析}

保守情境在第30--36個月達到損益平衡。基本情境在第18--24個月達到損益平衡。樂觀情境在第12--18個月達到損益平衡。此範圍為投資者提供了現實的預期,同時展示了可擴展性潛力。
