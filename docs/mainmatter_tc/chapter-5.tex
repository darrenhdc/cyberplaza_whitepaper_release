\chapter{CyberPlaza Token (CPT) 與代幣經濟學}
\subsection{CPT 代幣概覽與效用}

\subsubsection{支付系統}

平台使用 USDC 作為所有市場交易的主要支付貨幣。此做法消除了與私有穩定幣相關的監管風險,同時確保符合全球穩定幣框架的監管要求,具備熟悉的使用者體驗(USDC 被廣泛採用且受信任)、透明的美元計價定價、與現有 DeFi 基礎設施的無縫整合,以及沒有演算法穩定幣崩潰的風險。

\subsubsection{CyberPlaza Token (CPT)}

CPT 是平台的原生治理與效用代幣,旨在協調所有利害關係人之間的激勵措施,並捕捉平台價值增長。

\subsubsection{CPT 核心效用}

\paragraph{治理權利}

CPT 持有者可對平台參數(費用結構、收入分配比例等)進行投票,提出並投票支持新功能、夥伴關係與策略方向,並參與財庫管理與資本配置決策。投票權重基於質押的 CPT 數量與鎖倉期限(veToken 模型)。平台實施每季治理會議與透明的提案流程。

\paragraph{透過質押的收入分潤}

持有者可質押 CPT 以獲得平台收入分配(以 USDC 支付)。平台收入的 30\% 分配至質押獎勵池(針對永續性進行優化)。質押獎勵每週或每月發放(由治理決定)。較長的質押期限可獲得獎勵乘數(4 年鎖倉最高可達 2.5 倍)。根據平台績效(更具永續性)與質押比例,目標 APY 為 6–10\%。無無常損失風險(單資產質押)。

\paragraph{使用優惠}

質押 CPT 可享受平台服務 5–15\% 的折扣(分層制度),獲得包括進階分析、API 存取權與優先支援在內的高級功能,高交易量使用者可獲得交易費用減免,可提早使用新服務與測試版功能,以及優先分配高需求運算資源。

\paragraph{生態系激勵措施}

平台針對所有使用者類別提供激勵措施。使用者可獲得消費金額 1–3\% 的 CPT 回饋(現金回饋計畫)。服務供應商(SP)可獲得交易額 2–5\% 的 CPT 獎金。流動性提供者(LP)可獲得 2–4\% 的 CPT 代幣 APY 作為額外收益。推薦者可因介紹新使用者或 SP 至平台而獲得 CPT。社區貢獻透過漏洞獎勵、內容創作與程式碼貢獻獲得獎勵。

\paragraph{通縮機制}

平台收入的 20\% 用於從公開市場回購 CPT。購買的 CPT 代幣將被永久銷毀(發送至 0x0 位址),這將隨時間減少流通供應量,創造稀缺性。平台實施透明的每季銷毀活動,並透過鏈上驗證,預計在 5 年內將供應量減少 30–40\%。這將使所有 CPT 持有者受益,不僅僅是質押者。

\subsection{收入模型與分配機制}

\subsubsection{平台收入來源}

平台透過多種渠道產生收入,如下表 \ref{tab:revenue} 所示。

\begin{table}[htbp]
\centering
\caption{平台收入預測}
\label{tab:revenue}
\begin{tabular}{lccccr}
\hline
\textbf{收入來源} & \textbf{費率/金額} & \textbf{第1年} & \textbf{第2年} & \textbf{第3年} & \textbf{佔總額百分比} \\
\hline
\textbf{SaaS 訂閱} & \$50--500/month & \$1.5M & \$4M & \$8--10M & \textbf{40--50\%} \\
交易費用 & 2--5\% of GMV & \$0.8M & \$2.5M & \$5--7M & \textbf{25--30\%} \\
API \& 資料服務 & 變動 & \$0.3M & \$1.5M & \$3--4M & \textbf{15--20\%} \\
認證服務 & \$5K--50K per SP & \$0.3M & \$0.8M & \$1--2M & \textbf{5--8\%} \\
團購利潤 & 5--10\% 利潤率 & \$0.2M & \$0.7M & \$1.5--2M & \textbf{5--10\%} \\
\hline
\textbf{總收入} & --- & \textbf{\$3.1M} & \textbf{\$9.5M} & \textbf{\$19--25M} & \textbf{100\%} \\
\hline
\end{tabular}
\end{table}

與原始模型的主要變更包括,SaaS 訂閱現在作為主要收入來源(40–50\%)以確保可預測性,團購縮減為補充收入來源(5–10\%),這在早期規模下更為現實,以及強調 API 服務(15–20\%)作為高邊際、可擴展的收入。保守預測基於第 3 年 0.01\% 的市場滲透率。

\subsubsection{SaaS 訂閱層級}

平台提供分層訂閱方案,如下表 \ref{tab:saas} 所示。

\begin{table}[htbp]
\centering
\caption{SaaS 訂閱層級(說明性)}
\label{tab:saas}
\begin{tabularx}{\textwidth}{llXXr}
\hline
\textbf{層級} & \textbf{月費} & \textbf{目標使用者} & \textbf{功能} & \textbf{預估使用者(Y3)} \\
\hline
免費 & \$0 & 個人 & 2 個雲端帳戶、基礎監控 & 10,000+ \\
入門級 & \$50 & 小型團隊 & 5 個帳戶、成本追蹤、1\% CPT 回饋 & 2,000 \\
專業級 & \$200 & 開發團隊 & 10 個帳戶、AI 最佳化、API、3\% CPT & 500 \\
企業級 & \$500--2000 & 公司 & 無限、客製化整合、5\% CPT & 50--100 \\
\hline
\end{tabularx}
\end{table}

這種分層模型提供可預測的重複收入,同時仍允許免費增值使用者獲取。

\textbf{重要注意事項}:這些預測代表我們的目標情境。我們也建模了保守情境,第 1 年收入為 \$500K--1M,以確保即使初始成長較慢也能維持財務永續性。我們的商業模式不依賴立即實現大規模團購折扣。

\subsubsection{收入分配模型}

平台收入(100\%)分配如下:質押獎勵池獲得 30\%(為永續性而調降),並以 USDC 比例分配給 CPT 質押者。營運與開發獲得 35\%(為成長而調升),分配至工程與產品開發(15\%)、行銷與業務開發(10\%),以及基礎設施與安全性(5\%)。回購與銷毀獲得 20\%,其中 CPT 從去中心化交易所(DEX)購買並永久銷毀。團隊與基金會獲得 10\%,用於核心團隊報酬(5\%)與基金會營運(5\%)。緊急準備金獲得 5\%,作為對抗波動的新緩衝。

\subsubsection{質押獎勵計算範例}

考量第 3 年平台每月收入 \$1,500,000 的成熟平台情境。質押池分配(40\%)提供 \$600,000。如果質押的總 CPT 為 40,000,000(供應量的 40\%),而您的質押量為 10,000 CPT(質押供應量的 0.025\%),那麼您的每月獎勵為 \$600,000 × 0.025\% = \$150 USDC,年度獎勵為 \$150 × 12 = \$1,800 USDC。

若 CPT 價格 = \$2,則您的質押價值為 \$20,000,您的 APY 為 \$1,800 / \$20,000 = 9\%。此外的額外福利包括平台治理的投票權、服務折扣(5–15\%),以及回購/銷毀帶來的價格上漲。

\subsubsection{USDC 存款人流動性池報酬}

將 USDC 存入借貸池的流動性提供者可獲得如下表 \ref{tab:liquidity} 所示的報酬。

\begin{table}[htbp]
\centering
\caption{流動性提供者報酬}
\label{tab:liquidity}
\begin{tabular}{llll}
\hline
\textbf{組成部分} & \textbf{APY} & \textbf{支付貨幣} & \textbf{來源} \\
\hline
基礎利息 & 6--8\% & USDC & 平台營運利潤 \\
CPT 激勵 & 2--4\% & CPT & 代幣釋放(鎖倉釋放) \\
\textbf{預期總額} & \textbf{8--12\%} & \textbf{混合} & \textbf{永續收益} \\
\hline
\end{tabular}
\end{table}

主要特色包括存款用於團購營運(透過鏈上追蹤確保透明)、逐步提款系統防止擠兌情境、保險基金涵蓋最高 10\% 的池總鎖倉價值(TVL)、智慧合約由領先公司審計,以及基於池利用率的即時 APY 更新。

\subsection{代幣分配與鎖倉釋放計畫}

\subsubsection{總供應量與分配}

總供應量為 100,000,000 CPT(固定,無通貨膨脹)。分配細節如下表 \ref{tab:allocation} 所示。
\begin{table}[htbp]
\centering
\caption{CPT 代幣分配}
\label{tab:allocation}
\begin{tabularx}{\textwidth}{lrrr>{\raggedright\arraybackslash}X}
\hline
\textbf{類別} & \textbf{分配額} & \textbf{代幣數量} & \textbf{百分比} & \textbf{鎖倉與釋放條款} \\
\hline
\textbf{社群激勵} & \textbf{總額} & \textbf{55,000,000} & \textbf{55\%} & \textbf{基於績效的釋放} \\
\quad - 使用者獎勵 & & 25,000,000 & 25\% & 根據平台 GMV 里程碑釋放 \\
\quad - SP 激勵 & & 20,000,000 & 20\% & 根據交易額釋放 \\
\quad - LP 激勵 & & 10,000,000 & 10\% & 5 年釋放,前載式 \\
\textbf{基金會} & & \textbf{17,500,000} & \textbf{17.5\%} & \textbf{TGE 時釋放 10\%,剩餘 90\% 24 個月線性釋放} \\
\textbf{私募} & & \textbf{12,500,000} & \textbf{12.5\%} & \textbf{6 個月鎖定期,爾後 18 個月線性釋放} \\
\textbf{團隊} & & \textbf{15,000,000} & \textbf{15\%} & \textbf{12 個月鎖定期,爾後 36 個月線性釋放} \\
\hline
\textbf{總額} & & \textbf{100,000,000} & \textbf{100\%} & \\
\hline
\end{tabularx}
\end{table}

與原始模型的主要變更包括社群分配從 50\% 增加至 55\%(移除 USDC 持有者分配)、投資人分配從 15\% 減少至 12.5\%(以社群為優先的方針)、團隊分配從 17.5\% 減少至 15\%(更強的利益一致性),以及取消「流動性提供者」類別(替換為流動性提供者激勵)。

\subsubsection{釋放細節}

\paragraph{社群激勵(55\%)}

使用者獎勵(25M CPT)根據平台 GMV 目標每月釋放。公式為:每月釋放額 = 基礎金額 ×(實際 GMV / 目標 GMV)。分配期為 5 年,未領取的代幣將滾存至下一期。

SP 激勵(20M CPT)根據交易額每季釋放。高品質 SP(CSP)可獲得獎勵乘數。分配期為 5 年,可基於績效加速釋放。

LP 激勵(10M CPT)採用前載式釋放:第 1 年(40\%)、第 2 年(30\%)、第 3–5 年(30\%)。每周分配給活躍的流動性提供者,長期存款可獲得獎金。釋放為 50\% 立即釋放,50\% 在 6 個月內線性釋放。

\paragraph{團隊分配(15\%)}

團隊分配包括 12 個月的鎖定期(第一年無代幣釋放)。鎖定期結束後,進行 36 個月的線性釋放。總釋放期為 4 年。釋放合約透明且可公開驗證。

\paragraph{基金會分配(17.5\%)}

10\% 於 TGE 時釋放用於初始營運(由多重簽名控制)。剩餘 90\% 在 24 個月內線性釋放。這些資金用於夥伴關係、審計、法律、行銷與補助,並每季度發布透明報告。

\paragraph{私募(12.5\%)}

私募包括 6 個月的鎖定期,鎖定期結束後進行 18 個月的線性釋放。總釋放期為 2 年。反拋售機制限制每日交易量的最大賣出比例為 5\%。

\subsection{流動性激勵與 veToken 質押模型}

\subsubsection{veToken 機制(Vote-Escrowed CPT)}

我們實施了受 Curve Finance 啟發的 veToken 模型,該模型已被證明可協調長期激勵措施。使用者鎖倉 CPT 以獲得 veCPT(不可轉移)。鎖倉期限決定 veCPT 乘數,如下表 \ref{tab:vetoken} 所示。

\begin{table}[htbp]
\centering
\caption{按鎖倉期限劃分的 veToken 乘數}
\label{tab:vetoken}
\begin{tabular}{ll}
\hline
\textbf{鎖倉期限} & \textbf{veCPT 乘數} \\
\hline
1 週 & 0.01x \\
1 個月 & 0.04x \\
3 個月 & 0.25x \\
6 個月 & 0.50x \\
1 年 & 1.00x \\
2 年 & 1.50x \\
4 年 & 2.50x(最大值) \\
\hline
\end{tabular}
\end{table}

\subsubsection{veCPT 的優勢}

增強的治理權力提供 1 veCPT = 1 票(相較於標準 CPT:除非鎖倉否則無投票權),鎖倉期限越長,在平台方向上的發言權越強。

提升的質押獎勵包括 8–12\% 的基礎 APY(1 年鎖倉),最高 2.5 倍的提升(4 年鎖倉),以及最大鎖倉的提升後 APY 高達 20–30\%。

費用分潤優先順位意味著 veCPT 持有者首先獲得收入分配,且 veCPT 餘額越高,費用池的分額越高。

獨家優惠包括最高 15\% 的服務折扣、優先存取超額訂閱資源,以及獨家治理提案權(需達到最低 veCPT 要求)。

\subsubsection{流動性挖礦計畫}

第 1 階段:發布激勵(第 1–6 個月)以高 CPT 釋放來建立流動性。Uniswap V3 上的 CPT/USDC 池每日獲得 2000 CPT。CPT 單資產質押每日獲得 1500 CPT。USDC 借貸池獲得相當於每日 1000 CPT 的激勵。

第 2 階段:成長(第 7–24 個月)減少釋放,專注於永續收益。總釋放約為每日 2500 CPT,增加 USDC 借貸池的權重(激勵流動性)。

第 3 階段:成熟(第 25 個月起)新釋放極少(約每日 1000 CPT)。收入驅動的收益成為主要吸引力,回購與銷毀創造供應稀缺性。

\subsubsection{反大戶與公平發布機制}

平台實施多項保護機制,包括私募中的單筆最高購買限額為 \$100K、確保 TGE 時無大規模拋售的鎖倉釋放計畫、防止治理攻擊的時間加權投票、防止挖礦後拋售的逐步釋放,以及社群分配高於團隊+投資人(55\% > 27.5\%)。

\subsubsection{比較:傳統 vs. veCPT 模型}

下表 \ref{tab:comparison} 比較傳統質押與 veCPT 模型。

\begin{table}[htbp]
\centering
\caption{傳統質押 vs. veCPT 模型}
\label{tab:comparison}
\begin{tabular}{lll}
\hline
\textbf{指標} & \textbf{傳統質押} & \textbf{veCPT 模型} \\
\hline
最低承諾 & 無 & 1 週 \\
最高獎勵 & 固定 APY & 最高 2.5x 提升 \\
治理權力 & 線性(1 代幣 = 1 票) & 時間加權 \\
長期一致性 & 低 & 高 \\
僱傭資本風險 & 高 & 低 \\
價格穩定性 & 較低 & 較高 \\
\hline
\end{tabular}
\end{table}

為什麼此模型有效:它已被 Curve(\$CRV)證明,並自 2020 年以來經過戰場測試。它協調長期持有者的激勵措施,減少短期挖礦者的賣壓,創造強大的治理參與度,並提供不依賴永久通貨膨脹的永續代幣經濟學。

\subsection{上市策略與保守情境}

\subsubsection{冷啟動策略}

成功推出雙邊市場需要謹慎的順序規劃。我們的方法分為三個階段。

\paragraph{第 0 階段:種子使用者(第 1–3 個月)}

目標為 50–100 名付費使用者。來源包括 ClusterTech 現有客戶群與 Web3 專案。激勵措施包括 3 個月免費試用、早期採用者終生 50\% 折扣,以及初始 CPT 空投(總預算 100K CPT)。預算約為 \$150K(行銷 + 激勵措施)。

\paragraph{第 1 階段:早期採用者(第 3–12 個月)}

目標為 500–1000 名付費使用者與 10 家企業客戶。策略包括推薦計畫(推薦者與被推薦者各獲得 \$50 信用額)、透過技術部落格與 YouTube 教學的內容行銷、黑客松贊助(Web3 社群),以及雲端轉售商夥伴關係。預算約為 \$500K(行銷 + 銷售)。

\paragraph{第 2 階段:成長(第 12–24 個月)}

目標為 2000–5000 名使用者與 50 家企業客戶。策略包括全面啟動 CPT 質押激勵措施、策略夥伴關係(Infura、Alchemy 等),以及會議參與與思想領袖地位。預算為 \$1M+(隨收入擴展)。

\subsubsection{財務情境}

為向投資人提供透明度,我們建模了三種情境。

\paragraph{保守情境(高機率)}

下表 \ref{tab:conservative} 呈現保守財務情境。

\begin{table}[htbp]
\centering
\caption{保守財務情境}
\label{tab:conservative}
\begin{tabular}{lrrr}
\hline
\textbf{指標} & \textbf{第1年} & \textbf{第2年} & \textbf{第3年} \\
\hline
付費使用者 & 200 & 1,000 & 3,000 \\
ARPU (\$/month) & \$40 & \$60 & \$80 \\
MRR & \$8K & \$60K & \$240K \\
年度收入 & \$96K & \$720K & \$2.9M \\
營運成本 & \$600K & \$900K & \$1.5M \\
淨利 & -\$504K & -\$180K & +\$1.4M \\
累積現金 & -\$500K & -\$680K & +\$720K \\
\hline
\end{tabular}
\end{table}

\paragraph{基本情境(中機率)}

下表 \ref{tab:basecase} 呈現基本情境財務情境。

\begin{table}[htbp]
\centering
\caption{基本情境財務情境}
\label{tab:basecase}
\begin{tabular}{lrrr}
\hline
\textbf{指標} & \textbf{第1年} & \textbf{第2年} & \textbf{第3年} \\
\hline
付費使用者 & 500 & 2,500 & 8,000 \\
ARPU (\$/month) & \$50 & \$75 & \$100 \\
MRR & \$25K & \$188K & \$800K \\
年度收入 & \$300K & \$2.25M & \$9.6M \\
營運成本 & \$800K & \$1.5M & \$3M \\
淨利 & -\$500K & +\$750K & +\$6.6M \\
\hline
\end{tabular}
\end{table}

\paragraph{樂觀情境(低機率)}

下表 \ref{tab:optimistic} 呈現樂觀財務情境。

\begin{table}[htbp]
\centering
\caption{樂觀財務情境}
\label{tab:optimistic}
\begin{tabular}{lrrr}
\hline
\textbf{指標} & \textbf{第1年} & \textbf{第2年} & \textbf{第3年} \\
\hline
付費使用者 & 1,000 & 5,000 & 20,000 \\
ARPU (\$/month) & \$75 & \$100 & \$150 \\
MRR & \$75K & \$500K & \$3M \\
年度收入 & \$900K & \$6M & \$36M \\
營運成本 & \$1M & \$2.5M & \$8M \\
淨利 & -\$100K & +\$3.5M & +\$28M \\
\hline
\end{tabular}
\end{table}

\paragraph{主要假設}

情境反映不同的市場滲透率與定價能力。營運成本隨成長擴展,但受益於規模經濟。保守情境假設團購貢獻最小。所有情境假設主要收入來自 SaaS 與交易費用。CPT 激勵成本包含在營運成本中。

\paragraph{資金需求}

種子/天使投資 \$500K--1M 將涵蓋第 1 年的虧損與產品開發。若達到基本情境軌跡,計畫在第 2 年進行 \$3--5M 的 A 輪融資。計畫在第 3 年以後進行 \$10--20M 的 B 輪融資,用於國際擴張。

\paragraph{損益平衡分析}

保守情境在第 30--36 個月達到損益平衡。基本情境在第 18--24 個月達到損益平衡。樂觀情境在第 12--18 個月達到損益平衡。此範圍為投資人提供現實的預期,同時展示擴展潛力。
