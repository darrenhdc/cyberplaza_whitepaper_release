\chapter{运营概述}
\subsection{我们旨在解决的挑战}

\begin{enumerate}
\item \textbf{集中控制}:计算的日益重要性,特别是在云计算、高性能计算和人工智能(AI)领域,在现代社会中是不可否认的。然而,这些关键资源主要由大型企业控制,将优势限制在大多数用户之外。我们认为解决方案在于一个去中心化的市场,该市场能民主化地访问计算资源,营造更开放、更具包容性的环境。在这样的系统中,用户不仅是消费者,也是贡献者,他们可以影响计算发展的轨迹,并在计算的未来中拥有一席之地。

\item \textbf{低效问题}:当前的计算资源分配模式常常导致不平衡,造成资源利用不足或过度饱和。我们的项目旨在创建一个平台,高效匹配计算能力的需求与可用资源,从而优化利用率并减少浪费。

\item \textbf{高成本问题}:目前,大多数用户面临不必要的高额计算成本。我们的愿景是建立一个市场平台,以有竞争力的价格提供对广泛计算能力、存储方案、软件应用、数据和服务的直接访问。这不仅降低了总体成本,也扩大了用户基础。

\item \textbf{缺乏透明度}:现有的计算资源分配系统在定价、可用性和服务质量方面缺乏透明度。我们旨在构建一个开放、公正的平台,让用户能够根据关于资源、提供商和定价的可靠信息做出明智决策。

\item \textbf{用户权力不足}:对我们大多数人来说,执行需要计算的想法可能是一个繁琐的过程,往往需要依赖第三方服务。例如,人们必须依赖政府机构进行模拟后得出的电视天气预报,或者必须将个人数据委托给中心化实体才能为自己创建数字孪生。我们的项目旨在建立一个去中心化市场,提供所有必要的计算资源,使用户能够执行任何他们想要的计算,同时保持完全控制。
\end{enumerate}

对于现代社会这一重要的发展方向,我们需要解决建立去中心化综合生态系统的挑战,以实现更易获取、更高效的计算资源分配与利用。

\subsection{我们的解决方案概述}

\begin{enumerate}
\item 我们正在推出一个作为开放和民主组织运作的平台。该平台类似于一个计算资源市场,让人联想到淘宝等平台(即“算力淘宝平台”)。这一架构的所有权由CyberPlaza Token(CPT)的所有持有者共同持有,他们是我们平台的“股东”。

\item \textbf{支付系统}:我们的平台使用USDC进行所有交易,确保合规性、价格透明和熟悉的用户体验。这消除了与专有稳定币相关的风险,并与全球监管框架保持一致。

\item 在平台上,服务提供商(SPs)列出其计算资源——包括计算能力、存储、软件应用、数据和服务——供用户根据需求选择。作为服务的交换,SPs直接收到USDC支付,并根据其交易量获得CPT代币奖励。

\item 平台本身不拥有所列出的计算资源。不过,它可以通过“团购”采购计算资源,再转售给用户。这种模式类似于拼多多(Pinduoduo)的商业模式,采用去中心化流动性池,社区成员可以存入USDC以获得收益,同时支持平台运营。

\item 我们的平台是开放且包容的。对任何人担任四个角色中的任何一个或所有角色没有限制:平台“股东”、流动性提供者、SP和用户。这种灵活性使参与者能够以最符合其需求和能力的方式与平台互动。
\end{enumerate}
