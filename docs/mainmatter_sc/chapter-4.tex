\chapter{运营中的角色描述}

\section{运营的4种角色}

平台生态系统由四种不同的角色组成:平台角色、服务提供商(SP)角色、流动性提供商角色和用户角色。任何人担任或离开这4种角色中的任意一种或全部都没有限制。

\subsection{平台的角色}

\subsubsection{所有权与参与}

平台是由CyberPlaza Token(CPT)的所有持有者拥有的开放民主组织。任何人都可以通过以下方式参与项目获得CPT:(i) 为平台提供服务,(ii) 担任流动性提供商,(iii) 担任服务提供商(SP),(iv) 担任用户,或(v) 在二级市场购买CPT。

\subsubsection{平台功能}

平台充当分销商、匹配者和担保人,确保信任并促进用户、服务提供商(SP)与流动性提供商之间的交易。平台维护认证服务提供商(CSP)名单和“普通”服务提供商名单。CSP是那些提供的服务价值超过一定水平的提供商(目前定义为“未来10天内的销售将提供每月价值10,000美元以上USDC的服务”)。“普通”SP是那些提供的服务低于该阈值的提供商。平台在第一阶段将仅以CSP启动,随后引入SP。

平台将评估CSP,考虑其过往记录、声誉和CSP的绩效指标等因素,并将评估结果在平台上列出,以便用户做出知情决策。“普通”SP不会被评估,用户自主选择使用它们。平台作为可信赖的中介,增加了一层问责制,并提高了SP根据其SLA履行承诺的可能性。平台赚取交易费用的一部分(用户支付价格与SP获得价格之间的差额)。

\subsubsection{储备基金管理}

平台负责运营由存入USDC代币的流动性提供商设立的以USDC计价的“储备基金”,USDC是Web 3项目的货币,用于市场交易。储备基金将通过多种方式为USDC代币持有者产生利息,从而使USDC代币的铸造成为一种高收益投资。这些方式包括通过团购以折扣价获取计算资源并转售给用户,即利用储备基金开展“算力拼多多”业务。团购将来自全球主要云服务,包括AWS、Azure、Google Cloud、阿里云等,以及美国、欧洲和中国等国家和地区的超级计算中心。平台还通过投资可产生收入的计算资产(如比特币挖矿设施)获得利润,并通过具有高流动性的去中心化金融或传统金融投资获得利息。

\subsubsection{平台参与与未来扩展}

平台也可根据需要参与其他角色(流动性提供商、SP和用户),以引导流动性并确保服务质量。如果CPT持有者通过治理机制投票支持,平台未来可能将淘宝平台和拼多多运营模式扩展到算力之外。

\subsection{服务提供商(SP)的角色}

\subsubsection{注册与服务发布}

SP在平台上注册其服务,向用户提供算力(核时、存储、带宽、应用软件、数据和服务等)。SP在平台上列出其不同时段的计算资源可用性(例如,未来24小时内1,000核时的Intel Core i7,未来一个月内10,000核时)和价格表,供用户使用/预订。SP还将发布其提供的资源的各种基准(按照平台要求)及其SLA。

\subsubsection{支付与激励}

当SP的服务被用户选择和使用时,SP将直接收到USDC付款。此外,他们还将获得与其交易量成正比的CPT代币激励(交易价值的2--5%,以CPT等价物计算)。质押CPT代币的SP还可以获得降低的平台费用和在市场上的更高可见性。

\subsubsection{质量保证}

平台的评估系统验证CSP的质量和可靠性,确保列出的所有主要SP(CSP)都是可信赖的。通过利用声誉系统、用户评价和绩效指标,平台为CSP建立了基于功绩的排名系统。该评估系统让用户在选择SP进行任何大量使用时能做出知情决策,减少选择不可靠或不合适SP的机会。

\subsubsection{灵活的服务配置}

用户可以为一项任务选择组合使用多个SP,例如,大部分计算使用CSP,而数据分析的最后阶段使用“普通”SP(例如,用户自己提供的笔记本电脑)。平台为所选CSP提供评估,但不为非认证SP提供。

\subsection{流动性提供商的角色}

\subsubsection{概述}

流动性提供商是将USDC存入平台去中心化借贷池、为团购和平台运营提供运营资金支持的参与者。该角色为平台运营提供了一种透明且去中心化的融资模式。

\subsubsection{流动性提供机制}

流动性提供机制的运作方式如下:参与者将USDC存入经审计的智能合约,并获得代表其存款的rUSDC代币(收据代币)。平台将池化资本用于团购运营和运营资金。参与者可在池流动性可用的情况下提取存款。

\subsubsection{可参与性}

任何人都可以通过向池存入USDC成为流动性提供商,包括SP、用户和外部投资者。最低存款额的设计兼顾了可访问性和有意义的贡献。

\subsubsection{收益与福利}

流动性提供商通过多种机制获得收益。他们从平台运营利润中获得以USDC支付的6--8% APY的利息收益,以及以CPT代币形式支付的额外2--4% APY的CPT代币激励(带有锁定期),总预期收益率为8--12% APY。除财务收益外,他们还通过积累CPT获得治理权,拥有投票权,以及平台福利,包括降低的费用、优先访问权和早期产品发布。风险保护通过智能合约审计、保险基金(10%覆盖率)和透明跟踪来确保。

\subsection{用户的角色}

\subsubsection{获取计算资源}

用户可以通过以下简单流程在平台上获取计算资源:(i) 将USDC存入其平台钱包,(ii) 从市场浏览并选择服务提供商,以及(iii) 通过平台门户提交任务并使用USDC支付。

\subsubsection{具有竞争力的定价}

通过将平台作为算力淘宝平台使用,用户可以以具有竞争力的价格获取最适合自己的计算资源,由于团购福利,价格通常比直接从云提供商购买低10--30%。

\subsubsection{支付保护与透明度}

平台实施了全面的支付保护和透明度措施。智能合约托管会持有USDC付款,直到服务交付得到确认,如果SP未能满足SLA要求,则自动退款。该系统确保透明定价(无隐藏费用)、实时性能监控和报告,以及通过平台治理的争议解决机制。

\subsubsection{用户激励计划}

用户通过平台参与,通过多种赚取机制获得CyberPlaza代币(CPT)。消费奖励为用户提供消费金额的1--3%的CPT代币。推荐奖励允许用户因带来新用户或SP到平台而赚取CPT。忠诚度等级为持续平台使用提供更高奖励,质量反馈机制让用户因提供详细的服务评价而赚取CPT。

持有和质押CPT的好处是巨大的。使用折扣允许用户质押CPT以获得5--15%的服务折扣。收入分成使质押的CPT能够获得平台收入分配。治理权允许对平台参数和功能优先级进行投票。高级功能提供对高级工具、分析和API服务的访问。
