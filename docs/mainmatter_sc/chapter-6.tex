\chapter{技术与架构}

\section{项目治理基础设施}

\subsection{概述}

赛博广场网络(CyberPlaza Network)由赛博广场基金会(CyberPlaza Foundation)和赛博广场社区(CyberPlaza Community)组成。

赛博广场基金会是一个非营利性去中心化组织,致力于赛博广场平台的成功运营、计算技术与应用的推广和发展,以及支持平台上的去中心化社区建设和发展。基金会由赛博广场社区的CPT持有者所有和控制,由网络核心成员(见白皮书第8节)和基金会根据需要不时任命的顾问运营。基金会将成立赛博广场实验室,负责开发和研究新的计算资源技术与应用,以推动平台所需的技术创新和进步。

赛博广场社区是网络的社区部分,由流动性提供者、用户和服务提供者(SP)组成,他们共同参与基金会治理、开发和推广。社区成员可以通过参与治理、向基金会提出业务方向和技术发展相关提案、以及交流和分享经验来推动平台的发展和壮大。

赛博广场基金会与赛博广场社区之间的紧密联系对于实现网络的愿景和使命至关重要。

\subsection{智能合约模块}

我们将在Arbitrum(以太坊Layer 2)上部署符合ERC20标准的CPT智能合约,选择Arbitrum是因其低交易成本和高吞吐量。平台还将根据生态系统扩展的需要桥接到其他链。

CPT代币合约包含以下关键功能:标准ERC20功能(转账、授权等)、用于veToken机制的质押和锁定功能、治理投票集成、奖励分配机制、紧急暂停功能(由治理控制),以及用于未来增强的可升级代理模式。

\textbf{注意}:平台直接使用USDC进行支付,无需私有稳定币及相关监管风险。

\begin{verbatim}
// SPDX-License-Identifier: MIT
pragma solidity ^0.8.0;

import "@openzeppelin/contracts/token/ERC20/ERC20.sol";

contract CPTToken is ERC20 {
  struct LockInfo {
     uint256 amount;
     uint256 lockTimestamp;
     uint256 unlockTimestamp;
  }

   mapping (address => LockInfo[]) public locks;

   constructor(uint256 initialSupply) ERC20("CPT Token", "CPT") {
     _mint(msg.sender, initialSupply);
   }

   function lock(uint256 _amount, uint256 _lockTime) public {
     require(_amount <= balanceOf(msg.sender), "Not enough CPT to lock");
     require(_lockTime > 0, "Lock time must be positive");

       uint256 lockUntil = block.timestamp + _lockTime;
    
       LockInfo memory newLock = LockInfo({
           amount: _amount,
           lockTimestamp: block.timestamp,
           unlockTimestamp: lockUntil
       });
    
       locks[msg.sender].push(newLock);
    
       _burn(msg.sender, _amount);

   }

  function unlock(uint256 lockIndex) public {
    require(lockIndex < locks[msg.sender].length, 
            "No lock found at this index");
    require(block.timestamp >= locks[msg.sender][lockIndex].unlockTimestamp,
            "CPT still locked");

        uint256 amountToUnlock = locks[msg.sender][lockIndex].amount;
        locks[msg.sender][lockIndex] = 
            locks[msg.sender][locks[msg.sender].length - 1];
        locks[msg.sender].pop();
    
        _mint(msg.sender, amountToUnlock);
    }

   function calculateLockedAmount(address user, uint256 lockDuration) 
       public view returns (uint256) {
     uint256 totalLockedAmount = 0;

        for (uint256 i = 0; i < locks[user].length; i++) {
           if (block.timestamp - locks[user][i].lockTimestamp > lockDuration) {
               totalLockedAmount += locks[user][i].amount;
           }
        }
    
        return totalLockedAmount;
    }

}
\end{verbatim}

\subsection{代币标准与小数处理}

CPT代币遵循标准ERC20规范,使用18位小数,而USDC使用6位小数。平台对所有转换操作采用SafeMath库,以防止溢出和下溢错误。价格预言机包含小数归一化逻辑,最低交易阈值可缓解粉尘攻击向量。对于小数金额,协议采用保守的四舍五入机制。

\subsection{锁仓投票代币机制}

平台采用锁仓投票(ve)代币模型,以对齐长期利益相关者的激励。用户将CPT锁定1周到4年不等的时间,获得不可转让的veCPT代币,该代币决定治理权重和奖励分配。

veCPT余额遵循以下关系:
\begin{equation}
\text{veCPT} = \text{CPT}_{\text{locked}} \times \min\left(\frac{t_{\text{lock}}}{t_{\text{max}}}, 1\right) \times 2.5
\end{equation}
其中$t_{\text{lock}}$表示所选锁定期限,$t_{\text{max}} = 4$年为最大锁定期限。2.5倍的乘数为4年锁定承诺提供最大治理权重。

随着锁定期限接近到期,veCPT余额线性衰减:
\begin{equation}
\text{veCPT}(t) = \text{CPT}_{\text{locked}} \times \frac{t_{\text{remaining}}}{t_{\text{max}}} \times 2.5
\end{equation}
这种衰减机制通过锁定延期或代币重新锁定激励持续参与。

\subsubsection{奖励分配}

平台以USDC形式收取的收入中,30%将每周或每月分配到质押奖励池。个人奖励根据veCPT持有量按比例计算:
\begin{equation}
\text{Reward}_{\text{user}} = \text{Revenue}_{\text{pool}} \times \frac{V_{\text{user}}}{V_{\text{total}}}
\end{equation}
其中$V_{\text{user}}$表示用户的veCPT余额,$V_{\text{total}}$表示veCPT总供应量。有效年化收益率(APY)根据质押参与度和平台表现动态变化:
\begin{equation}
\text{APY} = \frac{\text{Annual Revenue Pool}}{\text{Total CPT Staked Value}} \times \frac{\text{veCPT Multiplier}}{\text{Average Multiplier}}
\end{equation}

\subsubsection{安全与优化}

智能合约遵循OpenZeppelin标准接受第三方审计,参数修改由多签治理控制。所有奖励分配在链上跟踪以确保透明度。安全功能包括外部调用的重入保护、基于角色的访问控制、紧急暂停功能和可升级代理模式。关键参数变更强制执行48小时时间锁。

 gas优化采用基于默克尔树的批量申领、veCPT余额的惰性评估、压缩存储变量和事件驱动的链下索引。这些技术在保持安全保障的同时降低了交易成本。

\subsection{预言机集成}

平台集成Chainlink去中心化预言机用于价格发现和数据聚合。CPT/USD价格馈源聚合Uniswap V3时间加权平均价格和中心化交易所报价的数据。USDC/USD验证采用Chainlink的经过验证的馈源,偏差阈值为0.5%。预言机每5分钟更新一次,或在价格变动超过1%时更新,并配备手动回退机制以确保冗余。

对于计算资源定价,链下聚合器监控主要云提供商(AWS、Azure、GCP、阿里云)的公开API,计算计算、存储和带宽的实时市场价格。聚合定价每天或在偏差超过5%时发布到链上预言机合约。

预言机安全依赖于至少7个独立Chainlink节点的共识。系统拒绝偏离中位数超过10%或超过1小时的旧数据。断路器在检测到操纵企图时自动暂停交易。

\subsection{治理架构}

关键平台操作需要通过Gnosis Safe实现的多签批准。超过100K USDC的 treasury转移需要5-of-9签名,而智能合约升级需要7-of-9批准并附带48小时时间锁。参数调整采用4-of-9共识,紧急安全响应采用3-of-5快速响应配置。

治理流程遵循结构化时间线:持有100K+ veCPT的用户可提交提案,随后是7天的社区讨论期和5天的链上投票期,其中1 veCPT等于1票。已批准的提案在48小时延迟后执行。多签委员会对恶意提案保留否决权,该权利每季度审查一次。

\subsection{跨链基础设施}

平台采用LayerZero全链协议进行多链部署。Arbitrum作为主链,因其低交易成本和高吞吐量。以太坊主网支持针对需要Layer-1安全性的机构用户,而Polygon集成则为对成本敏感的用户提供更低的交易成本。未来扩展包括Optimism(2024年第三季度)和Base(2024年第四季度),以实现更广泛的生态系统集成。

桥接安全包含多重保障:流动性上限将每条链的桥接供应量限制为10%,速率限制将吞吐量限制为每小时1M CPT,紧急暂停机制对异常情况做出响应,5%的保险基金为桥接价值提供抵押以抵御潜在漏洞。

\subsection{钱包基础设施}

作为标准ERC20代币,CPT支持所有兼容钱包,包括浏览器扩展(MetaMask、Rabby、Rainbow)、移动应用(Trust Wallet、Coinbase Wallet、imToken)、硬件设备(Ledger、Trezor)和智能合约钱包(Argent、Gnosis Safe)。计划未来部署与Fireblocks和Copper.co的机构托管集成。

Web门户采用WalletConnect和Web3Modal协议实现标准化钱包连接。授权连接后,平台查询用户余额、质押头寸和veCPT持有量,以启用完整功能访问。交易签名遵循EIP-712标准的类型化结构化数据,呈现人类可读的消息,提高对钓鱼向量的安全性。

\section{市场计算基础设施}

\subsection{系统架构}

平台采用三层架构。Web3接口层通过React.js和ethers.js框架管理钱包认证(WalletConnect)、USDC支付处理和CPT奖励分配。编排层协调CHESS集群管理系统、作业调度、资源分配、性能监控和服务提供者认证流程。计算资源层聚合CSP集群、公共云API(AWS、Azure、GCP、阿里云)、私有HPC中心和未来的边缘计算节点。

交易流程包括:提交作业并存入USDC,智能合约托管至完成,CHESS介导的资源匹配,在分配的服务提供者基础设施上执行,实时SLA合规性监控,结果交付与自动支付结算,以及CPT奖励的比例分配(用户1-3%,服务提供者2-5%)。

\subsection{HPC基础设施组件}

高性能计算(HPC)基础设施包含专用节点类型:计算节点使用多核处理器和大容量内存执行数值模拟和数据分析;可视化节点使用GPU加速渲染大型数据集;I/O节点管理存储与计算架构之间的数据传输;存储节点提供高并发文件系统;管理节点协调资源分配和作业调度。

网络架构采用高速互连技术(InfiniBand、以太网)用于节点间通信。并行文件系统支持大型数据集和中间结果的并发多节点读写操作。

\subsubsection{软件栈}

监控和管理工具为管理员提供系统组件的实时健康和性能数据,包括CPU利用率、内存消耗和网络流量模式。集群管理软件协调整个系统的操作,为地理分布的计算节点提供配置、监控和维护能力。

资源分配采用专用调度器管理CPU时间、内存和其他计算资源,以最大化系统利用率效率。用户界面包括命令行工具和Web门户,用于作业提交和管理。HPC应用中心聚合领域特定的应用和模板,使用户能够直接下载和部署计算工具。集成计费系统对不同资源类型和计费周期实施透明的定价策略,促进合理的资源利用和准确的成本核算。

\subsection{支付与结算基础设施}

\subsubsection{托管机制}

作业提交启动托管流程,用户批准将USDC支付到平台的智能合约。托管合约计算预计成本,包括资源类型(CPU/GPU/存储)、持续时间预测、预言机衍生的市场定价和20%的缓冲区以应对潜在超支。USDC在批准后转移到托管账户,针对唯一的作业标识符锁定。

作业完成后,实际资源消耗决定最终结算。服务提供者直接获得95-98%的费用(USDC),平台保留2-5%的交易费。多余的托管资金自动返还给用户,CPT奖励按比例分配给用户(1-3%)和服务提供者(2-5%)。

\subsubsection{争议解决协议}

SLA违规触发分级解决机制。5分钟内失败的作业有资格获得自动全额退款。部分完成的作业根据实际交付生成按比例退款。用户可在72小时内提交带有支持证据的争议。价值超过10K USDC的案件升级到平台治理仲裁,保险基金覆盖经核实的最高100K USDC的索赔。

\subsubsection{服务提供者认证}

服务提供者认证需要多阶段验证流程。初始注册需要公司验证文件、基础设施规格、支付钱包地址和安全合规证书(SOC 2、ISO 27001)。技术验证采用行业标准基准,包括High-Performance Linpack(HPL)、High-Performance Conjugate Gradient(HPCG)、STREAM内存带宽、用于AI工作负载的MLPerf以及网络延迟评估。安全审计验证AES-256加密、网络隔离和DDoS保护能力。

已批准的候选人进入30天的 probationary 期,进行增强监控和10个作业的并发限制。成功完成后授予认证服务提供者(CSP) status,能够访问机构客户和参与团购。CSP在高级目录中显示带有验证徽章。

持续合规要求月度99.5%的 uptime、低于5分钟的作业启动时间以及性能在宣传基准的10%以内。每季度重新认证验证持续能力。针对关键漏洞的安全补丁必须在48小时内部署。违规触发分级处罚:首次违规警告并给予7天整改期,第二次违规暂停30天,第三次违规撤销认证。

\subsection{技术栈}

平台采用React.js 18+和TypeScript进行前端开发,ethers.js v6和WalletConnect v2进行Web3集成,Material-UI确保界面一致性。后端架构使用Node.js/Express.js或Python FastAPI提供API服务,PostgreSQL用于关系型持久化,Redis用于缓存,RabbitMQ/Kafka用于异步作业队列,The Graph用于区块链事件索引,Prometheus/Grafana用于可观测性。DevOps基础设施通过Docker容器化所有服务,通过Kubernetes编排生产部署,通过GitHub Actions实现CI/CD,通过Cloudflare CDN分发内容,通过Nginx进行负载均衡。

入门级CSP需要100+ CPU核心(Intel Xeon/AMD EPYC)、500 GB RAM、10 TB NVMe SSD或50 TB HDD、10 Gbps网络上行链路,以及可选的4+ NVIDIA A100/H100 GPU。企业级CSP可扩展至10,000+ CPU核心、50 TB+聚合RAM、1 PB+并行文件系统存储(Lustre/GPFS)、100 Gbps InfiniBand骨干网和100+高端GPU。

\subsection{平台用户功能}

CPT门户服务三个主要群体:探索项目信息的访客、采购市场资源(公共云、HPC提供者、硬件、软件、存储)的用户,以及执行USDC存款和铸造操作的流动性提供者。

公共云消费者可以在包括FQ、Amazon和Huawei Cloud在内的供应商之间进行选择,定价以USDC计价并附带促销优惠。选择供应商会将用户重定向到原生门户(例如AWS),那里的标准操作通过CPT平台托管进行支付路由。平台随后以法定货币与供应商结算。

HPC资源消费者可以在供应商(CT集群、区域提供者、华为、AWS)之间比较价格点、硬件规格、性能指标和区域带宽。选择供应商和提交作业通过CHESS门户进行,需存入足够的USDC,资金托管至完成,随后进行法定结算。存储采购遵循相同的工作流程。

软件选项包括用户提供的应用或平台列出的来自Ansys、HPC软件供应商和CHESS应用中心的解决方案。供应商入驻支持硬件、存储、软件和辅助计算产品。当硬件和软件组件均来自平台列表时,系统验证硬件-软件兼容性,确保执行兼容性。该架构可根据需求变化适应未来的功能扩展。

\subsection{公共云集成}

市场聚合来自主要公共云供应商(AWS、Azure、Google Cloud、阿里云)的计算资源。定价以USDC计价,显示当前促销和可用性状态。

\subsubsection{供应商集成模型}

平台采用三种集成方法。直接API集成利用经销商凭证通过供应商API(AWS EC2、Azure Resource Manager、GCP Compute Engine)进行实时配置,支持自动实例生命周期管理。优惠券代码系统通过预先生成的代码防止超售,解决容量限制,有基于价值的(100美元通用积分)或基于资源的(1000 GPU小时、10 TB存储)格式。托管服务提供商模型将赛博广场定位为拥有批量定价协议的MSP,管理供应商账户并提供统一账单。

实时价格比较显示计算、存储和网络成本以及总拥有成本计算。团购折扣突出显示相对于直接采购的潜在节省。

\subsubsection{作业提交工作流程}

HPC作业提交流程包括:通过过滤的CSP列表选择资源(CPU类型、GPU可用性、区域、定价);通过应用中心模板或自定义代码配置作业,指定要求(节点、核心、内存、运行时间、GPU)和I/O位置;成本估算,包含USDC细分和预计CPT奖励;支付授权,将USDC转移到托管账户并附带应急缓冲区;通过CHESS调度器分配执行,实时状态监控;完成结算,交付结果并自动分配支付、退还超额款项和发放CPT奖励。

高级功能包括支持100+作业的批量提交(带参数扫描)、定义顺序执行的工作流依赖、用于容错的 checkpoint/restart、spot instance竞价(可获得50-70%的抢占式容量折扣),以及用于动态资源调整的自动缩放。

服务提供者通过中央仪表盘管理操作,包括资源分配、作业监督、财务跟踪(USDC收入、CPT积累)和性能分析(客户满意度、利用率指标)。

\subsection{多集群管理系统}

CHESS(Cluster High-performance Execution and Scheduling System)平台提供对地理分布的计算资源的统一管理。系统通过带有基于角色的访问控制的中央Web门户集成监控、调度和资源分配。

\subsubsection{核心功能}

平台通过Web界面和SSH协议支持全面的数据管理,实现文件操作,包括上传、下载、压缩和提取。节点管理通过批量命令控制电源状态、远程访问(VNC、shell),并支持异构硬件配置(CPU、GPU、FPGA)。资源配额执行有关存储和计算分配的管理政策,在超过阈值时自动生成警报。

高可用性架构通过冗余管理节点和数据库复制消除单点故障。系统协调多个地理分布的集群,在子集群之间统一用户角色传播。

\subsection{性能监控基础设施}

高性能和云计算系统聚合大量硬件资源,通过高速网络互连,形成低延迟、高容量的配置。有效的集群管理需要监控和管理工具提供资源配置、实时性能跟踪、故障检测与警报,以及使用状态可视化。

\subsubsection{CHESS监控功能}

CHESS监控系统通过聚合仪表盘提供全面的集群监督,显示Ethernet和InfiniBand架构的CPU和内存使用情况、负载状态、存储状态和网络吞吐量。自定义时间间隔选择支持历史趋势分析和性能跟踪。仪表盘显示提供可定制的大屏幕演示,动态更新存储使用、作业调度和网络统计等指标。

多集群监控扩展到地理分布的安装,具有自适应屏幕布局和分辨率优化。机架可视化渲染物理拓扑,集成电源管理和VNC远程访问控制。单节点监控捕获粒度CPU、内存、存储、负载和网络指标,同时提供故障诊断和恢复建议。GPU监控跟踪设备特定的使用率、内存利用率、温度和带宽。作业监控分析实时执行状态和队列组成,提供详细的CPU利用率、内存消耗和节点负载统计。集群警报实现可配置的阈值,通过电子邮件和系统通知路由。

性能指标以用户定义的间隔收集,捕获CPU、内存、磁盘和网络数据。物理拓扑可视化包含机架和节点布置,以及基于阈值的故障警报。

\subsubsection*{调度器与资源管理}

高效的调度和资源管理在多集群系统中至关重要。CHESS提供灵活的调度策略,包括FIFO、抢占和回填策略。系统支持带有服务质量(QoS)配置的资源预留、涵盖串行、并行和GPU工作负载的高级作业提交,以及用于负载平衡优化的队列管理。

\subsubsection{作业提交与管理}

用户通过命令行界面、基于Web的GUI或针对常见工作流的应用模板提交作业。管理员配置资源配额、优先级级别和提交政策,以管理系统访问和利用。

% \subsection*{6.2.5 Pricing Module}

% This module will focus on calculating costs for resource usage and presenting pricing details to users. It will integrate with job scheduling and monitoring systems for real-time cost tracking.

% \subsubsection*{6.2.5 User Interfaces and Operational Portals}

\subsubsection{用户管理}

平台支持自注册和管理员配置的账户,集成LDAP认证以实现集中管理。基于角色的访问控制实现默认角色(管理员、部门管理员、用户),并通过灵活的权限分配管理系统访问和功能。

\subsubsection{通知与消息}

用户会收到关于账单和使用情况的自动警报,以及管理公告。

\subsection{应用中心}

应用中心通过可浏览的库提供对预安装HPC应用(Ansys、MATLAB、TensorFlow)的访问。用户通过带有交互式参数配置的图形模板提交作业。输出管理包括日志查看、错误分析、性能指标跟踪和集成可视化工具(用于AI应用的TensorBoard)。

\subsection{硬件性能评估}

硬件性能评估模块执行基准测试,测量CPU和GPU性能以及网络吞吐量和延迟。资源效率分析根据工作负载特性优化分配策略。故障恢复指标评估硬件在故障情况下的可靠性和恢复性能。

\subsection{安全架构与合规性}

\subsubsection{多层安全模型}

平台在三层实施纵深防御安全。智能合约安全采用Certora或等效工具的形式验证、由CertiK、Trail of Bits或OpenZeppelin进行的年度第三方审计、为关键漏洞提供高达50万美元奖励的漏洞赏金计划、带有48小时时间锁的可升级透明代理模式,以及用于紧急漏洞响应的断路器。

平台安全包括通过OAuth 2.0和JWT认证的API保护(100次请求/分钟速率限制)、针对服务提供者访问的IP白名单,以及90天API密钥轮换。数据加密采用TLS 1.3(传输保护)、AES-256(静态数据)、敏感工作负载的端到端加密,以及用于密钥管理的硬件安全模块。基础设施安全部署Cloudflare DDoS保护、带有OWASP规则集的Web应用防火墙、季度渗透测试,以及用于事件监控的SIEM系统。

数据隐私与合规措施通过账户删除权、数据可移植性、隐私设计原则和欧盟数据驻留选项满足GDPR要求。KYC/AML程序对每月超过10K USDC的交易实施基本验证,对CSP认证实施增强验证,对可疑活动进行交易监控,并遵守FATF旅行规则。数据隔离采用容器化或基于VM的作业执行、网络分段、完成后自动数据擦除,以及跨用户泄漏预防。

\subsubsection{事件响应}

持续安全运营中心监控异常活动,包括异常提款、智能合约漏洞和API滥用。事件分类遵循四级严重程度模型(关键、高、中、低),评估目标为15分钟。关键事件触发立即合约暂停,并在1小时内发送多签通知。关键事件在24小时内公开披露,事后报告在7天内发布。恢复程序通过治理渠道部署补丁,并从保险基金赔偿受影响用户。

\subsubsection{监管合规性}

平台追求SOC 2 Type II认证(第1年目标)以确保数据安全和可用性,以及ISO 27001认证(第2年目标)以实现信息安全管理。Cloud Security Alliance STAR认证验证CSP安全态势。PCI DSS合规性正在考虑用于未来的支付方式扩展。

\subsection{可扩展性与性能优化}

\subsubsection{水平扩展架构}

平台通过分布式数据库架构进行水平扩展,在各区域部署PostgreSQL只读副本,按ID哈希对用户数据进行分片,使用Redis集群存储热数据(会话、定价),并通过Cloudflare CDN交付静态资产。

微服务架构将功能分解为可独立扩展的服务:用户服务(认证、配置文件)、作业服务(提交、调度、监控)、支付服务(USDC托管、结算、CPT奖励)、服务提供者服务(入驻、认证、评级)、定价服务(预言机聚合)和通知服务(电子邮件、推送、链上事件)。每个服务根据需求独立扩展。

负载均衡在US、EU和亚洲地区实现地理分布,使用Kubernetes Horizontal Pod Autoscaler进行动态容量调整,Hystrix断路器防止级联故障,RabbitMQ队列用于异步作业处理。

\subsubsection{性能目标}

\begin{center}
\begin{tabular}{|l|c|c|}
\hline
\textbf{指标} & \textbf{目标(第1年)} & \textbf{目标(第3年)} \\
\hline
API响应时间 & <200ms (p95) & <100ms (p95) \\
作业提交时间 & <5秒 & <2秒 \\
支付结算时间 & <30秒 & <10秒 \\
页面加载时间 & <2秒 & <1秒 \\
平台 uptime & 99.5\% & 99.9\% \\
并发用户数 & 10,000 & 100,000 \\
每日交易量 & 50,000 & 1,000,000 \\
\hline
\end{tabular}
\end{center}

\subsubsection{区块链可扩展性}

Arbitrum Layer 2部署为主要操作提供低于0.10美元的交易费用和40,000 TPS的吞吐量。批量交易处理将奖励分配分组,以摊薄gas成本。The Graph协议处理链下事件索引。未来开发包括为高频微支付场景提供状态通道。

gas优化技术通过基于默克尔证明的奖励申领(节省80%)、veCPT余额的惰性评估、压缩存储变量编码,以及在功能等效的情况下优先使用事件日志而非状态变量,降低交易成本。

\subsection{灾难恢复}

\subsubsection{备份基础设施}

数据库备份每天执行一次(完整备份),每六小时执行一次(增量备份),并进行持续的事务日志复制。系统在冷存储归档前保持30天的保留期。智能合约状态利用区块链的固有不可变性,辅以归档节点部署和每季度的去中心化存储快照(IPFS/Arweave)。用户作业结果备份到指定的存储端点,平台元数据保留90天,并具备GDPR合规的按需导出功能。

\subsubsection{恢复目标}

表~\ref{tab:recovery-targets}指定了组件级的恢复时间目标(RTO)和恢复点目标(RPO)。

\begin{table}[htbp]
\centering
\caption{恢复时间与恢复点目标}
\label{tab:recovery-targets}
\begin{tabular}{lcc}
\hline
\textbf{组件} & \textbf{RTO} & \textbf{RPO} \\
\hline
智能合约 & 不适用 & 0 \\
Web门户 & 1小时 & 6小时 \\
数据库 & 2小时 & 1小时 \\
作业调度器 & 30分钟 & 15分钟 \\
\hline
\end{tabular}
\end{table}

在美国和欧盟地区的双活部署,在主区域不可用5分钟后,可实现自动DNS故障转移。实时跨区域数据同步保持一致性,具备手动覆盖能力用于操作干预。

\subsection{发展路线图}

短期开发(6-12个月)优先考虑iOS和Android移动应用、用于第三方集成的增强API(RESTful、GraphQL)、基于机器学习的成本优化,以及额外的区块链桥接部署(Polygon、Optimism)。

中期目标(1-2年)通过边缘计算支持IoT部署、敏感工作负载的机密计算集成(Intel SGX、AMD SEV)、去中心化存储协议(Filecoin、Arweave)、专用AI/ML资源市场,以及探索性量子计算合作来扩展平台能力。

长期愿景(2-5年)包括全面过渡到DAO治理、开放去中心化计算协议开发、用于隐私增强的零知识证明实现、通过IBC或等效协议的跨链互操作性,以及基于NFT的物理计算资源代币化。

\subsection{总结}

本章详细介绍了将Web3区块链基础设施与成熟HPC系统集成的技术架构。混合设计将去中心化激励机制(CPT代币、锁仓投票治理)与经过验证的CHESS集群管理平台相结合。安全架构通过智能合约审计、基础设施加固和监管合规路径(SOC 2、ISO 27001)实现多层保护。系统可从数千个并发用户扩展到数十万个,同时保持低于200ms的API响应时间。

与现有去中心化计算项目(Golem、iExec、Render)相比,赛博广场的差异化在于成熟的基础设施(20+年CHESS平台历史)、企业合规导向、超越点对点架构的多云集成、预集成的应用生态系统,以及结合去中心化访问与专业服务提供者认证的混合市场。这种定位满足企业计算需求,同时支持Web3经济参与。
