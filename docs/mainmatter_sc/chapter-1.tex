\chapter{执行摘要}

计算在现代生活中发挥着日益关键的作用,且这一趋势预计在可预见的未来仍将持续。Web3 CyberPlaza 网络项目旨在让个人和机构都能以开放包容的方式从这一趋势中受益。

本项目推出 CyberPlaza 平台,这是一个去中心化市场,可被描述为“计算资源版淘宝”(算力淘宝平台)。该平台匹配用户与服务提供商(SPs)的需求,涵盖高性能计算、智能计算和云计算领域。用户可在一处获取多样化的算力、存储、软件应用、数据及计算服务,以高性价比满足其特定需求;而服务提供商则可获得无限制的全球销售渠道。服务提供商和用户均可通过持有 CyberPlaza 代币(CPTs\footnote{CPT 是物理学中的守恒量;所有物理定律都不得违反 CPT 守恒,这与能量守恒类似。CPT(Carriage Paid To,运费付至)也是一个国际贸易术语,意为卖方将承担货物运送给消费者的费用。我们的代币名称蕴含这两层隐喻。})分享平台成功,该代币代表“淘宝”平台的治理份额。

\textbf{支付与结算}:平台交易使用 USDC 结算,USDC 是一种被广泛采用且受监管的稳定币,可确保合规性与用户熟悉度。这种方式消除了与专有稳定币相关的复杂性和监管风险,同时保持透明的美元计价定价。

\textbf{收入模型}:平台通过多种渠道产生收入:SaaS 订阅(40--50\%),包含平台访问的月度/年度订阅;交易手续费(25--30\%),即计算资源购买的 2--5\% 手续费;API 与数据服务(15--20\%),提供高级 API 访问与分析服务;团购(5--10\%),通过批量采购产生利润作为补充收入。这些收入通过透明的质押奖励机制分配给 CPT 代币持有者,使参与者能够通过为平台提供流动性和治理贡献获得可持续收益(目标 6--10\% APY)。

\textbf{去中心化流动性池}:为支持平台的团购运营并确保具有竞争力的定价,项目将实施一个去中心化借贷池,参与者可存入 USDC 以赚取利息(5--7\% APY)和 CPT 激励(2--3\% APY),同时为平台提供运营资金。这种模式以更透明、可审计且去中心化的方式取代了传统储备金。

CyberPlaza 平台并不直接拥有其挂牌的计算资源。相反,为确保持续供应和具有竞争力的定价,平台借鉴拼多多的商业模式,利用社区提供的流动性,通过“团购”模式(团购)获取计算资源。

\textbf{团购注意事项}:尽管团购是我们战略的一部分,但它是\textbf{补充收入渠道}(占总收入的 5--10\%),而非核心商业模式。我们的主要价值来自 SaaS 订阅和智能云管理工具。随着平台规模扩大,我们将争取团购折扣,但我们的核心价值主张并不依赖于从云服务提供商处获得大幅批量折扣。

团购模式减少了消费者与服务提供商(即当前控制所有计算资源的行业巨头)之间的不平衡。这使平台能够批量获取计算资源并提供给用户,打造一个充满活力的繁荣市场。“拼多多模式”业务的收益通过质押奖励机制分配给 CPT 质押者,平台收入的 30\% 分配给质押池(为提升可持续性从 40\% 下调),35\% 用于运营和增长,20\% 用于回购与销毁,10\% 用于团队,5\% 用于应急储备。

项目核心团队在与项目相关的多个领域拥有丰富经验,包括分布式高性能计算、公共云服务、异构计算、AI 与大数据应用、分布式系统软件开发、DeFi 投资、商业犯罪预防以及计算资源的业务与营销。

总体而言,CyberPlaza 网络 Web 3 项目旨在为计算资源提供一个去中心化市场,让用户和服务提供商都能受益,同时使所有参与者能够通过持有 CPT 代币和质押奖励分享平台的成功。依托强大的核心团队以及受淘宝和拼多多等成功平台启发的创新商业模式,该项目为计算行业的所有参与者打造了一个充满活力、合规且可持续的生态系统。
