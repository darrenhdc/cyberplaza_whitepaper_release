\chapter{执行摘要}

计算在现代生活中扮演着日益重要的角色,且这一趋势预计将在可预见的未来持续。Web3 CyberPlaza 网络项目旨在让个人与机构都能以开放、包容的方式从这一趋势中受益。

本项目推出 CyberPlaza 平台——一个可被描述为“算力淘宝平台”的去中心化市场。该平台匹配用户与服务提供商(SP)的需求,覆盖高性能计算、智能计算与云计算领域。用户可在一个平台获取多样化的算力、存储、软件应用、数据及计算服务,以高性价比满足其特定需求。另一方面,服务提供商可获得无限制的全球销售渠道。服务提供商与用户均可通过持有 CyberPlaza 代币(CPTs\footnote{CPT 是物理学中的守恒量,所有物理定律必须不违反 CPT 守恒,类似于能量守恒。CPT(Carriage Paid To,运费付至)也是国际贸易术语,指卖方将支付货物交付至消费者的费用。我们的代币名称承载了这两种隐喻。})——即“淘宝”平台的治理份额,共享平台的成功。

\textbf{支付与结算}:平台上的交易使用 USDC(一种被广泛采用且受监管的稳定币)进行结算,确保合规性与用户熟悉度。这种方式消除了与专有稳定币相关的复杂性与监管风险,同时保持透明的美元计价。

\textbf{收入模式}:平台通过多种渠道产生收入:SaaS 订阅(40--50\%),包括平台访问的月度/年度订阅;交易手续费(25--30\%),对计算资源购买收取 2--5\% 的手续费;API \& 数据服务(15--20\%),提供高级 API 访问与分析服务;团购(5--10\%),从批量采购中获取差价作为补充收入。这些收入通过透明的质押奖励机制分配给 CPT 代币持有者,使参与者能够通过为平台提供流动性与治理贡献获得可持续的年化收益率(目标 6--10\% APY)。

\textbf{去中心化流动性池}:为支持平台的团购运营并确保有竞争力的定价,项目实现了一个去中心化借贷池,参与者可存入 USDC 以赚取利息(5--7\% APY)及 CPT 激励(2--3\% APY),同时为平台提供运营资金。该模式以更透明、可审计的去中心化方式取代了传统准备金。

CyberPlaza 平台并不直接拥有其平台上列出的计算资源。相反,为确保持续的供应与有竞争力的定价,平台借鉴拼多多的商业模式,利用社区提供的流动性通过“团购”模式(团购)获取计算资源。

\textbf{团购说明}:尽管团购是我们战略的一部分,但它是 \textbf{补充收入渠道}(占总收入的 5--10\%)而非核心商业模式。我们的主要价值来自 SaaS 订阅与智能云管理工具。团购折扣将在平台规模化时推行,但我们的核心价值主张并不依赖于从云提供商处获得大额批量折扣。

这种团购方式减少了消费者与目前控制所有计算资源的行业巨头之间的不平衡。这使平台能够批量获取计算资源并提供给用户,创建一个动态且繁荣的市场。团购业务的收益将通过质押奖励机制分配给 CPT 质押者,平台收入的 30\% 分配给质押池(从 40\% 下调以提升可持续性),35\% 用于运营与增长,20\% 用于回购 \& 销毁,10\% 用于团队,5\% 用于应急储备。

本项目的核心团队在与项目相关的领域拥有丰富经验,包括分布式高性能计算、公共云服务、异构计算、人工智能与大数据应用、分布式系统软件开发、DeFi 投资、商业防欺诈,以及计算资源的商业与营销。

总体而言,CyberPlaza Network Web 3 项目旨在提供一个去中心化的计算资源市场,让用户与服务提供商均能受益,同时让所有参与者通过 CPT 代币持有与质押奖励分享平台成功。凭借强大的核心团队以及借鉴淘宝、拼多多等成功平台的创新商业模式,该项目为计算行业的所有参与者构建了一个繁荣、合规且可持续的生态系统。
