\chapter{CyberPlaza 通证 (CPT) 与通证经济}
\subsection{CPT 通证概述与效用}

\subsubsection{支付系统}

该平台使用 USDC 作为所有市场交易的主要支付货币。这种方式消除了与专有稳定币相关的监管风险,同时确保符合全球稳定币框架的监管要求,提供熟悉的用户体验(USDC 被广泛采用且值得信赖)、透明的美元计价、与现有 DeFi 基础设施的无缝集成,以及不存在与算法稳定币失败相关的风险。

\subsubsection{CyberPlaza 通证 (CPT)}

CPT 是平台的原生治理与效用通证,旨在协调所有利益相关者的激励并捕捉平台价值增长。

\subsubsection{CPT 核心效用}

\paragraph{治理权}

CPT 持有者可以对平台参数(费用结构、收益分配比例等)进行投票,对新功能、合作伙伴关系和战略方向提出并投票决定,并参与国库管理和资本分配决策。投票权重基于质押的 CPT 数量和锁仓时长(veToken 模型)。平台每季度举行治理会议,并采用透明的提案流程。

\paragraph{质押收益分享}

持有者可质押 CPT 以赚取平台收益分配(以 USDC 支付)。平台收益的 30\% 分配给质押奖励池(为可持续性优化)。质押奖励按周或按月分配(由治理决定)。较长的质押期限可获得奖励乘数(4 年锁仓最高 2.5 倍)。目标 APY 为 6--10\%,基于平台表现(更可持续)和质押比例。无无常损失风险(单资产质押)。

\paragraph{使用福利}

质押 CPT 可享受平台服务 5--15\% 的折扣(分层系统),访问高级功能(包括高级分析、API 访问和优先支持),为高交易量用户降低交易费用,提前访问新服务和测试功能,以及高需求计算资源的优先分配。

\paragraph{生态系统激励}

平台为所有用户类别提供激励。用户可获得消费金额 1--3\% 的 CPT 返还(返现计划)。服务提供商根据交易量获得 2--5\% 的 CPT 奖励。流动性提供商获得 2--4\% 的 CPT 代币 APY 作为额外收益。推荐人因介绍新用户或服务提供商(SP)加入平台而获得 CPT。社区贡献通过漏洞赏金、内容创作和代码贡献获得奖励。

\paragraph{通缩机制}

平台收益的 20\% 用于从公开市场回购 CPT。购买的 CPT 代币将被永久销毁(发送至 0x0 地址),随着时间的推移减少流通供应量,创造稀缺性。平台实施具有链上验证的透明季度销毁事件,预计 5 年内供应量将减少 30--40\%。这将使所有 CPT 持有者受益,而不仅仅是质押者。

\subsection{收益模型与分配机制}

\subsubsection{平台收益来源}

平台通过如表~\ref{tab:revenue} 所示的多个来源产生收益。

\begin{table}[htbp]
\centering
\caption{平台收益预测}
\label{tab:revenue}
\begin{tabular}{lccccr}
\hline
\textbf{收益来源} & \textbf{费率/金额} & \textbf{第 1 年} & \textbf{第 2 年} & \textbf{第 3 年} & \textbf{占比} \\
\hline
\textbf{SaaS 订阅} & \$50--500/月 & \$1.5M & \$4M & \$8--10M & \textbf{40--50\%} \\
交易费用 & GMV 的 2--5\% & \$0.8M & \$2.5M & \$5--7M & \textbf{25--30\%} \\
API 与数据服务 & 可变 & \$0.3M & \$1.5M & \$3--4M & \textbf{15--20\%} \\
认证服务 & 每 SP \$5K--50K & \$0.3M & \$0.8M & \$1--2M & \textbf{5--8\%} \\
团购利润 & 5--10\% 利润率 & \$0.2M & \$0.7M & \$1.5--2M & \textbf{5--10\%} \\
\hline
\textbf{总收益} & --- & \textbf{\$3.1M} & \textbf{\$9.5M} & \textbf{\$19--25M} & \textbf{100\%} \\
\hline
\end{tabular}
\end{table}

与原始模型的主要变化包括:SaaS 订阅现在作为主要收益来源(40--50%)以确保可预测性;团购缩减为补充来源(5--10%),考虑到早期规模这一变化更为现实;API 服务作为高利润、可扩展的收益来源得到重视(15--20%)。保守预测基于第 3 年 0.01% 的市场渗透率。

\subsubsection{SaaS 订阅层级}

平台提供如表~\ref{tab:saas} 所示的分层订阅计划。

\begin{table}[htbp]
\centering
\caption{SaaS 订阅层级(示例)}
\label{tab:saas}
\begin{tabularx}{\textwidth}{llXXr}
\hline
\textbf{层级} & \textbf{价格/月} & \textbf{目标用户} & \textbf{功能} & \textbf{预计用户数(第 3 年)} \\
\hline
免费版 & \$0 & 个人用户 & 2 个云账户、基础监控 & 10,000+ \\
入门版 & \$50 & 小型团队 & 5 个账户、成本追踪、1\% CPT 返现 & 2,000 \\
专业版 & \$200 & 开发团队 & 10 个账户、AI 优化、API、3\% CPT 返现 & 500 \\
企业版 & \$500--2000 & 企业用户 & 无限制、定制集成、5\% CPT 返现 & 50--100 \\
\hline
\end{tabularx}
\end{table}

该分层模型提供可预测的经常性收益,同时仍允许免费增值用户获取。

\textbf{重要说明}:这些预测代表我们的目标场景。我们还模拟了保守场景,第 1 年收益为 \$500K--1M,以确保即使初始增长缓慢也能保持财务可持续性。我们的业务模型不依赖于立即实现大规模团购折扣。

\subsubsection{收益分配模型}

平台收益(100%)分配如下:质押奖励池获得 30%(为可持续性而降低),并以 USDC 形式按比例分配给 CPT 质押者。运营与开发获得 35%(为增长而增加),分配给工程与产品开发(15%)、营销与业务发展(10%)以及基础设施与安全(5%)。回购与销毁获得 20%,用于从 DEX 购买 CPT 并永久销毁。团队与基金会获得 10%,用于核心团队薪酬(5%)和基金会运营(5%)。应急储备金获得 5%,作为应对波动性的新缓冲。

\subsubsection{质押奖励计算示例}

考虑第 3 年的成熟平台场景,平台月收益为 \$1,500,000。质押池分配(40%)提供 \$600,000。如果质押的 CPT 总量为 40,000,000(供应量的 40%),而您的质押量为 10,000 CPT(质押总量的 0.025%),那么您的月度奖励为 \$600,000 $\times$ 0.025% = \$150 USDC,年度奖励为 \$150 $\times$ 12 = \$1,800 USDC。

如果 CPT 价格 = \$2,那么您的质押价值为 \$20,000,APY 为 \$1,800 / \$20,000 = 9%。此外,额外收益包括平台治理投票权、服务折扣(5--15%)以及回购/销毁带来的价格升值。

\subsubsection{USDC 存款者的流动性池收益}

将 USDC 存入借贷池的流动性提供商可获得如表~\ref{tab:liquidity} 所示的收益。

\begin{table}[htbp]
\centering
\caption{流动性提供商收益}
\label{tab:liquidity}
\begin{tabular}{llll}
\hline
\textbf{组成部分} & \textbf{APY} & \textbf{支付货币} & \textbf{来源} \\
\hline
基础利息 & 6--8\% & USDC & 平台运营利润 \\
CPT 激励 & 2--4\% & CPT & 通证释放(锁仓释放) \\
\textbf{总预期} & \textbf{8--12\%} & \textbf{混合} & \textbf{可持续收益} \\
\hline
\end{tabular}
\end{table}

核心功能包括:存款用于团购运营(透明链上跟踪);渐进式提款系统防止挤兑情况;保险基金覆盖最高 10% 的池 TVL;智能合约由领先公司审计;基于池利用率的实时 APY 更新。

\subsection{通证分配与锁仓释放计划}

\subsubsection{总供应量与分配}

总供应量为 100,000,000 CPT(固定,无通胀)。分配细分如表~\ref{tab:allocation} 所示。
\begin{table}[htbp]
\centering
\caption{CPT 通证分配}
\label{tab:allocation}
\begin{tabularx}{\textwidth}{lrrr>{\raggedright\arraybackslash}X}
\hline
\textbf{类别} & \textbf{分配} & \textbf{通证数量} & \textbf{占比} & \textbf{锁仓与释放条款} \\
\hline
\textbf{社区激励} & \textbf{总计} & \textbf{55,000,000} & \textbf{55\%} & \textbf{基于绩效的释放} \\
\quad - 用户奖励 & & 25,000,000 & 25\% & 基于平台 GMV 里程碑释放 \\
\quad - SP 激励 & & 20,000,000 & 20\% & 基于交易量释放 \\
\quad - LP 奖励 & & 10,000,000 & 10\% & 5 年释放,前期倾斜 \\
\textbf{基金会} & & \textbf{17,500,000} & \textbf{17.5\%} & \textbf{TGE 时释放 10\%,剩余 90\% 24 个月线性释放} \\
\textbf{私募} & & \textbf{12,500,000} & \textbf{12.5\%} & \textbf{6 个月锁定期,之后 18 个月线性释放} \\
\textbf{团队} & & \textbf{15,000,000} & \textbf{15\%} & \textbf{12 个月锁定期,之后 36 个月线性释放} \\
\hline
\textbf{总计} & & \textbf{100,000,000} & \textbf{100\%} & \\
\hline
\end{tabularx}
\end{table}

与原始方案的主要变化包括:社区分配从 50% 增加到 55%(移除 USDC 持有者分配);投资者分配从 15% 减少到 12.5%(社区优先方法);团队分配从 17.5% 减少到 15%(更强的一致性);取消“流动性提供商”类别(替换为流动性提供商激励)。

\subsubsection{锁仓释放细节}

\paragraph{社区激励 (55\%)}

用户奖励(2500 万 CPT)基于平台 GMV 目标每月释放。公式为:月度释放量 = 基础量 × (实际 GMV / 目标 GMV)。分配期为 5 年,未领取的代币结转至下一周期。

SP 激励(2000 万 CPT)基于交易量按季度释放。更高质量的服务提供商(CSP)可获得奖励乘数。分配期为 5 年,可能会有基于绩效的加速释放。

LP 奖励(1000 万 CPT)采用前期倾斜释放:第 1 年(40%)、第 2 年(30%)、第 3--5 年(30%)。每周分配给活跃流动性提供商,长期存款可获得奖励。锁仓释放为 50% 即时释放,剩余 50% 在 6 个月内线性释放。

\paragraph{团队分配 (15\%)}

团队分配包括 12 个月的锁定期(第一年无代币释放)。锁定期后,36 个月线性释放。总锁仓释放期为 4 年。锁仓释放合约透明且可公开验证。

\paragraph{基金会分配 (17.5\%)}

10% 在 TGE 时释放用于初始运营(多签控制)。剩余 90% 在 24 个月内线性释放。这些资金用于合作伙伴关系、审计、法律、营销和资助,每季度发布透明度报告。

\paragraph{私募 (12.5\%)}

私募包括 6 个月的锁定期,锁定期后 18 个月线性释放。总锁仓释放期为 2 年。反倾销机制将出售限制在每日交易量的 5% 以内。

\subsection{流动性激励与 veToken 质押模型}

\subsubsection{veToken 机制(投票托管 CPT)}

我们采用受 Curve Finance 启发的 veToken 模型,该模型已被证明能够协调长期激励。用户锁仓 CPT 以获得 veCPT(不可转让)。锁仓时长决定 veCPT 乘数如表~\ref{tab:vetoken} 所示。

\begin{table}[htbp]
\centering
\caption{veToken 乘数(按锁仓时长)}
\label{tab:vetoken}
\begin{tabular}{ll}
\hline
\textbf{锁仓时长} & \textbf{veCPT 乘数} \\
\hline
1 周 & 0.01x \\
1 个月 & 0.04x \\
3 个月 & 0.25x \\
6 个月 & 0.50x \\
1 年 & 1.00x \\
2 年 & 1.50x \\
4 年 & 2.50x(最高) \\
\hline
\end{tabular}
\end{table}

\subsubsection{veCPT 的优势}

增强的治理权:1 veCPT = 1 票(标准 CPT:除非锁仓否则 0 票),锁仓时间越长,在平台方向上的话语权越强。

增强的质押奖励:基础 APY 为 8--12%(1 年锁仓),最高 2.5 倍增强(4 年锁仓),最高锁仓可获得 20--30% 的增强 APY。

费用分享优先:veCPT 持有者优先获得收益分配,veCPT 余额越高,费用池份额越高。

专属福利:最高服务折扣(15%)、优先访问超额订阅资源、专属治理提案权(需最低 veCPT 门槛)。

\subsubsection{流动性挖矿计划}

第 1 阶段:启动激励(第 1--6 个月):以高 CPT 释放为流动性引导。Uniswap V3 上的 CPT/USDC 池每天获得 2000 CPT。CPT 单质押每天获得 1500 CPT。USDC 借贷池获得相当于每天 1000 CPT 的奖励。

第 2 阶段:增长(第 7--24 个月):减少释放,专注于可持续收益。总释放量约为每天 2500 CPT,增加 USDC 借贷池的权重(激励流动性)。

第 3 阶段:成熟(第 25 个月及以后):新释放量极小(约每天 1000 CPT)。收益驱动的收益率成为主要吸引力,回购/销毁创造供应量稀缺。

\subsubsection{反鲸鱼与公平启动机制}

平台实施多项保护机制,包括私募中单次购买的最高限额为 \$100K、锁仓释放确保 TGE 时无大规模抛售、时间加权投票防止治理攻击、渐进式释放防止挖矿抛售、社区分配大于团队 + 投资者(55\% > 27.5\%)。

\subsubsection{对比:传统质押 vs. veCPT 模型}

表~\ref{tab:comparison} 对比了传统质押与 veCPT 模型。

\begin{table}[htbp]
\centering
\caption{传统质押与 veCPT 模型对比}
\label{tab:comparison}
\begin{tabular}{lll}
\hline
\textbf{指标} & \textbf{传统质押} & \textbf{veCPT 模型} \\
\hline
最低承诺 & 无 & 1 周 \\
最高奖励 & 固定 APY & 最高 2.5x 增强 \\
治理权 & 线性(1 通证 = 1 票) & 时间加权 \\
长期一致性 & 低 & 高 \\
投机资本风险 & 高 & 低 \\
价格稳定性 & 较低 & 较高 \\
\hline
\end{tabular}
\end{table}

该模型有效的原因:Curve(\$CRV)已证明其有效性,且自 2020 年以来经过实战检验。它协调长期持有者的激励,减少短期挖矿者的抛售压力,创造强大的治理参与度,并提供不依赖于永久通胀的可持续通证经济。

\subsection{上市策略与保守场景}

\subsubsection{冷启动策略}

成功启动双边市场需要精心排序的阶段。我们的方法包括三个阶段。

\paragraph{阶段 0:种子用户(第 1--3 个月)}

目标是 50--100 名付费用户。来源包括 ClusterTech 现有客户群和 Web3 项目。激励措施包括 3 个月免费试用、早期采用者 50% 终身折扣以及初始 CPT 空投(总预算 10 万 CPT)。预算约为 \$150K(营销 + 激励)。

\paragraph{阶段 1:早期采用者(第 3--12 个月)}

目标是 500--1000 名付费用户和 10 家企业客户。策略包括推荐计划(推荐人和被推荐人各获得 \$50 信用额度)、通过技术博客和 YouTube 教程进行内容营销、黑客松赞助(Web3 社区)以及云经销商合作伙伴关系。预算约为 \$500K(营销 + 销售)。

\paragraph{阶段 2:增长(第 12--24 个月)}

目标是 2000--5000 名用户和 50 家企业客户。策略包括全面激活 CPT 质押激励、战略合作伙伴关系(Infura、Alchemy 等)以及会议出席和思想领导力。预算为 \$1M+(随收益增长)。

\subsubsection{财务场景}

为向投资者提供透明度,我们模拟了三种场景。

\paragraph{保守场景(高概率)}

表~\ref{tab:conservative} 展示了保守财务场景。

\begin{table}[htbp]
\centering
\caption{保守财务场景}
\label{tab:conservative}
\begin{tabular}{lrrr}
\hline
\textbf{指标} & \textbf{第 1 年} & \textbf{第 2 年} & \textbf{第 3 年} \\
\hline
付费用户 & 200 & 1,000 & 3,000 \\
ARPU(美元/月) & \$40 & \$60 & \$80 \\
MRR & \$8K & \$60K & \$240K \\
年度收益 & \$96K & \$720K & \$2.9M \\
运营成本 & \$600K & \$900K & \$1.5M \\
净收入 & -\$504K & -\$180K & +\$1.4M \\
累积现金 & -\$500K & -\$680K & +\$720K \\
\hline
\end{tabular}
\end{table}

\paragraph{基准场景(中概率)}

表~\ref{tab:basecase} 展示了基准财务场景。

\begin{table}[htbp]
\centering
\caption{基准财务场景}
\label{tab:basecase}
\begin{tabular}{lrrr}
\hline
\textbf{指标} & \textbf{第 1 年} & \textbf{第 2 年} & \textbf{第 3 年} \\
\hline
付费用户 & 500 & 2,500 & 8,000 \\
ARPU(美元/月) & \$50 & \$75 & \$100 \\
MRR & \$25K & \$188K & \$800K \\
年度收益 & \$300K & \$2.25M & \$9.6M \\
运营成本 & \$800K & \$1.5M & \$3M \\
净收入 & -\$500K & +\$750K & +\$6.6M \\
\hline
\end{tabular}
\end{table}

\paragraph{乐观场景(低概率)}

表~\ref{tab:optimistic} 展示了乐观财务场景。

\begin{table}[htbp]
\centering
\caption{乐观财务场景}
\label{tab:optimistic}
\begin{tabular}{lrrr}
\hline
\textbf{指标} & \textbf{第 1 年} & \textbf{第 2 年} & \textbf{第 3 年} \\
\hline
付费用户 & 1,000 & 5,000 & 20,000 \\
ARPU(美元/月) & \$75 & \$100 & \$150 \\
MRR & \$75K & \$500K & \$3M \\
年度收益 & \$900K & \$6M & \$36M \\
运营成本 & \$1M & \$2.5M & \$8M \\
净收入 & -\$100K & +\$3.5M & +\$28M \\
\hline
\end{tabular}
\end{table}

\paragraph{关键假设}

场景反映了不同的市场渗透率和定价权。运营成本随增长而增加,但受益于规模经济。保守场景假设团购贡献最小。所有场景均假设主要收益来自 SaaS 和交易费用。CPT 激励成本包含在运营成本中。

\paragraph{资金需求}

50 万至 100 万美元的种子/天使资金将覆盖第 1 年的亏损和产品开发。如果基准场景轨迹得到确认,计划在第 2 年进行 300 万至 500 万美元的 A 轮融资。计划在第 3 年及以后进行 1000 万至 2000 万美元的 B 轮融资,用于国际扩张。

\paragraph{盈亏平衡分析}

保守场景在第 30--36 个月达到盈亏平衡。基准场景在第 18--24 个月达到盈亏平衡。乐观场景在第 12--18 个月达到盈亏平衡。

这一范围为投资者提供了现实的预期,同时展示了可扩展潜力。
