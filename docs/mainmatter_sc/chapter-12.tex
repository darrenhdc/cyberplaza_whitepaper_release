\chapter{FAQ}

\section{常见问题}

\begin{enumerate}
\item \textbf{作为算力淘宝平台的用户,我能获得什么?}

\textbf{回答}:您可以从平台上列出的众多供应商中选择适合您使用的计算资源,包括CPU、GPU、FPGA算力、存储、应用软件和服务(例如,在特定硬件平台上优化您的软件,或将您的云应用从一个云厂商迁移到另一个云厂商)。您可以做出明智的服务选择,因为平台会评估并公示计算资源的性能,平台会保障服务提供商的服务水平协议(SLA),并以折扣价格(类似淘宝/京东)提供AWS、Azure、GCP以及众多计算中心和数据中心的资源。此外,通过使用平台,您将分享平台的所有权,并凭借获得的CPT代币分享平台的部分利润(一个您部分拥有的淘宝)。

\item \textbf{用户是谁?普通公众可能不是计算资源的主要用户。另一方面,许多机构客户可能无法参与代币经济。}

\textbf{回答}:目前,全球普通公众在公有云上的计算资源使用价值超过400亿美元,这确实只是机构客户使用量的一小部分。对于无法参与代币经济的机构客户,他们可以通过平台的渠道合作伙伴(见白皮书的合作章节)以通常的B2B方式购买计算服务,支付法定货币。

\item \textbf{一些机构服务提供商,例如AWS或中国的某超级计算中心,在提供服务时可能无法接受代币。用户如何通过平台使用它们的资源?}

\textbf{回答}:平台使用“储备基金”以法定货币购买这些服务提供商的服务。通过团购(拼多多)模式,平台可以提供折扣服务。

\item \textbf{为什么像AWS这样的云厂商会屈服于团购的压力?}

\textbf{回答}:我们的平台将成为AWS宝贵的销售渠道,为其提供接触Web 3和DeFi社区的机会。此外,在平台上众多服务提供商的竞争压力下,以及足够大的团购交易且“储备池”有一定预付款的情况下,折扣对所有云和计算资源厂商来说都是完全合理的。

\item \textbf{假设运营完美,算力淘宝平台可能有多少业务量?}

\textbf{回答}:AWS的年营收在2019年为350亿美元,2020年为450亿美元,2021年为620亿美元,2022年为814亿美元(根据Gartner的数据,其中约93%由机构客户消耗,7%由个人用户消耗)。如果我们将全球商业计算业务总价值视为AWS的7倍(2022年全球为5520亿美元,即AWS的7倍,来源:联合市场研究),那么2024年全球总营收将超过1万亿美元。如果算力淘宝平台能占据0.1%的市场份额,每年将超过10亿美元,且增长迅速。

\item \textbf{作为流动性提供者或CPT持有者参与,我能获得什么好处?}

\textbf{作为流动性提供者(存入USDC)}:参与者从平台运营利润中获得5--7\%的USDC年化收益率(APY),额外获得2--3\%的CPT代币年化收益率(带锁仓),总预期年化收益率为8--12\%,保持USDC流动性(可在提前通知的情况下提现),并在支持平台增长的同时获得可持续收益。

\textbf{作为CPT持有者/质押者}:参与者可以质押CPT获得8--12\%的年化收益率(锁定4年并使用boost可高达15--20\%),从平台利润的40\%中获得USDC收益分配,受益于通缩式回购销毁机制(20\%的营收),获得治理权(对平台方向进行投票),质押时可享受平台服务5--15\%的折扣,访问 premium 功能和优先支持,以及提前参与新产品发布。

\textbf{为什么这些收益是可持续的}:与算法稳定币或庞氏骗局不同,我们的收益来自真实的交易费用(市场活动的2--5\%)、团购利润(批量采购的10--20\%)、增值服务(认证、订阅、API),以及透明、可审计的收入流。

\item \textbf{为什么资金(无论是铸币者还是投资者)会选择加入算力淘宝平台,而不是其他Web 3项目?}

\textbf{回答}:详情请见“与其他‘资产代币化项目’的竞争分析”和“与其他Web 3计算资源项目的竞争分析”页面。简而言之:与其他资产相比,算力的价值增长更为迅速;我们的团队尤其具备建立算力淘宝平台的资质。

\item \textbf{一些潜在用户或服务提供商可能无法参与代币经济。他们如何参与?}

\textbf{回答}:这些客户在平台上找到合适的产品后,可以通过平台的合作伙伴(见第10节列出的代理商)购买。服务提供商也可以通过合作伙伴在平台上列出他们的产品。合作伙伴与用户/提供商之间的交易可以通过法定货币进行,无需涉及代币。

\item \textbf{一些消费者认为淘宝和拼多多上的产品质量较低。平台如何防范这种情况?}

\textbf{回答}:所有月列价超过10,000 USDC的产品必须经过平台认证。如上文第4节所述,平台要求服务提供商根据标准性能测试列出其服务的性能(包括High-Performance Linpack、High-Performance Conjugate Gradient、STREAM Sustainable Bandwidth、HPC Challenge、MLPerf、ResNet-50图像分类、BERT语言处理、CUDA Benchmark Suite、SPECviewperf图形性能、DeepBench等)。平台会定期验证服务提供商声称的性能,并将性能指标与服务价格一起列出供用户选择。

\item \textbf{为什么像AWS或华为云这样的公司想要在平台上销售他们的服务?}

\textbf{回答}:云计算公司目前向分销商提供折扣以销售他们的服务。分销商雇佣销售团队来销售这些服务。在某种意义上,平台充当这些厂商的分销商,不同之处在于,通过Web 3设置,厂商可以接触到Web 3和DeFi社区。

\end{enumerate}
