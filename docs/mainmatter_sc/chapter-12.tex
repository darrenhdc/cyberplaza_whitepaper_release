\chapter{FAQ}

\section{常见问题}

\begin{enumerate}
\item \textbf{作为算力淘宝平台的用户,我能获得什么?}

\textbf{答案}: 您可以选择平台上列出的众多提供商的计算资源供您使用,包括CPU、GPU、FPGA算力、存储、应用软件和服务(例如,在特定硬件平台上优化您的软件,或将您的云应用从一个云厂商迁移到另一个)。您可以做出明智的服务选择,因为计算资源的性能由平台评估并公示,SP的SLA由平台保障,且您可以以折扣价格使用AWS、Azure、GCP以及众多计算中心和数据中心的资源(类似淘宝/京东)。此外,通过使用平台,您将分享我们平台的所有权,并因此通过您获得的CPT分享平台的部分利润(一个您部分拥有的淘宝)。

\item \textbf{用户是谁?普通公众可能不是计算资源的主要用户。另一方面,许多机构客户可能无法参与代币经济。}

\textbf{答案}: 目前,全球普通公众在公共云上的计算使用价值超过400亿美元,这确实只是机构客户的一小部分。对于无法参与代币经济的机构客户,他们可以通过平台的渠道合作伙伴,以常规的B2B方式购买计算使用服务(见白皮书的合作部分),并支付法定货币。

\item \textbf{一些机构服务提供商,例如AWS或中国的某个超级计算中心,在提供服务时可能无法接受代币。我们的用户如何通过平台使用他们的资源?}

\textbf{答案}: 平台使用“储备基金”以法定货币购买这些服务提供商的服务。通过团购(拼多多),平台可以提供折扣服务。

\item \textbf{为什么像AWS这样的云厂商会屈服于团购的压力?}

\textbf{答案}: 我们的平台将成为AWS的宝贵销售渠道,为其提供接触Web 3和DeFi社区的机会。此外,由于平台上众多SP之间的竞争压力,以及有足够规模的团购交易(来自“储备池”的一定预付款),折扣对所有云和计算资源厂商来说都是完全合理的。

\item \textbf{假设运营完美,算力淘宝平台会有多少业务?}

\textbf{答案}: AWS的年度收入分别为2019年350亿美元、2020年450亿美元、2021年620亿美元、2022年814亿美元(根据Gartner的数据,其中约93%由机构客户消耗,7%由个人用户消耗)。如果我们将全球商业计算业务总价值定为AWS的7倍(2022年全球为5520亿美元,即AWS的7倍(Allied Market Research)),那么2024年全球总收入将超过1万亿美元。如果算力淘宝平台能占据总市场的0.1%,则每年收入将超过10亿美元,并将快速增长。

\item \textbf{作为流动性提供商或CPT持有者参与,我能获得什么收益?}

\textbf{作为流动性提供商(存入USDC)}: 参与者从平台运营利润中获得5--7\%的USDC年化收益率(APY),额外获得2--3\%的CPT代币年化收益率(带锁仓期),总预期年化收益率为8--12\%,保持USDC流动性(可在提前通知期后提取),并在获得可持续收益的同时支持平台增长。

\textbf{作为CPT持有者/质押者}: 参与者可以质押CPT以获得8--12\%的年化收益率(4年锁仓并附加激励可高达15--20\%),从平台40\%的利润中获得USDC收益分配,受益于通缩式回购销毁机制(20\%的收入),获得治理权(对平台方向进行投票),质押时可享受平台服务5--15\%的折扣,获得高级功能和优先支持,并能提前参与新产品发布。

\textbf{为什么这些收益可持续}: 与算法稳定币或庞氏骗局不同,我们的收益来自真实的交易手续费(市场活动的2--5\%)、团购利润(批量采购的10--20\%)、增值服务(认证、订阅、API)以及透明、可审计的收入流。

\item \textbf{为什么资金方(无论是铸币者还是投资者)愿意加入算力淘宝平台,而不是其他Web 3项目?}

\textbf{答案}: 详情请见“与其他‘资产代币化项目’的竞争分析”和“与其他Web 3计算资源项目的竞争分析”页面。简言之:与其他资产相比,算力的价值增长更快;我们的团队特别有资格建立一个算力淘宝平台。

\item \textbf{一些潜在用户或服务提供商可能无法参与代币经济。他们如何参与?}

\textbf{答案}: 这些客户在平台上找到合适的产品后,可以通过平台的合作伙伴购买(见第10节列出的代理商)。服务提供商也可以通过合作伙伴在平台上列出其产品。合作伙伴与用户/提供商之间的交易可以通过法定货币进行,不涉及代币。

\item \textbf{一些消费者认为淘宝和拼多多上的产品质量较低。平台如何防范这种情况?}

\textbf{答案}: 所有月标价超过\$10,000 USDC的产品必须经过平台认证。如上文第4节所述,平台要求服务提供商通过标准性能测试(包括High-Performance Linpack、High-Performance Conjugate Gradient、STREAM Sustainable Bandwidth、HPC Challenge、MLPerf、ResNet-50图像分类、BERT语言处理、CUDA Benchmark Suite、SPECviewperf图形性能、DeepBench等)列出其服务的性能。平台将定期验证服务提供商声称的性能,并将性能指标与服务价格一起列出,供用户选择。

\item \textbf{为什么像AWS或华为云这样的公司想在平台上销售他们的服务?}

\textbf{答案}: 云计算公司目前为销售其服务的分销商提供折扣。分销商雇佣销售团队来销售服务。从某种意义上说,平台充当这些厂商的分销商,只是借助Web 3的设置,厂商可以接触到Web 3和DeFi社区。

\end{enumerate}
