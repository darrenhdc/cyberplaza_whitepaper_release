\chapter{Overview of Operation}
\subsection{The Challenges we aim to address}

\begin{enumerate}
\item \textbf{Centralized Control}: The increasing significance of computing, specifically in the domains of Cloud Computing, High Performance Computing, and Artificial Intelligence (AI), is undeniable in our modern society. However, these crucial resources are predominantly controlled by major corporations, limiting the advantages to the majority of users. We believe that the solution lies in a decentralized marketplace, one that democratizes access to computing resources, fostering a more open and inclusive environment. In such a system, users are not just consumers, but also contributors who can influence the trajectory of computational development and have a stake in the future of computing.

\item \textbf{Inefficiency}: The current model of computing resource distribution often leads to imbalances, resulting in underutilization or oversaturation of resources. Our project aims to create a platform that efficiently matches the demand for computing power with available resources, thus optimizing utilization and minimizing waste.

\item \textbf{High Costs}: Currently, the majority of users face unnecessarily high computational costs. Our vision is to establish a marketplace platform that provides direct access to a wide range of computing power, storage solutions, software applications, data, and services at competitive prices. This not only reduces the overall costs but also widens the user base.

\item \textbf{Lack of Transparency}: The existing computing resource distribution system suffers from a lack of transparency regarding pricing, availability, and quality of service. We aim to build an open and impartial platform that empowers users to make well-informed decisions, backed by reliable information about resources, providers, and pricing.

\item \textbf{Lack of User Empowerment}: For most of us, executing ideas that require computation can be a cumbersome process, often requiring reliance on third-party services. For example, one has to rely on TV weather forecasts based on simulations carried out by a government agency, or one has to entrust personal data to a centralized entity in order to create a digital twin for oneself. Our project aims at a decentralized marketplace that provides all necessary computational resources, enabling users to perform any computation they want while maintaining complete control.
\end{enumerate}

For this important direction of development of the modern society, we need to solve the challenges of setting up a decentralized comprehensive ecosystem to enable a more accessible, and efficient allocation and utilization of computing resources.

\subsection{Outline of Our Solution}

\begin{enumerate}
\item We are launching a platform that operates as an open and democratic organization. The platform is akin to a marketplace for computing resources, reminiscent of platforms like Taobao (i.e., a ``算力淘宝平台''). Ownership of this setup is disseminated among all holders of the CyberPlaza Token (CPT), who are the ``shareholders'' of our platform.

\item \textbf{Payment System}: Our platform uses USDC for all transactions, ensuring regulatory compliance, price transparency, and familiar user experience. This eliminates risks associated with proprietary stablecoins and aligns with global regulatory frameworks.

\item On the Platform, Service Providers (SPs) list their computing resources---including computing power, storage, software applications, data and services---for users to select according to their needs. In exchange for their services, SPs receive USDC payments directly, along with CPT token incentives based on their transaction volume.

\item The platform itself does not own the computing resources listed. However, it can procure computing resources through ``group buying'' (團購) for reselling to users. This model is akin to the 拼多多 (Pinduoduo) business strategy, using a decentralized liquidity pool where community members can deposit USDC to earn returns while supporting platform operations.

\item Our platform is open and inclusive. There are no restrictions on anyone assuming any or all of the four roles: Platform ``shareholder'', Liquidity Provider, SP, and User. This flexibility enables participants to engage with the platform in ways that best suit their needs and capabilities.
\end{enumerate}