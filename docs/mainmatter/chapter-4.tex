\chapter{Description of the Roles in the Operation}

\section{The 4 roles of the Operation}

The platform ecosystem consists of four distinct roles: Platform Role, Service Provider (SP) Role, Liquidity Provider Role, and User Role. There is no restriction on anyone taking up or leaving any or all of the 4 roles.

\subsection{The Role of the Platform}

\subsubsection{Ownership and Participation}

The Platform is an open and democratic organization owned by all holders of the CyberPlaza Token, CPT. Anyone can obtain CPT by participating in the project through (i) providing services to the Platform, (ii) being a Liquidity Provider, (iii) being a Service Provider SP, (iv) being a User, or (v) buying CPT on the secondary market.

\subsubsection{Platform Functions}

The Platform serves as a distributor, match-maker and guarantor, ensuring trust and facilitating transactions between the Users, the Service Providers (SP's), and the Liquidity Provider. The Platform maintains a list of Certified SP's (CSP's), and a list of ``ordinary'' SP's. CSP's are those who offer value of service beyond a certain level (currently defined as ``providing services worth \$10,000+ USDC monthly for sales within the coming 10 days''). ``Ordinary'' SP's are those offering services below the threshold. The Platform will begin only with CSPs in its first stage, and introduce SPs later.

The Platform will evaluate the CSP's, considering factors such as their track records, reputation, and performance metrics of the CSPs, and list the evaluation results on the Platform so that Users can make informed decisions. The ``ordinary'' SP's are not evaluated and User uses them at their own choice. The Platform's role as a trusted intermediary adds a layer of accountability and increases the likelihood of the SPs fulfilling their commitments according to their SLA's. The Platform earns a portion of the transaction fee (difference between the price the Users paid and the price the SPs obtained).

\subsubsection{Reserve Fund Management}

The Platform is responsible for the operation of the ``Reserve Fund'' in USDC established by the Liquidity Provider who depositing USDC Tokens, which is the currency of the web 3 project, used for transactions in the marketplace. The Reserve Fund will generate interests for the USDC Tokens holders through various means, hence making the minting of the USDC Token a high-yield investment. These means include obtaining computing resources at a discount through group-buying (團購) for resell to the Users, i.e., carrying out a 算力拼多多(Pinduoduo) business with the Reserve Fund. The group-buying will be from major cloud services worldwide, including AWS, Azure, Google Cloud, Alibaba Cloud, etc., and supercomputing centers in the US, Europe and China etc. The platform also obtains profit through investing into revenue-generating computing assets, such as Bitcoin mining facilities, and obtains interests through decentralized finance or traditional finance investments which have high liquidity.

\subsubsection{Platform Participation and Future Expansion}

The Platform may also participate in the other roles (Liquidity Providers, SP and User) as needed to bootstrap liquidity and ensure service quality. The Platform may extend the 淘寶平台(Taobao) and the 拼多多 operation (Pinduoduo) to beyond 算力(computing resources) in the future, if the CPT holders vote for it through the governance mechanism.

\subsection{The Role of the Service Providers (SPs)}

\subsubsection{Registration and Service Listing}

The SPs register their services on the Platform to provide computility (core-hour, storage, bandwidth, application software, data and services etc.) to the Users. The SPs list on the Platform the availability of their computing resources for different periods (e.g. 1,000 core-hours of Intel Core i7 in the next 24 hours, 10,000 core-hours in the coming month) and the price list, for the Users to use/book. The SP will also post various benchmarks (as required by the Platform) of the resource they offer, as well as their SLA.

\subsubsection{Payment and Incentives}

The SPs receive USDC payments directly when their services are selected and used by Users. Additionally, they earn CPT token incentives proportional to their transaction volume (2--5\% of transaction value in CPT equivalent). SPs who stake CPT tokens can also receive reduced platform fees and enhanced visibility on the marketplace.

\subsubsection{Quality Assurance}

The platform's evaluation system verifies the quality and reliability of the CSPs, ensuring that all major SPs listed (CSPs) are trustworthy. By leveraging a reputation system, user reviews, and performance metrics, the Platform establishes a merit-based ranking system for the CSPs. The evaluation system let Users make informed decisions in selecting SP's for any substantial usage, reducing the chances of choosing an unreliable or unsuitable SP.

\subsubsection{Flexible Service Configuration}

An User may choose to use a combination of SP's for a job, e.g., the major part of the computation with a CSP, while the last leg of data analysis using an ``ordinary'' SP (e.g., the laptop put up by the User herself). The Platform provides evaluation for the CSP chosen but not non-certified SPs.

\subsection{The Role of the Liquidity Providers}

\subsubsection{Overview}

Liquidity Providers are participants who deposit USDC into the platform's decentralized lending pool to support operational capital for group-buying and platform operations. This role replaces the previous ``Enabler'' concept with a more transparent and decentralized model.

\subsubsection{How Liquidity Provision Works}

The liquidity provision mechanism operates as follows: Participants deposit USDC into audited smart contracts and receive rUSDC tokens (receipt tokens) representing their deposit. The Platform utilizes the pooled capital for group-buying operations and working capital. Participants can withdraw deposits subject to pool liquidity availability.

\subsubsection{Accessibility}

Anyone can become a Liquidity Provider by depositing USDC into the pool, including SPs, Users, and external investors. The minimum deposit is designed to be accessible while ensuring meaningful contribution.

\subsubsection{Returns and Benefits}

Liquidity Providers earn returns through multiple mechanisms. They receive Interest Earnings of 6--8\% APY paid in USDC from platform operational profits, and CPT Token Incentives providing an additional 2--4\% APY in CPT tokens (with vesting), resulting in a Total Expected Yield of 8--12\% APY combined. Beyond financial returns, they gain Governance Rights through CPT accumulation providing voting power, and Platform Benefits including reduced fees, priority access, and early product launches. Risk Protection is ensured through smart contract audits, an insurance fund (10\% coverage), and transparent tracking.

\subsection{The Role of the Users}

\subsubsection{Accessing Computing Resources}

Users can access computing resources on the Platform through a simple process: (i) Deposit USDC into their platform wallet, (ii) Browse and select Service Providers from the marketplace, and (iii) Submit jobs through the Platform Portal with USDC payment.

\subsubsection{Competitive Pricing}

By utilizing the Platform as a Taobao for computing resources 算力淘寶平台, Users have access to computing resources most suitable to them at competitive prices, often 10--30\% lower than direct purchasing from cloud providers due to group-buying benefits.

\subsubsection{Payment Protection and Transparency}

The platform implements comprehensive payment protection and transparency measures. Smart contract escrow holds USDC payments until service delivery is confirmed, with automatic refunds if SPs fail to meet SLA requirements. The system ensures transparent pricing with no hidden fees, real-time performance monitoring and reporting, and a dispute resolution mechanism through platform governance.

\subsubsection{User Incentive Program}

Users earn CyberPlaza Tokens (CPT) through platform engagement via multiple earning mechanisms. Consumption Rewards provide users with 1--3\% of spending amount in CPT tokens. Referral Bonuses allow users to earn CPT for bringing new users or SPs to the platform. Loyalty Tiers offer higher rewards for consistent platform usage, and Quality Feedback mechanisms enable users to earn CPT for providing detailed service reviews.

The benefits of holding and staking CPT are substantial. Usage Discounts allow users to stake CPT to receive 5--15\% discount on services. Revenue Sharing enables staked CPT to earn platform revenue distributions. Governance Rights permit voting on platform parameters and feature priorities. Premium Features provide access to advanced tools, analytics, and API services.